\documentclass[coverheight=240mm,
               coverwidth=175mm, 
               spinewidth=15mm,
               markcolor=black,
               bleedwidth=0mm,
               marklength=0mm]{bookcover}
% Font
\usepackage{fontspec}
\usepackage{tgadventor}
\usepackage{libertine}

\usepackage{lipsum,microtype}
\usepackage{xcolor}
\usepackage{fontawesome}
\usepackage{qrcode}

\begin{document}
    
% =================================
%              COVER
% =================================
\begin{bookcover}

% Font for the whole doc
\setmainfont{TeX Gyre Adventor}

% Color of the book, a kind of mustard
\definecolor{bookcolor}{HTML}{E1AD30}
%\definecolor{bookcolor}{HTML}{92000A}
\bookcovercomponent{color}{bg whole}{bookcolor!70}

% Color of the text
\definecolor{textcolor}{HTML}{000000}
\color{textcolor}


% -----------------------
%         SPINE
% -----------------------
\bookcovercomponent{center}{spine}[2mm, 3mm, 2mm, 3mm]{
    \noindent\rule[0.5em]{\partwidth}{1.5pt}
    \vfill
    \rotatebox[origin=c]{-90}{%
        \large\bfseries\scshape{%
        PhD Thesis \;\;--\;\;  LHC Effective Model For Optics Corrections \;\;--\;\; Maël Le Garrec}%
    }
    \vfill
    \noindent\rule[0.5em]{\partwidth}{1.5pt}
}


% -----------------------
%         FRONT
% -----------------------
% Text and picture on the front cover
% Minimum of 7mm margins with no text on it
\bookcovercomponent{normal}{front}[17mm, 20mm, 17mm, 20mm] % left, bottom, right, top
{
    % Redefine the geometry of the page
    % That's the same of the rest of the document, without the includehead and footer
    % Right and left are also the same
    % 3 lines
    \noindent\rule[0.5em]{\partwidth}{1.5pt}\vspace{-4pt}
    \noindent\rule[0.5em]{\partwidth}{1.5pt}\vspace{-4pt}
    \noindent\rule[0.5em]{\partwidth}{1.5pt}
    % Title
    \makebox[\partwidth][c]{\parbox{\partwidth-0.5cm}{
        \vspace{2.1cm}
        \begin{flushright}%
            \fontsize{35pt}{36pt}\selectfont%
            \bfseries
            \MakeUppercase{
                LHC Effective Model for Optics Corrections
            }%
        \end{flushright}
        % Small text
        \vspace{.1em}
        \fontsize{11pt}{15pt}\selectfont%
        \begin{flushright}%
            Measurements and corrections of high-order non-linear optics
        \end{flushright}
    }}
    % Big vertical space
    \vfill
    % Author
    \fontsize{15pt}{0pt}\selectfont%
    \bfseries\noindent\scshape Maël Le Garrec
    % 2 rules
    \par
    \vspace{0.3em}
    \noindent\rule[0.5em]{\partwidth}{1.5pt}\vspace{-2pt}
    \noindent\rule[0.5em]{\partwidth}{1.5pt}
}


% -----------------------
%         BACK
% -----------------------
% Text on the back cover
\bookcovercomponent{normal}{back}[17mm, 20mm, 17mm, 20mm]{
    % Lines
    \noindent\rule[0.5em]{\partwidth}{1.5pt}\vspace{-4pt}
    \noindent\rule[0.5em]{\partwidth}{1.5pt}
    \vspace{1.3cm}

    {\centering\bfseries\Large ABSTRACT\\[10mm]}%

    % Make a box slightly smaller than the width of the document
    \makebox[\partwidth][c]{\parbox{\partwidth-1.2cm}{
        %\fontspec{Linux Libertine O}
        \fontsize{10pt}{10pt}\selectfont
        This thesis investigates the crucial role of higher-order magnetic fields and non-linear optics in
        the stability and performance of particle accelerators, focusing on the Large Hadron Collider (LHC)
        at CERN. The control of non-linear optics, which deals with the interaction of charged particle 
        beams with complex magnetic fields such as sextupolar, octupolar, decapolar, and so on, is essential
        for managing beam dynamics. The LHC, as the world's most powerful accelerator, provides a unique
        opportunity to study these high-order effects, serving as a testbed for future accelerator designs.
        \\

        These higher-order fields significantly affect the beam's dynamic aperture and lifetime, especially
        at injection energy, where precise correction of magnetic field errors is required. Managing these
        challenges is not only vital for optimizing LHC performance but also for guiding the design and
        operation of next-generation machines.
        \\

        A key contribution of this work is the development of correction methods, based on a response matrix
        approach, for Resonance Driving Terms (RDTs), a critical factor in beam lifetime and dynamic
        aperture limitations. New corrective strategies for RDTs have led to notable improvements in beam
        lifetime and dynamic aperture at both injection and top energy operation. This thesis also addresses
        the discrepancies observed between experimental measurements and models of beam observables.
        \\

        These findings highlight the importance of precise modeling and correction of non-linear magnetic
        fields, offering insights that will benefit both the LHC and future high-energy particle
        accelerators.
    }}

    \vfill

    \centering
    \qrcode{https://github.com/Mael-Le-Garrec/PhD-Thesis/}
    \\[3mm]
    \faicon{github} Mael-Le-Garrec\\[0.5mm]
    \faicon{envelope} mael@legarrec.org

    \vspace{1cm}

    \noindent\rule[0.5em]{\partwidth}{1.5pt}%\vspace{-4pt}
    %\noindent\rule[0.5em]{\partwidth}{1.5pt}
}

\end{bookcover}
\end{document}