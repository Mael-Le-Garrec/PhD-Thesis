    % === Define the different glossaries
\newglossary*{nomenclature}{Nomenclature}
\newglossary*{symbols}{Symbols}

\makeglossaries


%==========================
%       Nomenclature
%==========================
\newglossaryentry{AC-Dipole}{
    type=nomenclature,
    name=AC-Dipole,
    description={
        Dipole magnet capable of generating variable oscillating fields. Used to increase the 
        transverse amplitude via forced oscillations for optics measurements.
    }
}
\newglossaryentry{Amplitude detuning}{
    type=nomenclature,
    name=Amplitude detuning,
    description={
        Tune shift dependent on the amplitude of transverse oscillations. Corrected via octupoles.
    }
}
\newglossaryentry{Chromatic Amplitude Detuning}{
    type=nomenclature,
    name=Chromatic Amplitude Detuning,
    description={
        Tune shift dependent on both the amplitude of transverse oscillations and the momentum
        offset. Corrected via decapoles.
    }
}
\newglossaryentry{Beta-beating}{
    type=nomenclature,
    name=Beta-beating,
    description={
        Relative difference of the beta function between measurement and model. Often expressed in 
        percents: $(\beta_{meas.} - \beta_{mdl})/\beta_{mdl.} \cdot 100$.
    }
}
\newglossaryentry{Beta-function}{
    type=nomenclature,
    name=Beta-function,
    description={
        Twiss parameter $\beta$ as a function of $s$, the longitudinal position. Related to the
        amplitude of transverse oscillations of the beam and its size.
    }
}
\newglossaryentry{Beam}{
    type=nomenclature,
    name=Beam,
    description={
        Short for "Particle beam". Beam 1 and Beam 2 refer to either of the two beams travelling in
        opposite directions in the LHC.
    }
}
\newglossaryentry{Drift}{
    type=nomenclature,
    name=Drift,
    description={
        Drift space, a field-free region.
    }
}
\newglossaryentry{Tune}{
    type=nomenclature,
    name=Tune,
    description={
        $Q$ -- Number of betatron oscillations per turn in a circular accelerator.
    }
}
\newglossaryentry{FeedDown}{
    type=nomenclature,
    name=Feed-down,
    description={
        Lower-order-like effects induced by a particle passing off-center through a multipole.
    }
}
\newglossaryentry{FeedUp}{
    type=nomenclature,
    name=Feed-up,
    description={
        Higher-order-like effects induced by a combination of multipoles.
    }
}
\newglossaryentry{Orbit}{
    type=nomenclature,
    name=Closed orbit,
    description={
        Path of the reference particle through the accelerator.
    }
}
\newglossaryentry{Laundau Octupole}{
    type=nomenclature,
    name=Laundau Octupole,
    description={
        Strong octupoles present in the LHC to introduce Landau Damping. The tune spread created via
        amplitude detuning helps damping the particles oscillations.
    }
}
\newglossaryentry{Crosstalk}{
    type=nomenclature,
    name=Crosstalk, 
    description={
        Unwanted magnetic interference between adjacent multipoles.
    }
}
\newglossaryentry{Chromaticity}{
    type=nomenclature,
    name=Chromaticity,
    description={
        Tune shift dependent on the momentum offset. Corrected via sextupoles to the first order.
    }
}
\newglossaryentry{Aperture}{
    type=nomenclature,
    name=Aperture,
    description={
        Physical aperture, i.e. are where the beam can pass, of an element in the accelerator.
    }
}
\newglossaryentry{Dynamic Aperture}{
    type=nomenclature,
    name=Dynamic Aperture,
    description={
        Maximum region in phase space where particle motion remains stable over time, beyond which
        particles may be lost.
    }
}
\newglossaryentry{Betatron coupling}{
    type=nomenclature,
    name=Coupling,
    description={
        Coupling of a particle's motion in the transverse planes. Corrected via skew quadrupoles.
    }
}
\newglossaryentry{MADX}{
    type=nomenclature,
    name=MAD-X,
    description={
        Current version of the Methodical Accelerator Design framework developed in BE-ABP. Used for
        beam dynamics simulations.
    }
}



%==========================
%         Acronyms
%==========================
\newacronym{lhc}{LHC}{
    Large Hadron Collider -- Largest and most powerful particle collider in the world.
}
\newacronym{bpm}{BPM}{
    Beam Position Monitor -- Intrumentation used to retrieve both position and intensity of 
    the beam via its induced electric field.
}
\newacronym{bbq}{BBQ}{
    Base Band Tune -- Precise tune measurement system consisting of a pick-up and filters.
}
\newacronym{ip}{IP}{
    Interaction Point -- Center of the straight arcs of the LHC. Beams collides in four of them
    where they cross (IP1, 2, 5, 8).
}
\newacronym{ir}{IR}{
    Interaction Region -- Vicinity of the interaction point. Often used interchangeably with "IP".
}
\newacronym{lsa}{LSA}{
    LHC Software Architecture -- Software used to operate the particle accelerators at CERN. Based
    on an online database to manage high and low level parameter settings.
}
\newacronym{md}{MD}{
    Machine Development -- Dedicated studies aimed at improving the accelerator parameters or
    testing new operational configurations.
}
\newacronym{omc}{OMC}{
    Optics Measurements and Corrections -- Name of the team dedicated to optics studies on the
    main CERN accelerators. Part of BE-ABP-LNO.
}
\newacronym{be}{BE}{
    Beams department -- Responsible for all aspects related to the production and delivery of
    particles, including beam physics, mechatronics, metrology, software development and operational
    management.
}
\newacronym{abp}{ABP}{
    Accelerators and Beam Physics group -- Responsible for studies and optimizations of beam
    dynamics over the complete CERN accelerator complex. Part of BE.
}
\newacronym{lno}{LNO}{
    Linear and Non-Linear Optics section -- In charge of optics studies for current and future
    accelerators at CERN. Part of BE-ABP.
}
\newacronym{rdt}{RDT}{
    Resonance Driving Term -- Coefficients related to the strength of a resonance.
}
\newacronym{RF}{RF}{
    Radio Frequency -- Shorthand for the acceleration system of the accelerator.
}
\newacronym{bch}{BCH}{
    Baker-Campbell-Hausdorff theorem -- Formula for the combination of exponentials in a Lie
    algebra, $e^X \cdot e^Y = e^Z$.
}
\newacronym{madx}{MAD-X}{
    Methodical Accelerator Design -- Current version of the framework developed in BE-ABP. Used for
    beam dynamics simulations.
}\newacronym{ptc}{PTC}{
    Polymorphic Tracking Code -- Framework used by MAD-X to perform calculations in the non-linear
    regime.
}


%==========================
%         Symbols
%==========================
\newglossaryentry{action}{
    type=symbols,
    name=$J_{x,y}$,
    description={
        Action - Phase space coordinate in the Courant-Snyder normalization $[\text{m}]$.
    }
}
\newglossaryentry{beta-star}{
    type=symbols,
    name=$\beta^*$,
    description={
       $\beta$-function at a given IP $[\text{m}]$.
    }
}
\newglossaryentry{brho}{
    type=symbols,
    name={$B \rho$},
    description={
        Magnetic rigidity. Quantifies a the ability of a field to deviate a particle $[\text{Tm}]$.
    }
}
\newglossaryentry{cminus}{
    type=symbols,
    name=$|C^-|$,
    description={
        Minimum tune separation. Global quantification of the linear coupling.
    }
}
\newglossaryentry{Jn}{
    type=symbols,
    name={$J_n$},
    description={
        Skew magnetic field strength. Skew field component of a multipole of order $n$, normalized
        to the magnetic rigidity $[\text{m}^{\text{-n}}]$.
    }
}
\newglossaryentry{Kn}{
    type=symbols,
    name=$K_n$,
    description={
        Normal magnetic field strength. Normal field component of a multipole of order $n$,
        normalized to the magnetic rigidity $[\text{m}^{\text{-n}}]$.
    }
}
\newglossaryentry{DQ}{
    type=symbols,
    name=$Q^{(n)}$,
    description={
        Chromaticity of order $n$. Orders up to three are generally denoted $Q'$, $Q''$ and $Q'''$.
    }
}