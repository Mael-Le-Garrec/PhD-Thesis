% === Define the different glossaries
\newglossary*{nomenclature}{Nomenclature}
\newglossary*{symbols}{Symbols}

\makeglossaries

% === Add the different entries
\newglossaryentry{Dipole}{type=nomenclature,name=Dipole, description={ Magnets with two poles, responsible for bending the particles in the accelerator. }}
\newglossaryentry{LBDS}{type=nomenclature,name=LBDS, description={LHC Beam Dump System }}
\newglossaryentry{Crosstalk}{type=nomenclature,name=Crosstalk, description={Interferences between two electronic circuits }}
\newglossaryentry{Laundau Octupole}{type=nomenclature,name=Laundau Octupole, description={Octupoles that introduce a spread in the beam, making it more stable }}
\newglossaryentry{BPM}{type=nomenclature,name=BPM, description={ Beam Position Monitor, gives the transverse position of the beam }}
\newglossaryentry{DOROS}{type=nomenclature,name=DOROS, description={ Low noise BPM. Currently can't be used with other BPMs due to synchronization issues  }}
\newglossaryentry{Dispersion}{type=nomenclature,name=Dispersion, description={ Change of orbit with momentum offset, mainly in the horizontal plane, created by the dipoles}}
\newglossaryentry{Coupling}{
    type=nomenclature,
    name=Coupling,
    description={ 
        Correlation between the motion of particles in horizontal or vertical plane to the other.
        Strong coupling negatively impacts the optics and is usually avoided. 
    }
}
\newglossaryentry{Emittance}{type=nomenclature,name=Emittance, description={ (\ensuremath{\epsilon}) Unit describing the beam in phase space. A low emittance indicates a beam with a small momentum offset and confined to a small distance }}
\newglossaryentry{beta-function}{
    type=nomenclature,
    name=Beta-function, 
    description={
        Variable of the twiss-parameters: $\beta$ as a function of the longitudinal position $s$.
        Related to the transverse beam size: $\sigma(s)= \sqrt{\epsilon \cdot \beta(s)}$ 
    }
}
\newglossaryentry{Chromaticity}{type=nomenclature,name=Chromaticity, description={ Tune change with momentum offset. Usually denoted as three orders: $Q'$, $Q''$ and $Q'''$ }}
\newglossaryentry{Aperture}{type=nomenclature,name=Aperture, description={ Maximum physical transverse size the beam can take in the accelerator without suffering losses }}
\newglossaryentry{Dynamic Aperture}{type=nomenclature,name=Dynamic Aperture, description={ Maximum stable aperture. Above that size, the particles become unstable and become lost }}
\newglossaryentry{Waist}{type=nomenclature,name=Waist, description={ Location where the $\beta$-function is at is minimum in an IP. $\beta^*$ refers to $\beta_{waist}$ }}
\newglossaryentry{Waist Shift}{type=nomenclature,name=Waist Shift, description={ Changing the waist to have $\beta^* = \beta_{IP}$ }}
\newglossaryentry{Rigid Waist Shift}{type=nomenclature,name=Rigid Waist Shift, description={ Doing a waist shift by powering all the triplets at once. No individual trim }}
\newglossaryentry{Orbit Feedback}{type=nomenclature,name=Orbit Feedback, description={ System responsible for acquisition and correction of the orbit }}
\newglossaryentry{ATS Factor}{type=nomenclature,name=ATS Factor, description={ Equivalent to the ratio of the virgin $\beta$-function to the $\beta$-function used in the current ATS scheme, at the edge of the arc  }}
\newglossaryentry{AC-Dipole}{
    type=nomenclature,
    name=AC-Dipole,
    description={
        Dipole magnet generating a variable oscillating field. Used to force beam oscillations for
        optics measurements.
    }
}


% ==== Acronyms
\newacronym{lhc}{LHC}{Large Hadron Collider}


% ==== Symbols
\newglossaryentry{action}
{
    type=symbols,
    name=action,
    symbol=$\mathcal{J}$,
    description={Action used as coordinate blabla},
}
