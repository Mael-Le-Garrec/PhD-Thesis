\chapter{\todo{Skew Octupolar Fields}}
\label{chapter:skew_octupole_fields}
\thumbforchapter{}
%\chaptertoc{}
%\newpage


%=============================
%        Introduction
%=============================
\section{\review{Introduction}}

The skew octupolar fields in the LHC have previously been identified as significant contributors to 
limits in forced dynamic aperture, being the dynamic aperture when kicking the beam with the
AC-Dipole~\cite{carlier_nonlinear_2020}. The skew octupolar correctors are positioned around the
ATLAS and CMS detectors, in Interaction Regions 1 and 5. Those correctors are \textit{common
aperture} magnets ; both beams are affected by the created magnetic fields. Unfortunately, one of
these four correctors, located to the left of ATLAS, is not functioning. As a result, although
corrections can be calculated, they will not effectively minimize the skew octupolar RDTs of
interest, $f_{1012,y}$ and $f_{1210,x}$. 
Their associated resonances and frequency lines are shown in
\cref{tab:skew_octupolar:resonances_rdts}.

\begin{table}[!htb]
    \centering
    \begin{tabular}{lccc}
      \toprule
      RDT         & Resonance                &  H-line                    & V-line         \\
      \midrule
      $f_{1012}$  & $\phantom{-}1Q_x - 1Q_y$ &  $\phantom{2Q_x-\ \,}1Q_y$ & $-1Q_x + 2Q_y$ \\
      $f_{1210}$  & $-1Q_x + 1Q_y$           &  $2Q_x - 1Q_y$             & $\phantom{-}1Q_x\phantom{+2Q_y\ \,}$    \\
      \bottomrule
    \end{tabular}
    \caption{Skew octupolar RDTs of interest, their associated resonances and the frequency spectrum
    lines they contribute to.}
    \label{tab:skew_octupolar:resonances_rdts}
\end{table}
  

%=============================
%         Top Energy
%=============================
\section{\todo{Corrections at Top Energy}}

The very first skew octupolar RDT corrections in the LHC were made in 2018 during
Run~2~\cite{carlier_nonlinear_2020}. Those corrections were computed by simulating the impact on
each corrector on the measured values, thus scanning the strengths of the correctors. This is
possible and remains viable as only three are used.
In this section, a different approach is taken, based on response matrices. This type of correction
is explained in details in \cref{correction_principle:response_matrix}. The real and imaginary 
responses of the RDTs for each corrector at a given strength are simulated through tracking.  These
responses are collected into a matrix, allowing the determination of the required strengths to match
the RDT level observed in the measurements. Inverting these values result in a correction.



%-----------------------------
%     Correctors Response
%-----------------------------
\subsection{\review{Correctors Response}}

To create a response matrix, simulations were conducted with the tunes and AC-Dipole deltas set to
those used for measurements. The natural tunes are $Q_x = 0.285$ and $Q_y = 0.292$ while the driven
tunes are $\Delta Q_x = -0.008$ and $\Delta Q_y = 0.01$.  Each corrector is then powered
individually for each tracking simulation. For this type of simulation, field errors are not
necessary, as only the RDT shift caused by the corrector relative to the baseline is needed.
\Cref{fig:skew_octupolar:response_correctors} shows the real part of the resulting RDTs from these
simulations for Beam 1. Beam 2 shows a similar level of response for these correctors.

\begin{figure}[!htb]
    \centering
    \begin{subfigure}{0.8\textwidth}
        \includegraphics[width=\textwidth]{./images/f1012_b1_correctors.pdf}
        \caption{$f_{1012,y}$}
    \end{subfigure}
    \par\bigskip 
    \begin{subfigure}{0.8\textwidth}
        \includegraphics[width=\textwidth]{./images/f1210_b1_correctors.pdf}
        \caption{$f_{1210,x}$}
    \end{subfigure}
    \caption{Simulation of the RDT response of the skew octupolar correctors at top energy for Beam
    1. Each corrector is powered at $J_4 = 1 [\text{m}^{-4}]$.}
    \label{fig:skew_octupolar:response_correctors}
\end{figure}

It can already be seen that the L5 and R5 correctors show a similar trend along the ring, with L5
showing a stronger response for the same strength, while the R1 corrector follows the opposite
trend. A polar plot at a given BPM can illustrate that trend, and be used to create manual
corrections. \Cref{fig:skew_octupolar:response_correctors_polar} shows the orthogonality of the
correctors for both beams and RDTs. L5 being stronger than R5 while having the same angle indicates
that only one of them is needed for corrections.

\begin{figure}[!htb]
    \centering
    \begin{subfigure}{0.8\textwidth}
        \includegraphics[width=\textwidth]{./images/orthogonal_a4_inj_f1012_y.pdf}
        \caption{$f_{1012,y}$}
    \end{subfigure}
    \par\bigskip 
    \begin{subfigure}{0.8\textwidth}
        \includegraphics[width=\textwidth]{./images/orthogonal_a4_inj_f1210_x.pdf}
        \caption{$f_{1210,x}$}
    \end{subfigure}
    \caption{Simulated RDTs response of the available skew octupolar correctors at top energy.  Each
    corrector is powered at $J_4 = 1 [\text{m}^{-4}]$. The orthogonality of R1 and L5/R5 allows to
    independently control the real and imaginary parts.}
    \label{fig:skew_octupolar:response_correctors_polar}
  \end{figure}



%-----------------------------
%         Correction
%-----------------------------
\subsection{\todo{Measurements and Corrections}}


\begin{figure}[!htb]
    \centering
    \begin{subfigure}{0.49\textwidth}
        \includegraphics[width=\textwidth]{./images/f1012_b1.pdf}
        \caption{$f_{1012,y}$ Beam 1}
    \end{subfigure}
    \hfill
    \begin{subfigure}{0.49\textwidth}
        \includegraphics[width=\textwidth]{./images/f1012_b2.pdf}
        \caption{$f_{1012,y}$ Beam 2}
    \end{subfigure}
    %
    \par\bigskip 
    % 
    \begin{subfigure}{0.49\textwidth}
        \includegraphics[width=\textwidth]{./images/f1210_b1.pdf}
        \caption{$f_{1210,x}$ Beam 1}
    \end{subfigure}
    \hfill
    \begin{subfigure}{0.49\textwidth}
        \includegraphics[width=\textwidth]{./images/f1210_b2.pdf}
        \caption{$f_{1210,x}$ Beam 2}
    \end{subfigure}
    \caption{Measured skew octupolar RDTs at top energy and $\beta^*=30\text{cm}$ before and after
    correction. A reduction if observed for all but one RDT in Beam 2.} 
    \label{fig:skew_octupolar:corrections_vs_bare}
\end{figure}


\begin{table}[!htb]
    \centering
    \begin{tabular}{lr}
      \toprule
      Corrector    &    Strength $[\text{m}^{-4}]$ \\
      \midrule
      MCOSX3.L1    &                —  \\
      MCOSX3.R1    &           $-0.50$ \\
      MCOSX3.L5    &           $ 0.42$ \\
      MCOSX3.R5    &           $-0.01$ \\
      \bottomrule
    \end{tabular}
    \caption{Computed corrections for skew octupolar RDTs at top energy. The corrector L1 has been 
    broken for several years and can not be used.}
    \label{tab:skew_octupolar:correction_strengths}
\end{table}
  


%\section{\todo{Correction of Skew Octupole Fields at Injection Energy}}
%
%\begin{figure}[H]
%    \includegraphics[width=\textwidth]{./chapters/07_octupoles/images/f1012_y_injection.pdf}
%    \caption{}
%    \label{fig:a4_injection_orthogonal_f1012}
%\end{figure}
%
%\begin{figure}[H]
%    \includegraphics[width=\textwidth]{chapters/07_octupoles/images/f1210_x_injection.pdf}
%    \caption{}
%    \label{fig:a4_injection_orthogonal_f1210}
%\end{figure}



\section{\todo{Landau Octupoles Contribution}}



%=============================
%        Conclusion
%=============================
\section{Summary}