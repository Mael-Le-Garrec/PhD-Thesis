\section{\review{Correction Principles}}

Several ways exist to correct the following observables: phase, $\beta$-beating, dispersion,
transverse coupling, amplitude detuning and RDTs. Two common methods to do so are here detailed.

% ===============================
%        Response Matrix
% ===============================
\subsection{\review{Response Matrix}}
\label{correction_principle:response_matrix}

A response matrix is a linear equation system that describes the change of an observable for a set
of individual multipole strengths. By taking the inverse of this matrix and multiplying it to the
measured observables, a set of corrector strengths is obtained that can replicate the measured
value. Taking the opposite sign then gives a correction. This technique is used to correct, amongst
others, $\beta$-beating, chromaticity as well as Resonance Driving Terms for the first time in the
LHC with this thesis. 
For some observables, like RDTs, it is common for the response matrix to contain over 500 values per
corrector and beam, one for each BPM.

Individual MAD-X simulations are run with a single specific circuit powered at a time. The resulting
parameter values (e.g. $\beta$-beating) are then compared to those obtained from a simulation
without any powering, allowing to determine the impact of each circuit on that observable.

A response matrix $R$ is thus created following \cref{eq:resp_matrix}, for a matrix of observables
$O$ with correctors powered on individually, a reference matrix of observables without any corrector
$O_R$, and a corrector strength $k$, the same for each simulation. Given measured data $M$, the set
of correctors needed to compensate the values can be obtained by taking the inverse of the response
matrix in \cref{eq:resp_matrix_inverted}.

\begin{equation}
  R = \left(O - O_R \right) \cdot \frac{1}{k}
  \label{eq:resp_matrix}
\end{equation}

\begin{equation}
  \begin{bmatrix}
    k_1 \\
    \vdots \\
    k_n \\
  \end{bmatrix}
  = -(R^{+} \cdot M)
  \label{eq:resp_matrix_inverted}
\end{equation}
 
Response matrices are very versatile and can combine several observables to be corrected by the same
multipoles. One example, detailed later in this thesis, is the correction of both the third order
chromaticity $Q'''$ and the Resonance Driving Term $f_{1004}$.

\subsubsection{\review{Example}}

In this example, developped for the decapolar corrections presented in \cref{chapter:decapoles},
simulations are run with MAD-X PTC to correct the third chromaticity in the LHC.  $Q'''$ is taken
from MADX-PTC for each beam and axis, with \verb|MCDs|, decapole correctors, powered with a
fixed strength one at a time. A scaling factor is applied to get the change of chromaticity for one
unit of $K_5$.  8 correctors are used, which strengths are denoted $k_1$ through $k_8$. Transposes
are only used to make the equations easier to display.\\
The values in \cref{table:resp_matrix_example} are corrected via
\cref{eq:resp_matrix_inverse_example} after having built the response matrix in
\cref{eq:resp_matrix_example}.

\begin{table}[H]
  \center
  \begin{tabular}{ccc}
    \toprule
      Observable & Value \\
    \midrule
      $Q'''_x$ & -666111 \\
      $Q'''_y$ &  121557 \\
    \bottomrule
  \end{tabular}
  \caption{Example of chromaticity values to be corrected via a response matrix.}
  \label{table:resp_matrix_example}
\end{table}

% ====
\vspace{0.4cm}
\begin{equation}
  R
  %
  =
  %
  \left(
    %{\text{Individual} \atop \text{simulations}}
    {\genfrac{}{}{0pt}{0}{\text{Individual}}{\text{simulations}}}
    \left\{
      \begin{bNiceMatrix}
       -155899  &  122004 \\  
       -254584  &  138368 \\
       -122715  &  106709 \\
       -218597  &  110686 \\
       -134140  &  106463 \\
       -245791  &  118951 \\
       -147035  &  116544 \\
       -219537  &  112317 \\
        \CodeAfter
        \OverBrace{1-1}{1-1}{Q'''_x}[yshift=2mm]
        \OverBrace{1-2}{1-2}{Q'''_y}[yshift=2mm]
      \end{bNiceMatrix}^T
    \right.
    -
    \left.
    \begin{bNiceMatrix}
       5135 \\
       8470 \\
      \CodeAfter
      \OverBrace{1-1}{1-1}{\scriptstyle \text{Reference}}[yshift=2mm]
    \end{bNiceMatrix}
    \right\}
    {\genfrac{}{}{0pt}{0}{Q'''_x}{Q'''_y}}
  \right)
  %
  \cdot
  %
  \underbrace{\frac{1}{-1000}}_{\text{Corrector strength}}
  \label{eq:resp_matrix_example}
\end{equation}
\vspace{0.5cm}


% Inverting the response matrix
\begin{equation}
    \begin{matrix}
      k_1 \\
      k_2 \\
      k_3 \\
      k_4 \\
      k_5 \\
      k_6 \\
      k_7 \\
      k_8 \\
    \end{matrix}
  \left\{
  \begin{pmatrix}
     -1235 \\
      1032   \\  
     -1394  \\ 
      1449   \\ 
     -1043  \\ 
      1864   \\ 
     -1187  \\ 
      1369   \\ 
  \end{pmatrix}
  \right.
  %
  =
  %
  -R^{+} 
  %
  \cdot
  %
  \left.
  \begin{pNiceMatrix}
      -666111 \\
      121557 \\
  \end{pNiceMatrix}
  \right\}
  %{\text{Measured} \atop \text{values}}
  {\genfrac{}{}{0pt}{0}{\text{Measured}}{\text{values}}}
  \label{eq:resp_matrix_inverse_example}
\end{equation}





% ===============================
%    Chromaticity Global Trim
% ===============================
\subsection{\review{Global Trims for Chromaticity}}
\label{subsection:correction_chromaticity}

% ~~~~~~~~~~~~~~~~~~~~~~~~~~~~~
% The script for the linearity can be found in
% /afs/cern.ch/work/m/mlegarr2/public/jupyter/chromaticity/simulations/linearity_dq3_mcd

As per the placement of the octupolar and decapolar correctors in the LHC 
layout~\cite{maclean_commissioning_2016-1}, $\beta$-functions at their location are slightly
different from arc to arc. This slight imbalance leads theoretically to the possibility of
correcting the horizontal and vertical axes of the second and third order chromaticity
independently, via a response matrix approach. In practice, the required strength to do so would
exceed those of the design of the correctors.

Another way to correct the chromaticity is via a global uniform trim, where every available
corrector is powered to the same strength. Simulations are run with MADX-PTC to obtain the response
in chromaticity for a given strength. Chromaticity being linear with multipole strength, a linear
function can be determined for each axis.
\Cref{fig:corrections-dq3_versus_k5} shows a simulation with several decapolar strengths,
highlighting this linear relation between the third order chromaticity $Q'''$ and $K_5$, while
\Cref{eq:corrections:chromaticity_affine_function_ptc} shows an example of such functions computed
at injection energy for the 2022 optics.

\begin{figure}[H]
  \centering
  \includegraphics[width=0.6\textwidth]{images/dq3_k5.pdf}
  \caption{Linear relation between the third order chromaticity and decapole corrector strengths,
           simulated with MADX-PTC.}
  \label{fig:corrections-dq3_versus_k5}
\end{figure}

\begin{equation}
  \begin{aligned}
    &Q'''_x = 1533 \cdot \Delta K_5 + 6680 \\
    &Q'''_y = -860 \cdot \Delta K_5 + 5647
  \end{aligned}
  \label{eq:corrections:chromaticity_affine_function_ptc}
\end{equation}

Only the linear part is relevant, as the offset is generated by other multipoles and field errors.
It is thus constant for a configuration where only the relevant correctors are used.
Corrections involve minimizing both $Q'''_x$ and $Q'''_y$, typically where both functions cross:

\begin{equation}
  \Delta K_5 = -\frac{(Q'''_x - Q'''_y)}{\text{slope}_{Q'''_x} - \text{slope}_{Q'''_y}}
  \label{eq:corrections:chromaticity_global_correction}
\end{equation}
