%=============================
%        Introduction
%=============================
\section{\review{Introduction}}

A key motivation for the non-linear optics studies conducted at the LHC is to demonstrate a strong
understanding and precise control of the optics through newly developed techniques, in order to
benefit and inform the design of future accelerators.

KEK, or the High Energy Accelerator Research Organization (Kō Enerugī Kasokuki Kenkyū Kikō), is a
Japanese institution operating the country's largest particle physics laboratory, located in
Tsukuba. KEK provides particle accelerators and essential infrastructure for a wide range of
research fields, including high-energy physics, material science, structural biology, and radiation
science. One of its most notable facilities is SuperKEKB, the world's most advanced
electron-positron collider. SuperKEKB is designed to reach an instantaneous luminosity of up to $80
\times 10^{34} \, \text{cm}^{-2}\text{s}^{-1}$ and has recently completed its commissioning phase.
It stores a 7 GeV electron beam in the High-Energy Ring (HER) and a 4 GeV positron beam in the
Low-Energy Ring (LER), which collide at a single interaction point where the Belle II experiment is
conducted. SuperKEKB currently holds the world record for instantaneous luminosity at $4.71 \times
10^{34} \, \text{cm}^{-2}\text{s}^{-1}$~\cite{zhou_luminosity_2023}, surpassing the previous record
set by the LHC. With a circumference of approximately 3 km, it is the largest lepton collider in
operation. Similar to CERN, KEK is a large complex housing multiple accelerators and facilities,
the injector chain for SuperKEKB being shown in \cref{fig:kek:layout_superkekb}.

\begin{figure}[!htb]
    \centering
    \includegraphics[width=0.8\textwidth]{./images/kek/layout_kekb.jpg}
    \caption{Schematic drawing of the accelerator complex at KEK~\cite{noauthor_operation_2024}.}
    \label{fig:kek:layout_superkekb}
\end{figure}

KEK maintains active collaborations with several organizations, including CERN, to advance research
in various areas of interest. SuperKEKB is of particular relevance to the FCC-ee, as it serves as a
prototype for testing innovative techniques in manufacturing, measurement, and analysis, whether
related to mechanical prototyping, element alignment, or beam dynamics. 

During a recent EAJADE~\cite{noauthor_eajade_nodate} secondment in Japan, optics measurement
techniques employed at CERN for the LHC, specifically turn-by-turn acquisitions, were applied and
tested on the two rings of SuperKEKB.  Previous studies on these techniques have been conducted and
are extended in this chapter. Instead of providing a detailed description, the focus is on updating
the methods previously used. Further details can be found
in~\cite{keintzel_jacqueline_beam_2022,keintzel_superkekb_2021,keintzel_impact_2021,thrane_measuring_2020}.



%=============================
%    Measurement Techniques
%=============================
\section{\review{Measurement Techniques for Linear Optics}}

In SuperKEKB, the beam optics are measured using either Closed Orbit
Distortion~\cite{ohnishi_optics_1999} or turn-by-turn measurements. As fast optics measurements can
be achieved with the Turn-By-Turn method, a measurement campaign was carried out to improve the
measurement quality. This includes investigating various beam excitation techniques and settings.
The necessary pre-processing steps for optics measurements have been detailed
in~\cite{keintzel_jacqueline_beam_2022}, and are not covered here.


%------------------------------
%    Closed orbit distortion
%------------------------------
\subsection{\review{Closed Orbit Distortion}}

Optics measurements using the COD method
\cite{harrison_global_1987,chung_measurement_1993,ohnishi_optics_1999} are well established and
routinely performed at SuperKEKB. In COD measurements, the beam is excited using six corrector
magnets, and the centroid orbit is recorded by 466 BPMs for the HER and 444 BPMs for the LER,
respectively. The optics of both transverse planes are then reconstructed using
analytical formulas. Since the correctors must be powered one at a time, the COD method is
relatively time-consuming. 
Since the average particle orbit is observed, the BPM readings are dependent on precise
calibration.
%One significant advantage of COD measurements at SuperKEKB is that approximately 6.5
%times more BPMs can be used compared to the TbT data.


%------------------------------
%        Turn-by-Turn
%------------------------------
\FloatBarrier
\subsection{\review{Turn-by-Turn}}

In SuperKEKB, 68 and 70 BPMs in the HER and LER rings are capable of recording turn-by-turn orbit
data, typically capturing several thousand turns in both transverse planes. Turn-by-turn
measurements are usually performed with a single bunch, with currents ranging from 0.2 mA to 1.5 mA.
The beam can be excited using three different methods.

First, an Injection Kicker (IK) delivers a single horizontal kick, causing the beam to oscillate to
large amplitudes. This amplitude then damps due to synchrotron radiation. The damping times for the
positron and electron rings are 46 ms and 53 ms, corresponding to 4600 and 5300 turns,
respectively~\cite{keintzel_jacqueline_beam_2022}. However, the IK only provides horizontal kicks,
limiting the precision of vertical optics measurements.

In contrast, the Phase-Locked Loop (PLL) method allows continuous excitation of the beam in both
horizontal and vertical planes. The PLL tracks the natural tune of the beam and drives it at the
corresponding frequency. A key advantage of the PLL system is that it can excite both planes
simultaneously, enabling measurements of transverse coupling and other resonance-driving terms
(RDTs). The PLL is has so far been the only method used to perform vertical turn-by-turn
measurements, and has not been utilized in this thesis.
%although data acquisition must be started manually, and a minimum bunch current of 0.5 mA is
%required. This method was not used for the data discussed in this thesis.
To induce oscillations in the vertical plane, the beam can be injected with an orbit offset. This
vertical offset causes the beam to oscillate as if it had received a single kick. The oscillations
then damp, and a new beam must be injected to repeat the measurement. This method has been
successfully used for the first time at SuperKEKB to measure vertical optics using turn-by-turn
data.

Once the turn-by-turn data is recorded, the standard analysis procedures outlined in
\cref{section:opticcs_meas} are applied. Specifically, the frequency spectrum is calculated before
reconstructing the optics. The model used for this analysis is generated with the SAD (Strategic
Accelerator Design) software \cite{noauthor_sad_nodate}.
A typical turn-by-turn signal is depicted in \cref{fig:kek:tbt_signal}.


\begin{figure}[!htb]
    \centering
    \begin{subfigure}[c]{0.47\textwidth}
        \includegraphics[width=\linewidth]{images/kek/horizontal_tbt_ler.pdf}
        \caption{Horizontal plane, oscillations are created by the Injection Kicker.}
    \end{subfigure}
    \hfill
    \begin{subfigure}[c]{0.48\textwidth}
        \includegraphics[width=\linewidth]{images/kek/vertical_tbt_ler.pdf}
        \caption{Vertical plane, oscillations are created via an injection offset.}
    \end{subfigure}
    \caption{Typical horizontal and vertical recorded turn-by-turn signals.}
    \label{fig:kek:tbt_signal}
\end{figure}



%------------------------------
%            GUI
%------------------------------
\subsection{\review{OMC3 Graphical User Interface}}

A crucial step in the analysis process is identifying faulty or noisy BPMs from the turn-by-turn
data. This task is often time consuming and susceptible to human error. In the LHC, a Graphical User
Interface (GUI) is employed to inspect the turn-by-turn signals before applying an FFT, which
reveals the spectral lines associated with the tunes. This tool plays a key role in the efficient
cleaning of outliers and noisy BPMs. As part of this thesis, the GUI has been updated to support
both the HER and LER rings. Using the latest version of the GUI provides access to recent bug fixes
and improvements, enhancing both usability and functionality. An example use case is shown in
\cref{fig:kek:gui_bad_bpms}, where BPMs delivering incorrect data across multiple measurements can
be easily detected.

\begin{figure}[!htb]
    \centering
    \includegraphics[width=0.8\textwidth]{./images/kek/GUIbadbpm.png}
    \caption{The identification of malfunctioning BPMs across measurements is easily done with the GUI.}
    \label{fig:kek:gui_bad_bpms}
\end{figure}



%=============================
%     Optics Observations
%=============================
\FloatBarrier
\section{\review{Optics Observations}}

All the measurements presented in this section have been performed in February 2024, during the
commissioning phase of SuperKEKB and are presented in \cref{tab:superkekb:configurations}.

%-----------------------------
%        Turn-by-Turn
%-----------------------------
\subsection{\review{Beta-Beating}}

Several configurations of the machine were measured using turn-by-turn acquisitions, as detailed in
\cref{tab:superkekb:configurations}. The configurations with $\beta_y^* = 48.6$~mm in LER and
$\beta_y^* = 81$~mm in HER are referred to as \textit{detuned}. Multiple kicks were performed for
each configuration to increase measurement precision. Although additional measurements were taken,
insufficient kick amplitude in some cases made it challenging to extract reliable linear optics data
and are thus here not included. Specifically, vertical measurements are challenging to perform as
oscillation amplitudes are often low due to the limited achievable injection offset. The low 
vertical action often makes measurements noisy.

\begin{table}
    \centering
    \begin{tabular}{llrrrrr}
    \hline
    Ring & Day & $\beta_x^*$ [mm] & $\beta_y^*$ [mm] & $Q_x$ & $Q_y$ & Kicks\\
    \hline
    LER        & 06 & 384 &\textbf{48.6} & 44.556 & 46.635 & H \& V \\
               & 09 & 384 &\textbf{48.6} & 44.553 & 46.621 & H  \\
               \hdashline
               & 20 & 200 & \textbf{8}   & 44.527 & 46.604 & H \\
               & 22 & 200 & \textbf{8}   & 44.535 & 46.590 & H \\
               &&&&&& \\
    HER        & 06 & 400 & \textbf{81}  & 45.572 & 43.616 & H \& V\\
               \hdashline
               & 20 & 200 & \textbf{8} & 45.530 & 43.595 & H \\
               & 22 & 200 & \textbf{8} & 45.535 & 43.596 & V \\
               & 26 & 200 & \textbf{8} & 45.535 & 43.596 & V \\
    \bottomrule
    \end{tabular}
  \caption{Configurations of the HER and LER rings for measurements performed via turn-by-turn
  acquisition. The day column informs on which date the measurement was performed in February 2024.}
  \label{tab:superkekb:configurations}
\end{table}

Both the \textit{detuned} and \text{8mm squeezed} optics are reliably measured for both rings, with
the exception of the vertical plane for the squeezed optics in LER due to the absence of injection
offset.
\Cref{fig:kek:beating_ler_detuned} and \cref{fig:kek:beating_her_detuned} show the measurements
taken with \textit{detuned} optics for LER and HER. For the squeezed optics, at 8mm, a comparison to
the measurements performed via COD is done in \cref{fig:kek:beating_ler_squeezed} and
\cref{fig:kek:beating_her_squeezed}, for LER and HER respectively. The agreement between the COD
and the turn-by-turn techniques is a good indication of their robustness, the difference between the
RMS difference of two methods being below 4\% for all measurements. The agreement found during these
measurements is similar to that obtained in previous studies~\cite{keintzel_jacqueline_beam_2022}.


% ==== LER

\begin{figure}[!htb]
    \centering
    \begin{subfigure}[b]{0.48\textwidth}
        \includegraphics[width=\linewidth]{images/kek/ler_06_09_bet_x.pdf}
        \caption{Horizontal beta-beating, RMS is $\approx 4\%$.}
    \end{subfigure}
    \hfill
    \begin{subfigure}[b]{0.48\textwidth}
        \includegraphics[width=\linewidth]{images/kek/ler_06_09_bet_y.pdf}
        \caption{Vertical beta-beating, RMS is $\approx 5\%$.}
    \end{subfigure}
    \caption{LER horizontal and vertical beta-beating for \textit{detuned} optics, reliably measured
    during two different days. The vertical plane is noticeably noisier as oscillation amplitudes
    are smaller.}
    \label{fig:kek:beating_ler_detuned}
\end{figure}

\begin{figure}[!htb]
    \centering
    \includegraphics[width=0.7\linewidth]{images/kek/ler_20_22_bet_x.pdf}
    \caption{LER horizontal beta-beating for \textit{8mm squeezed} optics, reliably measured during
    two different days. The vertical plane is absent due to amplitudes being too low to reconstruct
    linear optics. A comparison to the beating measured via COD is made. RMS is $\approx 6\%$.}
    \label{fig:kek:beating_ler_squeezed}
\end{figure}


% ==== LER

\begin{figure}[!htb]
    \centering
    \begin{subfigure}[b]{0.48\textwidth}
        \includegraphics[width=\linewidth]{images/kek/her_06_bet_x.pdf}
        \caption{Horizontal beta-beating, RMS is $\approx 6\%$.}
    \end{subfigure}
    \hfill
    \begin{subfigure}[b]{0.48\textwidth}
        \includegraphics[width=\linewidth]{images/kek/her_06_bet_y.pdf}
        \caption{Vertical beta-beating, RMS is $\approx 8\%$.}
    \end{subfigure}
    \caption{HER horizontal and vertical beta-beating for \textit{detuned} optics. The vertical
    plane is noticeably noisier as oscillation amplitudes are smaller.}
    \label{fig:kek:beating_her_detuned}
\end{figure}


\begin{figure}[!htb]
    \centering
    \begin{subfigure}[b]{0.48\textwidth}
        \includegraphics[width=\linewidth]{images/kek/her_20_bet_x_unzoomed.pdf}
        \caption{Horizontal beta-beating, RMS is $\approx 7\%$.}
    \end{subfigure}
    \hfill
    \begin{subfigure}[b]{0.48\textwidth}
        \includegraphics[width=\linewidth]{images/kek/her_22_26_bet_y_unzoomed.pdf}
        \caption{Vertical beta-beating, RMS is $\approx 5\%$.}
    \end{subfigure}
    \caption{HER horizontal and vertical beta-beating for \textit{8mm squeezed} optics. The vertical
    plane is noticeably noisier as oscillation amplitudes are smaller. A comparison to the beating
    measured via COD is made.}
    \label{fig:kek:beating_her_squeezed}
\end{figure}

\cref{tab:kek:summary_beating} shows a summary of the performed measurements and the RMS
$\beta$-beating.

\begin{table}
    \centering
    \begin{tabular}{llrr}
        \toprule
        Ring & Configuration & $\beta$-b. rms H & $\beta$-b. rms V \\
        \midrule
        LER  &  Detuned      & 4\%              & 5\%   \\
            &  Squeezed 8mm & 6\%              &       \\
        HER  &  Detuned      & 6\%              & 8\%  \\
            &  Squeezed 8mm & 7\%              & 5\%  \\
        \bottomrule
    \end{tabular}
    \caption{Summary of the measured $\beta$-beating with squeezed and detuned optics in the HER
    and LER rings.}
    \label{tab:kek:summary_beating}
\end{table}



%-----------------------------
%         LER Amp.Det.
%-----------------------------
\FloatBarrier
\subsection{\review{Amplitude Detuning}}

%-----------------------------
%         Tune Stability
\FloatBarrier
\subsubsection{\review{Tune Stability}}

To accurately measure linear and non-linear optics, reliable measurement of the tune is
essential~\cite{thrane_measuring_2020}. Ensuring tune stability across consecutive kicks is needed
to minimize measurement uncertainties and reflects the reproducibility of the machine. Using a
suitable kicker in the horizontal plane simplifies measurements, as indicated by the tune's 
stability across varying kick strengths. \Cref{fig:kek:shots} illustrates the variation in tune
from the first kick compared to subsequent kicks for both rings and planes.
While reproducibility is poor in the vertical plane of the LER, all other measurements agree within
few $10^{-3}$.

\begin{figure}[!htb]
    \centering
    \begin{subfigure}[b]{0.48\textwidth}
        \includegraphics[width=\linewidth]{images/kek/SUPERKEKBLER_shots.pdf}
        \caption{Tune stability for the LER ring.}
    \end{subfigure}
    \hfill
    \begin{subfigure}[b]{0.48\textwidth}
        \includegraphics[width=\linewidth]{images/kek/SUPERKEKBHER_shots.pdf}
        \caption{Tune stability for the HER ring.}
    \end{subfigure}
    \caption{Difference in tune between consecutive measurements, across several days.}
    \label{fig:kek:shots}
\end{figure}


%-----------------------------
%             LER
\FloatBarrier
\subsubsection{\review{Amplitude Detuning in LER}}

Measurement with higher kick amplitudes have been performed in the LER ring with \textit{detuned}
optics. These measurements stand out compared to previous ones due to their larger action range.
\Cref{fig:kek:ler_full_tune_ampdet} illustrates the various performed turn-by-turn measurements and
their corresponding action.
The tune is computed across each turn-by-turn measurement using a window of 200 turns, every 50
turns. A clear trend emerges with the tune increasing with the amplitude of the kicks. Over time,
the oscillations gradually dampen due to synchrotron radiation, and the tune reduces accordingly.

\begin{figure}[!htb]
    \centering
    \includegraphics[width=0.8\linewidth]{images/kek/LER_detuned_ampdet.pdf}
    \caption{Measured tune across the length of each turn-by-turn measurement.
             The kick action is shown in color. The tune is computed
             via a running window over 200 turns, every 50 turns.}
    \label{fig:kek:ler_full_tune_ampdet}
\end{figure}

The variation of the horizontal tune with the horizontal kick amplitude is related to the amplitude
detuning term $\frac{\partial Q_x}{\partial J_x}$. Its value can be retrieved by fitting the tune to
the action. \Cref{fig:kek:ler_ampdet} shows the various kicks along with a fit and a comparison to
the model. The measured value, $6500 \pm 500\;\text{m}^{-1}$ is one order of magnitude higher than
that simulated, of $600\;\text{m}^{-1}$. This observed discrepancy indicates possible unmodeled
octupolar-like sources. Such a discrepancy, although not as large, already has been observed with
different optics~\cite{keintzel_jacqueline_beam_2022}.

\begin{figure}[!htb]
    \centering
    \includegraphics[width=0.7\linewidth]{images/kek/amplitude_detuning.pdf}
    \caption{Amplitude dependence of the tune in the LER ring. The slope of the fitted line corresponds to
    the amplitude detuning term $\partial Q_x/\partial 2J_x$. For comparison, the tune computed 
    from the model is shown in green.}
    \label{fig:kek:ler_ampdet}
\end{figure}



%-----------------------------
%   Resonance Driving Terms
%-----------------------------
\FloatBarrier
\subsection{\review{Resonance Driving Terms}}

Resonance Driving Terms, coefficients linked to the strength of a resonance, can be measured via
their associated line amplitude in the frequency spectrum. To reliably measure RDTs, these
line amplitudes must be clearly distinguishable and above the noise level for every BPM. This can be
challenging to achieve and often requires high amplitude kicks.
For example, lines in the vertical spectrum, attributed to octupoles, were previously
observed~\cite{keintzel_jacqueline_beam_2022} at SuperKEKB but could not lead to a successful RDT
measurement due to kick amplitudes.

However, new measurements were taken with a different working point in order to be closer to the 
resonances with the hope of increasing their impact on particle motion. A resonance diagram,
illustrating those resonances and the typical working point of LER is shown in
\cref{fig:kek:tune_diagram}. Although several working points were tested, no clear correlation
between them and successful RDTs measurements could be established. The main determining factor
remained the kick amplitude. A typical frequency spectrum from a successful measurement in HER for
the vertical plane is shown in \cref{fig:kek:rdt_spectrum_HER}. The line $-1Q_x - 1Q_y$ is seen at
each BPM. Additionally, lines near the vertical tune are attributed to the synchrotron tune.

\begin{figure}[!htb]
    \centering
    \includegraphics[width=0.6\linewidth]{images/kek/tune_diagram.png}
    \caption{Resonance diagram highlighting the resonances close to the HER working point, in red, 
    and those often visible in the frequency spectra, in green.}
    \label{fig:kek:tune_diagram}
\end{figure}

\begin{figure}[!htb]
    \centering
    \includegraphics[width=0.8\linewidth]{images/kek/HER_2024-02-06_sextupoles_spectrum.pdf}
    \caption{Vertical frequency spectrum of HER, showing the natural tunes as well as a line
    excited by the sextupolar RDT $f_{1020}$.}
    \label{fig:kek:rdt_spectrum_HER}
\end{figure}

Successful measurements of sextupolar RDTs were indeed obtained. Specifically, the RDTs $f_{3000,x}$
in LER and $f_{1020,y}$ in HER were measured using the \textit{detuned} optics configuration. While
additional sextupolar lines were clearly visible in both the horizontal and vertical spectra, no
clear RDT measurements could be extracted as they were not seen for each BPM. The measured
amplitudes, along with comparisons to the model, are presented in \cref{fig:kek:rdt_f3000x_LER} and
\cref{fig:kek:rdt_f1020y_HER}. Notably, both measurements exhibit discrepancies with the model.  The
average amplitude of $f_{1020}$ in the vertical plane for HER is measured at $21\;\text{m}^{-1/2}$, 
compared to a modeled amplitude of $7\;\text{m}^{-1/2}$. Similarly, for $f_{3000}$, the average
values measured in the horizontal plane for LER are $2\;\text{m}^{-1/2}$ and $1.3\;\text{m}^{-1/2}$.
These discrepancies remain largely unexplored and may arise from decoherence
effects~\cite{tomas_direct_2003} that were not accounted for in the analysis software. 
Additionally, these could also arise from contributions from other multipoles through feed-down, as
observed in the LHC, or unknown sextupolar sources. The measured spikes in the LER RDT are not yet 
explained, but could come from a badly reconstructed phase space due to non-optimal phase advances 
between BPMs in that region or inverted BPM plane signal~\cite{frank2024private}.

\begin{figure}[!htb]
    \centering
    \includegraphics[width=0.8\linewidth]{images/kek/f1020y_HER.pdf}
    \caption{Sextupolar RDT $f_{1020,y}$ measured in HER with \textit{detuned} optics and compared to 
    the model.}
    \label{fig:kek:rdt_f1020y_HER}
\end{figure}

\begin{figure}[!htb]
    \centering
    \includegraphics[width=0.8\linewidth]{images/kek/f3000x_LER.pdf}
    \caption{Sextupolar RDT $f_{3000,x}$ measured in LER with \textit{detuned} optics and compared
    to the model.}
    \label{fig:kek:rdt_f3000x_LER}
\end{figure}



%-----------------------------
%        Chromaticity
%-----------------------------
\FloatBarrier
\subsection{\review{Chromaticity}}

Chromaticity measurements were performed using the method of varying the RF frequency over large
ranges while recording the tune via pickups, as described in
\cref{subsection:optics_corrections_chromaticity}. While this method has now been made standard at
the LHC, it is not used on the SuperKEKB rings, where only a few data points are collected over a
small momentum offset range.
%However, it provides the ability to measure the tune over a broader range of momentum offsets. 
%The tune was previously measured with turn-by-turn data, using the Injection %Kicker~\cite{keintzel_jacqueline_beam_2022} to produce fast kicks~\cite{keintzel_jacqueline_beam_2022} that limit losses. 
Similar to the LHC at injection energy, the Lorentz factor $\gamma^{-2}$ is small compared to the
momentum compaction factor $\alpha_c$ and can therefore be neglected.
\Cref{fig:kek:chroma_procedure} illustrates the RF frequency method.

\begin{figure}[!htb]
    \centering
    \includegraphics[width=0.8\linewidth]{images/kek/rf_qx.png}
    \caption{Horizontal tune change relative to the change in RF Frequency at SuperKEKB.}
    \label{fig:kek:chroma_procedure}
\end{figure}

Two measurements were taken with \textit{detuned} optics for the HER and LER rings and are shown in
\cref{fig:kek:chroma_HER_detuned} and \cref{fig:kek:chroma_LER_detuned}. The values of the fitted
chromaticity function are given in \cref{tab:kek:her_chroma_detuned} and
\cref{tab:kek:ler_chroma_detuned} for HER and LER respectively. In order to allow for a better
comparison the tune and the linear chromaticity are taken from the the measurements.

\begin{figure}[!htb]
    \centering
    \begin{subfigure}[b]{0.49\textwidth}
        \includegraphics[width=\linewidth]{images/kek/chromaticity/HER_09/qx_modelq0q1.pdf}
        \caption{Horizontal tune shift.}
    \end{subfigure}
    \begin{subfigure}[b]{0.49\textwidth}
        \includegraphics[width=\linewidth]{images/kek/chromaticity/HER_09/qy_modelq0q1.pdf}
        \caption{Vertical tune shift.}
    \end{subfigure}
    \caption{HER chromaticity measurements with \textit{detuned} optics.}
    \label{fig:kek:chroma_HER_detuned}
\end{figure}

\begin{figure}[!htb]
    \centering
    \begin{subfigure}[b]{0.49\textwidth}
        \includegraphics[width=\linewidth]{images/kek/chromaticity/LER_09/qx_modelq0q1.pdf}
        \caption{Horizontal tune shift.}
    \end{subfigure}
    \begin{subfigure}[b]{0.49\textwidth}
        \includegraphics[width=\linewidth]{images/kek/chromaticity/LER_09/qy_modelq0q1.pdf}
        \caption{Vertical tune shift.}
    \end{subfigure}
    \caption{LER chromaticity measurements with \textit{detuned} optics.}
    \label{fig:kek:chroma_LER_detuned}
\end{figure}

% HER Detuned
\begin{table}[!htb]
    \centering
    \begin{tabular}{cccc}
        \toprule
            & & \(Q'' \, [\times 10^3]\) & \(Q''' \, [\times 10^6]\) \\ 
        \midrule
        \multirow{2}{*}{X} & Meas. & $-0.51 \pm 0.01$ & $0.11 \pm 0.02$ \\
                        & Model & $-0.12 \pm 0.00$ & $0.09 \pm 0.00$ \\
        \midrule
        \multirow{2}{*}{Y} & Meas. & $0.00 \pm 0.0$ & \\
                        & Model & $-0.04 \pm 0.0$ & \\
        \bottomrule
    \end{tabular}
    \caption{Measured and modeled chromaticity in the HER ring with \textit{detuned} optics in both
    planes.}
    \label{tab:kek:her_chroma_detuned}
\end{table}

% LER Detuned
\begin{table}[!htb]
    \centering
    \begin{tabular}{ccrrr}
        \toprule
            & & \(Q'' \, [\times 10^3]\) & \(Q''' \, [\times 10^6]\) & $Q^{(4)} [\times 10^9]$\\ 
        \midrule
        \multirow{2}{*}{X} & Meas. & $-0.01 \pm 0.0$ & $ 0.02 \pm 0.0$ \\
                        & Model & $ 0.04 \pm 0.0$ & $-0.01 \pm 0.0$ \\
        \midrule
        \multirow{2}{*}{Y} & Meas. & $0.35 \pm 0.01$ & $0.25 \pm 0.01$ & $-0.09 \pm 0.01$\\
                        & Model & $0.10 \pm 0.00$ & $0.32 \pm 0.00$ & $-0.09 \pm 0.00$ \\
        \bottomrule
    \end{tabular}
    \caption{Measured and modeled chromaticity in the LER ring with \textit{detuned} optics in both
    planes.}
    \label{tab:kek:ler_chroma_detuned}
\end{table}

It can be noted that most of the measured third and fourth-order chromaticities agree well with the
model, whereas the second-order chromaticity appears to deviate. Discrepancies in $Q''$ were already
observed in 2022 with different optics~\cite{keintzel_jacqueline_beam_2022}. This discrepancy could
arise from octupolar sources, higher-order contributions from sextupoles or quadrupoles and
feed-down.



%=============================
%         Conclusion
%=============================
\FloatBarrier
\section{\review{Summary}}

The EAJADE secondment at SuperKEKB, during its February 2024 commissioning, focused on applying the
optics measurement techniques used at CERN to the HER and LER rings. SuperKEKB, an electron-positron
collider, serves as a testbed for the FCC-ee, which is currently in its design phase.

Linear optics measurements, performed with turn-by-turn acquisition, showed good agreement with the
conventional Closed Orbit Distortion (COD) method. These results were comparable to measurements
taken several years earlier, demonstrating good repeatability across consecutive kicks and over
multiple days. Additionally, vertical plane measurements with an injection offset were performed for
the first time on both rings, yielding promising results.

The scope of the optics measurements was then extended to include non-linear optics, covering
chromaticity, amplitude detuning, and Resonance Driving Terms (RDTs). Chromaticity measurements for
both rings were conducted over large momentum offset ranges, showing good agreement with the model
for the linear term, although some discrepancies were observed in $Q''$ for both rings.

Amplitude detuning was measured in the LER with detuned optics, and comparison with the model
revealed an order of magnitude difference. When combined with the chromaticity results, this
suggests potential errors due to unmodeled octupole-like sources, warranting further investigation
to resolve these discrepancies.

Measuring Resonance Driving Terms (RDTs) proved challenging, as they require high-amplitude kicks
and consistent repeatability across multiple kicks. Nevertheless, $f_{1020}$ and $f_{3000}$ were
successfully measured in the HER and LER rings for the first time. While certain discrepancies with
the model remain unresolved, they may be attributed to factors such as unaccounted-for decoherence,
damping, or unknown sextupole-like sources.

Overall, these measurements are consistent with those obtained through alternative methods at KEK.
The successful observation of RDTs further suggests that the techniques used at CERN are effective
and could enhance our understanding of not only SuperKEKB and its modeling, but also of future
accelerators such as the FCC-ee.
