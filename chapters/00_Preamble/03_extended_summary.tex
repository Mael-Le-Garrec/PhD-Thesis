% =============================
%       Extended Summary
% =============================
\chapter{\review{Extended Summary}}


\ifthenelse{\equal{\papersize}{A4}}{ % A4
    \newcommand{\fontsizesummary}{12pt}
    \newcommand{\fontskipsummary}{15pt}
}{  % else, B5
    \newcommand{\fontsizesummary}{11pt}
    \newcommand{\fontskipsummary}{11pt}
}



{
% 5 pages is reeaally long, let's just add some spacing so it's easier for the reader to follow
\fontsize{\fontsizesummary}{\fontskipsummary}\selectfont

% Introduction
The Large Hadron Collider (LHC) at CERN, situated in a 27-kilometer tunnel beneath the Swiss-French
border, is the world's largest and most powerful particle accelerator. Its primary mission is to
recreate the conditions of the universe just moments after the Big Bang, enabling scientists to
explore the fundamental forces and particles that constitute the cosmos. This remarkable facility
accelerates protons and heavy ions to nearly the speed of light before colliding them with immense
energy, providing profound insights into the building blocks of matter and the fundamental
interactions that govern our universe. The LHC represents not only a monumental engineering
achievement but also a crucial tool for advancing our understanding of high-energy physics. \\
\indent
Operating such a complex and powerful machine requires overcoming numerous technical challenges,
particularly in maintaining the precise control and stability of its particle beams. One significant
challenge lies in managing the effects of higher-order magnetic fields, which often arise from field
errors in the multipoles used to guide and focus the particle beams. These field errors,
including from quadrupolar up to decahexapolar components, can profoundly impact beam dynamics and
stability. Addressing these challenges is essential to ensuring that the LHC operates effectively
and continues to produce valuable scientific results. The following extended summary synthesizes
findings from three detailed studies investigating various aspects of higher-order multipole effects
and their implications for the LHC's performance.

% Concepts
This thesis employs the Hamiltonian formalism to describe particle motion in the transverse planes
under the influence of various multipole fields. In non-linear lattices, the complexity of beam
dynamics increases significantly, necessitating the use of advanced mathematical tools such as Lie
Algebra and Poisson Brackets to accurately characterize non-linear effects. The study derives
explicit higher-order non-linear transfer maps and provides a comprehensive summary of multipole
combinations. These non-linearities in the lattice lead to intricate phenomena such as chromaticity,
amplitude detuning, chromatic amplitude detuning, and resonances driven by Resonance Driving Terms
(RDTs), all of which are thoroughly derived and supported by detailed measurement techniques. \\
\indent
Optics measurements are conducted using a range of software tools and techniques. Turn-by-turn data
acquisition via Beam Position Monitors (BPMs) is emphasized as a crucial method for evaluating beam
optics, where an AC-dipole excites the beam, and the resulting oscillations are analyzed using
Fourier transforms to extract tunes and identify resonances.  Further treatment is done via the
oscillation amplitudes and the magnitude of spectral lines to retrieve linear and non-linear
observables such as the phase advance, beta function, dispersion, coupling, orbit, and RDTs.
Chromaticity measurements involve inducing momentum offsets by varying the RF frequency and
observing the corresponding tune shifts. The thesis also introduces a custom tool, the Non-Linear
Chromaticity GUI, which streamlines the analysis and correction of chromaticity by fitting measured
data to higher-order chromaticity functions and implementing necessary adjustments. A key correction
strategy discussed is the response matrix, a linear equation system that describes how variations in
multipole strengths impact observables, which has proven effective in correcting both linear and
non-linear observables in the LHC.

% First Study
The first study explores the origins and consequences of skew octupolar fields within the LHC. These
fields significantly influence the dynamic aperture of the accelerator, a critical parameter that
defines the range within which the particle beam remains stable. Skew octupolar correctors are
installed around key detectors, such as ATLAS and CMS, to manage these fields and mitigate their
effects on beam stability. The study focuses on measuring these fields with optics designed for top
energy, 6.8 TeV per particle. Corrections were performed using a response matrix method, a different
approach than what was used a few years prior. This method effectively addresses skew octupolar RDTs
using the available corrector magnets, although its performance is limited by the absence of one
corrector, which constrains the achievable correction strength. Consequently, certain RDTs, like
$f_{1012}$ and $f_{1210}$, can either be effectively corrected or maintained at a constant level
depending on the corrector configuration. The level of correction achieved is comparable to that
obtained using a different method in the last LHC Run.

\indent
Additionally, the study investigates the unexpected influence of Landau octupoles on skew octupolar
RDTs at injection energy, 450 GeV per particle. Landau octupoles are powerful octupoles used at 
injection energy to introduce damping through a tune spread.  During measurements, a significant
shift in skew octupolar RDTs was observed with various powerings of the Landau octupoles. Initially,
these octupoles were expected to generate skew fields only with misalignments, specifically roll
errors.  However, simulations indicate that octupole misalignments have minimal impact on skew
octupolar fields. Instead, coupling has been identified as a crucial factor. The combination of
coupling and Landau octupoles is expected to be a major contributor to skew octupolar RDTs at
injection energy, where octupoles are strongly powered. Accurate modeling of coupling is thus
essential for predicting the behavior of skew octupolar RDTs and managing their impact on beam
stability.

% Second Study
The second study focuses on decapolar fields in the LHC, particularly at injection energy. This
study addresses previously observed discrepancies between measurements and models related to
third-order chromaticity, a critical parameter that describes how particles with an energy deviation
experience different oscillation frequencies than the reference particle. Accurate control of
chromaticity is vital for maintaining beam stability. The introduction of previously unobserved
observables, has provided a clearer understanding of these discrepancies. Among these observables
are the bare chromaticity, which represents the chromaticity of the machine without any correctors
powered on to observe the bare influence of field errors, and chromatic amplitude detuning, a
detuning function of both momentum deviations and amplitude. Several approaches to measuring the
same fields help clarify the various contributions. The research reveals that the decay of the
decapolar component in the main dipoles is a significant factor contributing to these discrepancies.
When the LHC was designed, this decay was deemed too small to be significant and thus was not
included in the magnetic error tables used for simulation. However, as the machine's parameters are
pushed further each year and the effects of higher-order fields become better understood, it becomes
clear that accurate control and modeling of these fields are necessary.\\
\indent
For the first time, measurements and corrections of the decapolar Resonance Driving Term $f_{1004}$
were carried out at injection energy. Implementing combined corrections for third-order
chromaticity, chromatic amplitude detuning, and RDT $f_{1004}$ led to a 3\% improvement in beam
lifetime. Conversely, deliberately degrading the RDT alone resulted in a 10\% decrease in beam
lifetime, underscoring the importance of this resonance corrections for stable beam operation. The
study also explored how sextupoles and octupoles interact to generate decapolar-like fields. It was
found that sextupoles, both alone and in combination with Landau octupoles, produce a substantial
$f_{1004}$ decapolar RDT when powered to small currents. Therefore, in an operational context,
decapolar resonances, largely generated by strong octupoles, would benefit from adapted corrections.
These findings suggest that further advancements in correction methods could lead to even greater
improvements in beam stability.

% Third Study
The third study investigates higher-order fields in the LHC, specifically dodecapolar and
decatetrapolar fields. Using a newly implemented collimation setup and custom post-processing
techniques, this study successfully observed these higher-order fields. Studies were conducted to
estimate the effect of the non-linearity of the momentum compaction factor on the chromaticity
function during its computation from the RF frequency. The results indicate that while the momentum
compaction factor shows a second order in the LHC, its effects on the resulting chromaticity are
negligible even at large momentum offsets. Several chromaticity measurements with varying
configurations of octupolar and decapolar corrections then revealed the presence of fourth and
fifth-order terms ($Q^{(4)}$ and $Q^{(5)}$). These measurements consistently identified these
higher-order terms with similar values, demonstrating their robustness. Additionally, it is
emphasized that accurately characterizing the lower-order terms requires good measurement of these
higher-order terms. The study identifies, through simulations, dodecapolar and decatetrapolar fields
as primary contributors to these higher-order effects, originating from field errors in the main
dipoles and quadrupoles. The LHC's field error model appears to be in relative agreement with the
measurements once the decay of decatetrapolar components is considered. \\
\indent
For the first time at injection energy, the dodecapolar Resonance Driving Term $f_{0060}$ was
measured. This measurement shows clear repeatability, even when performed with different
configurations of octupolar and decapolar corrections. The measured values were found to be in good
agreement with the model. The research concludes that further investigations are needed to address
limitations in the measurement range of the chromaticity function and to refine estimates of
higher-order chromaticity terms. Additionally, studying the impact of lower-order multipoles on the
dodecapolar RDT and its effect on beam lifetime would be valuable for optimizing the LHC's
performance.

% Summary
In summary, the research detailed in these studies underscores the critical importance of
understanding and managing higher-order multipole effects in the LHC. Skew octupolar fields,
decapolar fields, and other higher-order terms have a significant impact on beam stability and
performance. Developing and implementing advanced diagnostic techniques and correction methods are
essential for enhancing simulation accuracy and operational strategies. The insights gained from
these studies are not only crucial for optimizing the LHC's performance but also for guiding the
design and operation of future accelerator projects.\\
\indent
As particle accelerators continue to evolve, the challenges associated with higher-order multipole
components will persist. Ongoing research in this field is vital for addressing these challenges and
ensuring that future accelerators achieve the precision required for groundbreaking scientific
discoveries. The lessons learned from the LHC's experience with the complex interactions of
multipole fields will inform the design and operation of next-generation accelerators, ensuring they
remain at the forefront of exploring fundamental questions about the universe.\\
\indent
The work presented in these chapters represents a significant contribution to the field of
accelerator physics by offering practical solutions for current operational challenges and paving
the way for future advancements.

} % last empty line is important to get an implicit \par


% =============================
%       Zusammenfassung
% =============================
\chapter{\review{Zusammenfassung}}

{
\fontsize{\fontsizesummary}{\fontskipsummary}\selectfont

% Einleitung
Der Large Hadron Collider (LHC) am CERN, der sich in einem 27 Kilometer langen Tunnel unter der
Schweizer-französischen Grenze befindet, ist der weltweit größte und leistungsstärkste
Teilchenbeschleuniger. Seine Hauptmission ist es, die Bedingungen des Universums kurz nach dem
Urknall nachzubilden und den Wissenschaftlern zu ermöglichen, die fundamentalen Kräfte und Teilchen
zu erforschen, die das Kosmos ausmachen. Diese bemerkenswerte Einrichtung beschleunigt Protonen und
schwere Ionen auf nahezu Lichtgeschwindigkeit, bevor sie mit enormer Energie kollidieren, was
tiefgreifende Einblicke in die Bausteine der Materie und die grundlegenden Wechselwirkungen bietet,
die unser Universum regieren. Der LHC stellt nicht nur ein monumentales Ingenieurwerk dar, sondern
auch ein entscheidendes Werkzeug zur Weiterentwicklung unseres Verständnisses der
Hochenergiephysik.\\
\indent
Der Betrieb einer so komplexen und leistungsstarken Maschine erfordert die Überwindung zahlreicher
technischer Herausforderungen, insbesondere bei der Aufrechterhaltung der präzisen Kontrolle und
Stabilität der Teilchenstrahlen. Eine wesentliche Herausforderung besteht darin, die Auswirkungen
höherer magnetischer Felder zu managen, die oft aus Feldfehlern in den Multipolen resultieren, die
zur Lenkung und Fokussierung der Teilchenstrahlen verwendet werden. Diese Feldfehler, von
quadrupolaren bis hin zu dekahexapolaren Komponenten, können die Strahldynamik und Stabilität
erheblich beeinflussen. Die Bewältigung dieser Herausforderungen ist entscheidend, um
sicherzustellen, dass der LHC effektiv arbeitet und weiterhin wertvolle wissenschaftliche Ergebnisse
liefert. Die folgende ausführliche Zusammenfassung synthetisiert Ergebnisse aus drei detaillierten
Studien, die verschiedene Aspekte der höherordentlichen Multipol-Effekte und deren Auswirkungen auf
die Leistung des LHC untersuchen.

% Konzepte
Diese Dissertation verwendet das Hamiltonsche Formalismus zur Beschreibung der Teilchenbewegung in
den transversalen Ebenen unter dem Einfluss verschiedener Multipolfelder. In nichtlinearen Lattices
steigt die Komplexität der Strahldynamik erheblich, was den Einsatz fortgeschrittener mathematischer
Werkzeuge wie Lie-Algebren und Poisson-Klammern erforderlich macht, um nichtlineare Effekte genau zu
charakterisieren. Die Studie leitet explizite höherordentliche nichtlineare Übertragungsabbildungen
ab und bietet eine umfassende Zusammenfassung von Multipol-Kombinationen. Diese Nichtlinearitäten im
Lattice führen zu komplexen Phänomenen wie Chromatizität, Amplitudenabstimmung, chromatischer
Amplitudenabstimmung und Resonanzen, die durch Resonance Driving Terms (RDTs) verursacht werden, die
gründlich abgeleitet und durch detaillierte Messtechniken unterstützt werden.\\
\indent
Optikmessungen werden mit einer Reihe von Software-Tools und Techniken durchgeführt. Die
Datenerfassung Turn-by-Turn über Beam Position Monitors (BPMs) wird als entscheidende Methode zur
Bewertung der Strahloptik hervorgehoben, wobei ein AC-Dipol den Strahl anregt und die resultierenden
Oszillationen mittels Fourier-Transformationen analysiert werden, um Tunes zu extrahieren und
Resonanzen zu identifizieren. Weitere Behandlungen erfolgen über die Oszillationsamplituden und die
Größe der Spektrallinien, um lineare und nichtlineare Observable wie die Phasenadvance, die
Beta-Funktion, Dispersion, Kopplung, Orbit und RDTs abzuleiten. Chromatizitätsmessungen beinhalten
die Erzeugung von Impulsabweichungen durch Variation der RF-Frequenz und Beobachtung der
entsprechenden Tune-Verschiebungen. Die Dissertation führt auch ein benutzerdefiniertes Werkzeug,
das Non-Linear Chromaticity GUI, ein, das die Analyse und Korrektur der Chromatizität vereinfacht,
indem gemessene Daten an höherordentliche Chromatizitätsfunktionen angepasst und notwendige
Anpassungen vorgenommen werden. Eine wichtige Korrekturstrategie, die diskutiert wird, ist die
Reaktionsmatrix, ein lineares Gleichungssystem, das beschreibt, wie Variationen in den
Multipolstärken die Observablen beeinflussen, und sich als effektiv bei der Korrektur sowohl
linearer als auch nichtlinearer Observablen im LHC erwiesen hat.

% Erste Studie
Die erste Studie untersucht die Ursprünge und Konsequenzen von schiefen Oktupolfeldern innerhalb des
LHC. Diese Felder beeinflussen erheblich die dynamische Apertur des Beschleunigers, ein kritischer
Parameter, der den Bereich definiert, innerhalb dessen der Teilchenstrahl stabil bleibt. Schiefe
Oktupol-Korrektoren sind um wichtige Detektoren wie ATLAS und CMS installiert, um diese Felder zu
managen und ihre Auswirkungen auf die Strahlinstabilität zu mindern. Die Studie konzentriert sich
auf die Messung dieser Felder mit Optiken, die für die Höchstenergie von 6,8 TeV pro Teilchen
ausgelegt sind. Korrekturen wurden unter Verwendung der Reaktionsmatrixmethode durchgeführt, einem
anderen Ansatz als der, der vor einigen Jahren verwendet wurde. Diese Methode adressiert effektiv
schiefe Oktupol-RDTs unter Verwendung der verfügbaren Korrektormagnete, obwohl ihre Leistung durch
das Fehlen eines Korrektors begrenzt ist, was die erreichbare Korrekturstärke einschränkt. Folglich
können bestimmte RDTs wie $f_{1012}$ und $f_{1210}$ je nach Korrektorkonfiguration entweder effektiv
korrigiert oder auf konstantem Niveau gehalten werden. Das erreichte Korrekturniveau ist
vergleichbar mit dem, das mit einer anderen Methode im letzten LHC-Betrieb erreicht wurde.\\
\indent
Zusätzlich untersucht die Studie den unerwarteten Einfluss von Landau-Oktupolen auf schiefe
Oktupol-RDTs bei der Injektionsenergie von 450 GeV pro Teilchen. Landau-Oktupole sind
leistungsstarke Oktupole, die bei der Injektionsenergie verwendet werden, um Dämpfung durch eine
Tune-Streuung einzuführen. Während der Messungen wurde eine signifikante Verschiebung in den
schiefen Oktupol-RDTs bei verschiedenen Stromstärken der Landau-Oktupole beobachtet. Zunächst wurde
erwartet, dass diese Oktupole nur durch Fehlanpassungen, insbesondere Rollfehler, schiefe Felder
erzeugen. Simulationen zeigen jedoch, dass Oktupol-Fehlanpassungen minimale Auswirkungen auf schiefe
Oktupolfelder haben. Stattdessen wurde die Kopplung als entscheidender Faktor identifiziert. Die
Kombination von Kopplung und Landau-Oktupolen wird als wesentlicher Beitrag zu schiefen Oktupol-RDTs
bei der Injektionsenergie erwartet, wo Oktupole stark betrieben werden. Eine genaue Modellierung der
Kopplung ist daher entscheidend für die Vorhersage des Verhaltens von schiefen Oktupol-RDTs und die
Verwaltung ihrer Auswirkungen auf die Strahlinstabilität.

% Zweite Studie
Die zweite Studie konzentriert sich auf dekapolare Felder im LHC, insbesondere bei der
Injektionsenergie. Diese Studie behandelt zuvor beobachtete Diskrepanzen zwischen Messungen und
Modellen im Zusammenhang mit der dritthöchsten Chromatizität, einem kritischen Parameter, der
beschreibt, wie Teilchen mit einer Energieabweichung unterschiedliche Oszillationsfrequenzen als das
Referenzteilchen erleben. Eine genaue Kontrolle der Chromatizität ist entscheidend für die
Aufrechterhaltung der Strahlinstabilität. Die Einführung zuvor unbeobachteter Observablen hat ein
klareres Verständnis dieser Diskrepanzen ermöglicht. Zu diesen Observablen gehören die rohe
Chromatizität, die die Chromatizität der Maschine ohne eingeschaltete Korrektoren darstellt, um den
reinen Einfluss der Feldfehler zu beobachten, und die chromatische Amplitudenabstimmung, eine
Abstimmungsfunktion sowohl für Impulsabweichungen als auch für Amplituden. Mehrere Ansätze zur
Messung derselben Felder helfen, die verschiedenen Beiträge zu klären. Die Forschung zeigt, dass der
Zerfall der dekapolaren Komponente in den Hauptdipolen ein wesentlicher Faktor für diese
Diskrepanzen ist. Als der LHC entworfen wurde, wurde dieser Zerfall als zu gering angesehen, um
signifikant zu sein, und daher nicht in die für Simulationen verwendeten Magnetfehler-Tabellen
aufgenommen. Da jedoch die Parameter der Maschine jedes Jahr weiter erhöht werden und die
Auswirkungen höherordentlicher Felder besser verstanden werden, wird klar, dass eine genaue
Kontrolle und Modellierung dieser Felder notwendig sind.\\
\indent
Erstmals wurden Messungen und Korrekturen des dekapolaren Resonance Driving Term $f_{1004}$ bei der
Injektionsenergie durchgeführt. Die Implementierung kombinierter Korrekturen für die dritthöchste
Chromatizität, chromatische Amplitudenabstimmung und RDT $f_{1004}$ führte zu einer Verbesserung der
Strahllaufzeit um 3 \%. Im Gegensatz dazu führte eine absichtliche Verschlechterung des RDTs allein
zu einem Rückgang der Strahllaufzeit um 10 \%, was die Bedeutung dieser Resonanzkorrekturen für einen
stabilen Betrieb des Strahls unterstreicht. Die Studie untersuchte auch, wie Sextupole und Oktupole
interagieren, um dekapolarähnliche Felder zu erzeugen. Es wurde festgestellt, dass Sextupole, sowohl
allein als auch in Kombination mit Landau-Oktupolen, bei kleinen Strömen einen erheblichen
$f_{1004}$-dekapolaren RDT erzeugen. Daher würden in einem betrieblichen Kontext dekapolare
Resonanzen, die größtenteils durch starke Oktupole erzeugt werden, von angepassten Korrekturen
profitieren. Diese Erkenntnisse legen nahe, dass weitere Fortschritte bei den Korrekturmethoden zu
noch größeren Verbesserungen der Strahlinstabilität führen könnten.

% Dritte Studie
Die dritte Studie untersucht höherordentliche Felder im LHC, insbesondere dodecapolare und
decatetrapolare Felder. Durch den Einsatz eines neu implementierten Kollimationssystems und
maßgeschneiderter Nachbearbeitungstechniken konnte diese Studie erfolgreich diese höherordentlichen
Felder beobachten. Studien wurden durchgeführt, um den Effekt der Nichtlinearität des
Impulskompaktionsfaktors auf die Chromatizitätsfunktion während ihrer Berechnung aus der RF-Frequenz
zu schätzen. Die Ergebnisse zeigen, dass der Impulskompaktionsfaktor zwar zweiter Ordnung im LHC
zeigt, seine Auswirkungen auf die resultierende Chromatizität jedoch selbst bei großen
Impulsabweichungen vernachlässigbar sind. Mehrere Chromatizitätsmessungen mit unterschiedlichen
Konfigurationen von Oktupol- und Dekapolarkorrekturen offenbarten dann die Präsenz von vierten und
fünften Ordnungsterminen ($Q^{(4)}$ und $Q^{(5)}$). Diese Messungen identifizierten konsequent diese
höherordentlichen Terme mit ähnlichen Werten und demonstrierten ihre Robustheit. Darüber hinaus wird
betont, dass eine genaue Charakterisierung der niedrigerordentlichen Terme eine gute Messung dieser
höherordentlichen Terme erfordert. Die Studie identifiziert durch Simulationen dodecapolare und
decatetrapolare Felder als Hauptbeiträge zu diesen höherordentlichen Effekten, die von Feldfehlern
in den Hauptdipolen und Quadrupolen ausgehen. Das Fehler-Modell des LHC scheint im Vergleich zu den
Messungen in relativer Übereinstimmung zu stehen, sobald der Zerfall der decatetrapolaren
Komponenten berücksichtigt wird.\\
\indent
Erstmals wurde bei der Injektionsenergie der dodecapolare Resonance Driving Term $f_{0060}$
gemessen. Diese Messung zeigt eine klare Wiederholbarkeit, selbst bei unterschiedlichen
Konfigurationen von Oktupol- und Dekapolarkorrekturen. Die gemessenen Werte erwiesen sich als gut
mit dem Modell übereinstimmend. Die Forschung kommt zu dem Schluss, dass weitere Untersuchungen
erforderlich sind, um Einschränkungen im Messbereich der Chromatizitätsfunktion zu adressieren und
Schätzungen höherordentlicher Chromatizitäts-Terme zu verfeinern. Darüber hinaus wäre es wertvoll,
den Einfluss niedrigerer Multipole auf den dodecapolaren RDT und dessen Auswirkungen auf die
Strahllaufzeit zu untersuchen, um die Leistung des LHC zu optimieren.

% Zusammenfassung
Zusammenfassend betont die Forschung, die in diesen Studien detailliert beschrieben wird, die
entscheidende Bedeutung des Verständnisses und Managements höherordentlicher Multipol-Effekte im
LHC. Schiefe Oktupolfelder, dekapolare Felder und andere höherordentliche Terme haben einen
signifikanten Einfluss auf die Strahlinstabilität und -leistung. Die Entwicklung und Implementierung
fortschrittlicher Diagnoseverfahren und Korrekturmethoden sind entscheidend für die Verbesserung der
Simulationsgenauigkeit und operativen Strategien. Die Erkenntnisse aus diesen Studien sind nicht nur
entscheidend für die Optimierung der LHC-Leistung, sondern auch für die Gestaltung und den Betrieb
zukünftiger Beschleunigerprojekte.\\
\indent
Da sich Teilchenbeschleuniger weiterentwickeln, werden die Herausforderungen im Zusammenhang mit
höherordentlichen Multipolkomponenten bestehen bleiben. Laufende Forschung auf diesem Gebiet ist
unerlässlich, um diese Herausforderungen zu bewältigen und sicherzustellen, dass zukünftige
Beschleuniger die Präzision erreichen, die für bahnbrechende wissenschaftliche Entdeckungen
erforderlich ist. Die aus den Erfahrungen des LHC mit den komplexen Wechselwirkungen der
Multipol-Felder gewonnenen Lektionen werden die Gestaltung und den Betrieb von Beschleunigern der
nächsten Generation informieren und sicherstellen, dass sie an der Spitze der Erforschung
fundamentaler Fragen zum Universum bleiben.\\
\indent
Die in diesen Kapiteln präsentierte Arbeit stellt einen bedeutenden Beitrag zum Bereich der
Beschleunigerphysik dar, indem sie praktische Lösungen für aktuelle betriebliche Herausforderungen
bietet und den Weg für zukünftige Fortschritte ebnet.

}