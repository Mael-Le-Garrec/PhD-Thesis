% =============================
%       Extended Summary
% =============================
\chapter{\review{Extended Summary}}


\ifthenelse{\equal{\papersize}{A4}}{ % A4
    \newcommand{\fontsizesummary}{12pt}
    \newcommand{\fontskipsummary}{14pt}
    \newcommand{\vspacesummary}{0cm}
}{  % else, B5
    \newcommand{\fontsizesummary}{11pt}
    \newcommand{\fontskipsummary}{10pt}
    \newcommand{\vspacesummary}{-0.5cm}
}

\vspace{\vspacesummary}

{
% 5 pages is reeaally long, let's just add some spacing so it's easier for the reader to follow
\fontsize{\fontsizesummary}{\fontskipsummary}\selectfont

% Introduction
The Large Hadron Collider (LHC) at CERN, situated in a 27-kilometer tunnel beneath the Swiss-French
border, is the world's largest and most powerful particle accelerator. Its primary mission is to
recreate the conditions of the universe just moments after the Big Bang, enabling scientists to
explore the fundamental forces and particles that constitute the cosmos. This remarkable facility
accelerates protons and heavy ions to nearly the speed of light before colliding them with immense
energy, providing profound insights into the building blocks of matter and the fundamental
interactions that govern our universe. The LHC represents not only a monumental engineering
achievement but also a crucial tool for advancing our understanding of high-energy physics.
\\
\indent
Operating such a complex and powerful machine requires overcoming numerous technical challenges,
particularly in maintaining the precise control and stability of its particle beams. One significant
challenge lies in managing the effects of higher-order magnetic fields, which often arise from field
errors in the magnets used to guide and focus the particle beams. These field errors can profoundly
impact beam dynamics and its lifetime. Addressing these challenges is essential to ensuring that
the LHC operates effectively and continues to produce valuable scientific results. Furthermore,
non-linear optics control is crucial for the success of future accelerators like the HL-LHC and FCC,
which will increasingly rely on high-order non-linear optics corrections to achieve their
performance targets. The following extended summary synthesizes findings from three detailed studies
investigating various aspects of higher-order multipole effects and their implications on the LHC's
beam dynamics.


% Concepts
This thesis employs the Hamiltonian formalism to describe particle motion in the transverse planes
under the influence of various multipole fields. In non-linear lattices, the complexity of beam
dynamics increases significantly, necessitating the use of advanced mathematical tools such as Lie
Algebra and Poisson Brackets to accurately characterize non-linear effects. The study derives
explicit higher-order non-linear transfer maps and provides a comprehensive summary of multipole
combinations. These non-linearities in the lattice lead to complex phenomena such as high-oder 
chromaticity, amplitude detuning, chromatic amplitude detuning, and resonances driven by Resonance
Driving Terms (RDTs), all of which are thoroughly derived and supported by detailed measurement
techniques.
\\
\indent
Optics measurements are conducted using a wide range of techniques and software tools. Turn-by-turn
data acquisition via Beam Position Monitors (BPMs) is emphasized as a crucial method for evaluating
beam optics, where an AC-dipole excites the beam, and the resulting oscillations are analyzed using
Fourier transforms to extract tunes and identify resonances. Further treatment is done via the
oscillation amplitudes and the magnitude of spectral lines to retrieve linear and non-linear
observables such as the phase advance, beta function, dispersion, coupling, orbit, and RDTs.
Chromaticity measurements involve inducing momentum offsets by varying the RF frequency and
observing the corresponding tune shifts.
\\
\indent
To advance the understanding of high-order non-linear fields, new measurement and analysis methods
have been developed. One such tool, the Non-Linear Chromaticity GUI, simplifies the process of
analyzing and correcting chromaticity during operation. Additionally, a novel response matrix
approach has been introduced, enabling efficient direct corrections of Resonance Driving Terms
(RDTs) in the LHC. This marks a shift from empirical correction adjustments to a more quantitative
and systematic method. These techniques have proven effective in correcting several key observables,
as discussed throughout this thesis. The development of these methods has been crucial in reducing
commissioning time, enabling a greater focus on high-order multipoles, from octupoles to
decatetrapoles, some of which have never been studied before. These investigations were further
facilitated by improvements in the collimator sequence, allowing the exploration of a higher action
and momentum offsets. Additionally, enhancements in dynamic aperture, as presented in this thesis,
provided the required amplitudes to probe these higher-order effects more effectively.


% First Study
The first chapter explores the origins and consequences of skew octupolar fields within the LHC.
These fields significantly influence the dynamic aperture of the accelerator, a parameter
that defines the amplitudes within which the particle beam remains stable. Skew octupolar correctors
are installed around key detectors, such as ATLAS and CMS, to manage these fields and mitigate their
effects on beam stability. The study focuses on measuring these fields with optics designed for top
energy, at 6.8 TeV per particle. Corrections were performed using a response matrix based approach,
a different method than what was used a few years prior, marking a shift from empirical corrections
to a more quantitative and systematic approach for the first time in the LHC. 
This method effectively addresses skew octupolar RDTs using the available corrector magnets,
although its performance is limited by the absence of one corrector, which constrains the achievable
correction strength. Consequently, the RDTs of interest, $f_{1012}$ and $f_{1210}$, can either be
effectively corrected or maintained at a constant level depending on the corrector configuration.
\\
\indent
Additionally, the study investigates the unexpected influence of Landau octupoles on skew octupolar
RDTs at injection energy, at 450 GeV per particle. Landau octupoles are powerful magnets used
at injection energy to introduce damping of multi-particle coherent instabilities via a tune
spread. They have been observed to generate a significant shift in skew octupolar RDTs during
measurements under various powering configurations. Skew octupolar resonances have previously been
identified as a source of emittance growth in the presence of electron clouds. The more intuitive
explanation for the generation of skew fields by normal multipoles had been the misalignments of
these octupoles, specifically roll errors. However, simulations indicate that octupole misalignments
have minimal impact on skew octupolar fields. Instead, transverse coupling has been identified as a
crucial factor. The combination of coupling and the strong powering of Landau octupoles at injection
energy is expected to be a major contributor to skew octupolar RDTs. Therefore, precise modeling of
coupling is essential for predicting the behavior of skew octupolar RDTs. Accurate modeling and
correction of skew octupolar fields in the LHC are essential to suppress resonances and improve both
the beam's dynamic stability and lifetime.


% Second Study
The second chapter focuses on decapolar fields in the LHC, particularly at injection energy. As the
FCC is expected to rely on precise control of decapolar fields, their accurate study in the LHC is
essential to increase confidence in its design and operational strategies.  Consequently, this study
addresses previously observed discrepancies between measurements and models related to third-order
chromaticity, a critical parameter that describes how particles with an energy deviation experience
different oscillation frequencies than the reference particle. Accurate control of chromaticity is
vital for maintaining beam stability. The introduction of previously unobserved observables, has
provided a clearer understanding of these discrepancies.  Among these observables are the bare
chromaticity, which represents the chromaticity of the machine without any correctors powered on to
observe the bare influence of field errors, and chromatic amplitude detuning, a detuning function of
both momentum deviations and oscillation amplitudes. Several approaches to measuring the same fields
help clarify the various contributions. The research reveals that the decay of the decapolar
component in the main dipoles is a significant factor contributing to these discrepancies. When the
LHC was designed, this decay was deemed too small to be significant and thus was not included in the
magnetic error tables used for simulation. However, as the machine's parameters are pushed further
each year and the effects of higher-order fields become better understood, it becomes clear that
accurate control and modeling of these fields are necessary.
\\
\indent
For the first time in the LHC, measurements and corrections of the decapolar Resonance Driving Term
$f_{1004}$ were carried out at injection energy. Corrections are based on a response matrix
approach, effectively implementing combined corrections of third-order chromaticity, chromatic
amplitude detuning, and RDT $f_{1004}$, leading to a 3\% improvement in beam lifetime.  Conversely,
deliberately degrading the RDT alone resulted in a 10\% decrease in beam lifetime, underscoring the
importance of this resonance corrections for stable beam operation. The study also explored how
sextupoles and octupoles interact to generate decapolar-like fields. It was found that sextupoles,
both alone and in combination with Landau octupoles, produce a substantial $f_{1004}$ decapolar RDT
when powered to small currents. Therefore, in an operational context, decapolar resonances, largely
generated by strong octupoles, would benefit from adapted corrections. These findings suggest that
further advancements in correction methods could lead to even greater improvements in beam
lifetime.

% Third Study
The third chapter investigates very-high-order fields in the LHC, specifically dodecapolar and
decatetrapolar fields. Using a newly implemented collimation setup and custom post-processing
techniques, this study successfully observed these higher-order fields. Studies were conducted to
estimate the effect of the non-linearity of the momentum compaction factor on the chromaticity
function during its computation from the RF frequency. The results indicate that while the momentum
compaction factor expansion shows a second order in the LHC, its effects on the resulting
chromaticity are negligible even at large momentum offsets. Several chromaticity measurements with
varying configurations of octupolar and decapolar corrections then revealed the presence of fourth
and fifth-order terms ($Q^{(4)}$ and $Q^{(5)}$). These measurements consistently identified these
higher-order terms with similar values, demonstrating their robustness. Additionally, it is
emphasized that accurately characterizing the lower-order terms requires good measurement of these
higher-order terms. The study identifies, through simulations, dodecapolar and decatetrapolar fields
as primary contributors to these higher-order effects, originating from field errors in the main
dipoles and quadrupoles. The LHC's field error model appears to be in relative agreement with the
measurements once the decay of decatetrapolar components is considered.
\\
\indent
For the first time at injection energy, the dodecapolar Resonance Driving Term $f_{0060}$ was
measured. This measurement shows clear repeatability, even when performed with different
configurations of octupolar and decapolar corrections. The measured values were found to be in good
agreement with the model. The research concludes that further investigations are needed to address
limitations in the measurement range of the chromaticity function and to refine estimates of
higher-order chromaticity terms. Additionally, studying the impact of lower-order multipoles on the
dodecapolar RDT and its effect on beam lifetime would be valuable for optimizing the LHC's
performance.

% SUPERKEKB
A EAJADE secondment at SuperKEKB during its February 2024 commissioning utilized optics
measurement techniques from CERN on the HER and LER rings. Linear optics measurements showed good
agreement with the Closed Orbit Distortion (COD) method, demonstrating repeatability over time. For
the first time, vertical plane measurements with an injection offset were conducted, yielding
promising results. The study extended to non-linear optics, including chromaticity and amplitude
detuning, revealing some discrepancies between measurements and model predictions, particularly
concerning potential unmodeled sources. Resonance Driving Terms (RDTs) were measured successfully
for the first time, although challenges remained due to factors like decoherence and damping.
Overall, the findings align with alternative KEK methods, indicating that CERN's techniques are
effective for enhancing understanding of SuperKEKB and future accelerators like the FCC-ee.


% Summary
In summary, the research detailed in these studies underscores the critical importance of
understanding and managing higher-order multipole effects in the LHC. Skew octupolar fields,
decapolar fields, and other higher-order fields have a significant impact on the beam's dynamic
aperture and lifetime. Developing and implementing advanced measurement techniques and correction
methods is essential for enhancing the understanding of non-linear optics. The insights gained from
these studies are crucial for optimizing the LHC's performance.
\\
\indent
As particle accelerators continue to evolve, the challenges associated with higher-order multipole
components will persist. Ongoing research in this field is vital for addressing these challenges and
ensuring that future accelerators achieve the precision required for new scientific discoveries. The
lessons learned from the LHC's experience with the complex interactions of multipole fields will
inform the design and operation of next-generation accelerators, such as the HL-LHC and FCC, which
will increasingly rely on precise non-linear optics control. These advancements will ensure that
these accelerators remain at the forefront of exploring fundamental questions about the universe.
\\
\indent
The work presented in these chapters represents a significant contribution to the field of
accelerator physics by offering practical solutions for current operational challenges and paving
the way for future advancements.

} % last empty line is important to get an implicit \par


% =============================
%       Zusammenfassung
% =============================
\chapter{\review{Zusammenfassung}}

\vspace{\vspacesummary}

{
\fontsize{\fontsizesummary}{\fontskipsummary}\selectfont

% Einführung
Der Large Hadron Collider (LHC) am CERN, der sich in einem 27 Kilometer langen Tunnel unter der
Schweizer-französischen Grenze befindet, ist der größte und leistungsstärkste Teilchenbeschleuniger
der Welt. Seine Hauptmission besteht darin, die Bedingungen des Universums kurz nach dem Urknall
nachzubilden, wodurch Wissenschaftler die grundlegenden Kräfte und Teilchen untersuchen können, aus
denen das Universum besteht. Diese bemerkenswerte Einrichtung beschleunigt Protonen und schwere
Ionen nahezu auf Lichtgeschwindigkeit, bevor sie mit enormer Energie kollidiert werden, was
tiefgreifende Einblicke in die Bausteine der Materie und die grundlegenden Wechselwirkungen, die
unser Universum regieren, ermöglicht. Der LHC stellt nicht nur einen monumentalen technischen Erfolg
dar, sondern ist auch ein entscheidendes Werkzeug für das Verständnis der Hochenergiephysik.
\\
\indent
Der Betrieb einer so komplexen und leistungsstarken Maschine erfordert das Überwinden zahlreicher
technischer Herausforderungen, insbesondere bei der präzisen Kontrolle und Stabilität ihrer
Teilchenstrahlen. Eine bedeutende Herausforderung besteht darin, die Auswirkungen höherer
magnetischer Felder zu verwalten, die häufig durch Feldfehler in den Magneten entstehen, die zur
Führung und Fokussierung der Teilchenstrahlen verwendet werden. Diese Feldfehler können die
Strahldynamik und dessen Lebensdauer erheblich beeinflussen. Die Bewältigung dieser
Herausforderungen ist entscheidend, um sicherzustellen, dass der LHC effektiv arbeitet und weiterhin
wertvolle wissenschaftliche Ergebnisse liefert. Darüber hinaus ist die Kontrolle nicht-linearer
Optik für den Erfolg zukünftiger Beschleuniger wie dem HL-LHC und FCC entscheidend, die zunehmend
auf Korrekturen höherer nicht-linearer Optik angewiesen sein werden, um ihre Leistungsziele zu
erreichen. Die folgende erweiterte Zusammenfassung synthetisiert die Ergebnisse von drei
detaillierten Studien, die verschiedene Aspekte höherer multipolarer Effekte und deren Auswirkungen
auf die Strahldynamik des LHC untersuchen.

% Konzepte
Diese Dissertation verwendet die Hamiltonsche Formulierung, um die Bewegung von Teilchen in den
transversalen Ebenen unter dem Einfluss verschiedener multipolarer Felder zu beschreiben. In
nicht-linearen Lattices erhöht sich die Komplexität der Strahldynamik erheblich, sodass der Einsatz
fortgeschrittener mathematischer Werkzeuge wie Lie-Algebra und Poisson-Klammern erforderlich ist, um
nicht-lineare Effekte genau zu charakterisieren. Die Studie leitet explizite höhere nicht-lineare
Transferkarten ab und bietet eine umfassende Zusammenfassung multipolarer Kombinationen. Diese
Nichtlinearitäten im Lattice führen zu komplexen Phänomenen wie höherer chromatischer Aberration,
Amplitudendämpfung, chromatischer Amplitudendämpfung und Resonanzen, die durch Resonance Driving
Terms (RDTs) hervorgerufen werden, die alle gründlich abgeleitet und durch detaillierte
Messtechniken unterstützt werden.
\\
\indent
Optikmessungen werden mit einer Vielzahl von Techniken und Software-Tools durchgeführt. Die
Datenerfassung „turn-by-turn“ über Strahlpositionsmonitore (BPMs) wird als entscheidende Methode zur
Evaluierung der Strahloptik hervorgehoben, bei der ein AC-Dipol den Strahl anregt und die
resultierenden Schwingungen mit Fourier-Transformationen analysiert werden, um die Tuningwerte zu
extrahieren und Resonanzen zu identifizieren. Weitere Behandlungen erfolgen über die
Schwingungsamplituden und die Größe der Spektrallinien, um lineare und nicht-lineare Observablen wie
die Phasenverschiebung, die Beta-Funktion, die Dispersion, die Kopplung, die Bahn und die RDTs zu
erhalten. Chromatisitätsmessungen beinhalten das Einführen von Impulsverschiebungen durch Variieren
der RF-Frequenz und das Beobachten der entsprechenden Tuningverschiebungen.
\\
\indent
Um das Verständnis höherer nicht-linearer Felder zu erweitern, wurden neue Mess- und Analysemethoden
entwickelt. Ein solches Werkzeug, die Non-Linear Chromaticity GUI, vereinfacht den Prozess der
Analyse und Korrektur der Chromatik während des Betriebs. Darüber hinaus wurde ein neuartiger
Response-matrix ansatz eingeführt, der effiziente direkte Korrekturen von Resonance Driving Terms
(RDTs) im LHC ermöglicht. Dies markiert einen Wechsel von empirischen Korrekturanpassungen zu einer
quantitativeren und systematischeren Methode. Diese Techniken haben sich als effektiv erwiesen, um
mehrere wichtige Observablen zu korrigieren, wie in dieser Dissertation besprochen wird. Die
Entwicklung dieser Methoden war entscheidend, um die Inbetriebnahmezeit zu verkürzen und einen
stärkeren Fokus auf höhere Multipole zu ermöglichen, von Oktupolen bis zu Dekatetrapolen, von denen
einige nie zuvor untersucht wurden. Diese Untersuchungen wurden durch Verbesserungen in der
Kollimatorsequenz weiter erleichtert, was die Erkundung höherer Aktionen und Impulsverschiebungen
ermöglichte. Darüber hinaus boten die in dieser Dissertation präsentierten Verbesserungen der
dynamischen Apertur die erforderlichen Amplituden, um diese höherwertigen Effekte effektiver zu
untersuchen.

% Erste Studie
Das erste Kapitel untersucht die Ursprünge und Konsequenzen skew Oktupolfelder im LHC. Diese
Felder beeinflussen erheblich die dynamische Apertur des Beschleunigers, ein Parameter, der die
Amplituden definiert, innerhalb derer der Teilchenstrahl stabil bleibt. Schiefe Oktupolkorrektoren
sind rund um wichtige Detektoren, wie ATLAS und CMS, installiert, um diese Felder zu verwalten und
deren Auswirkungen auf die Strahlstabilität zu mindern. Die Studie konzentriert sich auf die Messung
dieser Felder mit einer für die höchste Energie ausgelegten Optik bei 6,8 TeV pro Teilchen.
Korrekturen wurden unter Verwendung eines Ansatzes basierend auf der Response-matrix durchgeführt,
einer anderen Methode als die, die einige Jahre zuvor verwendet wurde, was einen Wechsel von
empirischen Korrekturen zu einem quantitativeren und systematischeren Ansatz im LHC darstellt. 
Diese Methode adressiert effektiv schiefe Oktupol-RDTs unter Verwendung der verfügbaren
Korrektormagneten, obwohl ihre Leistung durch das Fehlen eines Korrektors eingeschränkt ist, was die
erreichbare Korrekturkraft einschränkt. Folglich können die von Interesse sind, $f_{1012}$ und
$f_{1210}$, entweder effektiv korrigiert oder auf einem konstanten Niveau gehalten werden, abhängig
von der Konfiguration des Korrektors.
\\
\indent
Darüber hinaus untersucht die Studie den unerwarteten Einfluss von Landau-Oktupolen auf
schiefe Oktupol-RDTs bei Injektionsenergie von 450 GeV pro Teilchen. Landau-Oktupole, die bei
Injektionsenergie verwendet werden, um kohärente Instabilitäten durch eine Tuningstreuung zu
dämpfen, erzeugten während der Messungen unter verschiedenen Stromkonfigurationen eine signifikante
Verschiebung in den schiefen Oktupol-RDTs. Schiefe Oktupol-Resonanzen wurden zuvor als Quelle für
Emittanzwachstum in Anwesenheit von Elektronenwolken identifiziert. Die intuitivere Erklärung für
die Erzeugung skew Felder durch normale Multipole war die Fehlausrichtung dieser Oktupole,
insbesondere Rollfehler. Simulationen zeigen jedoch, dass die Fehlausrichtung von Oktupolen nur
minimale Auswirkungen auf schiefe Oktupolfelder hat. Stattdessen wurde die transversale Kopplung als
entscheidender Faktor identifiziert. Die Kombination aus Kopplung und der starken Stromversorgung
von Landau-Oktupolen bei Injektionsenergie wird als wesentlicher Beitrag zu schiefen Oktupol-RDTs
angesehen. Daher ist eine präzise Modellierung der Kopplung entscheidend, um das Verhalten der
schiefen Oktupol-RDTs vorherzusagen. Die genaue Modellierung und Korrektur der schiefen
Oktupolfelder im LHC sind entscheidend, um Resonanzen zu unterdrücken und sowohl die dynamische
Stabilität als auch die Lebensdauer des Strahls zu verbessern.

% Zweite Studie
Das zweite Kapitel konzentriert sich auf Dekapolfelder im LHC, insbesondere bei Injektionsenergie.
Da der FCC voraussichtlich auf eine präzise Kontrolle von Dekapolfeldern angewiesen ist, ist ihre
genaue Untersuchung im LHC entscheidend, um das Vertrauen in sein Design und seine
Betriebsstrategien zu erhöhen. Folglich befasst sich diese Studie mit zuvor beobachteten
Diskrepanzen zwischen Messungen und Modellen, die sich auf die dritte chromatische Aberration
beziehen, einen kritischen Parameter, der beschreibt, wie Teilchen mit einer Energiedifferenz
unterschiedliche Schwingungsfrequenzen im Vergleich zum Referenzteilchen erfahren. Eine präzise
Kontrolle der Chromatik ist für die Aufrechterhaltung der Strahlstabilität von entscheidender
Bedeutung. Die Einführung zuvor unobservierter Observablen hat ein klareres Verständnis dieser
Diskrepanzen ermöglicht. Zu diesen Observablen gehört die rohe Chromatik, die die Chromatik der
Maschine darstellt, ohne dass Korrektoren eingeschaltet sind, um den rohen Einfluss von Feldfehlern
zu beobachten, und die chromatische Amplitudendämpfung, eine Dämpfungsfunktion sowohl von
Impulsabweichungen als auch von Schwingungsamplituden. Mehrere Ansätze zur Messung derselben Felder
helfen, die verschiedenen Beiträge zu klären. Die Forschung zeigt, dass der Rückgang der dekapolaren
Komponente in den Hauptdipolen ein wesentlicher Faktor ist, der zu diesen Diskrepanzen beiträgt. Als
der LHC entworfen wurde, wurde dieser Rückgang als zu klein erachtet, um signifikant zu sein, und
daher nicht in die für Simulationen verwendeten Tabellen der magnetischen Fehler aufgenommen. Da
jedoch die Parameter der Maschine jedes Jahr weiter vorangetrieben werden und die Auswirkungen
höherer Felder besser verstanden werden, wird klar, dass eine präzise Kontrolle und Modellierung
dieser Felder notwendig sind.
\\
\indent
Erstmals im LHC wurden Messungen und Korrekturen des dekapolaren Resonance Driving Terms
$f_{1004}$ bei Injektionsenergie durchgeführt. Korrekturen basieren auf einem Ansatz der
Response-matrix, der effektiv kombinierte Korrekturen der dritten chromatischen Aberration, der
chromatischen Amplitudendämpfung und des RDT $f_{1004}$ implementiert, was zu einer Verbesserung der
Strahllebensdauer um 3\% führt. Umgekehrt führte eine absichtliche Verschlechterung des RDT allein
zu einem Rückgang der Strahllebensdauer um 10\%, was die Bedeutung dieser Resonanzkorrekturen für
einen stabilen Strahlbetrieb unterstreicht. Die Studie untersuchte auch, wie Sextupole und Oktupole
interagieren, um dekapolarenähnliche Felder zu erzeugen. Es wurde festgestellt, dass Sextupole,
sowohl allein als auch in Kombination mit Landau-Oktupolen, einen erheblichen $f_{1004}$ dekapolaren
RDT erzeugen, wenn sie auf kleine Ströme betrieben werden. Daher würden im Betrieb dekapolare
Resonanzen, die größtenteils durch starke Oktupole erzeugt werden, von angepassten Korrekturen
profitieren. Diese Ergebnisse deuten darauf hin, dass weitere Fortschritte in den Korrekturmethoden
zu noch größeren Verbesserungen der Strahllebensdauer führen könnten.

% Dritte Studie
Das dritte Kapitel untersucht sehr hochgradige Felder im LHC, insbesondere dodecapolare und
dekatetrapolare Felder. Mithilfe einer neu implementierten Kollimationsanordnung und
maßgeschneiderter Nachbearbeitungstechniken konnte diese Studie erfolgreich diese höhergradigen
Felder beobachten. Es wurden Studien durchgeführt, um den Effekt der Nichtlinearität des
Impulsabhängige pfadlänge auf die chromatische Funktion während ihrer Berechnung aus der RF-Frequenz
zu schätzen. Die Ergebnisse zeigen, dass, obwohl die Erweiterung des Impulsabhängige pfadlänge im
LHC zweiter Ordnung zeigt, seine Auswirkungen auf die resultierende Chromatik selbst bei großen
Impulsverschiebungen vernachlässigbar sind. Mehrere Chromatisitätsmessungen mit variierenden
Konfigurationen von oktupolarer und dekapolarer Korrektur ergaben dann das Vorhandensein von vierten
und fünften Ordnungstermen ($Q^{(4)}$ und $Q^{(5)}$). Diese Messungen identifizierten diese
höhergradigen Terme konsistent mit ähnlichen Werten und demonstrieren ihre Robustheit. Darüber
hinaus wird betont, dass eine genaue Charakterisierung der niedergradigen Terme eine gute Messung
dieser höhergradigen Terme erfordert. Die Studie identifiziert durch Simulationen dodecapolare und
dekatetrapolare Felder als Hauptbeiträge zu diesen höhergradigen Effekten, die aus Feldfehlern in
den Hauptdipolen und Quadrupolen stammen. Das Modell der Feldfehler des LHC scheint in relativer
Übereinstimmung mit den Messungen zu sein, sobald der Rückgang der dekatetrapolaren Komponenten
berücksichtigt wird.
\\
\indent Erstmals bei Injektionsenergie wurde der dodecapolare Resonance Driving Term $f_{0060}$
gemessen. Diese Messung zeigt eine klare Wiederholbarkeit, selbst wenn sie mit unterschiedlichen
Konfigurationen von oktupolarer und dekapolarer Korrektur durchgeführt wird. Die gemessenen Werte
stimmten gut mit dem Modell überein. Die Studie kommt zu dem Schluss, dass weitere Untersuchungen
erforderlich sind, um Einschränkungen im Messbereich der chromatischen Funktion zu beheben und
Schätzungen höhergradiger chromatischer Terme zu verfeinern. Darüber hinaus wäre es wertvoll, den
Einfluss niedergradiger Multipole auf den dodecapolaren RDT und dessen Auswirkungen auf die
Strahllebensdauer zu untersuchen, um die Leistung des LHC zu optimieren.

% SUPERKEKB
Ein EAJADE-Transfer bei SuperKEKB während der Inbetriebnahme im Februar 2024 nutzte
Optikmesstechniken vom CERN an den HER- und LER-Ringen. Lineare Optikmessungen zeigten eine gute
Übereinstimmung mit der Methode der geschlossenen Bahnverzerrung (COD) und demonstrierten die
Wiederholbarkeit über die Zeit. Erstmals wurden vertikale Messungen im Plange mit einer
Injektionsabweichung durchgeführt, die vielversprechende Ergebnisse erbrachten. Die Studie
erstreckte sich auf nicht-lineare Optik, einschließlich Chromatik und Amplitudendämpfung, und zeigte
einige Diskrepanzen zwischen Messungen und Modellvorhersagen, insbesondere hinsichtlich potenzieller
nicht modellierter Quellen. Resonance Driving Terms (RDTs) wurden erstmals erfolgreich gemessen,
obwohl Herausforderungen aufgrund von Faktoren wie Dekohärenz und Dämpfung blieben. Insgesamt
stimmen die Ergebnisse mit alternativen KEK-Methoden überein, was darauf hindeutet, dass die
Techniken des CERN effektiv sind, um das Verständnis von SuperKEKB und zukünftigen Beschleunigern
wie dem FCC-ee zu verbessern.

% Zusammenfassung
Zusammenfassend verdeutlicht die in diesen Studien detaillierte Forschung die entscheidende
Bedeutung des Verständnisses und der Verwaltung höhergradiger multipolarer Effekte im LHC. Schiefe
Oktupolfelder, Dekapolfelder und andere höhergradige Felder haben einen erheblichen Einfluss auf die
dynamische Apertur und die Lebensdauer des Strahls. Die Entwicklung und Implementierung
fortschrittlicher Mess- und Korrekturtechniken sind entscheidend, um das Verständnis der
nicht-linearen Optik zu verbessern. Die aus diesen Studien gewonnenen Erkenntnisse sind von
entscheidender Bedeutung für die Optimierung der Leistung des LHC.
\\
\indent Da Teilchenbeschleuniger weiterhin weiterentwickelt werden, werden die Herausforderungen im
Zusammenhang mit höheren multipolaren Komponenten bestehen bleiben. Fortlaufende Forschung in diesem
Bereich ist von entscheidender Bedeutung, um diese Herausforderungen zu bewältigen und
sicherzustellen, dass zukünftige Beschleuniger die erforderliche Präzision für neue
wissenschaftliche Entdeckungen erreichen. Die aus den Erfahrungen des LHC mit den komplexen
Wechselwirkungen der multipolaren Felder gewonnenen Erkenntnisse werden das Design und den Betrieb
zukünftiger Beschleuniger der nächsten Generation wie den HL-LHC und FCC informieren, die zunehmend
auf eine präzise Kontrolle der nicht-linearen Optik angewiesen sein werden. Diese Fortschritte
werden sicherstellen, dass diese Beschleuniger an der Spitze der Erforschung grundlegender Fragen
zum Universum bleiben.
\\
\indent Die in diesen Kapiteln präsentierte Arbeit stellt einen bedeutenden Beitrag zum Bereich der
Beschleunigerphysik dar, indem sie praktische Lösungen für aktuelle Betriebsherausforderungen bietet
und den Weg für zukünftige Fortschritte ebnet.


}