% =============================
%       Extended Summary
% =============================
\chapter{\review{Extended Summary}}


\ifthenelse{\equal{\papersize}{A4}}{ % A4
    \newcommand{\fontsizesummary}{12pt}
    \newcommand{\fontskipsummary}{14pt}
}{  % else, B5
    \newcommand{\fontsizesummary}{11pt}
    \newcommand{\fontskipsummary}{11pt}
}



{
% 5 pages is reeaally long, let's just add some spacing so it's easier for the reader to follow
\fontsize{\fontsizesummary}{\fontskipsummary}\selectfont

% Introduction
The Large Hadron Collider (LHC) at CERN, situated in a 27-kilometer tunnel beneath the Swiss-French
border, is the world's largest and most powerful particle accelerator. Its primary mission is to
recreate the conditions of the universe just moments after the Big Bang, enabling scientists to
explore the fundamental forces and particles that constitute the cosmos. This remarkable facility
accelerates protons and heavy ions to nearly the speed of light before colliding them with immense
energy, providing profound insights into the building blocks of matter and the fundamental
interactions that govern our universe. The LHC represents not only a monumental engineering
achievement but also a crucial tool for advancing our understanding of high-energy physics.
\\
\indent
Operating such a complex and powerful machine requires overcoming numerous technical challenges,
particularly in maintaining the precise control and stability of its particle beams. One significant
challenge lies in managing the effects of higher-order magnetic fields, which often arise from field
errors in the magnets used to guide and focus the particle beams. These field errors can profoundly
impact beam dynamics and its lifetime. Addressing these challenges is essential to ensuring that
the LHC operates effectively and continues to produce valuable scientific results. Furthermore,
non-linear optics control is crucial for the success of future accelerators like the HL-LHC and FCC,
which will increasingly rely on high-order non-linear optics corrections to achieve their
performance targets. The following extended summary synthesizes findings from three detailed studies
investigating various aspects of higher-order multipole effects and their implications on the LHC's
beam dynamics.


% Concepts
This thesis employs the Hamiltonian formalism to describe particle motion in the transverse planes
under the influence of various multipole fields. In non-linear lattices, the complexity of beam
dynamics increases significantly, necessitating the use of advanced mathematical tools such as Lie
Algebra and Poisson Brackets to accurately characterize non-linear effects. The study derives
explicit higher-order non-linear transfer maps and provides a comprehensive summary of multipole
combinations. These non-linearities in the lattice lead to complex phenomena such as high-oder 
chromaticity, amplitude detuning, chromatic amplitude detuning, and resonances driven by Resonance
Driving Terms (RDTs), all of which are thoroughly derived and supported by detailed measurement
techniques.
\\
\indent
Optics measurements are conducted using a wide range of techniques and software tools. Turn-by-turn
data acquisition via Beam Position Monitors (BPMs) is emphasized as a crucial method for evaluating
beam optics, where an AC-dipole excites the beam, and the resulting oscillations are analyzed using
Fourier transforms to extract tunes and identify resonances. Further treatment is done via the
oscillation amplitudes and the magnitude of spectral lines to retrieve linear and non-linear
observables such as the phase advance, beta function, dispersion, coupling, orbit, and RDTs.
Chromaticity measurements involve inducing momentum offsets by varying the RF frequency and
observing the corresponding tune shifts.
\\
\indent
To advance the understanding of high-order non-linear fields, new measurement and analysis methods
have been developed. One such tool, the Non-Linear Chromaticity GUI, simplifies the process of
analyzing and correcting chromaticity during operation. Additionally, a novel response matrix
approach has been introduced, enabling efficient direct corrections of Resonance Driving Terms
(RDTs) in the LHC. This marks a shift from empirical correction adjustments to a more quantitative
and systematic method. These techniques have proven effective in correcting several key observables,
as discussed throughout this thesis. The development of these methods has been crucial in reducing
commissioning time, enabling a greater focus on high-order multipoles, from octupoles to
decatetrapoles, some of which have never been studied before. These investigations were further
facilitated by improvements in the collimator sequence, allowing the exploration of a higher action
and momentum offsets. Additionally, enhancements in dynamic aperture, as presented in this thesis,
provided the required amplitudes to probe these higher-order effects more effectively.


% First Study
The first chapter explores the origins and consequences of skew octupolar fields within the LHC.
These fields significantly influence the dynamic aperture of the accelerator, a parameter
that defines the amplitudes within which the particle beam remains stable. Skew octupolar correctors
are installed around key detectors, such as ATLAS and CMS, to manage these fields and mitigate their
effects on beam stability. The study focuses on measuring these fields with optics designed for top
energy, at 6.8 TeV per particle. Corrections were performed using a response matrix based approach,
a different method than what was used a few years prior, marking a shift from empirical corrections
to a more quantitative and systematic approach for the first time in the LHC. 
This method effectively addresses skew octupolar RDTs using the available corrector magnets,
although its performance is limited by the absence of one corrector, which constrains the achievable
correction strength. Consequently, the RDTs of interest, $f_{1012}$ and $f_{1210}$, can either be
effectively corrected or maintained at a constant level depending on the corrector configuration.
\\
\indent
Additionally, the study investigates the unexpected influence of Landau octupoles on skew octupolar
RDTs at injection energy, at 450 GeV per particle. Landau octupoles, which are powerful magnets used
at injection energy to introduce multi-particle coherent instabilities damping through a tune
spread, generated a significant shift in skew octupolar RDTs during measurements under various
powering configurations. Skew octupolar resonances have previously been identified as a source of
emittance growth in the presence of electron clouds. The more intuitive explanation for the
generation of skew fields by normal multipoles had been the misalignments of these octupoles,
specifically roll errors. However, simulations indicate that octupole misalignments have minimal
impact on skew octupolar fields. Instead, transverse coupling has been identified as a crucial
factor. The combination of coupling and the strong powering of Landau octupoles at injection energy
is expected to be a major contributor to skew octupolar RDTs. Therefore, precise modeling of
coupling is essential for predicting the behavior of skew octupolar RDTs. Accurate modeling and
correction of skew octupolar fields in the LHC are essential to suppress resonances and improve both
the beam's dynamic stability and lifetime.


% Second Study
The second chapter focuses on decapolar fields in the LHC, particularly at injection energy. As the
FCC is expected to rely on precise control of decapolar fields, their accurate study in the LHC is
essential to increase confidence in its design and operational strategies.  Consequently, this study
addresses previously observed discrepancies between measurements and models related to third-order
chromaticity, a critical parameter that describes how particles with an energy deviation experience
different oscillation frequencies than the reference particle. Accurate control of chromaticity is
vital for maintaining beam stability. The introduction of previously unobserved observables, has
provided a clearer understanding of these discrepancies.  Among these observables are the bare
chromaticity, which represents the chromaticity of the machine without any correctors powered on to
observe the bare influence of field errors, and chromatic amplitude detuning, a detuning function of
both momentum deviations and oscillation amplitudes. Several approaches to measuring the same fields
help clarify the various contributions. The research reveals that the decay of the decapolar
component in the main dipoles is a significant factor contributing to these discrepancies. When the
LHC was designed, this decay was deemed too small to be significant and thus was not included in the
magnetic error tables used for simulation. However, as the machine's parameters are pushed further
each year and the effects of higher-order fields become better understood, it becomes clear that
accurate control and modeling of these fields are necessary.
\\
\indent
For the first time in the LHC, measurements and corrections of the decapolar Resonance Driving Term
$f_{1004}$ were carried out at injection energy. Corrections are based on a response matrix
approach, effectively implementing combined corrections of third-order chromaticity, chromatic
amplitude detuning, and RDT $f_{1004}$, leading to a 3\% improvement in beam lifetime.  Conversely,
deliberately degrading the RDT alone resulted in a 10\% decrease in beam lifetime, underscoring the
importance of this resonance corrections for stable beam operation. The study also explored how
sextupoles and octupoles interact to generate decapolar-like fields. It was found that sextupoles,
both alone and in combination with Landau octupoles, produce a substantial $f_{1004}$ decapolar RDT
when powered to small currents. Therefore, in an operational context, decapolar resonances, largely
generated by strong octupoles, would benefit from adapted corrections. These findings suggest that
further advancements in correction methods could lead to even greater improvements in beam
lifetime.

% Third Study
The third chapter investigates very-high-order fields in the LHC, specifically dodecapolar and
decatetrapolar fields. Using a newly implemented collimation setup and custom post-processing
techniques, this study successfully observed these higher-order fields. Studies were conducted to
estimate the effect of the non-linearity of the momentum compaction factor on the chromaticity
function during its computation from the RF frequency. The results indicate that while the momentum
compaction factor expansion shows a second order in the LHC, its effects on the resulting
chromaticity are negligible even at large momentum offsets. Several chromaticity measurements with
varying configurations of octupolar and decapolar corrections then revealed the presence of fourth
and fifth-order terms ($Q^{(4)}$ and $Q^{(5)}$). These measurements consistently identified these
higher-order terms with similar values, demonstrating their robustness. Additionally, it is
emphasized that accurately characterizing the lower-order terms requires good measurement of these
higher-order terms. The study identifies, through simulations, dodecapolar and decatetrapolar fields
as primary contributors to these higher-order effects, originating from field errors in the main
dipoles and quadrupoles. The LHC's field error model appears to be in relative agreement with the
measurements once the decay of decatetrapolar components is considered.
\\
\indent
For the first time at injection energy, the dodecapolar Resonance Driving Term $f_{0060}$ was
measured. This measurement shows clear repeatability, even when performed with different
configurations of octupolar and decapolar corrections. The measured values were found to be in good
agreement with the model. The research concludes that further investigations are needed to address
limitations in the measurement range of the chromaticity function and to refine estimates of
higher-order chromaticity terms. Additionally, studying the impact of lower-order multipoles on the
dodecapolar RDT and its effect on beam lifetime would be valuable for optimizing the LHC's
performance.

% SUPERKEKB
A EAJADE secondment at SuperKEKB during its February 2024 commissioning utilized optics
measurement techniques from CERN on the HER and LER rings. Linear optics measurements showed good
agreement with the Closed Orbit Distortion (COD) method, demonstrating repeatability over time. For
the first time, vertical plane measurements with an injection offset were conducted, yielding
promising results. The study extended to non-linear optics, including chromaticity and amplitude
detuning, revealing some discrepancies between measurements and model predictions, particularly
concerning potential unmodeled sources. Resonance Driving Terms (RDTs) were measured successfully
for the first time, although challenges remained due to factors like decoherence and damping.
Overall, the findings align with alternative KEK methods, indicating that CERN's techniques are
effective for enhancing understanding of SuperKEKB and future accelerators like the FCC-ee.


% Summary
In summary, the research detailed in these studies underscores the critical importance of
understanding and managing higher-order multipole effects in the LHC. Skew octupolar fields,
decapolar fields, and other higher-order fields have a significant impact on the beam's dynamic
aperture and lifetime. Developing and implementing advanced measurement techniques and correction
methods is essential for enhancing the understanding of non-linear optics. The insights gained from
these studies are crucial for optimizing the LHC's performance.
\\
\indent
As particle accelerators continue to evolve, the challenges associated with higher-order multipole
components will persist. Ongoing research in this field is vital for addressing these challenges and
ensuring that future accelerators achieve the precision required for new scientific discoveries. The
lessons learned from the LHC's experience with the complex interactions of multipole fields will
inform the design and operation of next-generation accelerators, such as the HL-LHC and FCC, which
will increasingly rely on precise non-linear optics control. These advancements will ensure that
these accelerators remain at the forefront of exploring fundamental questions about the universe.
\\
\indent
The work presented in these chapters represents a significant contribution to the field of
accelerator physics by offering practical solutions for current operational challenges and paving
the way for future advancements.

} % last empty line is important to get an implicit \par


% =============================
%       Zusammenfassung
% =============================
\chapter{\review{Zusammenfassung}}

{
\fontsize{\fontsizesummary}{\fontskipsummary}\selectfont

% Einleitung
Der Large Hadron Collider (LHC) am CERN, der sich in einem 27 Kilometer langen Tunnel unter der Grenze zwischen der Schweiz und Frankreich befindet, ist der größte und leistungsstärkste Teilchenbeschleuniger der Welt. Seine Hauptaufgabe besteht darin, die Bedingungen des Universums unmittelbar nach dem Urknall nachzubilden, um Wissenschaftler*innen die Erforschung der fundamentalen Kräfte und Teilchen zu ermöglichen, die das Universum ausmachen. Diese bemerkenswerte Anlage beschleunigt Protonen und schwere Ionen nahezu auf Lichtgeschwindigkeit, bevor sie mit enormer Energie kollidieren, was tiefgreifende Einblicke in die Bausteine der Materie und die grundlegenden Wechselwirkungen, die unser Universum bestimmen, liefert. Der LHC stellt nicht nur eine monumentale ingenieurtechnische Leistung dar, sondern ist auch ein wesentliches Instrument zur Förderung unseres Verständnisses der Hochenergiephysik.
\\
\indent
Der Betrieb einer derart komplexen und leistungsfähigen Maschine erfordert die Bewältigung zahlreicher technischer Herausforderungen, insbesondere in Bezug auf die präzise Steuerung und Stabilität der Teilchenstrahlen. Eine wesentliche Herausforderung besteht darin, die Auswirkungen höherer magnetischer Felder zu beherrschen, die häufig durch Feldfehler in den Magneten entstehen, die zur Führung und Fokussierung der Teilchenstrahlen verwendet werden. Diese Feldfehler können die Strahldynamik und die Strahllebensdauer erheblich beeinflussen. Die Bewältigung dieser Herausforderungen ist entscheidend, um den LHC effektiv zu betreiben und weiterhin wertvolle wissenschaftliche Ergebnisse zu erzielen. Darüber hinaus ist die Kontrolle nichtlinearer Optiken von entscheidender Bedeutung für den Erfolg zukünftiger Beschleuniger wie den HL-LHC und den FCC, die zunehmend auf nichtlineare Korrekturen höherer Ordnung angewiesen sein werden, um ihre Leistungsziele zu erreichen. Die folgende erweiterte Zusammenfassung fasst die Ergebnisse aus drei detaillierten Studien zusammen, die verschiedene Aspekte der höheren Multipol-Effekte und deren Auswirkungen auf die Strahldynamik des LHC untersuchen.

% Konzepte
Diese Dissertation verwendet den Hamiltonschen Formalismus, um die Teilchenbewegung in den transversalen Ebenen unter dem Einfluss verschiedener Multipolfelder zu beschreiben. In nichtlinearen Gittern steigt die Komplexität der Strahldynamik erheblich an, weshalb fortschrittliche mathematische Werkzeuge wie die Lie-Algebra und Poisson-Klammern notwendig sind, um nichtlineare Effekte genau zu charakterisieren. Die Arbeit leitet explizite nichtlineare Übertragungsabbildungen höherer Ordnung her und bietet eine umfassende Zusammenfassung von Multipolkombinationen. Diese Nichtlinearitäten im Gitter führen zu komplexen Phänomenen wie hoher Chromatizität, Amplitudentuning, chromatischem Amplitudentuning und Resonanzen, die durch Resonanzanregungsbegriffe (RDTs) getrieben werden. All diese Phänomene werden ausführlich abgeleitet und durch detaillierte Messtechniken unterstützt.
\\
\indent
Optikmessungen werden mit einer Vielzahl von Techniken und Software-Tools durchgeführt. Die turn-by-turn-Datenerfassung über Beam Position Monitors (BPMs) wird als entscheidende Methode zur Bewertung der Strahloptik hervorgehoben, bei der ein AC-Dipol den Strahl anregt und die resultierenden Oszillationen mittels Fourier-Transformationen analysiert werden, um Tunes zu extrahieren und Resonanzen zu identifizieren. Eine weiterführende Analyse erfolgt über die Oszillationsamplituden und die Größe der Spektrallinien, um lineare und nichtlineare Observablen wie Phasenvorschub, Betafunktion, Dispersion, Kopplung, Orbit und RDTs zu bestimmen. Chromatizitätsmessungen beinhalten das Induzieren von Impulsabweichungen durch Änderung der HF-Frequenz und die Beobachtung der entsprechenden Tune-Verschiebungen.
\\
\indent
Um das Verständnis nichtlinearer Felder höherer Ordnung zu verbessern, wurden neue Mess- und Analysemethoden entwickelt. Eines dieser Werkzeuge, das Non-Linear Chromaticity GUI, vereinfacht den Prozess der Analyse und Korrektur der Chromatizität während des Betriebs. Zusätzlich wurde ein neues Ansatzverfahren mit Antwortmatrizen eingeführt, das effiziente direkte Korrekturen der Resonanzanregungsbegriffe (RDTs) im LHC ermöglicht. Dies stellt einen Übergang von empirischen Korrekturanpassungen hin zu einer quantitativeren und systematischeren Methode dar. Diese Techniken haben sich als effektiv bei der Korrektur mehrerer wichtiger Observablen erwiesen, wie im Laufe dieser Dissertation erörtert wird. Die Entwicklung dieser Methoden war entscheidend für die Reduzierung der Inbetriebnahmezeit, was einen stärkeren Fokus auf höhere Multipole, von Oktupolen bis hin zu Dekatetrapolen, ermöglichte – einige davon wurden zuvor nie untersucht. Diese Untersuchungen wurden zudem durch Verbesserungen in der Kollimatorschaltung erleichtert, die die Erforschung größerer Aktionen und Impulsabweichungen ermöglichten. Darüber hinaus trugen die in dieser Dissertation vorgestellten Verbesserungen der dynamischen Apertur dazu bei, die erforderlichen Amplituden bereitzustellen, um diese höherordnigen Effekte effektiver zu untersuchen.

% Erste Studie
Das erste Kapitel untersucht die Ursprünge und Auswirkungen schiefer Oktupolfelder innerhalb des LHC. Diese Felder beeinflussen erheblich die dynamische Apertur des Beschleunigers, ein Parameter, der die Amplituden definiert, innerhalb derer der Teilchenstrahl stabil bleibt. Schiefe Oktupolkorrektoren sind um wichtige Detektoren wie ATLAS und CMS installiert, um diese Felder zu steuern und ihre Auswirkungen auf die Strahlstabilität zu mindern. Die Studie konzentriert sich darauf, diese Felder mit Optiken zu messen, die für die Höchstenergie von 6,8 TeV pro Teilchen ausgelegt sind. Korrekturen wurden mithilfe eines antwortmatrixbasierten Ansatzes durchgeführt, was eine andere Methode darstellt als die, die einige Jahre zuvor verwendet wurde. Dies markiert einen Übergang von empirischen Korrekturen zu einem quantitativeren und systematischeren Ansatz für den LHC.
Diese Methode adressiert effektiv schiefe Oktupol-RDTs unter Verwendung der verfügbaren Korrektormagnete, obwohl ihre Leistung durch das Fehlen eines Korrektors eingeschränkt ist, was die erreichbare Korrekturstärke begrenzt. Folglich können die von Interesse stehenden RDTs, $f_{1210}$ und $f_{1012}$ je nach Korrektorkonfiguration entweder effektiv korrigiert oder auf einem konstanten Niveau gehalten werden.
\\
\indent
Darüber hinaus untersucht die Studie den unerwarteten Einfluss von Landau-Oktupolen auf die schiefen Oktupol-RDTs bei Einspeiseenergie von 450 GeV pro Teilchen. Landau-Oktupole, die bei Einspeiseenergie als leistungsstarke Magnete eingesetzt werden, um durch eine Tune-Verbreiterung Dämpfung einzuführen, erzeugten während der Messungen unter verschiedenen Einspeisekonfigurationen eine signifikante Verschiebung der schiefen Oktupol-RDTs. Schiefe Oktupolresonanzen wurden zuvor als eine Quelle für Emittanzwachstum in Gegenwart von Elektronenwolken identifiziert. Die intuitivere Erklärung für die Erzeugung schiefer Felder durch normale Multipole war die Fehljustierung dieser Oktupole, insbesondere Rollfehler. Simulationen zeigen jedoch, dass Fehljustierungen von Oktupolen einen minimalen Einfluss auf schiefe Oktupolfelder haben. Stattdessen wurde die Kopplung als ein entscheidender Faktor identifiziert. Die Kombination aus Kopplung und der starken Einspeisung von Landau-Oktupolen bei Einspeiseenergie wird als Hauptursache für die schiefen Oktupol-RDTs angesehen. Daher ist eine präzise Modellierung der Kopplung unerlässlich, um das Verhalten der schiefen Oktupol-RDTs vorherzusagen. Eine genaue Modellierung und Korrektur der schiefen Oktupolfelder im LHC sind entscheidend, um Resonanzen zu unterdrücken und sowohl die dynamische Stabilität als auch die Lebensdauer des Strahls zu verbessern.

% Zweite Studie
Das zweite Kapitel konzentriert sich auf Dekapolfelder im LHC, insbesondere bei Einspeiseenergie. Da beim FCC eine präzise Kontrolle der Dekapolfelder erwartet wird, ist ihre genaue Untersuchung im LHC entscheidend, um das Vertrauen in das Design und die Betriebsstrategien zu erhöhen. Folglich behandelt diese Studie zuvor beobachtete Diskrepanzen zwischen Messungen und Modellen in Bezug auf die chromatische Aberration dritter Ordnung, einen kritischen Parameter, der beschreibt, wie Teilchen mit einer Energieabweichung unterschiedliche Oszillationsfrequenzen als das Referenzteilchen erfahren. Eine genaue Kontrolle der Chromatizität ist entscheidend für die Aufrechterhaltung der Strahlstabilität. Die Einführung zuvor nicht beobachteter Observablen hat ein klareres Verständnis dieser Diskrepanzen ermöglicht. Zu diesen Observablen gehören die nackte Chromatizität, die die Chromatizität der Maschine ohne aktivierte Korrektoren darstellt, um den unmittelbaren Einfluss von Feldfehlern zu beobachten, sowie das chromatische Amplitudentuning, eine Tuningfunktion sowohl von Impulsabweichungen als auch von Oszillationsamplituden. Mehrere Ansätze zur Messung derselben Felder helfen, die verschiedenen Beiträge zu klären. Die Forschung zeigt, dass der Abfall der dekapolaren Komponente in den Hauptdipolen ein signifikanter Faktor ist, der zu diesen Diskrepanzen beiträgt. Als der LHC entworfen wurde, wurde dieser Abfall als zu gering erachtet, um signifikant zu sein, und wurde daher nicht in den Magnetfehlertabellen berücksichtigt, die für Simulationen verwendet wurden. Mit der zunehmenden Beanspruchung der Maschinenparameter und dem besseren Verständnis der Auswirkungen höherer Felder wird jedoch deutlich, dass eine genaue Kontrolle und Modellierung dieser Felder notwendig ist.
\\
\indent
Erstmals im LHC wurden Messungen und Korrekturen des dekopolar Resonance Driving Term $f_{1004}$ bei Einspeiseenergie durchgeführt. Die Korrekturen basieren auf einem antwortmatrixbasierten Ansatz, der die kombinierten Korrekturen der chromatischen Aberration dritter Ordnung, des chromatischen Amplitudentunings und des RDT $f_{1004}$ effektiv umsetzt, was zu einer Verbesserung der Strahllaufzeit um $3\%$ führt. Im Gegensatz dazu führte die absichtliche Verschlechterung des RDT allein zu einem Rückgang der Strahllaufzeit um 10 %, was die Bedeutung dieser Resonanzkorrekturen für den stabilen Strahlbetrieb unterstreicht. Die Studie untersuchte auch, wie Sextupole und Oktupole interagieren, um dekabolare ähnliche Felder zu erzeugen. Es wurde festgestellt, dass Sextupole, sowohl allein als auch in Kombination mit Landau-Oktupolen, bei kleinen Strömen eine substanzielle dekabolare RDT f1004f1004​ erzeugen. Daher würden im operativen Kontext dekabolare Resonanzen, die hauptsächlich durch starke Oktupole erzeugt werden, von angepassten Korrekturen profitieren. Diese Ergebnisse deuten darauf hin, dass weitere Fortschritte bei den Korrekturmethoden zu noch größeren Verbesserungen der Strahllaufzeit führen könnten.

% Dritte Studie
Das dritte Kapitel untersucht sehr-hochordentliche Felder im LHC, speziell dodecapolare und decatetrapolare Felder. Mit einer neu implementierten Kollimationsanordnung und benutzerdefinierten Nachbearbeitungstechniken konnte diese Studie erfolgreich diese höherordentlichen Felder beobachten. Es wurden Studien durchgeführt, um den Effekt der Nichtlinearität des Impulskompaktierungsfaktors auf die Chromatizitätsfunktion während ihrer Berechnung aus der RF-Frequenz zu schätzen. Die Ergebnisse zeigen, dass der Impulskompaktierungsfaktor zwar im LHC eine zweite Ordnung aufweist, seine Auswirkungen auf die resultierende Chromatizität jedoch vernachlässigbar sind, selbst bei großen Impulsabweichungen. Mehrere Chromatizitätsmessungen mit unterschiedlichen Konfigurationen von oktupolarer und dekpolarer Korrektur offenbarten dann das Vorhandensein von vierten und fünften Ordnungsbegriffen ($Q^{(4)}$ und $Q^{(5)}$). Diese Messungen identifizierten konsistent diese höherordentlichen Terme mit ähnlichen Werten, was ihre Robustheit demonstriert. Darüber hinaus wird betont, dass eine genaue Charakterisierung der niedrigeren Ordnungen eine gute Messung dieser höherordentlichen Terme erfordert. Die Studie identifiziert durch Simulationen dodecapolare und decatetrapolare Felder als primäre Beiträge zu diesen höherordentlichen Effekten, die aus Feldfehlern in den Hauptdipolen und Quadrupolen stammen. Das Modell der Feldfehler des LHC scheint relativ gut mit den Messungen übereinzustimmen, wenn der Abfall der decatetrapolaren Komponenten berücksichtigt wird.  
\\
\indent
Für die erste Messung bei Einspeiseenergie wurde der dodepolare Resonance Driving Term $f_{0060}$ gemessen. Diese Messung zeigt eine klare Wiederholbarkeit, selbst wenn sie mit unterschiedlichen Konfigurationen von oktupolarer und dekpolarer Korrektur durchgeführt wird. Die gemessenen Werte wurden als gut mit dem Modell übereinstimmend befunden. Die Forschung kommt zu dem Schluss, dass weitere Untersuchungen erforderlich sind, um die Einschränkungen im Messbereich der Chromatizitätsfunktion anzugehen und die Schätzungen höherordentlicher Chromatizitätsbegriffe zu verfeinern. Darüber hinaus wäre es wertvoll, den Einfluss niedrigerer Multipole auf den dodepolaren RDT und dessen Auswirkungen auf die Strahllaufzeit zu untersuchen, um die Leistung des LHC zu optimieren.

% Zusammenfassung
Zusammenfassend unterstreicht die in diesen Studien detaillierte Forschung die kritische Bedeutung des Verständnisses und der Steuerung höherordentlicher Multipoleffekte im LHC. Schiefe Oktupolfelder, Dekapolfelder und andere höherordentliche Felder haben einen signifikanten Einfluss auf die dynamische Apertur und die Lebensdauer des Strahls. Die Entwicklung und Implementierung fortschrittlicher Messmethoden und Korrekturmethoden ist entscheidend für das Verständnis der nichtlinearen Optik. Die aus diesen Studien gewonnenen Erkenntnisse sind entscheidend für die Optimierung der Leistung des LHC.  
\\
\indent
Da sich Teilchenbeschleuniger weiterhin entwickeln, werden die Herausforderungen im Zusammenhang mit höheren Multipolen bestehen bleiben. Laufende Forschung in diesem Bereich ist von entscheidender Bedeutung, um diese Herausforderungen zu bewältigen und sicherzustellen, dass zukünftige Beschleuniger die für neue wissenschaftliche Entdeckungen erforderliche Präzision erreichen. Die aus den Erfahrungen des LHC mit den komplexen Wechselwirkungen von Multipolfeldern gewonnenen Erkenntnisse werden das Design und den Betrieb künftiger Beschleuniger, wie HL-LHC und FCC, die zunehmend auf eine präzise Kontrolle der nichtlinearen Optik angewiesen sein werden, informieren. Diese Fortschritte werden sicherstellen, dass diese Beschleuniger an der Spitze der Erforschung fundamentaler Fragen zum Universum bleiben.  
\\
\indent
Die in diesen Kapiteln präsentierte Arbeit stellt einen bedeutenden Beitrag zum Bereich der Beschleunigerphysik dar, indem sie praktische Lösungen für aktuelle Betriebsherausforderungen bietet und den Weg für zukünftige Fortschritte ebnet.


}