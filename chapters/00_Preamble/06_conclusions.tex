\chapter{\review{Conclusions}}
\label{chapter:conclusions}

This thesis has provided a comprehensive analysis of higher-order magnetic fields in the Large
Hadron Collider (LHC) and their significant impact on beam dynamics and stability. The research
underscores the critical importance of understanding and correcting skew octupolar, decapolar,
dodecapolar, and decatetrapolar fields to ensure optimal performance of the LHC, particularly at
injection energy.

The investigation into skew octupolar fields made use a response matrix approach for correcting
resonance driving using corrector magnets. The research further identified the interplay between
Landau octupoles and skew octupolar RDTs, highlighting the essential role of accurate coupling
modeling.

The study of decapolar fields addressed discrepancies between measured and predicted third-order
chromaticity, pinpointing the decay of decapolar components in the main dipoles as a key factor. The
development of new correction strategies led to tangible improvements in beam lifetime and
stability, demonstrating the effectiveness of these approaches.

Finally, the thesis introduced innovative measurement techniques for higher-order chromaticity
terms, revealing the contribution of dodecapolar and decatetrapolar fields errors. Furthermore,
first measurements of dodecapolar resonance driving terms were made.  These findings reinforce the
need for further exploration of higher-order fields to enhance the LHC's operational efficiency.

Overall, the research provides valuable insights into the complex interplay of magnetic fields
within the LHC and lays the groundwork for future advancements in collider performance and
stability.