\chapter{\review{Conclusions}}
\label{chapter:conclusions}

The primary aim of this thesis has been to extend and refine methods for the quantitative analysis
of non-linear optics, an area of long-standing interest not only for the LHC but also for the field
of accelerator physics in general. By pushing these studies into regimes where limited or no prior
research has been conducted, this work significantly enhances the understanding of error sources in
the LHC and improves its magnetic model.

The investigation into skew octupolar fields introduced, for the first time at the LHC, a response
matrix approach for correcting Resonance Driving Terms (RDTs) using the available corrector magnets.
This method enabled faster and more efficient commissioning of the low-$\beta$ optics, with
corrections resulting in a net improvement in dynamic aperture. Further research explored the
interaction between normal Landau octupoles and skew octupolar RDTs. One key insight from this work
is that even small changes in coupling at injection can have a substantial impact, underscoring the
need for careful consideration of this parameter in future studies and operational correction
strategies. This is especially relevant for non-linear studies, which often assume a static
reference value for coupling.

By leading studies on decapolar perturbations and their interaction with various multipole
components, this thesis has refined the magnetic model for decapolar errors. This improved
understanding has allowed for more effective correction strategies, leading to both increased
machine lifetime and greater dynamic aperture, key factors for non-linear optics studies.
Furthermore, previously unaccounted-for decapolar decay in the main dipoles has now been identified
as the main contributor to the discrepancies observed in third-order chromaticity ($Q'''$)
corrections.

In addition, the thesis presents the first measurement of dodecapolar ($b_6$) RDTs at injection
energy, made possible by improvements in lifetime and dynamic aperture. The introduction of
innovative measurement techniques for higher-order chromaticity terms revealed, for the first time,
the contribution of both dodecapolar and decatetrapolar fields. These findings underscore the
importance of further exploring higher-order fields to optimize the dynamic aperture, especially in
preparation for future colliders such as the FCC-ee, which will rely heavily on the precise control
of high-order fields for operational efficiency.

Overall, this research provides valuable insights into the complex interplay of magnetic fields
within the LHC, laying crucial groundwork for future advancements in collider performance and
stability.
