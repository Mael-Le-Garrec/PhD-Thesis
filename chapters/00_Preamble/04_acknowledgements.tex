% ===========================================================================
%                  Acknowledgements
% ===========================================================================
\chapter{\review{Acknowledgements}}

\ifthenelse{\equal{\papersize}{A4}}{ % A4
}{  % else, B5
    \vspace{-0.4cm}
}

{
\fontsize{\fontsizeabstract}{\fontskipabstract}\selectfont

Ok, so this is the place where I get to talk about my life and where you'll be looking for you name
to see what I've got to say about you. This would need to be a blog post given how many things I
want to say, so it will be \textit{relatively} brief.
Let's start first by saying that I'm incredibly happy to have to done this thesis at CERN, which I
honestly found quite fun. Not only because the subject was interesting, but also because I have been
surrounded by kind, caring and smart people through all these years.\\
\indent
To give a little perspective, I have originally studied computer science at EPITA (École Pour
l'Informatique et les Techniques Avancées) in Paris and earned an engineering degree in that field.
Nothing really predestined me to work in any field related to physics, let alone to do a PhD on 
accelerator physics, but here we are! After my degree, I applied for a trainee position at CERN, 
basically working on software development for optics measurements and corrections. After two years
of training, I was proposed with a PhD on the topics I've been learning, which I gladly accepted.
This is how I embarked on a 3-year journey about accelerator physics. For this, I would never be 
thankful enough to Rogelio Tomas, who recruited and trusted me for that first contract, without
which my life would probably be vastly different.
I am also immensely grateful to Ewen Maclean, who has been a fantastic supervisor, even probably the
best I've had yet to meet. He has always been present to help with me accelerator concepts in a very
intelligible manner. Being kind, following up with my studies and always proposing new ideas made me
feel like a complete member of the OMC team. On the university side, at Goethe-Universität
Frankfurt, I would like to thank Giuliano Franchetti for his guidance and ideas about this thesis.

Through my now five years at CERN, I have had the privilege to meet incredible people who changed
not only my career path but also who I am for the best. And even if it sounds a bit weird, I am
incredibly happy you have all been there.\\
\indent
Upon starting my very first contract, I was paired with Max Mihailescu and Sébastien Joly in an
office, all three beginning our journey at CERN. I am very thankful to Max for having been a good
friend during his short stay here, and for his insights on mathematics. I consider myself
remarkably lucky to have met Sébastien, who probably should be listed as a supervisor here given how 
much he helped me when I struggled to learn the basics of accelerator physics.
I would also like to thank Michael Hofer and Félix Carlier, who were there to answer my countless
questions, be it when I started or even now at the end of my PhD.
Closely related to my thesis subject were Félix Soubelet and Joschua Dilly who greatly helped me in
various ways.
Thanks a lot as well to Jacqueline Keintzel and Frank Zimmerman, to have given me the opportunity to
work for a month at KEK in Japan, and expand my knowledge of lepton accelerators. 

Many thanks to all the members of the OMC team who were great colleagues, be it in the office or
during barely-legal night and week-end shifts in the control room. Specifically, I would like to
thank Wietse Van Goethem, Leon Van Riesen-Haupt, Elena Fol, Andreas Wegscheider, Vittorio Ferrentino
and Tobias Persson. Talking about the control room, I want to thank the LHC operation team and
specifically Michi Hostettler for our conversations and his deep knowledge of CERN and the LHC.
\noindent
CERN being a vast laboratory, I had the pleasure to meet remarkable people who I enjoyed to be
around. I cannot detail how much you all mean to me as this is already quite long, but you get the 
idea. For this, thanks a lot to Sofia, Christophe, David, Jean-Baptiste, Lisa, Joanna, Roxana, Dora,
Joséphine, Jack, Ellie, Tirsi, Kostas, Björn, Christian, Laura, Roland, Luca, Wainer, Tiziana and
Pierre.

This thesis would not have been possible without the support of people from outside of CERN. Namely
I would like to thank my IRC friends, bonswouar, WhatIsGoingDown, pankkake, ShameOnYou, href and
bishop. Special thanks to a large friend group I have always loved to be part of, \textit{la crème
de la crème de la cataphilie}.
Thanks a ton as well to Clotilde, Matthias, Alpha, Paul, Antoine, Nicolas, Youness, Ksenia, Gérard, Vivien,
Baptiste, Félix, Benjamin, Fabian and Tsu for everything that happened during those years,
making them memorable.
I would like to thank a lot my parents and family who have always been there to support me in every
possible way. Thanks Antje, Michel, and Yoann.

}