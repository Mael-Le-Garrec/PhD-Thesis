\chapter{\review{Abstract}}

% That's so the extended summary does not feel too weird
\ifthenelse{\equal{\papersize}{A4}}{ % A4
    \newcommand{\fontsizeabstract}{12pt}
    \newcommand{\fontskipabstract}{14pt}
}{  % else, B5
    \newcommand{\fontsizeabstract}{11pt}
    \newcommand{\fontskipabstract}{11pt}

    \vspace{-0.5cm}
}

{
\fontsize{\fontsizeabstract}{\fontskipabstract}\selectfont

This thesis investigates the crucial role of higher-order magnetic fields and non-linear optics in
the stability and performance of particle accelerators, focusing on the Large Hadron Collider (LHC)
at CERN. The control of non-linear optics, which deals with the interaction of charged particle 
beams with complex magnetic fields such as sextupolar, octupolar, decapolar, and so on, is essential
for managing beam dynamics. The LHC, as the world's most powerful accelerator, provides a unique
opportunity to study these high-order effects, serving as a testbed for future accelerator designs.

These higher-order fields significantly affect the beam's dynamic aperture and lifetime, especially
at injection energy, where precise correction of magnetic field errors is required. Managing these
challenges is not only vital for optimizing LHC performance but also for guiding the design and
operation of next-generation machines.

A key contribution of this work is the development of correction methods for Resonance Driving Terms
(RDTs), a critical factor in beam lifetime and dynamic aperture limitations. New corrective
strategies for RDTs have led to notable improvements in beam lifetime and dynamic aperture at both
injection and top energy operation. This thesis also addresses the discrepancies observed between
experimental measurements and models of beam observables.

These findings highlight the importance of precise modeling and correction of non-linear magnetic
fields, offering insights that will benefit both the LHC and future high-energy particle
accelerators.
}