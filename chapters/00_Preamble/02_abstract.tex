\chapter{\review{Abstract}}

This thesis explores the impact of higher-order magnetic fields on the optics and stability of the
Large Hadron Collider at CERN, with a focus on octupolar, decapolar, dodecapolar, and
decatetrapolar fields. These fields significantly influence beam dynamics, particularly at injection
energy, where precise correction is critical for optimal performance.

A key aspect of the research is the development of correction methods for Resonance Driving Terms
(RDTs), crucial for managing dynamic aperture limitations. Additionally, the work addresses
discrepancies between measurements and models of beam observables, identifying the major
contributors. New corrective strategies for RDTs have led to measurable improvements in beam
lifetime and stability.

The findings underscore the importance of accurate modeling and correction of these fields to
enhance the LHC's operational efficiency and beam stability.