\appendix  % Declare we started the appendix, that switches the numbering to letters


% ==============================================================
%                      Units and Conversions
% ==============================================================
\chapter{Units and Conversions}
\thumbforappendix


% ---------------------------------------
%          Units and Conversions
% ---------------------------------------
\section{Physical Constants}

\begin{table}[H]
    \centering
    \begin{tabular}{lcrl}
    Name                        &    Symbol     &    Value                        &     Unit      \\
    Speed of light in vacuum    &     $c$       &   $2.99792458 \times 10^8$      &      m/s      \\
    Elementary charge           &     $e$       &   $1.60217663 \times 10^9$      &       C       \\
        
    \end{tabular}
    \caption{Physical Constants}
    \label{table:appendix:physical_constants}
\end{table}


\section{Units}


\section{Conversions}


% ==============================================================
%                 Hamiltonians and Transfer Maps
% ==============================================================
\chapter{Hamiltonians and Transfer Maps}
\label{appendix:transfer_maps}
\thumbforappendix

This appendix is intended to gather and explicit the Hamiltonians of the elements used in this 
thesis. Non-linear transfer maps are also described for some of those elements from the first to
the second order.

% ---------------------------------------
%              Hamiltonians
% ---------------------------------------
\section{Hamiltonians of Elements}

The Hamiltonian of a \textit{multipole} is the
following~\cite{keintzel_jacqueline_beam_nodate,tomas_direct_2003,franchi_studies_2006}:

\begin{equation}
    H = \Re \left[ \sum_{n>1} (K_n + iJ_n) \frac{(x+iy)^n}{n!} \right].
    \label{eq:appendix:transfer_maps:hamiltonian}
\end{equation}

From this, normal and skew fields can be separated:

\begin{equation}
    \begin{aligned}
        N_n &= \frac{1}{n!} K_n \Re \left[ (x+iy)^n \right] \\
        S_n &= -\frac{1}{n!} J_n \Im \left[ (x+iy)^n \right],
    \end{aligned}
\end{equation}

where $K$ and $J$ are the normalized strength of the multipole and $x,y$ the transverse coordinates.
\cref{table:appendix:hamiltonians} explicits the normal and skew Hamiltonians of multipoles up to
order 8.

\begin{table}[H]
    \centering
    \begin{tabular}{l  c  p{7.8cm}}
      \hline
       Name & Order & Normal and Skew Hamiltonians\\
      \hline
      \midrule
        Drift          & - & $H = \frac{1}{2} (p_x^2 + p_y^2)$ \\
      %\midrule
      %  Dipole         & 1 & $N_1 = K_{1} x$                                                                                            \newline $S_1 = - J_{1} y$ \\
      \midrule
        Quadrupole     & 2 & $N_2 = \frac{1}{2!} K_{2} \left(x^{2} - y^{2}\right)$                                                       \newline $S_2 = - J_{2} x y$ \\
      \midrule
        Sextupole      & 3 & $N_3 = \frac{1}{3!}K_{3} \left(x^{3} - 3 x y^{2}\right)$                                                    \newline $S_3 = - \frac{1}{3!}J_{3} \cdot \left(3 x^{2} y - y^{3}\right)$ \\
      \midrule
        Octupole       & 4 & $N_4 = \frac{1}{4!}K_{4} \left(x^{4} - 6 x^{2} y^{2} + y^{4}\right)$                                       \newline $S_4 = - \frac{1}{4!}J_{4} \cdot \left(4 x^{3} y - 4 x y^{3}\right)$ \\
      \midrule
        Decapole       & 5 & $N_5 = \frac{1}{5!}K_{5} \left(x^{5} - 10 x^{3} y^{2} + 5 x y^{4}\right)$                                 \newline $S_5 = - \frac{1}{5!}J_{5} \cdot \left(5 x^{4} y - 10 x^{2} y^{3} + y^{5}\right)$ \\
      \midrule
        Dodecapole     & 6 & $N_6 = \frac{1}{6!}K_{6} \left(x^{6} - 15 x^{4} y^{2} + 15 x^{2} y^{4} - y^{6}\right)$                    \newline $S_6 = - \frac{1}{6!}J_{6} \cdot \left(6 x^{5} y - 20 x^{3} y^{3} + 6 x y^{5}\right)$ \\
      \midrule
        Decatetrapole  & 7 & $N_7 = \frac{1}{7!}K_{7} \left(x^{7} - 21 x^{5} y^{2} + 35 x^{3} y^{4} - 7 x y^{6}\right)$               \newline $S_7 = - \frac{1}{7!}J_{7} \cdot \left(7 x^{6} y - 35 x^{4} y^{3} + 21 x^{2} y^{5} - y^{7}\right)$ \\
      \midrule
        Decahexapole   & 8 & $N_8 = \frac{1}{8!}K_{8} \left(x^{8} - 28 x^{6} y^{2} + 70 x^{4} y^{4} - 28 x^{2} y^{6} + y^{8}\right)$ \newline $S_8 = - \frac{1}{8!}J_{8} \cdot \left(8 x^{7} y - 56 x^{5} y^{3} + 56 x^{3} y^{5} - 8 x y^{7}\right)$ \\
      \midrule
      \end{tabular}
  \caption{Normal and skew Hamiltonians of multipoles up to order 8.}
  \label{table:appendix:hamiltonians}
\end{table}



% ---------------------------------------
%             Transfer Maps
% ---------------------------------------
\section{Transfer Maps}

This sections goes more in depth regarding the derivation of the examples of transfer maps 
introduced in~\cref{subsection:coordinate_systems:example_of_maps}.

As a reminder, the BCH of two elements up to order 3 is given below,
\small
\begin{equation}
    \begin{aligned}
      Z = \underbrace{H_{2} + H_{1}}_{\text{First order}}
        + \underbrace{\frac{\left[H_{2},H_{1}\right]}{2}}_{\text{Second order}}
        + \underbrace{\frac{\left[H_{2},\left[H_{2},H_{1}\right]\right]}{12} - \frac{\left[H_{1},\left[H_{2},H_{1}\right]\right]}{12}}_{\text{Third order}}
    \end{aligned}.
\end{equation}
\normalsize


% --------- Generic Expression
\subsection{Generic Effective Hamiltonian of Two Elements}

In order not to have to derive every combination of multipoles, a generic approach can be taken.
Two elements of orders $n$ and $m$ can indeed be combined together via the BCH. Their hamiltonians
is thus the one from \cref{eq:appendix:transfer_maps:hamiltonian} with orders $m$ and $n$.
A drift is put in between the two multipoles to change the coordinates. This results in non-zero
Poisson brackets as the momentum is propagated,

\begin{equation}
  [H_2,H_1] \rightarrow [H_{2}, D \cdot H_{1}].
\end{equation}

Higher orders will arise with derivations when applying the Poisson brackets. Those orders can be 
found by counting the number of Poisson brackets, and which operators are implicated.
A Poisson bracket implicating $H_2$ will increase the resulting order by $m$, and likewise for $H_1$
and $n$. The resulting order is then decreased by the number of poisson brackets times $2$.
\cref{eq:appendix:bch_formula_resulting_order} gives how this can be calculated when considering
the written form of the Poisson brackets (eg. $[H_2, [H_2, H_1]]$).

\begin{equation}
  \text{resulting order} = \text{count}("H_2") \cdot m + \text{count}("H_1") \cdot{n} - 2 \cdot \text{count}("[")
  \label{eq:appendix:bch_formula_resulting_order}
\end{equation}

Although this formula is quite helpful, it is not trivial to compute which orders will arise from
each BCH order. Indeed, not all combinations of $H_1$ and $H_2$ appear, as explained in details 
in~\cite{casas_efficient_2009}. \cref{table:appendix:transfer_maps:bch_resulting_orders} shows the
Poisson brackets corresponding to each order of the BCH, up to order 6, along with the resulting
multipole order. To keep the table readable, Poisson Brackets are not shown above that order, as
duplicates of resulting orders become quite frequent.
Similarly, \cref{table:appendix:transfer_maps:bch_resulting_orders_combination} shows how fields
can be generated depending on the orders of multipoles and that of the BCH.

\begin{table}[H]
  \small
  \centering
  \begin{tabular}{ccc}
    \toprule
      BCH Order & Poisson Brackets & Resulting Order \\
      \midrule
      1 & $A$ & $m$ \\
        & $B$ & $n$ \\
      \midrule
      2 & $\frac{\left[A,B\right]}{2}$ & $m + n - 2$ \\
      \midrule
      3 & $- \frac{\left[B,\left[A,B\right]\right]}{12}$ & $m + 2 n - 4$ \\
        & $\frac{\left[A,\left[A,B\right]\right]}{12}$ & $2 m + n - 4$ \\
      \midrule
      4 & $- \frac{\left[B,\left[A,\left[A,B\right]\right]\right]}{24}$ & $2 m + 2 n - 6$ \\
      \midrule
      5 & $- \frac{\left[A,\left[B,\left[A,\left[A,B\right]\right]\right]\right]}{120}$ & $3 m + 2 n - 8$ \\
        & $- \frac{\left[B,\left[A,\left[A,\left[A,B\right]\right]\right]\right]}{180}$ & $3 m + 2 n - 8$ \\
        & $- \frac{\left[A,\left[B,\left[B,\left[A,B\right]\right]\right]\right]}{360}$ & $2 m + 3 n - 8$ \\
        & $- \frac{\left[A,\left[A,\left[A,\left[A,B\right]\right]\right]\right]}{720}$ & $4 m + n - 8$ \\
        & $\frac{\left[B,\left[B,\left[A,\left[A,B\right]\right]\right]\right]}{180}$ & $2 m + 3 n - 8$ \\
        & $\frac{\left[B,\left[B,\left[B,\left[A,B\right]\right]\right]\right]}{720}$ & $m + 4 n - 8$ \\
      \midrule
      6 & $- \frac{\left[A,\left[A,\left[B,\left[B,\left[A,B\right]\right]\right]\right]\right]}{240}$ & $3 m + 3 n - 10$ \\
        & $\frac{\left[A,\left[B,\left[B,\left[A,\left[A,B\right]\right]\right]\right]\right]}{240}$ & $3 m + 3 n - 10$ \\
        & $\frac{\left[B,\left[B,\left[A,\left[A,\left[A,B\right]\right]\right]\right]\right]}{360}$ & $3 m + 3 n - 10$ \\
        & $\frac{\left[A,\left[B,\left[B,\left[B,\left[A,B\right]\right]\right]\right]\right]}{720}$ & $2 m + 4 n - 10$ \\
        & $\frac{\left[B,\left[A,\left[A,\left[A,\left[A,B\right]\right]\right]\right]\right]}{1440}$ & $4 m + 2 n - 10$ \\
        & $\frac{\left[B,\left[B,\left[B,\left[A,\left[A,B\right]\right]\right]\right]\right]}{1440}$ & $2 m + 4 n - 10$ \\
    %
    %
    %
      \midrule
      7 & $\cdots$ & $2m + 5n - 12$ \\
        &  & $m + 6n - 12$ \\
        &  & $3m + 4n - 12$ \\
        &  & $5m + 2n - 12$ \\
        &  & $6m + n - 12$ \\
        &  & $4m + 3n - 12$ \\
      \midrule
      8 & $\cdots$ & $6m + 2n - 14$ \\
        &  & $2m + 6n - 14$ \\
        &  & $5m + 3n - 14$ \\
        &  & $3m + 5n - 14$ \\
        &  & $4m + 4n - 14$ \\
      %\midrule
      %9 & $\cdots$ & $7m + 2n - 16$ \\
      %  &  & $8m + n - 16$ \\
      %  &  & $6m + 3n - 16$ \\
      %  &  & $5m + 4n - 16$ \\
      %  &  & $2m + 7n - 16$ \\
      %  &  & $m + 8n - 16$ \\
      %  &  & $4m + 5n - 16$ \\
      %  &  & $3m + 6n - 16$ \\
      \bottomrule
  \end{tabular}
  \caption{Resulting multipole orders arising from the Poisson brackets of a given BCH order for two
  multipoles $A$ and $B$ of order $m$ and $n$. Starting from order 7, the Poisson brackets are not
  given as duplicates grow the list significantly.}
  \label{table:appendix:transfer_maps:bch_resulting_orders}
\end{table}


\begin{table}[H]
  \centering
  \footnotesize
  \def\arraystretch{1.7}
  \begin{tabular}{ccccccc}
  \multicolumn{1}{c}{} & \multicolumn{6}{c}{BCH Order} \\
  Field & 1 & 2 & 3 & 4 & 5 & 6 \\
  \midrule
  3 & $\begin{aligned}&H_{3}\end{aligned}$ &  &  &  &  &  \\
  \hline
  4 & $\begin{aligned}&H_{4}\end{aligned}$ & $\begin{aligned}&(H_{3})^2\end{aligned}$ &  &  &  &  \\
  \hline
  5 & $\begin{aligned}&H_{5}\end{aligned}$ & $\begin{aligned}&H_{3}H_{4}\end{aligned}$ & $\begin{aligned}&(H_{3})^3\end{aligned}$ &  &  &  \\
  \hline
  6 & $\begin{aligned}&H_{6}\end{aligned}$ & $\begin{aligned}&H_{3}H_{5},\\&(H_{4})^2\end{aligned}$ & $\begin{aligned}&(H_{3})^2H_{4}\end{aligned}$ & $\begin{aligned}&(H_{3})^4\end{aligned}$ &  &  \\
  \hline
  7 & $\begin{aligned}&H_{7}\end{aligned}$ & $\begin{aligned}&H_{3}H_{6},\\&H_{4}H_{5}\end{aligned}$ & $\begin{aligned}&(H_{3})^2H_{5},\\&H_{3}(H_{4})^2\end{aligned}$ &  & $\begin{aligned}&(H_{3})^5\end{aligned}$ &  \\
  \hline
  8 & $\begin{aligned}&H_{8}\end{aligned}$ & $\begin{aligned}&(H_{5})^2,\\&H_{3}H_{7},\\&H_{4}H_{6}\end{aligned}$ & $\begin{aligned}&(H_{3})^2H_{6},\\&(H_{4})^3\end{aligned}$ & $\begin{aligned}&(H_{3})^2(H_{4})^2\end{aligned}$ & $\begin{aligned}&(H_{3})^4H_{4}\end{aligned}$ & $\begin{aligned}&(H_{3})^6\end{aligned}$ \\
  \hline
  9 & $\begin{aligned}&H_{9}\end{aligned}$ & $\begin{aligned}&H_{4}H_{7},\\&H_{3}H_{8},\\&H_{5}H_{6}\end{aligned}$ & $\begin{aligned}&(H_{3})^2H_{7},\\&(H_{4})^2H_{5},\\&H_{3}(H_{5})^2\end{aligned}$ &  & $\begin{aligned}&(H_{3})^3(H_{4})^2,\\&(H_{3})^4H_{5}\end{aligned}$ &  \\
  \hline
  10 & $\begin{aligned}&H_{10}\end{aligned}$ & $\begin{aligned}&H_{4}H_{8},\\&H_{5}H_{7},\\&H_{3}H_{9},\\&(H_{6})^2\end{aligned}$ & $\begin{aligned}&(H_{3})^2H_{8},\\&H_{4}(H_{5})^2,\\&(H_{4})^2H_{6}\end{aligned}$ & $\begin{aligned}&(H_{4})^4,\\&(H_{3})^2(H_{5})^2\end{aligned}$ & $\begin{aligned}&(H_{3})^2(H_{4})^3,\\&(H_{3})^4H_{6}\end{aligned}$ & $\begin{aligned}&(H_{3})^4(H_{4})^2\end{aligned}$ \\
  \hline
  11 & $\begin{aligned}&H_{11}\end{aligned}$ & $\begin{aligned}&H_{4}H_{9},\\&H_{3}H_{10},\\&H_{6}H_{7},\\&H_{5}H_{8}\end{aligned}$ & $\begin{aligned}&(H_{4})^2H_{7},\\&H_{3}(H_{6})^2,\\&(H_{5})^3,\\&(H_{3})^2H_{9}\end{aligned}$ &  & $\begin{aligned}&(H_{3})^4H_{7},\\&H_{3}(H_{4})^4,\\&(H_{3})^3(H_{5})^2\end{aligned}$ & $\begin{aligned}&(H_{3})^3(H_{4})^3\end{aligned}$ \\
  \hline
  12 & $\begin{aligned}&H_{12}\end{aligned}$ & $\begin{aligned}&H_{6}H_{8},\\&(H_{7})^2,\\&H_{5}H_{9},\\&H_{3}H_{11},\\&H_{4}H_{10}\end{aligned}$ & $\begin{aligned}&H_{4}(H_{6})^2,\\&(H_{4})^2H_{8},\\&(H_{5})^2H_{6},\\&(H_{3})^2H_{10}\end{aligned}$ & $\begin{aligned}&(H_{3})^2(H_{6})^2,\\&(H_{4})^2(H_{5})^2\end{aligned}$ & $\begin{aligned}&(H_{3})^4H_{8},\\&(H_{4})^5\end{aligned}$ & $\begin{aligned}&(H_{3})^4(H_{5})^2,\\&(H_{3})^2(H_{4})^4\end{aligned}$ \\
  \bottomrule
  \end{tabular}
  %
  \caption{Correspondence of a combination of multipoles from a BCH order to multipole-like fields.
  The exponents indicate the order of the BCH for individual components.}
  \label{table:appendix:transfer_maps:bch_resulting_orders_combination}
\end{table}


%Below are detailed the multipole-like orders that are generated depending on the BCH order.

%\paragraph{First Order}
%
%\begin{equation}
%  \text{multipolar-like orders} =
%  \begin{cases}
%    m\\
%    n
%  \end{cases}.
%\end{equation}
%
%
%\paragraph{Second Order}
%
%\begin{equation}
%  \text{multipolar-like orders} =
%  \begin{cases}
%    m + n - 2
%  \end{cases}.
%\end{equation}
%
%%The resulting effective Hamiltonian is of higher order for $n > 2$ and $m > 2$.
%
%
%\paragraph{Third Order}
%
%\begin{equation}
%  \text{multipolar-like orders} =
%  \begin{cases}
%    m+ 2n-4 \\
%    2m + n-4
%  \end{cases}.
%\end{equation}

%The resulting effective Hamiltonian is again here of higher order for $m + n > 4$.


% --------- Two sextupoles 
\subsection{Transfer Map of Two Sextupoles}
\label{appendix:transfer_map:two_sextupoles}

The transfer map of two sextupoles $H_1$ and $H_2$ of strength $K_1$ and $K_2$, separated by a drift
to introduce a change of coordinates in $H_1$ is the following,

\begin{equation}
    \mathcal{M} = e^{:Z:} = e^{:H_{2}:} \cdot e^{D:H_{1}:},
\end{equation}

After such application of the drift on $H_1$, the two hamiltonians read,

\begin{equation}
  \begin{aligned}
    H_1 &= \frac{1}{3!} K_{1} L_{1} \left(\left(L_{D} p_{x} + x\right)^{3} - 3 \left(L_{D} p_{x} + x\right) \left(L_{D} p_{y} + y\right)^{2}\right) \\
    H_2 &= \frac{1}{3!} K_{2} L_{2} \left(x^{3} - 3 x y^{2}\right).
  \end{aligned}
\end{equation}

Below are detailed each term of the BCH. Each term should be added together in order to obtain the
whole effective Hamiltonian $Z$.

\paragraph{First Order}

\footnotesize
\begin{equation}
    \begin{aligned}
      & \left.
      \begin{aligned}
        &K_1 L_{1} L_{D} \left(
        %
        \begin{aligned}
          &\frac{L_{D}^{2} p_{x}^{3}}{6} - \frac{L_{D}^{2} p_{x} p_{y}^{2}}{2} + \frac{L_{D} p_{x}^{2} x}{2} - L_{D} p_{x} p_{y} y \\
          &- \frac{L_{D} p_{y}^{2} x}{2} + \frac{p_{x} x^{2}}{2} - \frac{p_{x} y^{2}}{2} - p_{y} x y
        \end{aligned}\right)\\
        %
        &+ K_1 L_{1} \left(\frac{x^{3}}{6} - \frac{x y^{2}}{2}\right) + K_{2} L_{2} \left(\frac{x^{3}}{6} - \frac{x y^{2}}{2}\right)
      \end{aligned}
      \;\,\right\} \text{sextupolar}
    \end{aligned}
\end{equation}
\normalsize

\paragraph{Second Order}

\footnotesize
\begin{equation}
    \begin{aligned}
      &\left.
      K_1 K_2 L_{1} L_{2} L_{D} \left(
      %
      \begin{aligned}
        &\frac{L_{D}^{2} p_{x}^{2} x^{2}}{8} - \frac{L_{D}^{2} p_{x}^{2} y^{2}}{8} + \frac{L_{D}^{2} p_{x} p_{y} x y}{2} - \frac{L_{D}^{2} p_{y}^{2} x^{2}}{8} \\
        &+ \frac{L_{D}^{2} p_{y}^{2} y^{2}}{8} + \frac{L_{D} p_{x} x^{3}}{4} + \frac{L_{D} p_{x} x y^{2}}{4} + \frac{L_{D} p_{y} x^{2} y}{4} \\
        &+ \frac{L_{D} p_{y} y^{3}}{4} + \frac{x^{4}}{8} + \frac{x^{2} y^{2}}{4} + \frac{y^{4}}{8}
      \end{aligned} \right) \;\,\right\} \; \text{octupolar-like}\\
      %
    \end{aligned}
\end{equation}
\normalsize

\paragraph{Third Order}

\footnotesize
\begin{equation}
  \begin{aligned}
      &\left.
      \begin{aligned}
      &K_{1}^{2} K_{2} L_{1}^{2} L_{2} L_{D} \left(
     \begin{aligned}
       &\frac{L_{D}^{5} p_{x}^{4} x}{48} + \frac{L_{D}^{5} p_{x}^{3} p_{y} y}{12} - \frac{L_{D}^{5} p_{x}^{2} p_{y}^{2} x}{8} - \frac{L_{D}^{5} p_{x} p_{y}^{3} y}{12} \\
       &+ \frac{L_{D}^{5} p_{y}^{4} x}{48} + \frac{L_{D}^{4} p_{x}^{3} x^{2}}{12} + \frac{L_{D}^{4} p_{x}^{3} y^{2}}{12}- \frac{L_{D}^{4} p_{x} p_{y}^{2} x^{2}}{4} \\
       &- \frac{L_{D}^{4} p_{x} p_{y}^{2} y^{2}}{4} + \frac{L_{D}^{3} p_{x}^{2} x^{3}}{8} + \frac{L_{D}^{3} p_{x}^{2} x y^{2}}{8} - \frac{L_{D}^{3} p_{x} p_{y} x^{2} y}{4} \\
       &- \frac{L_{D}^{3} p_{x} p_{y} y^{3}}{4} - \frac{L_{D}^{3} p_{y}^{2} x^{3}}{8} - \frac{L_{D}^{3} p_{y}^{2} x y^{2}}{8} + \frac{L_{D}^{2} p_{x} x^{4}}{12} \\
       &- \frac{L_{D}^{2} p_{x} y^{4}}{12} - \frac{L_{D}^{2} p_{y} x^{3} y}{6} - \frac{L_{D}^{2} p_{y} x y^{3}}{6} + \frac{L_{D} x^{5}}{48} \\
       &- \frac{L_{D} x^{3} y^{2}}{24} - \frac{L_{D} x y^{4}}{16}
     \end{aligned} \right) \\
       %
       %
       % Decapole again
      &+ K_{1} K_{2}^{2} L_{1} L_{2}^{2} L_{D} \left(
      \begin{aligned}
        &\frac{L_{D}^{2} p_{x} x^{4}}{48} - \frac{L_{D}^{2} p_{x} x^{2} y^{2}}{8} + \frac{L_{D}^{2} p_{x} y^{4}}{48} + \frac{L_{D}^{2} p_{y} x^{3} y}{12} \\
        &- \frac{L_{D}^{2} p_{y} x y^{3}}{12} + \frac{L_{D} x^{5}}{48} - \frac{L_{D} x^{3} y^{2}}{24} - \frac{L_{D} x y^{4}}{16}
      \end{aligned}\right)
      \end{aligned}
      \; \right\} \; \text{decapolar-like} \\
  \end{aligned}
\end{equation}
\normalsize


% --------- Sextupole and Octupole
\subsection{Transfer Map of a Sextupole and Octupole}
\label{appendix:transfer_map:sextupole_and_octupole}

The transfer map of a sextupole $H_1$ and octupole $H_2$ of strength $K_1$ and $K_2$, separated by
a drift like in the previous example is given by

\begin{equation}
  \mathcal{M} = e^{:Z:} = e^{:H_2:} \cdot e^{D:H_1:}
\end{equation}

with $H_1$ and $H_2$ having as final expressions,

\begin{equation}
  \begin{aligned}
    H_1 &= \frac{1}{3!} K_{3,h1} L_{1} \left(\left(L_{D} p_{x} + x\right)^{3} - 3 \left(L_{D} p_{x} + x\right) \left(L_{D} p_{y} + y\right)^{2}\right) \\
    H_2 &= \frac{1}{4!} K_{2} L_{2} \left(x^{4} - 6 x^{2} y^{2} + y^{4}\right).
  \end{aligned}
\end{equation}

The first two orders of the BCH of those two elements is given below.

\paragraph{First Order}

\footnotesize
\begin{equation}
  \begin{aligned}
    &K_{3} \left. \left(
    \begin{aligned}
      &\frac{L_{D}^{3} p_{x}^{3}}{6} - \frac{L_{D}^{3} p_{x} p_{y}^{2}}{2} + \frac{L_{D}^{2} p_{x}^{2} x}{2} - L_{D}^{2} p_{x} p_{y} y - \frac{L_{D}^{2} p_{y}^{2} x}{2} \\
      &+ \frac{L_{D} p_{x} x^{2}}{2} - \frac{L_{D} p_{x} y^{2}}{2} - L_{D} p_{y} x y + \frac{x^{3}}{6} - \frac{x y^{2}}{2}
    \end{aligned}
    \right) \;\right\} \; \text{sextupolar}\\
    \\[-1.5em]
    &+ K_{4} \left. \left(
      \frac{x^{4}}{24} - \frac{x^{2} y^{2}}{4} + \frac{y^{4}}{24}
    \right) \;\right\} \; \text{octupolar}
  \end{aligned}
\end{equation}
\normalsize


\paragraph{Second Order}

\footnotesize
\begin{equation}
  \begin{aligned}
    &K_{3} K_{4} L_{D} \left.\left(
    \begin{aligned}
      &\frac{L_{D}^{2} p_{x}^{2} x^{3}}{24} - \frac{L_{D}^{2} p_{x}^{2} x y^{2}}{8} + \frac{L_{D}^{2} p_{x} p_{y} x^{2} y}{4} - \frac{L_{D}^{2} p_{x} p_{y} y^{3}}{12} \\
      &- \frac{L_{D}^{2} p_{y}^{2} x^{3}}{24} + \frac{L_{D}^{2} p_{y}^{2} x y^{2}}{8} + \frac{L_{D} p_{x} x^{4}}{12} - \frac{L_{D} p_{x} y^{4}}{12} \\
      &+ \frac{L_{D} p_{y} x^{3} y}{6} + \frac{L_{D} p_{y} x y^{3}}{6} + \frac{x^{5}}{24} + \frac{x^{3} y^{2}}{12} + \frac{x y^{4}}{24}
    \end{aligned} 
    \right) \;\right\} \; \text{decapolar-like}\\
  \end{aligned}
\end{equation}
\normalsize



% --------- Skew Quad and Octupole
\subsection{Transfer Map of a Skew Quadrupole and Octupole}
\label{appendix:transfer_map:skew_quadrupole_and_octupole}

The transfer map of a skew quadrupole $H_1$ and octupole $H_2$ of strength $K_1$ and $K_2$,
separated by a drift like in the previous examples is given by

\begin{equation}
  \mathcal{M} = e^{:Z:} = e^{:H_2:} \cdot e^{D:H_1:}
\end{equation}

with $H_1$ and $H_2$ having as final expressions,

\begin{equation}
  \begin{aligned}
    H_1 &= - J_{1} L_{1} \left(L_{D} p_{x} + x\right) \left(L_{D} p_{y} + y\right) \\
    H_2 &= \frac{1}{4!} K_{2} L_{2} \left(x^{4} - 6 x^{2} y^{2} + y^{4}\right).
  \end{aligned}
\end{equation}

The first two orders of the BCH of those two elements is given below.

\paragraph{First Order}

\begin{equation}
  \begin{aligned}
    &\left. J_{1} L_{1} \left(- L_{D}^{2} p_{x} p_{y} - L_{D} p_{x} y - L_{D} p_{y} x - x y\right) \; \right\} \; \text{skew quadrupolar}\\
    &\left. + K_{2} L_{2} \left(\frac{x^{4}}{24} - \frac{x^{2} y^{2}}{4} + \frac{y^{4}}{24}\right) \; \right\} \; \text{octupolar}
  \end{aligned}
\end{equation}


\paragraph{Second Order}

\begin{equation}
  \left. J_{1} K_{2} L_{1} L_{2} L_{D} \left(
  \begin{aligned}
      &\frac{L_{D} p_{x} x^{2} y}{4} - \frac{L_{D} p_{x} y^{3}}{12} - \frac{L_{D} p_{y} x^{3}}{12} \\
      &+ \frac{L_{D} p_{y} x y^{2}}{4} + \frac{x^{3} y}{6} + \frac{x y^{3}}{6}
  \end{aligned}
  \right) \; \right\} \; \text{skew octupolar-like}
\end{equation}




% ==============================================================
%                    Chromatic Amplitude Detuning
% ==============================================================
\chapter{Chromatic Amplitude Detuning}
\label{appendix:chromatic_amplitude_detuning}

This appendix details the derivations of chromatic amplitude detuning from sextupoles up to
dodecapoles. As chromaticity and amplitude detuning are part of it, they will therefore be detailed
here as well.

\newpage
Up to the third order, the expression of the Taylor expansion of the Chromatic Amplitude Detuning
around $\epsilon_x$, $\epsilon_y$ and $\delta$, for a tune $Q_z$, $z \in \{x, y\}$ reads:

\begin{equation}
\begin{aligned}
Q_z(\epsilon_x, \epsilon_y, \delta) = Q_{z0} &+ \left[\frac{\partial Q_z}{\partial \epsilon_x} \epsilon_x
                                                 + \frac{\partial Q_z}{\partial \epsilon_y} \epsilon_y
                                                 + \colorbox{yellow!0}{$\displaystyle \frac{\partial Q_z}{\partial \delta}$} \delta
                                                \right] \\
                                             &+ \frac{1}{2!} \biggl[\frac{\partial^2Q_z}{\partial \epsilon_x^2}\epsilon_x^2 
                                                 + \frac{\partial^2Q_z}{\partial \epsilon_y^2}\epsilon_y^2
                                                 + \colorbox{yellow!0}{$\displaystyle \frac{\partial^2 Q_z}{\partial \delta^2}$} \delta^2  \\
                                             &\;\begin{aligned}
                                             \phantom{+ \frac{1}{2!} \biggl[}
                                               &+ 2 \frac{\partial^2Q_z}{\partial \epsilon_x \partial \epsilon_y}\epsilon_x \epsilon_y
                                                  + 2 \frac{\partial^2Q_z}{\partial \epsilon_x \partial \delta}\epsilon_x \delta
                                                  + 2 \frac{\partial^2Q_z}{\partial \delta \partial \epsilon_y} \delta \epsilon_y
                                             \biggr] \\
                                             \end{aligned} \\
                                             &+ \frac{1}{3!}
                                             \biggl[
                                                  \colorbox{yellow!0}{$\displaystyle \frac{\partial^3 Q_z}{\partial \delta^3}$}\delta^{3}
                                                  + \frac{\partial^{3} Q_z}{\partial \epsilon_{x}^{3}}  \epsilon_{x}^{3} 
                                                  + \frac{\partial^{3} Q_z}{\partial \epsilon_{y}^{3}}  \epsilon_{y}^{3} \\
                                             &\;\begin{aligned}
                                             \phantom{+ \frac{1}{3!} \biggl[} 
                                               &+ 3 \frac{\partial^{3} Q_z}{\partial \epsilon_{x}\partial \delta^{2}} \delta^{2} \epsilon_{x} 
                                                + 3  \frac{\partial^{3} Q_z}{\partial \epsilon_{y}\partial \delta^{2}}  \delta^{2} \epsilon_{y}
                                                + 3 \frac{\partial^{3} Q_z}{\partial \epsilon_{x}^{2}\partial \delta}  \delta \epsilon_{x}^{2} \\
                                               &+ 3 \frac{\partial^{3} Q_z}{\partial \epsilon_{y}^{2}\partial \delta} \delta \epsilon_{y}^{2}  
                                                + 3  \frac{\partial^{3} Q_z}{\partial \epsilon_{y}\partial \epsilon_{x}^{2}} \epsilon_{x}^{2} \epsilon_{y} 
                                                + 3 \frac{\partial^{3} Q_z}{\partial \epsilon_{y}^{2}\partial \epsilon_{x}} \epsilon_{x} \epsilon_{y}^{2} \\
                                               &+ 6 \frac{\partial^{3} Q_z}{\partial \epsilon_{y}\partial  \epsilon_{x}\partial \delta} \delta \epsilon_{x} \epsilon_{y} 
                                             \biggr] + \cdots \\
                                             \end{aligned}
\end{aligned}
\end{equation}


\section{Principle}

From~\cite{dilly_derivation_2023}, the detuning caused by a magnet of length L can be described by
its hamiltonian with 

\begin{equation}
  \Delta Q_z = \frac{1}{2 \pi} \int_L \frac{\partial \langle H \rangle}{\partial J_z} \diff s.
\end{equation}

The usual variables $x$ and $y$ of \cref{eq:normal_skew_hamiltonian_magnet} can be replaced by
\textit{action-angle} variables to introduce the action:

\begin{equation}
  \begin{aligned}
    x \rightarrow \sqrt{2J_x \beta_x} \cos{\phi_x} \\
    y \rightarrow \sqrt{2J_y \beta_y} \cos{\phi_x}
  \end{aligned}
  \label{appendix:chromatic_ampdet:action_angle}
\end{equation}

A momentum dependence can be introduced for a particle with a different orbit
($\Delta z$)~\cite{wiedemann_particle_1999} via dispersion. Combined with
\cref{appendix:chromatic_ampdet:action_angle}, a dependence on all required components is achieved:

\begin{equation}
  \begin{aligned}
    x + \Delta x \rightarrow \sqrt{2J_x \beta_x} \cos{\phi_x} + D_x \delta \\
    y + \Delta y \rightarrow \sqrt{2J_y \beta_y} \cos{\phi_y} + D_y \delta
  \end{aligned}
\end{equation}


After averaging over the phase variable, all that is left is to compute the partial derivatives.

\todo{The following derivations are just copy-paster from old text}

% ======================
\section{Sextupole}

The change of tune induced by a sextupole is:

\begin{equation}Q_x = \frac{1}{4\pi} K_3 \beta_x \eta \delta L\end{equation}
\begin{equation}Q_y = - \frac{1}{4\pi} K_3 \beta_y \eta \delta L\end{equation}

We can now get the terms we're interested in:

\begin{equation}\begin{aligned}
\frac{\partial Q_x}{\partial J_x} = 0 \quad;&& \frac{\partial Q_x}{\partial J_y} = 0 \quad;&& \frac{\partial Q_x}{\partial \delta} = \frac{1}{4\pi}K_3\beta_x\eta L = Q_x'\\
\frac{\partial Q_y}{\partial J_x} = 0 \quad;&& \frac{\partial Q_y}{\partial J_y} = 0 \quad;&& \frac{\partial Q_y}{\partial \delta} = -\frac{1}{4\pi}K_3\beta_y\eta L = Q_y'\\
\end{aligned}\end{equation}

Contribution to the Chromatic Amplitude Detuning:

\begin{equation}
\begin{aligned}
Q_z(\epsilon_x, \epsilon_y, \delta) = \color{gray}Q_{z0} &+ \color{gray}\left[\frac{\partial Q_z}{\partial \epsilon_x} \epsilon_x
                                                 + \frac{\partial Q_z}{\partial \epsilon_y} \epsilon_y
                                                 + \textcolor{orange}{\frac{\partial Q_z}{\partial \delta} \delta}
                                                \right] \\
\end{aligned}
\end{equation}

\newpage

\hypertarget{octupole-2}{%
\subsection{Octupole}\label{octupole-2}}

From the hamiltonian of a normal octupole, with a displacement in \(x\)
(\cref{eq:hamiltonian_displacement_octupole}), we can change the
variable (\(x = \sqrt{2 J_x \beta_x} \cos \phi_x\) and
\(y = \sqrt{2 J_y \beta_y} \cos \phi_y\)):

\begin{equation}
\begin{aligned}
\mathcal{N}_4 = \frac{1}{24} K_4 \biggl[& \left(\sqrt{2J_x\beta_x} cos\phi_x\right)^4 \\
                                        & +4 \left(\sqrt{2 J_x \beta_x} \cos \phi_x\right)^3 \eta \delta \\
                                        & +6 \left(\sqrt{2 J_x \beta_x} \cos \phi_x\right)^2 \eta^2 \delta^2 \\
                                        & +4 \left(\sqrt{2 J_x \beta_x} \cos \phi_x\right) \eta^2 \delta \\
                                        & +\eta^4 \delta^4 \\
                                        & -6 \left(\sqrt{2 J_x \beta_x} \cos \phi_x\right)^2 \left(\sqrt{2 J_y \beta_y}\cos \phi_y\right)^2\\
                                        & -6 \left(\sqrt{2 J_y \beta_y} \cos \phi_y\right)^2 \cdot 2 \left(\sqrt{2 J_x \beta_x}\cos \phi_x\right) \eta \delta \\
                                        & -6 \left(\sqrt{2 J_y \beta_y} \cos \phi_y\right)^2 \eta^2 \delta^2 \\
                                        & + \left(\sqrt{2 J_y \beta_y} \cos \phi_y\right)^4
                                  \biggr]\\
\end{aligned}
\end{equation}

We can now average over the phase variables: \begin{equation}
\begin{aligned}
\left< \mathcal{N}_4 \right> = \frac{1}{24} K_4 \biggl[& \frac{3}{2} J_x^2\beta_x^2 \\
                                                       & +6 J_x \beta_x \eta^2 \delta^2 \\
                                                       & +\eta^4 \delta^4 \\
                                                       & -6 J_x \beta_x J_y \beta_y\\
                                                       & -6 J_y \beta_y \eta^2 \delta^2 \\
                                                       & + \frac{3}{2} J_y^2 \beta_y^2
                                                 \biggr]\\
\end{aligned}
\end{equation}

The tunes then are:

\begin{equation}\begin{aligned}
Q_x = \frac{1}{2\pi} \frac{\partial \left< \mathcal{N_4} \right>}{\partial J_x} &= \frac{1}{48\pi} K_4 \biggl[3 J_x \beta_x^2
                                                                                                             +6 \beta_x \eta^2 \delta^2
                                                                                                             -6 \beta_x J_y  \beta_y 
                                                                                                      \biggr]\\
Q_y = \frac{1}{2\pi} \frac{\partial \left< \mathcal{N_4} \right>}{\partial J_y} &= \frac{1}{48\pi} K_4 \biggl[-6 J_x \beta_x \beta_y
                                                                                                             -6 \beta_y \eta^2 \delta^2
                                                                                                             +3 J_y \beta_y^2
                                                                                                      \biggr]
\end{aligned}\end{equation}

\begin{equation}\begin{aligned}
  \frac{\partial Q_x}{\partial J_x} =& \frac{1}{16\pi} K_4 \beta_x^2 &&;\quad 
  \frac{\partial Q_x}{\partial J_y} =& -\frac{1}{8\pi} K_4 \beta_x\beta_y &&;\quad
  \frac{\partial^2 Q_x}{\partial \delta^2} =& \frac{1}{4\pi} K_4 \beta_x \eta^2  = Q_x''
\\
  \frac{\partial Q_y}{\partial J_x} =& -\frac{1}{8\pi} K_4 \beta_x \beta_y &&;\quad
  \frac{\partial Q_y}{\partial J_y} =& \frac{1}{16\pi} K_4 \beta_y^2 &&;\quad
  \frac{\partial^2 Q_y}{\partial \delta^2} =& -\frac{1}{4\pi} K_4 \beta_y \eta^2  = Q_y''
\\
\end{aligned}\end{equation}

Contribution to the Chromatic Amplitude Detuning:

\begin{equation}
\begin{aligned}
Q_z(\epsilon_x, \epsilon_y, \delta) = \color{gray}Q_{z0} &\color{gray}+
                                                \textcolor{orange}{\biggl[}
                                                   \textcolor{orange}{\frac{\partial Q_z}{\partial \epsilon_x} \epsilon_x}
                                                 + \textcolor{orange}{\frac{\partial Q_z}{\partial \epsilon_y} \epsilon_y}
                                                 + \frac{\partial Q_z}{\partial \delta} \delta
                                                \textcolor{orange}{\biggr]} \\
                                             &\color{gray}
                                             + \textcolor{orange}{\frac{1}{2!} \biggl[}
                                                   \frac{\partial^2Q_z}{\partial \epsilon_x^2}\epsilon_x^2 
                                                 + \frac{\partial^2Q_z}{\partial \epsilon_y^2}\epsilon_y^2
                                                 + \textcolor{orange}{\frac{\partial^2 Q_z}{\partial \delta^2} \delta^2}  \\
                                             &\;\begin{aligned}
                                             \phantom{+ \frac{1}{2!} \biggl[}
                                               & \color{gray}
                                               + 2 \frac{\partial^2Q_z}{\partial \epsilon_x \partial \epsilon_y}\epsilon_x \epsilon_y
                                                  + 2 \frac{\partial^2Q_z}{\partial \epsilon_x \partial \delta}\epsilon_x \delta
                                                  + 2 \frac{\partial^2Q_z}{\partial \delta \partial \epsilon_y} \delta \epsilon_y
                                             \textcolor{orange}{\biggr]} \\
                                             \end{aligned} \\
                                             &\color{gray}+ \frac{1}{3!}
                                             \biggl[
                                                  \frac{\partial^3 Q_z}{\partial \delta^3} \delta^{3}
                                                  + \frac{\partial^{3} Q_z}{\partial \epsilon_{x}^{3}}  \epsilon_{x}^{3} 
                                                  + \frac{\partial^{3} Q_z}{\partial \epsilon_{y}^{3}}  \epsilon_{y}^{3} \\
                                             &\;\begin{aligned}
                                             \phantom{+ \frac{1}{3!} \biggl[} 
                                               &\color{gray}
                                               + 3 \frac{\partial^{3} Q_z}{\partial \epsilon_{x}\partial \delta^{2}} \delta^{2} \epsilon_{x} 
                                                + 3  \frac{\partial^{3} Q_z}{\partial \epsilon_{y}\partial \delta^{2}}  \delta^{2} \epsilon_{y}
                                                + 3 \frac{\partial^{3} Q_z}{\partial \epsilon_{x}^{2}\partial \delta}  \delta \epsilon_{x}^{2} \\
                                               &\color{gray}
                                               + 3 \frac{\partial^{3} Q_z}{\partial \epsilon_{y}^{2}\partial \delta} \delta \epsilon_{y}^{2}  
                                                + 3  \frac{\partial^{3} Q_z}{\partial \epsilon_{y}\partial \epsilon_{x}^{2}} \epsilon_{x}^{2} \epsilon_{y} 
                                                + 3 \frac{\partial^{3} Q_z}{\partial \epsilon_{y}^{2}\partial \epsilon_{x}} \epsilon_{x} \epsilon_{y}^{2} \\
                                               &\color{gray}
                                               + 6 \frac{\partial^{3} Q_z}{\partial \epsilon_{y}\partial  \epsilon_{x}\partial \delta} \delta \epsilon_{x} \epsilon_{y} 
                                             \biggr]\\
                                             \end{aligned}
\end{aligned}
\end{equation}

\newpage

\hypertarget{decapole-2}{%
\subsection{Decapole}\label{decapole-2}}

The normal field of a decapole has been calculated in
\cref{eq:decapole_expanded}:

\[\begin{aligned}
\mathcal{N_5}(x, y) = \frac{1}{120} K_{5} \biggl[&
  \eta^5\delta^5 + 5\eta^4\delta^4x + 10\eta^3\delta^3x^2 + 10\eta^2\delta^2 x^3 + 5\eta\delta x^4 + x^5 \\
  & -10y^2 (\eta^3\delta^3 + 3\eta^2\delta^2x + 3\eta\delta x^2 + x^3)\\
  & +5y^4 (x + \eta\delta) \biggr]
\end{aligned}\]

Changing variables (\(x \rightarrow \sqrt{2 J_x \beta_x} \cos\phi_x\);
\(y \rightarrow \sqrt{2 J_y \beta_y} \cos\phi_y\)):

\begin{equation}\begin{aligned}
\mathcal{N_5}(x, y) = \frac{1}{120} K_{5} 
  \biggl[&
        \eta^5\delta^5 + 5\eta^4\delta^4\left(\sqrt{2 J_x \beta_x} \cos \phi_x\right) \\
        &+ 10\eta^3\delta^3\left(\sqrt{2 J_x \beta_x} \cos \phi_x\right)^2 + 10\eta^2\delta^2 \left(\sqrt{2 J_x \beta_x} \cos \phi_x\right)^3 \\
        &+ 5\eta\delta \left(\sqrt{2 J_x \beta_x} \cos \phi_x\right)^4 + \left(\sqrt{2 J_x \beta_x} \cos \phi_x\right)^5 \\
        &- 10\left(\sqrt{2 J_y \beta_y} \cos \phi_y\right)^2 \biggl[(\eta^3\delta^3 + 3\eta^2\delta^2\left(\sqrt{2 J_x \beta_x} \cos \phi_x\right) \\
        &\phantom{- 10\left(\sqrt{2 J_y \beta_y} \cos \phi_y\right)^2 \biggl[}+ 3\eta\delta \left(\sqrt{2 J_x \beta_x} \cos \phi_x\right)^2 + \left(\sqrt{2 J_x \beta_x} \cos \phi_x\right)^3\biggr]\\
        & +5\left(\sqrt{2 J_y \beta_y} \cos \phi_y\right)^4 \left(\left(\sqrt{2 J_x \beta_x} \cos \phi_x\right) + \eta\delta\right) 
  \biggr]
\end{aligned}\end{equation}

Averaging over the phase variables: \begin{equation}\begin{aligned}
\mathcal{N_5}(x, y) = \frac{1}{120} K_{5} 
  \biggl[
         & \eta^5\delta^5 
          + 10 \eta^3 \delta^3 J_x \beta_x \\
         & + \frac{15}{2} \eta \delta J_x^2 \beta_x^2
          - 10 J_y \beta_y \eta^3 \delta^3 \\
         & - 30 J_y \beta_y \eta \delta J_x \beta_x 
          + \frac{15}{2} J_y^2 \beta_y^2 \eta \delta
  \biggr]
\end{aligned}\end{equation}

The tunes then are:

\begin{equation}\begin{aligned}
Q_x = \frac{1}{2\pi} \frac{\partial \left< \mathcal{N_5} \right>}{\partial J_x} &= \frac{1}{240\pi} K_5 \biggl[10 \eta^3 \delta^3 \beta_x
                                                                                                               + 15 \eta \delta J_x \beta_x^2
                                                                                                               - 30 J_y \beta_y \beta_x \eta \delta
                                                                                                      \biggr]\\
Q_y = \frac{1}{2\pi} \frac{\partial \left< \mathcal{N_5} \right>}{\partial J_y} &= \frac{1}{240\pi} K_5 \biggl[-10 \eta^3 \delta^3 \beta_y
                                                                                                               + 15 \eta \delta J_y \beta_y^2
                                                                                                               -30 J_x \beta_y \beta_x \eta \delta
                                                                                                      \biggr]
\end{aligned}\end{equation}

We can now calculate our chromatic amplitude detuning terms:

\begin{equation}\begin{aligned}
  \frac{\partial^2 Q_x}{\partial J_x \partial \delta} =& \frac{1}{16 \pi} K_5 \beta_x^2 \eta &&;\quad 
  \frac{\partial^2 Q_x}{\partial J_y \partial \delta} =& -\frac{1}{8\pi} K_5 \beta_x \beta_y \eta&&;\quad
  \frac{\partial^3 Q_x}{\partial \delta^3} =& \frac{1}{4\pi} K_5 \beta_x \eta^3 = Q_x''' 
\\
  \frac{\partial^2 Q_y}{\partial J_x \partial \delta} =& -\frac{1}{8\pi} K_5 \beta_x \beta_y \eta&&;\quad
  \frac{\partial^2 Q_y}{\partial J_y \partial \delta} =& \frac{1}{16 \pi} K_5 \beta_y^2 \eta &&;\quad 
  \frac{\partial^3 Q_y}{\partial \delta^3} =& -\frac{1}{4\pi} K_5 \beta_y \eta^3 = Q_y'''
\\
\end{aligned}\end{equation}

Contribution to the Chromatic Amplitude Detuning:

\begin{equation}
\begin{aligned}
Q_z(\epsilon_x, \epsilon_y, \delta) = \color{gray}Q_{z0} &\color{gray}+
                                                \left[
                                                   \frac{\partial Q_z}{\partial \epsilon_x} \epsilon_x
                                                 + \frac{\partial Q_z}{\partial \epsilon_y} \epsilon_y
                                                 + \frac{\partial Q_z}{\partial \delta} \delta
                                                \right] \\
                                             &\color{gray}
                                             + \textcolor{orange}{\frac{1}{2!} \biggl[}
                                                   \frac{\partial^2Q_z}{\partial \epsilon_x^2}\epsilon_x^2 
                                                 + \frac{\partial^2Q_z}{\partial \epsilon_y^2}\epsilon_y^2
                                                 + \frac{\partial^2 Q_z}{\partial \delta^2} \delta^2  \\
                                             &\;\begin{aligned}
                                             \phantom{+ \frac{1}{2!} \biggl[}
                                               & \color{gray}
                                               + 2 \frac{\partial^2Q_z}{\partial \epsilon_x \partial \epsilon_y}\epsilon_x \epsilon_y
                                               + \textcolor{orange}{2 \frac{\partial^2Q_z}{\partial \epsilon_x \partial \delta}\epsilon_x \delta}
                                               + \textcolor{orange}{2 \frac{\partial^2Q_z}{\partial \delta \partial \epsilon_y} \delta \epsilon_y}
                                             \textcolor{orange}{\biggr]} \\
                                             \end{aligned} \\
                                             &\color{gray}+ \textcolor{orange}{\frac{1}{3!}
                                             \biggl[}
                                                  \textcolor{orange}{\frac{\partial^3 Q_z}{\partial \delta^3} \delta^{3}}
                                                  + \frac{\partial^{3} Q_z}{\partial \epsilon_{x}^{3}}  \epsilon_{x}^{3} 
                                                  + \frac{\partial^{3} Q_z}{\partial \epsilon_{y}^{3}}  \epsilon_{y}^{3} \\
                                             &\;\begin{aligned}
                                             \phantom{+ \frac{1}{3!} \biggl[} 
                                               &\color{gray}
                                               + 3 \frac{\partial^{3} Q_z}{\partial \epsilon_{x}\partial \delta^{2}} \delta^{2} \epsilon_{x} 
                                                + 3  \frac{\partial^{3} Q_z}{\partial \epsilon_{y}\partial \delta^{2}}  \delta^{2} \epsilon_{y}
                                                + 3 \frac{\partial^{3} Q_z}{\partial \epsilon_{x}^{2}\partial \delta}  \delta \epsilon_{x}^{2} \\
                                               &\color{gray}
                                               + 3 \frac{\partial^{3} Q_z}{\partial \epsilon_{y}^{2}\partial \delta} \delta \epsilon_{y}^{2}  
                                                + 3  \frac{\partial^{3} Q_z}{\partial \epsilon_{y}\partial \epsilon_{x}^{2}} \epsilon_{x}^{2} \epsilon_{y} 
                                                + 3 \frac{\partial^{3} Q_z}{\partial \epsilon_{y}^{2}\partial \epsilon_{x}} \epsilon_{x} \epsilon_{y}^{2} \\
                                               &\color{gray}
                                               + 6 \frac{\partial^{3} Q_z}{\partial \epsilon_{y}\partial  \epsilon_{x}\partial \delta} \delta \epsilon_{x} \epsilon_{y} 
                                             \textcolor{orange}{\biggr]}\\
                                             \end{aligned}
\end{aligned}
\end{equation}

\newpage

\hypertarget{dodecapole-2}{%
\subsection{Dodecapole}\label{dodecapole-2}}

The main normal field of a dodecapole is:

\begin{equation}\mathcal{N_6}(x,y) = \frac{1}{720} K_6 (x^6 - 15x^4y^2 + 15x^2y^4 -y^6)\end{equation}

With a displacement in \(x \rightarrow x + \eta \delta\):
\begin{equation}\mathcal{N_6}(x,y) = \frac{1}{720} K_6 \biggl[(x + \eta \delta)^6 - 15(x + \eta \delta)^4y^2 + 15(x + \eta \delta)^2y^4 -y^6\biggr]\end{equation}

Expanded form, after having removed odd exponents. Those exponents
would yield an average of \(0\) for the cosines:

\begin{equation}\begin{aligned}
  \mathcal{N_6}(x,y) = \frac{1}{720} K_6 
                                \biggl[
                                  &x^6 + 15x^2 \eta^4 \delta^4 + 15 x^4 \eta^2 \delta^2 + \eta^6 \delta^6 \\
                                  &-15y^2 (\eta^4 \delta^4 + 4x \eta^3 \delta^3 + 6x^2 \eta^2 \delta^2 + 4x^3 \eta \delta+ x^4) \\
                                  &+15y^4 (x^2 + \eta^2 \delta^2) \\
                                  &-y^6
                                \biggr]
\end{aligned}\end{equation}

Changing variables (\(x \rightarrow \sqrt{2 J_x \beta_x} \cos\phi_x\);
\(y \rightarrow \sqrt{2 J_y \beta_y} \cos\phi_y\)):
\begin{equation}\begin{aligned}
  \mathcal{N_6}(x,y) = \frac{1}{720} K_6 
                                \biggl[
                                  &\left(\sqrt{2 J_x \beta_x} \cos \phi_x\right)^6 \\
                                  &+ 15\left(\sqrt{2 J_x \beta_x} \cos \phi_x\right)^2 \eta^4 \delta^4 \\
                                  &+ 15 \left(\sqrt{2 J_x \beta_x} \cos \phi_x\right)^4 \eta^2 \delta^2 \\
                                  &+ \eta^6 \delta^6 \\
                                  &-15\left(\sqrt{2 J_y \beta_y} \cos \phi_y\right)^2 \bigg(\eta^4 \delta^4 \\
                                      &\;\begin{aligned}
                                      \phantom{-15\left(\sqrt{2 J_y \beta_y} \cos \phi_y\right)^2 \bigg(}
                                        &+ 6\left(\sqrt{2 J_x \beta_x} \cos \phi_x\right)^2 \eta^2 \delta^2 \\
                                        &+  \left(\sqrt{2 J_x \beta_x} \cos \phi_x\right)^4
                                      \biggr) \\
                                      \end{aligned} \\
                                  &+15\left(\sqrt{2 J_y \beta_y} \cos \phi_y\right)^4 \biggl(\left(\sqrt{2 J_x \beta_x} \cos \phi_x\right)^2 + \eta^2 \delta^2 \biggr) \\
                                  &-  \left(\sqrt{2 J_y \beta_y} \cos \phi_y\right)^6
                                \biggr]
\end{aligned}\end{equation}

Averaging over the phase variables: \begin{equation}\begin{aligned}
\left< \mathcal{N_6}(x,y) \right> = \frac{1}{720} K_6 
                                \biggl[
                                  &\frac{5}{2} J_x^3 \beta_x^3 \\
                                  &+ 15 \cdot J_x \beta_x \eta^4 \delta^4 \\
                                  &+ 15 \cdot J_x^2 \beta_x^2 \frac{3}{2} \eta^2 \delta^2 \\
                                  &+ \eta^6 \delta^6 \\
                                  &-15\left(J_y \beta_y \right) \bigg(\eta^4 \delta^4 \\
                                      &\;\begin{aligned}
                                      \phantom{-15\left(J_y \beta_y \right) \bigg(}
                                        &+ 6\left(J_x \beta_x \right) \eta^2 \delta^2 \\
                                        &+  \left(J_x^2 \beta_x^2 \frac{3}{2}\right)
                                      \biggr) \\
                                      \end{aligned} \\
                                  &+ 15\left(J_y^2 \beta_y^2 \frac{3}{2} \right) \biggl(J_x \beta_x + \eta^2 \delta^2 \biggr) \\
                                  &- \frac{5}{2} J_y^3 \beta_y^3
                                \biggr]
\end{aligned}\end{equation}

The tunes then are:

\begin{equation}\begin{aligned}
Q_x = \frac{1}{2\pi} \frac{\partial \left< \mathcal{N_6} \right>}{\partial J_x} = \frac{1}{1440\pi} K_6 \biggl[
                                 & \frac{15}{2} J_x^2 \beta_x^3\\
                                 & +15 \beta_x \eta^4 \delta^4 \\
                                 & +45 J_x \beta_x^2 \eta^2 \delta^2 \\
                                 & -90 J_y \beta_y \beta_x \eta^2 \delta^2 \\
                                 & -45 J_y \beta_y J_x \beta_x^2\\
                                 & + 15 \cdot \frac{3}{2} J_y^2 \beta_y^2 \beta_x
                                                                                                      \biggr]\\
Q_y = \frac{1}{2\pi} \frac{\partial \left< \mathcal{N_6} \right>}{\partial J_y} = \frac{1}{1440\pi} K_6 \biggl[
                                 & -15 \beta_y \eta^4 \delta^4 \\
                                 & -90 \beta_y J_x \beta_x \eta^2 \delta^2 \\
                                 & -15 \cdot \frac{3}{2}  \beta_y J_x^2 \beta_x^2 \\
                                 & +45 J_y \beta_y^2 J_x \beta_x \\
                                 & +45 J_y \beta_y^2 \eta^2 \delta^2 \\
                                 & -\frac{15}{2} J_y^2 \beta_y ^3 
                                                                                                      \biggr]
\end{aligned}\end{equation}

We can now calculate our chromatic amplitude detuning terms. Since there
are many terms, I'm going to split them here. First, \(Q_x\):

\begin{equation}\begin{aligned}
  \frac{\partial^2 Q_x}{\partial J_x^2} =&\; \frac{1}{96\pi} K_6 \beta_x^3 \\
  \frac{\partial^3 Q_x}{\partial J_x \partial \delta^2} =&\; \frac{1}{16\pi} K_6 \beta_x^2 \eta^2\\
  \cline{1-2}\\[-5\jot]
  \frac{\partial^2 Q_x}{\partial J_y^2} =&\; \frac{1}{32\pi} K_6 \beta_y^2\beta_x \\
  \frac{\partial^3 Q_x}{\partial J_y \partial \delta^2} =&\; -\frac{1}{8\pi} K_6 \beta_y \beta_x \eta^2\\
  \cline{1-2}\\[-5\jot]
  \frac{\partial^2 Q_x}{\partial J_x \partial J_y} =&\; -\frac{1}{32\pi} K_6 \beta_y \beta_x^2\\[2\jot]
\end{aligned}\end{equation}

Then \(Q_y\): \begin{equation}\begin{aligned}
  \frac{\partial^2 Q_y}{\partial J_y^2} =&\; -\frac{1}{96\pi} K_6 \beta_y^3 \\
  \frac{\partial^3 Q_y}{\partial J_y \partial \delta^2} =&\; \frac{1}{16\pi} K_6 \beta_y^2 \eta^2\\
  \cline{1-2}\\[-5\jot]
  \frac{\partial^2 Q_y}{\partial J_x^2} =&\; -\frac{1}{32\pi} K_6 \beta_y\beta_x^2 \\
  \frac{\partial^3 Q_y}{\partial J_x \partial \delta^2} =&\; -\frac{1}{8\pi} K_6 \beta_y \beta_x \eta^2\\
  \cline{1-2}\\[-5\jot]
  \frac{\partial^2 Q_y}{\partial J_y \partial J_x} =&\; \frac{1}{32\pi} K_6 \beta_y^2 \beta_x\\[2\jot]
\end{aligned}\end{equation}

Then the chromaticity: \begin{equation}\begin{aligned}
  \frac{\partial^4 Q_x}{\partial \delta^4} =&\; \frac{1}{4\pi} K_6 \beta_x \eta^4 = Q_x''''\\
  \frac{\partial^4 Q_y}{\partial \delta^4} =&\; -\frac{1}{4\pi} K_6 \beta_y \eta^4 = Q_y''''\\
\end{aligned}\end{equation}

\vspace{2cm}

Contribution to the Chromatic Amplitude Detuning:

\begin{equation}
\begin{aligned}
Q_z(\epsilon_x, \epsilon_y, \delta) = \color{gray}Q_{z0} &\color{gray}+
                                                \left[
                                                   \frac{\partial Q_z}{\partial \epsilon_x} \epsilon_x
                                                 + \frac{\partial Q_z}{\partial \epsilon_y} \epsilon_y
                                                 + \frac{\partial Q_z}{\partial \delta} \delta
                                                \right] \\
                                             &\color{gray}
                                             + \textcolor{orange}{\frac{1}{2!} \biggl[}
                                                   \textcolor{orange}{\frac{\partial^2Q_z}{\partial \epsilon_x^2}\epsilon_x^2}
                                                 + \textcolor{orange}{\frac{\partial^2Q_z}{\partial \epsilon_y^2}\epsilon_y^2}
                                                 + \frac{\partial^2 Q_z}{\partial \delta^2} \delta^2  \\
                                             &\;\begin{aligned}
                                             \phantom{+ \frac{1}{2!} \biggl[}
                                               & \color{gray}
                                               + \textcolor{orange}{2 \frac{\partial^2Q_z}{\partial \epsilon_x \partial \epsilon_y}\epsilon_x \epsilon_y}
                                               + 2 \frac{\partial^2Q_z}{\partial \epsilon_x \partial \delta}\epsilon_x \delta
                                               + 2 \frac{\partial^2Q_z}{\partial \delta \partial \epsilon_y} \delta \epsilon_y
                                             \textcolor{orange}{\biggr]} \\
                                             \end{aligned} \\
                                             &\color{gray}+ \textcolor{orange}{\frac{1}{3!}
                                             \biggl[}
                                                  \frac{\partial^3 Q_z}{\partial \delta^3} \delta^{3}
                                                  + \frac{\partial^{3} Q_z}{\partial \epsilon_{x}^{3}}  \epsilon_{x}^{3} 
                                                  + \frac{\partial^{3} Q_z}{\partial \epsilon_{y}^{3}}  \epsilon_{y}^{3} \\
                                             &\;\begin{aligned}
                                             \phantom{+ \frac{1}{3!} \biggl[} 
                                               &\color{gray}
                                                + \textcolor{orange}{3 \frac{\partial^{3} Q_z}{\partial \epsilon_{x}\partial \delta^{2}} \delta^{2} \epsilon_{x}}
                                                + \textcolor{orange}{3  \frac{\partial^{3} Q_z}{\partial \epsilon_{y}\partial \delta^{2}}  \delta^{2} \epsilon_{y}}
                                                + 3 \frac{\partial^{3} Q_z}{\partial \epsilon_{x}^{2}\partial \delta}  \delta \epsilon_{x}^{2} \\
                                               &\color{gray}
                                               + 3 \frac{\partial^{3} Q_z}{\partial \epsilon_{y}^{2}\partial \delta} \delta \epsilon_{y}^{2}  
                                                + 3  \frac{\partial^{3} Q_z}{\partial \epsilon_{y}\partial \epsilon_{x}^{2}} \epsilon_{x}^{2} \epsilon_{y} 
                                                + 3 \frac{\partial^{3} Q_z}{\partial \epsilon_{y}^{2}\partial \epsilon_{x}} \epsilon_{x} \epsilon_{y}^{2} \\
                                               &\color{gray}
                                               + 6 \frac{\partial^{3} Q_z}{\partial \epsilon_{y}\partial  \epsilon_{x}\partial \delta} \delta \epsilon_{x} \epsilon_{y} 
                                              \textcolor{orange}{\biggr]}\\
                                             \end{aligned}\\
                                         &\color{gray}
                                         + \textcolor{orange}{\frac{1}{4!} \biggl[ \frac{\partial^4 Q_z}{\partial \delta^4} \delta^4 \biggr] }
\end{aligned}
\end{equation}

\newpage

\hypertarget{ptc-check}{%
\subsection{PTC check}\label{ptc-check}}

A simulation has been done with PTC to assess that those equations are
correct. A dodecapole has been added to the lattice with a strength
\(KL = 1e^6\). Here are the results, confirming PTC works as intended.

The ANH numbers refer to the partial derivative relative to \(J_x\),
\(J_y\) and \(\delta\). So ANHX 021 would for example be
\(\dfrac{\partial^3 Q_x}{\partial J_y^2 \partial \delta}\).

\begin{center}
\begin{tabular}{lrrr}
\toprule
       Term &         Analytical &      Simulation & Rel. Diff [\%] \\
\midrule
  ANH X 200 &      4782639.96971 &      4782639.97 &           0.0 \\
  ANH X 102 &       86945.930342 &        86945.93 &          -0.0 \\
  ANH X 020 &   593469879.552116 &    593469880.01 &           0.0 \\
  ANH X 012 &    -1118366.433407 &    -1118366.433 &          -0.0 \\
  ANH X 110 &   -92277073.535598 &     -92277073.6 &           0.0 \\
  ANH X 004 &        1053.754809 &       1053.7548 &     -0.000001 \\
            &                    &                 &               \\
  ANH Y 200 &   -92277073.535598 &     -92277073.6 &           0.0 \\
  ANH Y 102 &    -1118366.433407 &    -1118366.433 &          -0.0 \\
  ANH Y 020 & -1272278817.264865 & -1272278818.913 &           0.0 \\
  ANH Y 012 &     3596325.539479 &     3596325.543 &           0.0 \\
  ANH Y 110 &   593469879.552116 &    593469880.01 &           0.0 \\
  ANH Y 004 &       -6777.108503 &      -6777.1085 &          -0.0 \\
\bottomrule
\end{tabular}
\end{center}



% ==============================================================
%                    Resonance Driving Terms
% ==============================================================
\chapter{Resonance Driving Terms}
\label{appendix:rdts}
\thumbforappendix

\todo{derivations\\
plots of phase spaces}

This appendix intends to clarify where Resonance Driving Terms can be seen in the frequency
spectrum, what resonance they contribute to and what their action dependance is.  
The number of valid RDTs indeed grows rapidly with the magnet order $n$, as shows
\cref{table:appendix:number_rdts}, and is given by the following combinations:

\begin{equation}
    C(n+3, 3) - C(n+1,1) - \left[(n+1) \;\mathrm{mod}\; 2\right] \cdot C\left(\floor*{\frac{n}{2}}+1, 1\right).
    \label{eq:number_rdts}
\end{equation}

\begin{table}[H]
  \centering
  \begin{tabular}{lccc}
  Multipole     & Order & Number of poles & Number of RDTs             \\
  \hline
  Quadrupole    & 2     & 4    & $5  $                                 \\ 
  Sextupole     & 3     & 6    & $16  $                                \\ 
  Octupole      & 4     & 8    & $27  $                                \\ 
  Decapole      & 5     & 10   & $50  $                                \\ 
  Dodecapole    & 6     & 12   & $73  $                                \\ 
  Decatetrapole & 7     & 14   & $112  $                               \\ 
  Decahexapole  & 8     & 16   & $151  $                               \\ 
  Hectopole     & 50    & 100  & $23349  $                             \\
  Kilopole      & 500   & 1000 & $2.1 \times 10^7$ \\ \hline
  \end{tabular}
  \caption{Number of valid RDTs for a given multipole order}
  \label{table:appendix:number_rdts}
\end{table}


Several different RDTs can contribute to the same line, which can be observed in the horizontal or vertical spectrum. The tables below describe which RDTs contribute to a specific combination of line and plane.
All tables have been computed up to the order 6, for decapoles.
The line columns represents ($Q_x$, $Q_y$). For example (-1, 2) is \(-1Q_x + 2Qy\).

As a reminder, for a given RDT $f_{jklm}$, we will observe:

\begin{equation}\begin{aligned}
& (j-k)Q_x + (l-m)Q_y = p \in \mathbb{N} \quad\quad& \mbox{excited resonance}\\
& H(1 - j + k, m - l) \quad\quad& \mbox{horizontal line, if } j \ne 0 \\
& V(k - j, 1 - l + m) \quad\quad& \mbox{vertical line, if } l \ne 0. \\
\end{aligned}
\label{eq:reminder_rdt}
\end{equation}

The amplitude of each line is given by:
\begin{equation}
    \begin{aligned}
    &|H_{f_{jklm}}| = 2 j (2 I_x)^\frac{j+k-1}{2} (2 I_y)^\frac{l+m}{2} |f_{jklm}| \\
    &|V_{f_{jklm}}| = 2 l (2 I_x)^\frac{j+k}{2} (2 I_y)^\frac{l+m-1}{2} |f_{jklm}|.
    \label{eq:amplitude_fjklm}
    \end{aligned}
\end{equation}

According to equations \ref{eq:reminder_rdt} and \ref{eq:amplitude_fjklm}, it can be seen that many RDTs will no generate any line and thus can not be observed.

\section{Frequency Spectrum Lines}

\subsection{Horizontal Axis}

\begin{longtable}[]{@{}ll@{}}
\toprule()
H-line & RDTs \\
\midrule()
\endhead
(-5, 0) & f6000 \\
(-4, -1) & f5010 \\
(-4, 0) & f5000 \\
(-4, 1) & f5001 \\
(-3, -2) & f4020 \\
(-3, -1) & f4010 \\
(-3, 0) & f4000, f4011, f5100 \\
(-3, 1) & f4001 \\
(-3, 2) & f4002 \\
(-2, -3) & f3030 \\
(-2, -2) & f3020 \\
(-2, -1) & f3010, f3021, f4110 \\
(-2, 0) & f3000, f3011, f4100 \\
(-2, 1) & f3001, f3012, f4101 \\
(-2, 2) & f3002 \\
(-2, 3) & f3003 \\
(-1, -4) & f2040 \\
(-1, -3) & f2030 \\
(-1, -2) & f2020, f2031, f3120 \\
(-1, -1) & f2010, f2021, f3110 \\
(-1, 0) & f2000, f2011, f3100, f2022, f3111, f4200 \\
(-1, 1) & f2001, f2012, f3101 \\
(-1, 2) & f2002, f2013, f3102 \\
(-1, 3) & f2003 \\
(-1, 4) & f2004 \\
(0, -5) & f1050 \\
(0, -4) & f1040 \\
(0, -3) & f1030, f1041, f2130 \\
(0, -2) & f1020, f1031, f2120 \\
(0, -1) & f1010, f1021, f2110, f1032, f2121, f3210 \\
(0, 0) & f1011, f2100, f1022, f2111, f3200 \\
(0, 1) & f1001, f1012, f2101, f1023, f2112, f3201 \\
(0, 2) & f1002, f1013, f2102 \\
(0, 3) & f1003, f1014, f2103 \\
(0, 4) & f1004 \\
(0, 5) & f1005 \\
(1, -4) & f1140 \\
(1, -3) & f1130 \\
(1, -2) & f1120, f1131, f2220 \\
(1, -1) & f1110, f1121, f2210 \\
(1, 1) & f1101, f1112, f2201 \\
(1, 2) & f1102, f1113, f2202 \\
(1, 3) & f1103 \\
(1, 4) & f1104 \\
(2, -3) & f1230 \\
(2, -2) & f1220 \\
(2, -1) & f1210, f1221, f2310 \\
(2, 0) & f1200, f1211, f2300 \\
(2, 1) & f1201, f1212, f2301 \\
(2, 2) & f1202 \\
(2, 3) & f1203 \\
(3, -2) & f1320 \\
(3, -1) & f1310 \\
(3, 0) & f1300, f1311, f2400 \\
(3, 1) & f1301 \\
(3, 2) & f1302 \\
(4, -1) & f1410 \\
(4, 0) & f1400 \\
(4, 1) & f1401 \\
(5, 0) & f1500 \\
\bottomrule()
\end{longtable}

\subsection{Vertical Axis}

\begin{longtable}[]{@{}ll@{}}
\toprule()
V-line & RDTs \\
\midrule()
\endhead
(-5, 0) & f5010 \\
(-4, -1) & f4020 \\
(-4, 0) & f4010 \\
(-4, 1) & f4011 \\
(-3, -2) & f3030 \\
(-3, -1) & f3020 \\
(-3, 0) & f3010, f3021, f4110 \\
(-3, 1) & f3011 \\
(-3, 2) & f3012 \\
(-2, -3) & f2040 \\
(-2, -2) & f2030 \\
(-2, -1) & f2020, f2031, f3120 \\
(-2, 0) & f2010, f2021, f3110 \\
(-2, 1) & f2011, f2022, f3111 \\
(-2, 2) & f2012 \\
(-2, 3) & f2013 \\
(-1, -4) & f1050 \\
(-1, -3) & f1040 \\
(-1, -2) & f1030, f1041, f2130 \\
(-1, -1) & f1020, f1031, f2120 \\
(-1, 0) & f1010, f1021, f2110, f1032, f2121, f3210 \\
(-1, 1) & f1011, f1022, f2111 \\
(-1, 2) & f1012, f1023, f2112 \\
(-1, 3) & f1013 \\
(-1, 4) & f1014 \\
(0, -5) & f0060 \\
(0, -4) & f0050 \\
(0, -3) & f0040, f0051, f1140 \\
(0, -2) & f0030, f0041, f1130 \\
(0, -1) & f0020, f0031, f1120, f0042, f1131, f2220 \\
(0, 0) & f0021, f1110, f0032, f1121, f2210 \\
(0, 2) & f0012, f0023, f1112 \\
(0, 3) & f0013, f0024, f1113 \\
(0, 4) & f0014 \\
(0, 5) & f0015 \\
(1, -4) & f0150 \\
(1, -3) & f0140 \\
(1, -2) & f0130, f0141, f1230 \\
(1, -1) & f0120, f0131, f1220 \\
(1, 0) & f0110, f0121, f1210, f0132, f1221, f2310 \\
(1, 1) & f0111, f0122, f1211 \\
(1, 2) & f0112, f0123, f1212 \\
(1, 3) & f0113 \\
(1, 4) & f0114 \\
(2, -3) & f0240 \\
(2, -2) & f0230 \\
(2, -1) & f0220, f0231, f1320 \\
(2, 0) & f0210, f0221, f1310 \\
(2, 1) & f0211, f0222, f1311 \\
(2, 2) & f0212 \\
(2, 3) & f0213 \\
(3, -2) & f0330 \\
(3, -1) & f0320 \\
(3, 0) & f0310, f0321, f1410 \\
(3, 1) & f0311 \\
(3, 2) & f0312 \\
(4, -1) & f0420 \\
(4, 0) & f0410 \\
(4, 1) & f0411 \\
(5, 0) & f0510 \\
\bottomrule()
\end{longtable}

\newpage

\section{Amplitude, Resonances and Lines}

This part focuses on individual Resonance Drivings Terms, expliciting what magnet they originate from, what resonance they excite, how they can be observed and what kicks are needed in order to measure them.
The amplitude columns implicitly omits the term $|f_{jklm}|$, which depends on $K$ and $J$.

Amplitude legend:

\begin{itemize}
\tightlist
\item
  \colorbox{orange!20}{$I_x$}: depends only on horizontal amplitude
\item
  \colorbox{red!20}{$I_y$}: depends only on vertical amplitude
\item
  \colorbox{blue!20}{$I_x I_y$}: depends on both horizontal and vertical
  amplitude
\end{itemize}

%\small
{\scriptsize
\begin{longtable}{llllllll}
\toprule()
n & jklm &   type & resonance &   H-line &   V-line &                                       Amplitude H &                                       Amplitude V \\
\midrule()
\endhead
2 & 0020 & normal &    (0, 2) &          &  (0, -1) &                                                   &              \colorbox{red!10}{$4  (2I_y)^ {1/2}$} \\
2 & 2000 & normal &    (2, 0) &  (-1, 0) &          &           \colorbox{orange!10}{$4 (2I_x)^ {1/2} $} &                                                   \\
2 & 0110 &   skew &   (-1, 1) &          &   (1, 0) &                                                   &           \colorbox{orange!10}{$2 (2I_x)^ {1/2} $} \\
2 & 1001 &   skew &   (1, -1) &   (0, 1) &          &              \colorbox{red!10}{$2  (2I_y)^ {1/2}$} &                                                   \\
2 & 1010 &   skew &    (1, 1) &  (0, -1) &  (-1, 0) &              \colorbox{red!10}{$2  (2I_y)^ {1/2}$} &           \colorbox{orange!10}{$2 (2I_x)^ {1/2} $} \\
\midrule()
3 & 0111 & normal &   (-1, 0) &          &   (1, 1) &                                                   & \colorbox{blue!10}{$2 (2I_x)^ {1/2} (2I_y)^ {1/2}$} \\
3 & 0120 & normal &   (-1, 2) &          &  (1, -1) &                                                   & \colorbox{blue!10}{$4 (2I_x)^ {1/2} (2I_y)^ {1/2}$} \\
3 & 1002 & normal &   (1, -2) &   (0, 2) &          &                    \colorbox{red!10}{$2  (2I_y)$} &                                                   \\
3 & 1011 & normal &    (1, 0) &   (0, 0) &  (-1, 1) &                    \colorbox{red!10}{$2  (2I_y)$} & \colorbox{blue!10}{$2 (2I_x)^ {1/2} (2I_y)^ {1/2}$} \\
3 & 1020 & normal &    (1, 2) &  (0, -2) & (-1, -1) &                    \colorbox{red!10}{$2  (2I_y)$} & \colorbox{blue!10}{$4 (2I_x)^ {1/2} (2I_y)^ {1/2}$} \\
3 & 1200 & normal &   (-1, 0) &   (2, 0) &          &                 \colorbox{orange!10}{$2 (2I_x) $} &                                                   \\
3 & 2100 & normal &    (1, 0) &   (0, 0) &          &                 \colorbox{orange!10}{$4 (2I_x) $} &                                                   \\
3 & 3000 & normal &    (3, 0) &  (-2, 0) &          &                 \colorbox{orange!10}{$6 (2I_x) $} &                                                   \\
3 & 0012 &   skew &   (0, -1) &          &   (0, 2) &                                                   &                    \colorbox{red!10}{$2  (2I_y)$} \\
3 & 0021 &   skew &    (0, 1) &          &   (0, 0) &                                                   &                    \colorbox{red!10}{$4  (2I_y)$} \\
3 & 0030 &   skew &    (0, 3) &          &  (0, -2) &                                                   &                    \colorbox{red!10}{$6  (2I_y)$} \\
3 & 0210 &   skew &   (-2, 1) &          &   (2, 0) &                                                   &                 \colorbox{orange!10}{$2 (2I_x) $} \\
3 & 1101 &   skew &   (0, -1) &   (1, 1) &          & \colorbox{blue!10}{$2 (2I_x)^ {1/2} (2I_y)^ {1/2}$} &                                                   \\
3 & 1110 &   skew &    (0, 1) &  (1, -1) &   (0, 0) & \colorbox{blue!10}{$2 (2I_x)^ {1/2} (2I_y)^ {1/2}$} &                 \colorbox{orange!10}{$2 (2I_x) $} \\
3 & 2001 &   skew &   (2, -1) &  (-1, 1) &          & \colorbox{blue!10}{$4 (2I_x)^ {1/2} (2I_y)^ {1/2}$} &                                                   \\
3 & 2010 &   skew &    (2, 1) & (-1, -1) &  (-2, 0) & \colorbox{blue!10}{$4 (2I_x)^ {1/2} (2I_y)^ {1/2}$} &                 \colorbox{orange!10}{$2 (2I_x) $} \\
\midrule()
4 & 0013 & normal &   (0, -2) &          &   (0, 3) &                                                   &              \colorbox{red!10}{$2  (2I_y)^ {3/2}$} \\
4 & 0031 & normal &    (0, 2) &          &  (0, -1) &                                                   &              \colorbox{red!10}{$6  (2I_y)^ {3/2}$} \\
4 & 0040 & normal &    (0, 4) &          &  (0, -3) &                                                   &              \colorbox{red!10}{$8  (2I_y)^ {3/2}$} \\
4 & 0211 & normal &   (-2, 0) &          &   (2, 1) &                                                   &       \colorbox{blue!10}{$2 (2I_x) (2I_y)^ {1/2}$} \\
4 & 0220 & normal &   (-2, 2) &          &  (2, -1) &                                                   &       \colorbox{blue!10}{$4 (2I_x) (2I_y)^ {1/2}$} \\
4 & 1102 & normal &   (0, -2) &   (1, 2) &          &       \colorbox{blue!10}{$2 (2I_x)^ {1/2} (2I_y)$} &                                                   \\
4 & 1120 & normal &    (0, 2) &  (1, -2) &  (0, -1) &       \colorbox{blue!10}{$2 (2I_x)^ {1/2} (2I_y)$} &       \colorbox{blue!10}{$4 (2I_x) (2I_y)^ {1/2}$} \\
4 & 1300 & normal &   (-2, 0) &   (3, 0) &          &           \colorbox{orange!10}{$2 (2I_x)^ {3/2} $} &                                                   \\
4 & 2002 & normal &   (2, -2) &  (-1, 2) &          &       \colorbox{blue!10}{$4 (2I_x)^ {1/2} (2I_y)$} &                                                   \\
4 & 2011 & normal &    (2, 0) &  (-1, 0) &  (-2, 1) &       \colorbox{blue!10}{$4 (2I_x)^ {1/2} (2I_y)$} &       \colorbox{blue!10}{$2 (2I_x) (2I_y)^ {1/2}$} \\
4 & 2020 & normal &    (2, 2) & (-1, -2) & (-2, -1) &       \colorbox{blue!10}{$4 (2I_x)^ {1/2} (2I_y)$} &       \colorbox{blue!10}{$4 (2I_x) (2I_y)^ {1/2}$} \\
4 & 3100 & normal &    (2, 0) &  (-1, 0) &          &           \colorbox{orange!10}{$6 (2I_x)^ {3/2} $} &                                                   \\
4 & 4000 & normal &    (4, 0) &  (-3, 0) &          &           \colorbox{orange!10}{$8 (2I_x)^ {3/2} $} &                                                   \\
4 & 0112 &   skew &  (-1, -1) &          &   (1, 2) &                                                   &       \colorbox{blue!10}{$2 (2I_x)^ {1/2} (2I_y)$} \\
4 & 0121 &   skew &   (-1, 1) &          &   (1, 0) &                                                   &       \colorbox{blue!10}{$4 (2I_x)^ {1/2} (2I_y)$} \\
4 & 0130 &   skew &   (-1, 3) &          &  (1, -2) &                                                   &       \colorbox{blue!10}{$6 (2I_x)^ {1/2} (2I_y)$} \\
4 & 0310 &   skew &   (-3, 1) &          &   (3, 0) &                                                   &           \colorbox{orange!10}{$2 (2I_x)^ {3/2} $} \\
4 & 1003 &   skew &   (1, -3) &   (0, 3) &          &              \colorbox{red!10}{$2  (2I_y)^ {3/2}$} &                                                   \\
4 & 1012 &   skew &   (1, -1) &   (0, 1) &  (-1, 2) &              \colorbox{red!10}{$2  (2I_y)^ {3/2}$} &       \colorbox{blue!10}{$2 (2I_x)^ {1/2} (2I_y)$} \\
4 & 1021 &   skew &    (1, 1) &  (0, -1) &  (-1, 0) &              \colorbox{red!10}{$2  (2I_y)^ {3/2}$} &       \colorbox{blue!10}{$4 (2I_x)^ {1/2} (2I_y)$} \\
4 & 1030 &   skew &    (1, 3) &  (0, -3) & (-1, -2) &              \colorbox{red!10}{$2  (2I_y)^ {3/2}$} &       \colorbox{blue!10}{$6 (2I_x)^ {1/2} (2I_y)$} \\
4 & 1201 &   skew &  (-1, -1) &   (2, 1) &          &       \colorbox{blue!10}{$2 (2I_x) (2I_y)^ {1/2}$} &                                                   \\
4 & 1210 &   skew &   (-1, 1) &  (2, -1) &   (1, 0) &       \colorbox{blue!10}{$2 (2I_x) (2I_y)^ {1/2}$} &           \colorbox{orange!10}{$2 (2I_x)^ {3/2} $} \\
4 & 2101 &   skew &   (1, -1) &   (0, 1) &          &       \colorbox{blue!10}{$4 (2I_x) (2I_y)^ {1/2}$} &                                                   \\
4 & 2110 &   skew &    (1, 1) &  (0, -1) &  (-1, 0) &       \colorbox{blue!10}{$4 (2I_x) (2I_y)^ {1/2}$} &           \colorbox{orange!10}{$2 (2I_x)^ {3/2} $} \\
4 & 3001 &   skew &   (3, -1) &  (-2, 1) &          &       \colorbox{blue!10}{$6 (2I_x) (2I_y)^ {1/2}$} &                                                   \\
4 & 3010 &   skew &    (3, 1) & (-2, -1) &  (-3, 0) &       \colorbox{blue!10}{$6 (2I_x) (2I_y)^ {1/2}$} &           \colorbox{orange!10}{$2 (2I_x)^ {3/2} $} \\
\midrule()
5 & 0113 & normal &  (-1, -2) &          &   (1, 3) &                                                   & \colorbox{blue!10}{$2 (2I_x)^ {1/2} (2I_y)^ {3/2}$} \\
5 & 0122 & normal &   (-1, 0) &          &   (1, 1) &                                                   & \colorbox{blue!10}{$4 (2I_x)^ {1/2} (2I_y)^ {3/2}$} \\
5 & 0131 & normal &   (-1, 2) &          &  (1, -1) &                                                   & \colorbox{blue!10}{$6 (2I_x)^ {1/2} (2I_y)^ {3/2}$} \\
5 & 0140 & normal &   (-1, 4) &          &  (1, -3) &                                                   & \colorbox{blue!10}{$8 (2I_x)^ {1/2} (2I_y)^ {3/2}$} \\
5 & 0311 & normal &   (-3, 0) &          &   (3, 1) &                                                   & \colorbox{blue!10}{$2 (2I_x)^ {3/2} (2I_y)^ {1/2}$} \\
5 & 0320 & normal &   (-3, 2) &          &  (3, -1) &                                                   & \colorbox{blue!10}{$4 (2I_x)^ {3/2} (2I_y)^ {1/2}$} \\
5 & 1004 & normal &   (1, -4) &   (0, 4) &          &                \colorbox{red!10}{$2  (2I_y)^ {2}$} &                                                   \\
5 & 1013 & normal &   (1, -2) &   (0, 2) &  (-1, 3) &                \colorbox{red!10}{$2  (2I_y)^ {2}$} & \colorbox{blue!10}{$2 (2I_x)^ {1/2} (2I_y)^ {3/2}$} \\
5 & 1022 & normal &    (1, 0) &   (0, 0) &  (-1, 1) &                \colorbox{red!10}{$2  (2I_y)^ {2}$} & \colorbox{blue!10}{$4 (2I_x)^ {1/2} (2I_y)^ {3/2}$} \\
5 & 1031 & normal &    (1, 2) &  (0, -2) & (-1, -1) &                \colorbox{red!10}{$2  (2I_y)^ {2}$} & \colorbox{blue!10}{$6 (2I_x)^ {1/2} (2I_y)^ {3/2}$} \\
5 & 1040 & normal &    (1, 4) &  (0, -4) & (-1, -3) &                \colorbox{red!10}{$2  (2I_y)^ {2}$} & \colorbox{blue!10}{$8 (2I_x)^ {1/2} (2I_y)^ {3/2}$} \\
5 & 1202 & normal &  (-1, -2) &   (2, 2) &          &             \colorbox{blue!10}{$2 (2I_x) (2I_y)$} &                                                   \\
5 & 1211 & normal &   (-1, 0) &   (2, 0) &   (1, 1) &             \colorbox{blue!10}{$2 (2I_x) (2I_y)$} & \colorbox{blue!10}{$2 (2I_x)^ {3/2} (2I_y)^ {1/2}$} \\
5 & 1220 & normal &   (-1, 2) &  (2, -2) &  (1, -1) &             \colorbox{blue!10}{$2 (2I_x) (2I_y)$} & \colorbox{blue!10}{$4 (2I_x)^ {3/2} (2I_y)^ {1/2}$} \\
5 & 1400 & normal &   (-3, 0) &   (4, 0) &          &             \colorbox{orange!10}{$2 (2I_x)^ {2} $} &                                                   \\
5 & 2102 & normal &   (1, -2) &   (0, 2) &          &             \colorbox{blue!10}{$4 (2I_x) (2I_y)$} &                                                   \\
5 & 2111 & normal &    (1, 0) &   (0, 0) &  (-1, 1) &             \colorbox{blue!10}{$4 (2I_x) (2I_y)$} & \colorbox{blue!10}{$2 (2I_x)^ {3/2} (2I_y)^ {1/2}$} \\
5 & 2120 & normal &    (1, 2) &  (0, -2) & (-1, -1) &             \colorbox{blue!10}{$4 (2I_x) (2I_y)$} & \colorbox{blue!10}{$4 (2I_x)^ {3/2} (2I_y)^ {1/2}$} \\
5 & 2300 & normal &   (-1, 0) &   (2, 0) &          &             \colorbox{orange!10}{$4 (2I_x)^ {2} $} &                                                   \\
5 & 3002 & normal &   (3, -2) &  (-2, 2) &          &             \colorbox{blue!10}{$6 (2I_x) (2I_y)$} &                                                   \\
5 & 3011 & normal &    (3, 0) &  (-2, 0) &  (-3, 1) &             \colorbox{blue!10}{$6 (2I_x) (2I_y)$} & \colorbox{blue!10}{$2 (2I_x)^ {3/2} (2I_y)^ {1/2}$} \\
5 & 3020 & normal &    (3, 2) & (-2, -2) & (-3, -1) &             \colorbox{blue!10}{$6 (2I_x) (2I_y)$} & \colorbox{blue!10}{$4 (2I_x)^ {3/2} (2I_y)^ {1/2}$} \\
5 & 3200 & normal &    (1, 0) &   (0, 0) &          &             \colorbox{orange!10}{$6 (2I_x)^ {2} $} &                                                   \\
5 & 4100 & normal &    (3, 0) &  (-2, 0) &          &             \colorbox{orange!10}{$8 (2I_x)^ {2} $} &                                                   \\
5 & 5000 & normal &    (5, 0) &  (-4, 0) &          &            \colorbox{orange!10}{$10 (2I_x)^ {2} $} &                                                   \\
5 & 0014 &   skew &   (0, -3) &          &   (0, 4) &                                                   &                \colorbox{red!10}{$2  (2I_y)^ {2}$} \\
5 & 0023 &   skew &   (0, -1) &          &   (0, 2) &                                                   &                \colorbox{red!10}{$4  (2I_y)^ {2}$} \\
5 & 0032 &   skew &    (0, 1) &          &   (0, 0) &                                                   &                \colorbox{red!10}{$6  (2I_y)^ {2}$} \\
5 & 0041 &   skew &    (0, 3) &          &  (0, -2) &                                                   &                \colorbox{red!10}{$8  (2I_y)^ {2}$} \\
5 & 0050 &   skew &    (0, 5) &          &  (0, -4) &                                                   &               \colorbox{red!10}{$10  (2I_y)^ {2}$} \\
5 & 0212 &   skew &  (-2, -1) &          &   (2, 2) &                                                   &             \colorbox{blue!10}{$2 (2I_x) (2I_y)$} \\
5 & 0221 &   skew &   (-2, 1) &          &   (2, 0) &                                                   &             \colorbox{blue!10}{$4 (2I_x) (2I_y)$} \\
5 & 0230 &   skew &   (-2, 3) &          &  (2, -2) &                                                   &             \colorbox{blue!10}{$6 (2I_x) (2I_y)$} \\
5 & 0410 &   skew &   (-4, 1) &          &   (4, 0) &                                                   &             \colorbox{orange!10}{$2 (2I_x)^ {2} $} \\
5 & 1103 &   skew &   (0, -3) &   (1, 3) &          & \colorbox{blue!10}{$2 (2I_x)^ {1/2} (2I_y)^ {3/2}$} &                                                   \\
5 & 1112 &   skew &   (0, -1) &   (1, 1) &   (0, 2) & \colorbox{blue!10}{$2 (2I_x)^ {1/2} (2I_y)^ {3/2}$} &             \colorbox{blue!10}{$2 (2I_x) (2I_y)$} \\
5 & 1121 &   skew &    (0, 1) &  (1, -1) &   (0, 0) & \colorbox{blue!10}{$2 (2I_x)^ {1/2} (2I_y)^ {3/2}$} &             \colorbox{blue!10}{$4 (2I_x) (2I_y)$} \\
5 & 1130 &   skew &    (0, 3) &  (1, -3) &  (0, -2) & \colorbox{blue!10}{$2 (2I_x)^ {1/2} (2I_y)^ {3/2}$} &             \colorbox{blue!10}{$6 (2I_x) (2I_y)$} \\
5 & 1301 &   skew &  (-2, -1) &   (3, 1) &          & \colorbox{blue!10}{$2 (2I_x)^ {3/2} (2I_y)^ {1/2}$} &                                                   \\
5 & 1310 &   skew &   (-2, 1) &  (3, -1) &   (2, 0) & \colorbox{blue!10}{$2 (2I_x)^ {3/2} (2I_y)^ {1/2}$} &             \colorbox{orange!10}{$2 (2I_x)^ {2} $} \\
5 & 2003 &   skew &   (2, -3) &  (-1, 3) &          & \colorbox{blue!10}{$4 (2I_x)^ {1/2} (2I_y)^ {3/2}$} &                                                   \\
5 & 2012 &   skew &   (2, -1) &  (-1, 1) &  (-2, 2) & \colorbox{blue!10}{$4 (2I_x)^ {1/2} (2I_y)^ {3/2}$} &             \colorbox{blue!10}{$2 (2I_x) (2I_y)$} \\
5 & 2021 &   skew &    (2, 1) & (-1, -1) &  (-2, 0) & \colorbox{blue!10}{$4 (2I_x)^ {1/2} (2I_y)^ {3/2}$} &             \colorbox{blue!10}{$4 (2I_x) (2I_y)$} \\
5 & 2030 &   skew &    (2, 3) & (-1, -3) & (-2, -2) & \colorbox{blue!10}{$4 (2I_x)^ {1/2} (2I_y)^ {3/2}$} &             \colorbox{blue!10}{$6 (2I_x) (2I_y)$} \\
5 & 2201 &   skew &   (0, -1) &   (1, 1) &          & \colorbox{blue!10}{$4 (2I_x)^ {3/2} (2I_y)^ {1/2}$} &                                                   \\
5 & 2210 &   skew &    (0, 1) &  (1, -1) &   (0, 0) & \colorbox{blue!10}{$4 (2I_x)^ {3/2} (2I_y)^ {1/2}$} &             \colorbox{orange!10}{$2 (2I_x)^ {2} $} \\
5 & 3101 &   skew &   (2, -1) &  (-1, 1) &          & \colorbox{blue!10}{$6 (2I_x)^ {3/2} (2I_y)^ {1/2}$} &                                                   \\
5 & 3110 &   skew &    (2, 1) & (-1, -1) &  (-2, 0) & \colorbox{blue!10}{$6 (2I_x)^ {3/2} (2I_y)^ {1/2}$} &             \colorbox{orange!10}{$2 (2I_x)^ {2} $} \\
5 & 4001 &   skew &   (4, -1) &  (-3, 1) &          & \colorbox{blue!10}{$8 (2I_x)^ {3/2} (2I_y)^ {1/2}$} &                                                   \\
5 & 4010 &   skew &    (4, 1) & (-3, -1) &  (-4, 0) & \colorbox{blue!10}{$8 (2I_x)^ {3/2} (2I_y)^ {1/2}$} &             \colorbox{orange!10}{$2 (2I_x)^ {2} $} \\
\midrule()
6 & 0015 & normal &   (0, -4) &          &   (0, 5) &                                                   &              \colorbox{red!10}{$2  (2I_y)^ {5/2}$} \\
6 & 0024 & normal &   (0, -2) &          &   (0, 3) &                                                   &              \colorbox{red!10}{$4  (2I_y)^ {5/2}$} \\
6 & 0042 & normal &    (0, 2) &          &  (0, -1) &                                                   &              \colorbox{red!10}{$8  (2I_y)^ {5/2}$} \\
6 & 0051 & normal &    (0, 4) &          &  (0, -3) &                                                   &             \colorbox{red!10}{$10  (2I_y)^ {5/2}$} \\
6 & 0060 & normal &    (0, 6) &          &  (0, -5) &                                                   &             \colorbox{red!10}{$12  (2I_y)^ {5/2}$} \\
6 & 0213 & normal &  (-2, -2) &          &   (2, 3) &                                                   &       \colorbox{blue!10}{$2 (2I_x) (2I_y)^ {3/2}$} \\
6 & 0222 & normal &   (-2, 0) &          &   (2, 1) &                                                   &       \colorbox{blue!10}{$4 (2I_x) (2I_y)^ {3/2}$} \\
6 & 0231 & normal &   (-2, 2) &          &  (2, -1) &                                                   &       \colorbox{blue!10}{$6 (2I_x) (2I_y)^ {3/2}$} \\
6 & 0240 & normal &   (-2, 4) &          &  (2, -3) &                                                   &       \colorbox{blue!10}{$8 (2I_x) (2I_y)^ {3/2}$} \\
6 & 0411 & normal &   (-4, 0) &          &   (4, 1) &                                                   &   \colorbox{blue!10}{$2 (2I_x)^ {2} (2I_y)^ {1/2}$} \\
6 & 0420 & normal &   (-4, 2) &          &  (4, -1) &                                                   &   \colorbox{blue!10}{$4 (2I_x)^ {2} (2I_y)^ {1/2}$} \\
6 & 1104 & normal &   (0, -4) &   (1, 4) &          &   \colorbox{blue!10}{$2 (2I_x)^ {1/2} (2I_y)^ {2}$} &                                                   \\
6 & 1113 & normal &   (0, -2) &   (1, 2) &   (0, 3) &   \colorbox{blue!10}{$2 (2I_x)^ {1/2} (2I_y)^ {2}$} &       \colorbox{blue!10}{$2 (2I_x) (2I_y)^ {3/2}$} \\
6 & 1131 & normal &    (0, 2) &  (1, -2) &  (0, -1) &   \colorbox{blue!10}{$2 (2I_x)^ {1/2} (2I_y)^ {2}$} &       \colorbox{blue!10}{$6 (2I_x) (2I_y)^ {3/2}$} \\
6 & 1140 & normal &    (0, 4) &  (1, -4) &  (0, -3) &   \colorbox{blue!10}{$2 (2I_x)^ {1/2} (2I_y)^ {2}$} &       \colorbox{blue!10}{$8 (2I_x) (2I_y)^ {3/2}$} \\
6 & 1302 & normal &  (-2, -2) &   (3, 2) &          &       \colorbox{blue!10}{$2 (2I_x)^ {3/2} (2I_y)$} &                                                   \\
6 & 1311 & normal &   (-2, 0) &   (3, 0) &   (2, 1) &       \colorbox{blue!10}{$2 (2I_x)^ {3/2} (2I_y)$} &   \colorbox{blue!10}{$2 (2I_x)^ {2} (2I_y)^ {1/2}$} \\
6 & 1320 & normal &   (-2, 2) &  (3, -2) &  (2, -1) &       \colorbox{blue!10}{$2 (2I_x)^ {3/2} (2I_y)$} &   \colorbox{blue!10}{$4 (2I_x)^ {2} (2I_y)^ {1/2}$} \\
6 & 1500 & normal &   (-4, 0) &   (5, 0) &          &           \colorbox{orange!10}{$2 (2I_x)^ {5/2} $} &                                                   \\
6 & 2004 & normal &   (2, -4) &  (-1, 4) &          &   \colorbox{blue!10}{$4 (2I_x)^ {1/2} (2I_y)^ {2}$} &                                                   \\
6 & 2013 & normal &   (2, -2) &  (-1, 2) &  (-2, 3) &   \colorbox{blue!10}{$4 (2I_x)^ {1/2} (2I_y)^ {2}$} &       \colorbox{blue!10}{$2 (2I_x) (2I_y)^ {3/2}$} \\
6 & 2022 & normal &    (2, 0) &  (-1, 0) &  (-2, 1) &   \colorbox{blue!10}{$4 (2I_x)^ {1/2} (2I_y)^ {2}$} &       \colorbox{blue!10}{$4 (2I_x) (2I_y)^ {3/2}$} \\
6 & 2031 & normal &    (2, 2) & (-1, -2) & (-2, -1) &   \colorbox{blue!10}{$4 (2I_x)^ {1/2} (2I_y)^ {2}$} &       \colorbox{blue!10}{$6 (2I_x) (2I_y)^ {3/2}$} \\
6 & 2040 & normal &    (2, 4) & (-1, -4) & (-2, -3) &   \colorbox{blue!10}{$4 (2I_x)^ {1/2} (2I_y)^ {2}$} &       \colorbox{blue!10}{$8 (2I_x) (2I_y)^ {3/2}$} \\
6 & 2202 & normal &   (0, -2) &   (1, 2) &          &       \colorbox{blue!10}{$4 (2I_x)^ {3/2} (2I_y)$} &                                                   \\
6 & 2220 & normal &    (0, 2) &  (1, -2) &  (0, -1) &       \colorbox{blue!10}{$4 (2I_x)^ {3/2} (2I_y)$} &   \colorbox{blue!10}{$4 (2I_x)^ {2} (2I_y)^ {1/2}$} \\
6 & 2400 & normal &   (-2, 0) &   (3, 0) &          &           \colorbox{orange!10}{$4 (2I_x)^ {5/2} $} &                                                   \\
6 & 3102 & normal &   (2, -2) &  (-1, 2) &          &       \colorbox{blue!10}{$6 (2I_x)^ {3/2} (2I_y)$} &                                                   \\
6 & 3111 & normal &    (2, 0) &  (-1, 0) &  (-2, 1) &       \colorbox{blue!10}{$6 (2I_x)^ {3/2} (2I_y)$} &   \colorbox{blue!10}{$2 (2I_x)^ {2} (2I_y)^ {1/2}$} \\
6 & 3120 & normal &    (2, 2) & (-1, -2) & (-2, -1) &       \colorbox{blue!10}{$6 (2I_x)^ {3/2} (2I_y)$} &   \colorbox{blue!10}{$4 (2I_x)^ {2} (2I_y)^ {1/2}$} \\
6 & 4002 & normal &   (4, -2) &  (-3, 2) &          &       \colorbox{blue!10}{$8 (2I_x)^ {3/2} (2I_y)$} &                                                   \\
6 & 4011 & normal &    (4, 0) &  (-3, 0) &  (-4, 1) &       \colorbox{blue!10}{$8 (2I_x)^ {3/2} (2I_y)$} &   \colorbox{blue!10}{$2 (2I_x)^ {2} (2I_y)^ {1/2}$} \\
6 & 4020 & normal &    (4, 2) & (-3, -2) & (-4, -1) &       \colorbox{blue!10}{$8 (2I_x)^ {3/2} (2I_y)$} &   \colorbox{blue!10}{$4 (2I_x)^ {2} (2I_y)^ {1/2}$} \\
6 & 4200 & normal &    (2, 0) &  (-1, 0) &          &           \colorbox{orange!10}{$8 (2I_x)^ {5/2} $} &                                                   \\
6 & 5100 & normal &    (4, 0) &  (-3, 0) &          &          \colorbox{orange!10}{$10 (2I_x)^ {5/2} $} &                                                   \\
6 & 6000 & normal &    (6, 0) &  (-5, 0) &          &          \colorbox{orange!10}{$12 (2I_x)^ {5/2} $} &                                                   \\
6 & 0114 &   skew &  (-1, -3) &          &   (1, 4) &                                                   &   \colorbox{blue!10}{$2 (2I_x)^ {1/2} (2I_y)^ {2}$} \\
6 & 0123 &   skew &  (-1, -1) &          &   (1, 2) &                                                   &   \colorbox{blue!10}{$4 (2I_x)^ {1/2} (2I_y)^ {2}$} \\
6 & 0132 &   skew &   (-1, 1) &          &   (1, 0) &                                                   &   \colorbox{blue!10}{$6 (2I_x)^ {1/2} (2I_y)^ {2}$} \\
6 & 0141 &   skew &   (-1, 3) &          &  (1, -2) &                                                   &   \colorbox{blue!10}{$8 (2I_x)^ {1/2} (2I_y)^ {2}$} \\
6 & 0150 &   skew &   (-1, 5) &          &  (1, -4) &                                                   &  \colorbox{blue!10}{$10 (2I_x)^ {1/2} (2I_y)^ {2}$} \\
6 & 0312 &   skew &  (-3, -1) &          &   (3, 2) &                                                   &       \colorbox{blue!10}{$2 (2I_x)^ {3/2} (2I_y)$} \\
6 & 0321 &   skew &   (-3, 1) &          &   (3, 0) &                                                   &       \colorbox{blue!10}{$4 (2I_x)^ {3/2} (2I_y)$} \\
6 & 0330 &   skew &   (-3, 3) &          &  (3, -2) &                                                   &       \colorbox{blue!10}{$6 (2I_x)^ {3/2} (2I_y)$} \\
6 & 0510 &   skew &   (-5, 1) &          &   (5, 0) &                                                   &           \colorbox{orange!10}{$2 (2I_x)^ {5/2} $} \\
6 & 1005 &   skew &   (1, -5) &   (0, 5) &          &              \colorbox{red!10}{$2  (2I_y)^ {5/2}$} &                                                   \\
6 & 1014 &   skew &   (1, -3) &   (0, 3) &  (-1, 4) &              \colorbox{red!10}{$2  (2I_y)^ {5/2}$} &   \colorbox{blue!10}{$2 (2I_x)^ {1/2} (2I_y)^ {2}$} \\
6 & 1023 &   skew &   (1, -1) &   (0, 1) &  (-1, 2) &              \colorbox{red!10}{$2  (2I_y)^ {5/2}$} &   \colorbox{blue!10}{$4 (2I_x)^ {1/2} (2I_y)^ {2}$} \\
6 & 1032 &   skew &    (1, 1) &  (0, -1) &  (-1, 0) &              \colorbox{red!10}{$2  (2I_y)^ {5/2}$} &   \colorbox{blue!10}{$6 (2I_x)^ {1/2} (2I_y)^ {2}$} \\
6 & 1041 &   skew &    (1, 3) &  (0, -3) & (-1, -2) &              \colorbox{red!10}{$2  (2I_y)^ {5/2}$} &   \colorbox{blue!10}{$8 (2I_x)^ {1/2} (2I_y)^ {2}$} \\
6 & 1050 &   skew &    (1, 5) &  (0, -5) & (-1, -4) &              \colorbox{red!10}{$2  (2I_y)^ {5/2}$} &  \colorbox{blue!10}{$10 (2I_x)^ {1/2} (2I_y)^ {2}$} \\
6 & 1203 &   skew &  (-1, -3) &   (2, 3) &          &       \colorbox{blue!10}{$2 (2I_x) (2I_y)^ {3/2}$} &                                                   \\
6 & 1212 &   skew &  (-1, -1) &   (2, 1) &   (1, 2) &       \colorbox{blue!10}{$2 (2I_x) (2I_y)^ {3/2}$} &       \colorbox{blue!10}{$2 (2I_x)^ {3/2} (2I_y)$} \\
6 & 1221 &   skew &   (-1, 1) &  (2, -1) &   (1, 0) &       \colorbox{blue!10}{$2 (2I_x) (2I_y)^ {3/2}$} &       \colorbox{blue!10}{$4 (2I_x)^ {3/2} (2I_y)$} \\
6 & 1230 &   skew &   (-1, 3) &  (2, -3) &  (1, -2) &       \colorbox{blue!10}{$2 (2I_x) (2I_y)^ {3/2}$} &       \colorbox{blue!10}{$6 (2I_x)^ {3/2} (2I_y)$} \\
6 & 1401 &   skew &  (-3, -1) &   (4, 1) &          &   \colorbox{blue!10}{$2 (2I_x)^ {2} (2I_y)^ {1/2}$} &                                                   \\
6 & 1410 &   skew &   (-3, 1) &  (4, -1) &   (3, 0) &   \colorbox{blue!10}{$2 (2I_x)^ {2} (2I_y)^ {1/2}$} &           \colorbox{orange!10}{$2 (2I_x)^ {5/2} $} \\
6 & 2103 &   skew &   (1, -3) &   (0, 3) &          &       \colorbox{blue!10}{$4 (2I_x) (2I_y)^ {3/2}$} &                                                   \\
6 & 2112 &   skew &   (1, -1) &   (0, 1) &  (-1, 2) &       \colorbox{blue!10}{$4 (2I_x) (2I_y)^ {3/2}$} &       \colorbox{blue!10}{$2 (2I_x)^ {3/2} (2I_y)$} \\
6 & 2121 &   skew &    (1, 1) &  (0, -1) &  (-1, 0) &       \colorbox{blue!10}{$4 (2I_x) (2I_y)^ {3/2}$} &       \colorbox{blue!10}{$4 (2I_x)^ {3/2} (2I_y)$} \\
6 & 2130 &   skew &    (1, 3) &  (0, -3) & (-1, -2) &       \colorbox{blue!10}{$4 (2I_x) (2I_y)^ {3/2}$} &       \colorbox{blue!10}{$6 (2I_x)^ {3/2} (2I_y)$} \\
6 & 2301 &   skew &  (-1, -1) &   (2, 1) &          &   \colorbox{blue!10}{$4 (2I_x)^ {2} (2I_y)^ {1/2}$} &                                                   \\
6 & 2310 &   skew &   (-1, 1) &  (2, -1) &   (1, 0) &   \colorbox{blue!10}{$4 (2I_x)^ {2} (2I_y)^ {1/2}$} &           \colorbox{orange!10}{$2 (2I_x)^ {5/2} $} \\
6 & 3003 &   skew &   (3, -3) &  (-2, 3) &          &       \colorbox{blue!10}{$6 (2I_x) (2I_y)^ {3/2}$} &                                                   \\
6 & 3012 &   skew &   (3, -1) &  (-2, 1) &  (-3, 2) &       \colorbox{blue!10}{$6 (2I_x) (2I_y)^ {3/2}$} &       \colorbox{blue!10}{$2 (2I_x)^ {3/2} (2I_y)$} \\
6 & 3021 &   skew &    (3, 1) & (-2, -1) &  (-3, 0) &       \colorbox{blue!10}{$6 (2I_x) (2I_y)^ {3/2}$} &       \colorbox{blue!10}{$4 (2I_x)^ {3/2} (2I_y)$} \\
6 & 3030 &   skew &    (3, 3) & (-2, -3) & (-3, -2) &       \colorbox{blue!10}{$6 (2I_x) (2I_y)^ {3/2}$} &       \colorbox{blue!10}{$6 (2I_x)^ {3/2} (2I_y)$} \\
6 & 3201 &   skew &   (1, -1) &   (0, 1) &          &   \colorbox{blue!10}{$6 (2I_x)^ {2} (2I_y)^ {1/2}$} &                                                   \\
6 & 3210 &   skew &    (1, 1) &  (0, -1) &  (-1, 0) &   \colorbox{blue!10}{$6 (2I_x)^ {2} (2I_y)^ {1/2}$} &           \colorbox{orange!10}{$2 (2I_x)^ {5/2} $} \\
6 & 4101 &   skew &   (3, -1) &  (-2, 1) &          &   \colorbox{blue!10}{$8 (2I_x)^ {2} (2I_y)^ {1/2}$} &                                                   \\
6 & 4110 &   skew &    (3, 1) & (-2, -1) &  (-3, 0) &   \colorbox{blue!10}{$8 (2I_x)^ {2} (2I_y)^ {1/2}$} &           \colorbox{orange!10}{$2 (2I_x)^ {5/2} $} \\
6 & 5001 &   skew &   (5, -1) &  (-4, 1) &          &  \colorbox{blue!10}{$10 (2I_x)^ {2} (2I_y)^ {1/2}$} &                                                   \\
6 & 5010 &   skew &    (5, 1) & (-4, -1) &  (-5, 0) &  \colorbox{blue!10}{$10 (2I_x)^ {2} (2I_y)^ {1/2}$} &           \colorbox{orange!10}{$2 (2I_x)^ {5/2} $} \\
\bottomrule
\end{longtable}
}