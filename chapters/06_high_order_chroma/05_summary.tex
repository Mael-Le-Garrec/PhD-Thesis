%=============================
%        Conclusion
%=============================
\section{\review{Summary}}

This chapter explores the measurement and analysis of higher-order fields in the LHC at injection 
energy, with a focus on dodecapolar and decatetrapolar fields. Leveraging a newly implemented
collimation setup and a custom post-processing technique, these higher-order fields have been
successfully observed.

Chromaticity measurements were conducted with varying momentum offsets, revealing fourth and
fifth-order terms $Q^{(4)}$ and $Q^{(5)}$. The repeated measurements consistently identified these
higher-order terms, demonstrating their robustness. Measuring across such extended ranges and
including higher orders is essential for accurately characterizing the lower orders. The analysis
also shows that the primary contributors to these terms are dodecapolar and decatetrapolar fields,
which originate from fields errors of the main dipoles.

For the first time, the dodecapolar Resonance Driving Term $f_{0060}$ was measured. This
measurement, repeated for both beams at different configurations, shows a good agreement with the
model.

It is then concluded that further investigations could be conducted to address limitations in the
measurement range of the chromaticity function and to refine estimates of higher-order chromaticity
terms. Dodecapolar RDT studies could benefit from lifetime measurements accompanied by trims of
the relevant correctors situated in the Interaction Regions (IRs). Additionally, investigating the 
impact of lower-order multipoles on the RDT would be valuable.
Overall, gaining a thorough understanding of the higher-order field errors is essential for
optimizing the LHC's performance.