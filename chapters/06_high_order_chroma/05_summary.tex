%=============================
%        Conclusion
%=============================
\section{\review{Summary}}


% Intro and Improvements
Building on previous efforts at the LHC to understand non-linear multipoles like
sextupoles and octupoles, this chapter extends the analysis to higher-order fields, particularly
dodecapolar and decatetrapolar components at injection energy. With the development of improved
measurement techniques and analysis procedures, alongside enhancements in dynamic aperture via skew
octupolar and decapolar corrections, these higher-order fields have been successfully observed for
the first time. The use of a newly implemented collimation setup has played a crucial role in this,
representing a significant advancement in the study of non-linear effects in the LHC.

% RDT
Leveraging large kick amplitudes, the dodecapolar Resonance Driving Term $f_{0060}$ was measured, at
injection energy, for the first time at the LHC. This measurement, performed across various
non-linear corrector configurations, shows strong agreement with the magnetic model. The primary
source of this RDT is expected to be the dodecapolar ($b_6$) field errors in the main dipoles,
for which the decay has been modeled.

% Chromaticity
Chromaticity measurements over a wide range of momentum offsets, facilitated by the new collimation
setup, revealed fourth and fifth order terms ($Q^{(4)}$ and $Q^{(5)}$). Repeated measurements
under different LHC configurations consistently identified these higher-order terms, underscoring
the robustness of their identification. The long-term effort to measure these terms, spanning
several years, has been crucial in refining the understanding of high-order fields and
benchmarking the magnetic model. Discrepancies observed between measurements and model predictions
may stem from unmodeled systematic errors or contributions from lower-order multipoles.

% Further Machines
The advancements in these measurement techniques not only enabled breakthroughs in RDT studies but
also opened up the exploration of previously inaccessible higher-order non-linear regimes. These
findings provide new pathways for understanding non-linear beam dynamics in the LHC. Improved
comprehension of non-linear errors and their measurement will be crucial for ensuring the optimal
performance of future LHC upgrades and next-generation accelerators, where non-linear effects are
expected to have a substantial impact on dynamic aperture.