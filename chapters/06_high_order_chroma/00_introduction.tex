%====================
% 	Introduction
%====================
\section{\review{Introduction}}

The preceding chapters of this thesis presented significant advancements in the understanding and
modeling of skew octupoles and normal decapoles, along with the consequent improvements in
dynamic aperture through their correction. Non-linear optics measurements have been conducted in the
LHC since its initial Run~\cite{maclean_non-linear_2011, maclean_commissioning_2016-1}, though only
recently has attention been focused on higher-order multipoles, such as
dodecapoles~\cite{dilly_joschua_corrections_nodate}. These indirect measurements, which utilize
feed-down to amplitude detuning, are limited by their sensitivity to local errors and are applicable
only in the interaction regions.

Given that field errors of this nature are anticipated to have a significant impact on beam dynamics
and dynamic aperture in future accelerators such as the FCC, it becomes essential to develop methods
for their study. This will not only allow benchmarking but also enhance the global magnetic
model of the LHC.

Recent attempts to measure high-order multipoles have been made possible by the improvements in
dynamic aperture mentioned earlier, which allowed for larger oscillation amplitudes. Additionally,
the introduction of a new collimator sequence, developed by collimation experts, enabled
measurements at unprecedented amplitude levels, thus allowing for higher-order multipoles to be
investigated.

For the first time in the LHC, direct measurements of dodecapolar Resonance Driving Terms (RDTs)
were successfully performed at injection energy, utilizing high-amplitude kicks provided by the
AC-Dipole. Furthermore, fourth and fifth order chromaticities, which are related to dodecapolar and
decatetrapolar errors, were also measured for the first time at injection energy. 
These direct measurements of novel observables related to very-high-order multipoles open up new
avenues for improving the understanding of non-linear errors in the LHC, as well as
for refining the techniques involved in their measurement and analysis, that could benefit future
machines.


%Those measurements, made by varying the RF frequency while observing the
%resulting tune change, have been performed with a momentum offset of up to $\delta = \pm 2.2 \times
%10^{-3}$, which led to the observation of the third order term of the non-linear chromaticity.
%
%During the commissioning of Run~3 in 2022, a new collimator sequence has been introduced, allowing wider
%momentum offset measurements, within $\delta \in [-3.2\times 10^{-3},3.7 \times 10^{-3}]$. This
%improved setup led to the observation of the fourth and fifth order terms at injection energy.
%Those terms, denoted $Q^{(4)}$ and $Q^{(5)}$ respectively in
%\cref{eq:very_high_orders:chromaticity_high_orders}, are produced to first order by dodecapoles and
%decatetrapoles. Dodecapoles being powered off at injection and decatetrapoles being absent from the
%lattice, those fields originate from the field errors of the various magnets installed in the LHC.

%\begin{equation}
%  \begin{aligned}
%    Q(\delta) = Q_0 + Q'\delta &+ \frac{1}{2!}Q''\delta^2 + \frac{1}{3!}Q'''\delta^3
%                                + \frac{1}{4!}Q^{(4)}\delta^4  + \frac{1}{5!}Q^{(5)}\delta^5
%                                + \mathcal{O}(\delta^6).
%  \end{aligned}
%  \label{eq:very_high_orders:chromaticity_high_orders}
%\end{equation}
%
%
%In addition to completing the measurements of high-order fields through chromaticity scans,
%turn-by-turn measurements were also conducted. High amplitude kicks indeed made it possible to
%observe dodecapolar RDTs in the LHC for the first time. Such fields were never before observed
%directly, but rather only via feed-down to amplitude detuning~\cite{dilly_corrections_2022}.
%