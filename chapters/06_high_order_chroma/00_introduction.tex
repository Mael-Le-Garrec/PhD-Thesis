%====================
% 	Introduction
%====================
\section{\review{Introduction}}

Beam-based high order field measurements have been carried out in the LHC since its first
Run~\cite{maclean_non-linear_2011, maclean_commissioning_2016-1}, via
chromaticity studies. Those measurements, made by varying the RF frequency while observing the
resulting tune change, have been performed with a momentum offset of up to $\delta = \pm 2.2 \times
10^{-3}$, which led to the observation of the third order term of the non-linear chromaticity.

During the commissioning of Run~3 in 2022, a new collimator sequence has been introduced, allowing wider
momentum offset measurements, within $\delta \in [-3.2\times 10^{-3},3.7 \times 10^{-3}]$. This
improved setup led to the observation of the fourth and fifth order terms at injection energy.
Those terms, denoted $Q^{(4)}$ and $Q^{(5)}$ respectively in
\cref{eq:very_high_orders:chromaticity_high_orders}, are produced to first order by dodecapoles and
decatetrapoles. Dodecapoles being powered off at injection and decatetrapoles being absent from the
lattice, those fields originate from the field errors of the various magnets installed in the LHC.

\begin{equation}
  \begin{aligned}
    Q(\delta) = Q_0 + Q'\delta &+ \frac{1}{2!}Q''\delta^2 + \frac{1}{3!}Q'''\delta^3
                                + \frac{1}{4!}Q^{(4)}\delta^4  + \frac{1}{5!}Q^{(5)}\delta^5
                                + \mathcal{O}(\delta^6).
  \end{aligned}
  \label{eq:very_high_orders:chromaticity_high_orders}
\end{equation}


In addition to completing the measurements of high-order fields through chromaticity scans,
turn-by-turn measurements were also conducted. High amplitude kicks indeed made it possible to
observe dodecapolar RDTs in the LHC for the first time. Such fields were never before observed
directly, but rather only via feed-down to amplitude detuning~\cite{dilly_corrections_2022}.
