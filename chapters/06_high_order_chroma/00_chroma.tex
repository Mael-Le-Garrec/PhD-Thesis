\chapter{Very High Order Field Measurement in the LHC}
\thumbforchapter{}
\chaptertoc{}
\newpage

\section{First Measurement of Fourth and Fifth Order Chromaticity}

\begin{enumerate}
\color{red}
    \item IPAC paper basically
\end{enumerate}


\section{INTRODUCTION}

Non-Linear chromaticity measurements at injection have been performed since Run~1~\cite{maclean:ipac11-wepc078,maclean:ipac16-thpmr039,maclean_commissioning_2016,maclean_measurement_2014-1}. Those measurements, made by varying the RF frequency while observing the resulting tune change, have been
carried out with a momentum offset of up to $\delta = \pm 2.2 \times 10^{-3}$, which led to the
observation of the third order term of the non-linear chromaticity.

During the commissioning of Run~3, a new collimator sequence was introduced, allowing wider
momentum offset measurements, within $\delta \in [-3.2\times 10^{-3},3.7 \times 10^{-3}]$.
This improved setup led to the observation of the fourth and fifth order
terms at injection energy, denoted $Q^{(4)}$ and $Q^{(5)}$ respectively, produced to first order by dodecapoles and tetradecapoles (see section \nameref{sec:nl_chroma_model}):
\begin{equation}
\begin{aligned}
Q(\delta) = Q_0 + Q'\delta &+ \frac{1}{2!}Q''\delta^2 + \frac{1}{3!}Q'''\delta^3 \\
                           &+ \frac{1}{4!}Q^{(4)}\delta^4  + \frac{1}{5!}Q^{(5)}\delta^5 + \mathcal{O}(\delta^6).
\end{aligned}
\end{equation}

The momentum offset $\delta$ is  related to the RF frequency and the momentum compaction factor:
$$
\delta = -\frac{1}{\alpha_c} \frac{\Delta f_{RF}}{f_{RF,nominal}}.
$$
The model $\alpha_c$ used is $3.48 \times 10^{-4}$ for beam 1 and $3.47 \times 10^{-4}$ for beam 2.
Via this relation, a change of 140Hz of the RF frequency corresponds to a momentum offset of about $-0.001$.

Results of $Q^{(4)}$ and $Q^{(5)}$ measurements are presented, along with a comparison to the model.

%========================================================================================
%	NON-LINEAR MEASUREMENTS
%========================================================================================

\section{NL-CHROMATICITY MEASUREMENTS}

Two chromaticity measurements were performed with different settings. The first one used the 
nominal correction strengths for octupole and decapole corrector magnets, derived from magnetic measurements, where the second one used beam-based corrections for the same elements, computed from measurements. 
Those two measurements have a respective momentum-offset range of $[-3.1 \times 10^{-3}, 3.1 \times 10^{-3}]$ and $[-3.2 \times 10^{-3}, 3.7 \times 10^{-3}]$.

%----------------------------------------
%     Nominal Corrections
%----------------------------------------
\subsection{Nominal Corrections}

A first chromaticity measurement was performed during the LHC beam commissioning in April 2022. The
horizontal and vertical tunes were set to 0.28 and 0.31. $Q'$ was reduced to a value of $2$ to allow
for a better identification of the higher order terms. The standard measurement procedure was then applied,    
by varying the RF frequency to induce a change in momentum offset. Frequency steps of 20Hz were taken roughly every
30 seconds, to allow for a precise tune measurement. Once beam losses, registered by the beam
loss monitors (BLM), are deemed too high the frequency is reverted back to its nominal frequency in
larger steps.
Figure \ref{rf_scan} shows a typical RF scan performed to measure chromaticity.

\begin{figure}[tbh]
    \centering
    \includegraphics[width=\columnwidth]{images/MOPL027_f1-1.pdf}
    \caption{Observation of the tune dependence on momentum offset, created by a shift of RF frequency.}
    \label{rf_scan}
\end{figure}

At very high momentum-offsets, the Base-Band Tune system (BBQ)~\cite{gasior_principle_2005, boccardi_first_2009}
was found not to give reliable tune measurements. A new approach using custom post-processing of the raw BBQ 
turn-by-turn data was therefore developed, giving more precise tune measurements by performing
spectral analysis with an increased number of turns to improve the signal to noise ratio. Further cleaning is
then applied by removing outliers and identified noise lines.

The octupole and decapole correctors were set to their nominal settings.
Results of this initial measurement are shown in Tab. \ref{chroma_fidel}. Lower order chromaticities such as
$Q'$ and $Q''$ are consistent with previous measurements~\cite{maclean_commissioning_2016}.

\begin{table}[tbh]
    \centering
    \small
    \setlength{\tabcolsep}{4.2pt}
    \begin{tabular}{|l||r|r|r|r|}
    \hline
                  & $Q^{(2)} [10^3]$ & $Q^{(3)} [10^6]$ & $Q^{(4)} [10^9]$ & $Q^{(5)} [10^{12}]$ \\ \hline\hline
        B1        &              &               &              & \\
        X         & -2.44 ± 0.02 & -3.36 ± 0.04 & -0.56 ± 0.02  &  1.20 ± 0.07 \\
        Y         &  0.97 ± 0.02 &  1.62 ± 0.05  &  0.15 ± 0.03 & -0.88 ± 0.09 \\ \hline
        B2        &              &               &              & \\
        X         & -2.45 ± 0.03 & -2.72 ± 0.08 & -1.00 ± 0.05  &  0.15 ± 0.14 \\
        Y         &  0.79 ± 0.03 & 1.54 ± 0.06  &  0.24 ± 0.04  & -0.74 ± 0.13 \\ \hline
    \end{tabular}
    \caption{Terms of the high order chromaticity obtained during Run~3 commissioning in April 2022, with nominal corrections.}
    \label{chroma_fidel}
\end{table}

Due to the momentum offset being zero several times during the measurement, it was possible to determine that
the tune drift is negligible. The measurement was also performed after an extended period at injection
energy, where the $b_3$ decay is small and not causing any change in the first order chromaticity.
The fitted curve for the chromaticity function is shown in Fig. \ref{chroma_before_correction}.
It can be seen that a higher order polynomial is beneficial for the fit, as discussed further in~"\nameref{subsection:q4q5_quality}".


\begin{figure}[tbh]
    \centering
    \includegraphics[width=1\columnwidth]{images/MOPL027_f2-1.pdf}
    \caption{Beam 1 measurement of higher order chromaticity terms with nominal corrections used during operation. Fits are up to the third and fifth order.}
    \label[type]{chroma_before_correction}
\end{figure}


%----------------------------------------
%     Beam-Based Corrections
%----------------------------------------
\subsection{Beam-Based Corrections}

After correcting the second and third order chromaticities via the octupole and decapole correctors, a
second measurement was performed.
A uniform trim on all the correctors of each class was applied for each beam, resulting in a global correction. A total of four circuits were unavailable for the octupoles, three for beam 1 and one for beam 2, resulting in larger corrections for beam 1.
Corrections applied on top of the nominal settings~\cite{maclean_commissioning_2016} for the octupoles and decapoles are shown in Tab. \ref{mcdo_values_corr}.

\begin{table}[tbh]
    \centering
    \begin{tabular}{|l||r|r|}
    \hline
      Beam  &    $K_4 [\mathrm{m}^{-4}]$      &  $K_5 [\mathrm{m}^{-5}]$  \\ \hline\hline
        1   &  +3.2973     &  +1610   \\ \hline
        2   &   +2.1716    &  +1618   \\ \hline
    \end{tabular}
    \caption{Corrections applied on top of the nominal octupole and decapole correctors strengths.}
    \label{mcdo_values_corr}
\end{table}

Figure \ref{chroma_after_correction} shows the chromaticity fit after the beam-based minimization of $Q''$ and $Q'''$,
while Tab.~\ref{chroma_table_after} shows the measured chromaticity.

Previous studies of chromaticity in the LHC only considered fits up to third-order.
Including fits up to a fifth order increases the $Q'''$ estimate of both measurements, while improving the fit quality. $Q'''$ for beam 1 with only a fit to the third order would have a value of $-0.38 \times 10^6$
instead of the $-1.02 \times 10^6$ obtained with a fifth order fit. 
Accurately measuring the third order chromaticity is essential in order to correct it.

\begin{figure}[tbh]
    \centering
    \includegraphics[width=1\columnwidth]{images/MOPL027_f3-1.pdf}
    \caption{Beam 1 measurement of high order chromaticity terms after application of $Q''$ and $Q'''$ 
             beam-based corrections on octupole and decapole correctors.}
    \label[type]{chroma_after_correction}
\end{figure}

\begin{table}[tbh]
    \centering
    \small
    \setlength{\tabcolsep}{4.2pt}
    \begin{tabular}{|l||r|r|r|r|}
    \hline
                 & $Q^{(2)} [10^3]$ & $Q^{(3)} [10^6]$ & $Q^{(4)} [10^9]$ & $Q^{(5)} [10^{12}]$ \\ \hline\hline
        B1    &           &          &              &              \\
        X         & -0.62 ± 0.01     & -1.02 ± 0.03 & -0.63 ± 0.02 &  1.22 ± 0.05 \\
        Y         & -0.24 ± 0.01     & 0.12 ± 0.02 &  0.04 ± 0.02 & -0.56 ± 0.04 \\ \hline
        B2    &           &          &              &              \\
        X         & -0.85 ± 0.01     & -0.64 ± 0.03 & -0.58 ± 0.02 &  1.07 ± 0.06 \\
        Y         & -0.30 ± 0.02     & 0.14 ± 0.03 &  0.16 ± 0.02 & -0.66 ± 0.05 \\ \hline
    \end{tabular}
    \caption{Terms of higher order chromaticity obtained during Run~3 commissioning in April 2022, with beam-based corrections for $Q''$ and $Q'''$.}
    \label{chroma_table_after}
\end{table}


%----------------------------------------
%     Q4 and Q5 confirmation
%----------------------------------------
\subsection{\texorpdfstring{$Q^{(4)}$ and $Q^{(5)}$}{Q4 and Q5} fit quality}
\label{subsection:q4q5_quality}

The values measured for $Q^{(4)}$ and $Q^{(5)}$ are similar across the two measurements, with nominal and beam-based corrections performed with very different lower order chromaticity and several hours apart.
This reproducibility gives confidence that the measured values are robust.
It is to be noted that one exception exists, for the horizontal plane of beam 2, where the measurement with nominal correction settings showed a high correlation between the fourth and fifth order terms, making the fit less reliable.

The reduced chi-square for the last measurement for each fit order is detailed in Tab.~\ref{table_chisquare}, where it can be seen that a fit above fifth order does not improve the fit quality.
%Figure \ref{chroma_comparison} shows a comparison of the measured chromaticity before and after
%beam-based corrections for the horizontal axis of beam 1.

\begin{table}[tbh]
    \centering
    \begin{tabular}{|l||c|c|c|c|}
    \hline
        Plane     &  $\chi^2_\nu$ $Q^{(3)}$ & $\chi^2_\nu$ $Q^{(4)}$ &  $\chi^2_\nu$ $Q^{(5)}$ &  $\chi^2_\nu$ $Q^{(6)}$  \\ \hline\hline
        Beam 1    &   &   &   & \\
        % The commented out lines are from measurement with the regular BBQ data
        % The un-commented lines are from using the raw BBQ data
        %X         & 7.62  & 4.07 & 0.62 &\\             % regular
        %Y         & 0.72  & 0.57 & 0.15 &\\ \hline      % regular
        X         & 17.9  & 12.1 & 1.8 & 1.47 \\               % raw bbq
        Y         &  3.0  & 2.2  & 0.7 & 0.7 \\ \hline        % raw bbq
        Beam 2    &    &    &   &\\
        %X         & 7.60  & 3.10 & 0.73 &\\             % regular
        %Y         & 0.48  & 0.46 & 0.16 &\\ \hline      % regular
        X         & 17.3 & 7.1 & 1.8 & 1.76 \\             % raw bbq
        Y         & 2.9  & 2.8 & 1.0 & 1.0 \\ \hline      % raw bbq
    \end{tabular}
    \caption{Reduced $\chi^2_\nu$ values for each order of fit, taken from the last commissioning measurement.}
    \label{table_chisquare}
\end{table}


%\begin{figure}[!ht]
%    \centering
%    \includegraphics[width=1\columnwidth]{images/comparison_b1x.png}
%    \caption{Comparison of the chromaticity function with nominal and beam-based corrections.}
%    \label[type]{chroma_comparison}
%\end{figure}


%----------------------------------------------------------------------------------------
%	    MODEL EXPECTATIONS
%----------------------------------------------------------------------------------------

\section{NL-CHROMATICITY MODEL}
\label{sec:nl_chroma_model}

The model of the LHC is based on MADX and WISE field errors~\cite{p_hagen_wise_2006}. To compute the
chromaticity, simulations are run via the Polymorphic Tracking Code (PTC), with field errors from
sextupole to hexadecapole loaded and applied on all magnets. Simulation results are shown in Tab.~\ref{ptc_values}.

Table \ref{ptc_values_ratios} shows the ratio between measured and simulated high-order chromaticity. The measured $Q^{(5)}$ shows a consistent discrepancy with the model, larger by about a factor 2.

\begin{table}[tbh]
    \centering
    \small
    \begin{tabular}{|l||r|r|}
    \hline
        Plane     &  $Q^{(4)} [10^9]$  &  $Q^{(5)} 
        [10^{12}]$ \\\hline\hline
        Beam 1    &              &               \\
        X         & -0.2 ± 0.1 & 0.7 ± 0.1  \\
        Y         &  0.1 ± 0.1 & -0.3 ± 0.1  \\ \hline
        Beam 2    &  &   \\
        X         & -0.2 ± 0.1 &  0.8 ± 0.1  \\
        Y         &  0.1 ± 0.1 & -0.4 ± 0.1 \\ \hline
    \end{tabular}
    \caption{Simulated high order chromaticity terms via PTC, including field errors from $b_3$ to $b_8$ with the previous beam-based corrections.}
    \label{ptc_values}
\end{table}

%\begin{table}[tbh]
%    \centering
%    \footnotesize
%    \begin{tabular}{|l||c|c|c|c|}
%    \hline
%        Plane     &  \multicolumn{2}{c|}{$Q^{(4)}$ ratio}   &  \multicolumn{2}{c|}{$Q^{(5)}$ ratio} \\
%        \cline{2-5}
%        Measurement &   first    &    second   &    first   &    second\\ \hline\hline
%        Beam 1    &              &             &            & \\
%        X         &  3.2 ± 0.3   & 3.6 ± 0.3  & 1.8 ± 0.1  & 1.8 ± 0.1  \\
%        Y         &  7.9 ± 2.0   & 2.1 ± 1.1   & 2.7 ± 0.3  & 1.7 ± 0.1  \\ \hline
%        Beam 2    &              &             &            & \\ 
%        X         &              & 4.3 ± 0.4   &            & 1.6 ± 0.1  \\
%        Y         &  18 ± 5      & 12 ± 3      & 2.2 ± 0.4  & 1.9 ± 0.2 \\ \hline
%    \end{tabular}
%    \caption{Ratios of the simulated and measured high order chromaticity terms for both first and second measurements.  The values are taken from tables \ref{chroma_fidel}, \ref{chroma_table_after} and \ref{ptc_values}. The fit with high correlation was not included.}
%    \label{ptc_values_ratios}
%\end{table}
\begin{table}[tbh]
    \centering
    \begin{tabular}{|l||c|c|}
    \hline
        Plane     &  \multicolumn{2}{c|}{$Q^{(5)}$ ratio} \\
        \cline{2-3}
        Measurement &    first   &    second\\ \hline\hline
        Beam 1    &            & \\
        X         &1.8 ± 0.1  & 1.8 ± 0.1  \\
        Y         & 2.7 ± 0.3  & 1.7 ± 0.1  \\ \hline
        Beam 2    &            & \\ 
        X         &            & 1.6 ± 0.1  \\
        Y         & 2.2 ± 0.4  & 1.9 ± 0.2 \\ \hline
    \end{tabular}
    \caption{Ratios of the measured to simulated fifth order chromaticity term for both first and second measurements.  The values are taken from tables~\ref{chroma_fidel}, \ref{chroma_table_after} and \ref{ptc_values}. The fit with high correlation was not included.}
    \label{ptc_values_ratios}
\end{table}


Simulations with only $b_6$ and $b_7$ field errors have been run to assess the contribution of lower order magnets to the fifth order chromaticities. The results strongly imply that the tetradecapole errors are the main contributors to $Q^{(5)}$, as can be seen in Fig.~\ref{beam1_q5x_ptc}.
Fringe fields and skew multipoles have been found to have a negligible impact.
Ongoing studies are assessing the contribution of $\beta$-beating, linear coupling and alignment errors to those estimates.

\begin{figure}[tbh]
    \centering
    \includegraphics[width=1\columnwidth]{images/MOPL027_f4-1.pdf}
    \caption{Measured and simulated fifth order chromaticity. 
             The simulations are done via PTC and include different multipole errors, some of them further
             include the nominal corrections for $b_3$, $b_4$ and $b_5$.
             The $b_2$ errors, applied on dipoles and quadrupoles, generate beta-beating.
             The measurement with a high correlation is not included.}
    \label{beam1_q5x_ptc}
\end{figure}

%----------------------------------------------------------------------------------------
%	CONCLUSION
%----------------------------------------------------------------------------------------

\section{CONCLUSIONS AND OUTLOOK}

A wider momentum offset range, combined with new analysis techniques permitted the observation of fourth and fifth order chromaticity for the first time in the LHC. Reproducible values were measured with different machine configurations.
Preliminary simulations show that the observed values do not match well with the LHC non-linear model. A factor 2 is observed between beams and planes for $Q^{(5)}$, which may point to a systematic error in the b7 error model.

Correction of the measured higher order chromaticity terms is not possible, due to the lack of adequate
correctors in the LHC. It is nevertheless interesting to characterize the higher order errors for an effective model and understand the effect a higher order fit has on lower order terms.
Precise measurement of those lower chromaticity terms is required in order to effectively correct them. 
Higher order terms have thus to be taken into account.

The current range of momentum offset is deemed sufficient to measure higher order chromaticity. Attempts will, however,
be taken to increase that range and assess if such a wider range can refine the estimate of $Q^{(4)}$ and
$Q^{(5)}$.


\section{First Measurement of Dodecapole RDTs}