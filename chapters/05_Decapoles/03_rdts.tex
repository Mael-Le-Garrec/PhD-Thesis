% === RDTs
\section{\review{Resonance Driving Terms}}

Decapoles, due to their order, contribute to many RDTs. Indeed, 25 of them can be theoretically 
observed in simulations and measurements. In practice, the contributions of individual multipoles
become indistinguishable as many resonances or lines overlap, making it impossible to isolate
certain terms. Some resonances, described in~\cref{appendix:rdts}, are unique to certain multipoles
when considering not too high orders. Those resonances, provided that they are sufficiently strong
and the beam close to them, can be measured via their RDTs.

\begin{figure}[!htb]
    \centering
    \includegraphics[width=0.8\textwidth]{./images/tune_diagram_f1004.pdf}
    \caption{Frequency map at injection energy, with decapolar field errors and nominal settings for
    landau octupoles. The highlighted resonance (1,-4), excited by decapoles, shows a degradation
    over 20,000 turns. The tune shift $\Delta Q$ between the start and the end of the simulation is
    indicated in color.}
    \label{fig:decapoles:rdts:tune_diagram}
\end{figure}

Of particular interest to LHC operation is the Resonance Driving Term $f_{1004}$, which drives
the resonance $1Q_x - 4Q_y$. This RDT appears in the horizontal frequency spectrum at $4Q_y$ with
an amplitude dependence on $J_y^2$. 
\Cref{fig:decapoles:rdts:tune_diagram} shows a frequency
map~\cite{yannis_papaphilippou_detecting_2014} from a simulation that includes decapolar field
errors, where their impact on the beam dynamics is clearly visible. Multiple particles were tracked
under varying conditions within the bunch distribution. The color map represents tune diffusion over
the course of the simulation, with blue indicating small diffusion and stable particle trajectories,
while red represents large diffusion, suggesting unstable or chaotic trajectories that are unlikely
to survive in a real machine.

Measurements were taken for the first time in the LHC to observe the $f_{1004}$ RDT
at injection energy. The frequency line of the resonance $1Q_x - 4Q_y$ is seen at $4Q_y$ in the
horizontal spectrum, as shows \cref{fig:decapoles:rdts:spectrum_f1004}.

\begin{figure}[!htb]
    \centering
    \includegraphics[width=0.9\textwidth]{./images/f1004x_spectrum.pdf}
    \caption{Horizontal frequency spectrum of turn-by-turn data, with nominal and beam-based
    corrections for the third order chromaticity $Q'''$. The $1Q_x - 4Q_y$ resonance can be seen
    at $4Q_y$ with different amplitudes for each correction scheme.}
    \label{fig:decapoles:rdts:spectrum_f1004}
\end{figure}

%Moreover, \cref{fig:decapoles:rdts:spectrum_f1004} shows that the amplitude of this resonance line
%decreases upon application of beam-based corrections for $Q'''$. This translates to the amplitude
%of the RDT $f_{1004}$, as seen in \cref{fig:decapoles:rdts:f1004_dq3}.
%
%\begin{figure}[!htb]
%    \centering
%    \includegraphics[width=0.9\textwidth]{./images/f1004_dq3.pdf}
%    \caption{Amplitude of the RDT $f_{1004}$ generated by normal decapoles, measured before and
%    after having applied beam-based corrections of the third order chromaticity $Q'''$.}
%    \label{fig:decapoles:rdts:f1004_dq3}
%\end{figure}


% ---------------------------------------
%         Decapole Contribution
% ---------------------------------------
\subsection{\review{Measurement and Correction}}
\label{section:decapoles:decapolar_contribution_correction}

Decapolar fields are expected to be the main contributors to the RDT $f_{1004}$. As such, powering
the decapolar correctors is a good way to correct the related resonances.
Being linear with the strength of the correctors, the RDT can be corrected via a response matrix,
as previously detailed in \cref{correction_principle:response_matrix}.

Measurements were conducted to establish a baseline for the amplitude of the RDT without octupolar
or decapolar correctors and with the nominal FiDeL corrections. Such measurement is shown in
\cref{fig:decapoles:rdt:b1_fidel_vs_bare} for Beam 1, similar results are observed for Beam 2. It
can be there observed that the FiDeL corrections degrade the RDT.

\begin{figure}[!htb]
    \centering
    \includegraphics[width=0.9\textwidth]{./images/f1004/f1004x_corrections_B1_fidel_vs_bare.pdf}
    \caption{Amplitude of the decapolar RDT $f_{1004}$ measured with nominal FiDeL corrections for 
    $Q'''$ and without.}
    \label{fig:decapoles:rdt:b1_fidel_vs_bare}
\end{figure}


Corrections were then attempted based on the measurements obtained with the nominal FiDeL settings
and were subsequently applied on top of these measurements.
The strength of the decapolar correctors is shown in \cref{tab:decapoles:rdts:correction_f1004_k5}
for the FiDeL settings, the delta applied on top, and the final correctors values.


\begin{table}[!htb]
    \centering
    \begin{tabular}{lrrr}
    \toprule
    Circuit & FiDeL $K_5 [\textrm{m}^{-5}]$ & $\Delta K_5 [\textrm{m}^{-5}]$ & $K_5 [\textrm{m}^{-5}]$\\
    \midrule
    Beam 1 & \\
    \hspace{2mm}RCD.A12B1 &$-4582$ & $6055 $ &  $ 1473 $\\
    \hspace{2mm}RCD.A23B1 &$-5106$ & $7    $ &  $-5099 $\\
    \hspace{2mm}RCD.A34B1 &$-4855$ & $3827 $ &  $-1028 $\\
    \hspace{2mm}RCD.A45B1 &$-4577$ & $-4746$ &  $-9323 $\\
    \hspace{2mm}RCD.A56B1 &$-4125$ & $-4903$ &  $-9028 $\\
    \hspace{2mm}RCD.A67B1 &$-5166$ & $2961 $ &  $-2205 $\\
    \hspace{2mm}RCD.A78B1 &$-6827$ & $3593 $ &  $-3234 $\\
    \hspace{2mm}RCD.A81B1 &$-5500$ & $2380 $ &  $-3120 $\\
    \hspace{2mm}Total     &$-40738$& $9174 $ &  $-31564$\\
    Beam 2 & \\  % inverted the signs of the correction
    \hspace{2mm}RCD.A12B2 &$-4490$ & $3639 $ &  $-851  $\\
    \hspace{2mm}RCD.A23B2 &$-5155$ & $-1147$ &  $-6302 $\\
    \hspace{2mm}RCD.A34B2 &$-4825$ & $-1038$ &  $-5863 $\\
    \hspace{2mm}RCD.A45B2 &$-4619$ & $3986 $ &  $-633  $\\
    \hspace{2mm}RCD.A56B2 &$-4064$ & $2944 $ &  $-1120 $\\
    \hspace{2mm}RCD.A67B2 &$-5066$ & $2357 $ &  $-2709 $\\
    \hspace{2mm}RCD.A78B2 &$-6866$ & $-2952$ &  $-9818 $\\
    \hspace{2mm}RCD.A81B2 &$-5446$ & $1825 $ &  $-3621 $\\
    \hspace{2mm}Total     &$-40531$& $9614 $ &  $-30917$      \\
    \bottomrule
    \end{tabular}
    \caption{Strength of decapolar correctors with nominal FiDeL settings and after application of
    corrections aiming at reducing both the RDT $f_{1004}$ and the third order chromaticity $Q'''$. The total value has a direct
    incidence on $Q'''$.}
    \label{tab:decapoles:rdts:correction_f1004_k5}
\end{table}

This RDT correction also serves as a partial $Q'''$ correction. To fully correct $Q'''$ indeed
approximately requires a strength of $+13,000 \;K_5$ distributed amongst the correctors. Therefore,
this new approach reduces $Q'''$ by about $70\%$ compared to the previous method. The chromatic
amplitude detuning terms are also expected to be decreased.
Result of these measurements, as well as the inverse of the correction, are shown in
\cref{fig:decapoles:rdts:f1004_correction_B2}.

\begin{figure}[!htb]
    \centering
    \begin{subfigure}{0.48\textwidth}
        \includegraphics[width=1\textwidth]{./images/f1004/f1004x_corrections_B1.pdf}
        \caption{$|f_{1004}|$ for Beam 1}
    \end{subfigure}
    \hfill
    \begin{subfigure}{0.48\textwidth}
        \includegraphics[width=1\textwidth]{./images/f1004/f1004x_corrections_B2.pdf}
        \caption{$|f_{1004}|$ for Beam 2}
    \end{subfigure}
    \caption{Measured $f_{1004}$ with nominal settings, and
    combined RDT \& $Q'''$ correction with normal and opposite signs.}
    \label{fig:decapoles:rdts:f1004_correction_B2}
\end{figure}


Although the FiDeL scheme was not intended to correct the RDT but rather only from $Q'$ to $Q'''$,
it would be expected for it to lower the amplitude of the RDT. It can though be seen that it
degrades the resonance compared to the machine with no decapolar correctors. On the other hand, the
newly computed RDT correction does lower the amplitude of $f_{1004}$ as expected.  Its inverse has
the opposite effect.

Simulations were run with decapolar correctors turned off and with the RDT
correction. The response of the RDT between these two schemes is shown in
\cref{fig:decapoles:rdt:b1_response_corr}. The difference between their RMS value ratio is $\approx
6\%$, indicating that simulations correctly model the decapolar correctors.

\begin{figure}[!htb]
    \centering
    \includegraphics[width=0.9\textwidth]{./images/f1004/b1_response_rdt_corr.pdf}
    \caption{Comparison for measurement and simulation of the response of the imaginary part of
    $f_{1004}$ upon application on unpowered correctors of the RDT corrections.}
    \label{fig:decapoles:rdt:b1_response_corr}
\end{figure}


% ==== Correction
In order to understand what can be gained from correcting decapolar fields, a lifetime measurement
was taken with the corrections previously described in
\cref{section:decapoles:decapolar_contribution_correction}. This scheme corrects the three decapolar 
observables, being the RDT $f_{1004}$ linked to the resonance $1Q_x - 4Q_y$, the third order
chromaticity $Q'''$ and the chromatic amplitude detuning terms.
\Cref{fig:decapoles:impact:b5_lifetime_rdt_corr} shows the evolution of the lifetime, starting with
corrections applied, removed and then trimmed to their opposite. A net change in lifetime for Beam 1
can be measured after each application. Acquired signal for Beam 2 has been deemed too noisy to be 
relevant, due to the shortness of the measurement.

\begin{figure}[!htb]
    \centering
    \includegraphics[width=0.8\textwidth]{./images/b5_lifetime_rdt_corr.pdf}
    \caption{Measured lifetime of Beam 1 with the nominal corrections for $Q'''$, combined
    correction of $f_{1004}$ and $Q'''$, and its inverse.}
    \label{fig:decapoles:impact:b5_lifetime_rdt_corr}
\end{figure}

It is apparent here that the corrections have a beneficial effect on the beam. The lifetime
improvement is of $\approx 3 \%$, while the degradation after applying the opposite is of $\approx
5\%$. Further developments in the correction scheme and lengthier measurements could easily improve
this lifetime gain.






% ---------------------------------------
%        Lower Order Contribution
% ---------------------------------------
\subsection{\review{Feed-Up Contributions}}
\label{section:decapoles:feed_up}

% http://localhost:8888/lab/workspaces/auto-d/tree/work_afs2/jupyter/resonance_driving_terms/measurements/2024-03-13_b3_b4_effect_on_b5/Sextupoles_and_Octupoles.ipynb

% ------- Introduction
\subsubsection{\review{First Observation}}

As described in \cref{appendix:transfer_maps}, multipoles can combine to create perturbations that
are seen as higher orders when considering higher orders of the BCH expansion.
For decapolar RDTs, combinations of several sextupoles and sextupoles with octupoles give rise to
decapolar-like fields, as described in
\cref{table:appendix:transfer_maps:bch_resulting_orders_combination}. 
Such contributions were already observed in the LHC with sextupoles contributing to amplitude
detuning~\cite{soubelet_simulations_nodate}.

The effective Hamiltonian $h$ of the transfer map of two elements $h1$ and $h2$, $e^{:h1:} \cdot
e^{:h2:} = e^{:h:}$, can be expanded to an arbitrary order, which will then create perturbations
akin to certain multipoles, as reminded by the following,

\begin{equation}
    \begin{aligned}
        h &= h_1 + h_2 \quad &\Rightarrow \text{1\textsuperscript{st} order} \\
        &\quad + \frac{1}{2} [h_1, h_2] \quad &\Rightarrow \text{2\textsuperscript{nd} order} \\
        &\quad + \frac{1}{12} [h_1, [h_1, h_2]] - \frac{1}{12} [h_2, [h_1, h_2]] \quad &\Rightarrow \text{3\textsuperscript{rd} order} \\
        &\quad + \cdots,
    \end{aligned}
    \nonumber
\end{equation}

where sextupoles would be expected to contribute to decapolar RDTs at the third order, and a
combination of sextupoles and octupoles to the second.

In previous section, it was observed that the RDT response of decapolar correctors was properly
modelled. However, RDT measurements taken before and after corrections of the non-linear
chromaticity $Q''$ and $Q'''$ during the 2022's commissioning exhibited an unexpected RDT behavior.
As the non-linear chromaticity corrections were applied, it was expected that the RDT $f_{1004}$
would also lower with the reduction of the decapolar strengths $K_5$. However, an increase of the
RDT was observed, as shows \cref{fig:decapoles:f1004_dq2_dq3}.

\begin{figure}[!htb]
    \centering
    \includegraphics[width=0.9\textwidth]{./images/f1004_dq2_dq3_2022.pdf}    
    \caption{Non-intuitive increase of the RDT $f_{1004}$ after application of both the $Q''$ and
    $Q'''$ corrections.}
    \label{fig:decapoles:f1004_dq2_dq3}
\end{figure}


While decapolar errors are expected to be the main contributors to decapolar RDTs, other strong
sources can indeed be identified. \Cref{fig:decapoles:rdts:contributions} shows the average
amplitude of the RDT $f_{1004}$ depending on the error sources introduced in the simulations. 
Significant contributions to the Resonance Driving Terms (RDTs) appear to come from the sextupolar
and octupolar field errors present in the main dipoles. Interestingly, the octupolar errors induce
an RDT that is even larger than that generated by the decapolar errors. 
The following sections will provide a detailed analysis of the contributions from the sextupolar and
octupolar field components to this RDT, before presenting experimental measurements.

\begin{figure}[!htb]
    \centering
    \includegraphics[width=0.8\textwidth]{./images/f1004/f1004_several_factors.pdf}
    \caption{Simulation of the amplitude of the decapolar RDT $f_{1004}$ depending on the field
             errors applied on main dipoles as well as coupling ($C^-$). Sextupolar ($b_3$) and
             octupolar ($b_4$) fields have a clear impact on this amplitude.}
    \label{fig:decapoles:rdts:contributions}
\end{figure}


%% ------- Action dependance
%\subsubsection{\review{Action Dependance and Analysis}}
%
%Resonance lines in the frequency spectrum are often contributed to by several multipoles. Some lines
%start getting a contribution with rather high multipole orders, like the RDT $f_{1004}$ considered
%here. The line $4Q_y$ in the horizontal spectrum is indeed contributed to by decapoles and then only
%by decatetrapoles. When the main contributing field alone is varied, it is easy to reconstruct the
%RDT, as its fit is only dependant its action dependance ($\propto J_x^{*} J_y^{*}$). Several
%turn-by-turn measurements at the same configuration can be taken wit varying kick amplitudes,
%refining the RDT value with more data points for the fit.
%
%Considering the contribution of lower order multipoles is a bit trickier, as the second order RDTs
%change the dependance of the frequency line~\cite{franchi_first_2014}. In order to be able to
%compare the RDT from several turn by turn measurements, the same kick amplitude must then be used.
%Failing to do so would lead to a poor fit of the line amplitude relative to the action, resulting in
%an RDT with incorrect amplitude and significant noise.


% ------- Sextupoles ----------
\FloatBarrier
\subsubsection{\review{Combination of Sextupoles}}

At the third order of the BCH expansion, the combination of two sextupoles yields a decapolar-like
expression. This means that, during normal operation of the machine, decapolar observables will be
altered when adjusting parameters such as the linear chromaticity $Q'$. 
Derivation of such a combination can be found in \cref{appendix:transfer_map:two_sextupoles}. The
resulting Hamiltonian indeed is similar to the terms of a decapole, dropping the $p_{x,y}$ terms for
readability:

\begin{equation}
    \begin{aligned}
         (H_3)^3 &\propto \frac{1}{48} \left(x^5 - 2x^3y^2 - 3xy^4 \right)\\
                 &\sim    x^5 - 10x^3y^2 + 5xy^4.
    \end{aligned}
    \label{eq:decapoles:sextupoles_b5}
\end{equation}

To quantify the actual impact of such an equation on the LHC, a simulation was run with injection
optics while varying this same linear chromaticity $Q'$. No field components higher than
sextupoles, nor any additional field errors, have been introduced. The resulting effect on the RDT
$f_{1004}$ can be seen in \cref{fig:decapoles:rdts:simulated_f1004_from_sextupoles}.


\begin{figure}[H]
    \centering
    \includegraphics[width=0.7\textwidth]{./images/f1004/f1004_dq.pdf}
    \caption{Simulated change of the decapolar RDT $f_{1004}$ with varying linear
    chromaticity $Q'$ generated by sextupoles. The combination of sextupolar fields clearly shows an 
    increase in decapolar RDT.}
    \label{fig:decapoles:rdts:simulated_f1004_from_sextupoles}
\end{figure}

As the overall $K_3$ strength of sextupoles changes, so does the linear chromaticity. 
Considering the previous \cref{eq:decapoles:sextupoles_b5}, a higher chromaticity
is expected to increase, in that configuration of the LHC, the amplitude of the RDT $f_{1004}$,
related to the last term $xy^4$. \Cref{fig:decapoles:sextupoles_k3_f1004} shows how the RDT is
expected to vary, depending on the overall sextupoles strength and the linear chromaticity. It can
be noted that although the relation between $K_3$ and $Q'$ is linear, that of $K_3$ and the RDT
varies with the cubed strength. Using the sum of the cubed strength is possible due to the
chromaticity knob being a factor applied on all sextupoles at the same time.

\begin{figure}[!htb]
    \centering
    \includegraphics[width=0.9\textwidth]{./images/f1004/avg_f1004_k3.pdf}
    \caption{Average amplitude of the decapolar RDT $f_{1004}$ depending on the overall strength
    of the sextupoles used to control the linear chromaticity $Q'$. This two-plot combination is
    only valid for a certain LHC configuration.}
    \label{fig:decapoles:sextupoles_k3_f1004}
\end{figure}




% ------- Sextupole + Octupole ----------
\subsubsection{\review{Combination Sextupoles and Octupoles}}


At the second order of the BCH expansion, the combination of a sextupole and an octupole yields a
decapolar-like expression.
Like sextupoles, octupoles are used in operation, thus contributing to decapolar RDTs. This
happens, amongst other, when correcting the second order chromaticity $Q''$ and most importantly with
the Landau Octupoles, which are powered to high strengths at high strengths through the whole LHC
cycle to introduce Landau damping~\cite{gareyte_landau_1997}.
Derivation of such a combination can be found in
\cref{appendix:transfer_map:sextupole_and_octupole}. The resulting Hamiltonian indeed is similar to
the terms of a decapole, dropping the $p_{x,y}$ terms for readability:

\begin{equation}
    \begin{aligned}
         H_3 H_4 &\propto \frac{1}{24} \left(x^5 + 2x^3y^2 + xy^4 \right)\\
                   &\sim    x^5 - 10x^3y^2 + 5xy^4.
    \end{aligned}
    \label{eq:decapoles:sextupole_octupole_b5}
\end{equation}

In order to assess the previous equation, simulations were run with several configurations.  A set
of two configurations was run to check the impact of octupoles alone. The first configuration is run
with all sextupoles of the machine turned off, while octupoles are powered. The second configuration
turns off all sextupoles and octupoles. \Cref{fig:decapoles:rdts:sectupole_octupole_no_diff} shows
the resulting RDT $f_{1004}$ from these simulations. It is there apparent that varying octupoles
without sextupoles does not have any effect on this RDT.

\begin{figure}[!htb]
    \centering
    \includegraphics[width=0.8\textwidth]{./images/f1004/f1004_no_ms.pdf}
    \caption{Simulated decapolar RDT $f_{1004}$ with two different schemes. First scheme has
    lattice sextupoles turned off and octupoles turned on. Second scheme has all sextupoles of the
    lattice turned off and octupoles turned off as well. No difference is seen, as expected from
    the equations.}
    \label{fig:decapoles:rdts:sectupole_octupole_no_diff}
\end{figure}

The most powerful octupoles used in operation are the lattice octupoles, used for Landau damping.
\Cref{fig:decapoles:rdts:simulation_mo_powered} shows a simulation ran with varying strengths of
those magnets. It can be noted here that the shift of the RDT is almost of an order of magnitude,
making octupoles a large contributor to the decapolar fields.

\begin{figure}[!htb]
    \centering
    \includegraphics[width=0.8\textwidth]{./images/f1004/f1004_mo.pdf}
    \caption{Simulated change of the decapolar RDT $f_{1004}$ depending on the strength of the
    lattice octupoles used for Landau damping.}
    \label{fig:decapoles:rdts:simulation_mo_powered}
\end{figure}


%\begin{wraptable}{r}{0.4\textwidth}
\begin{table}[!htb]
    \centering
    \begin{tabular}{rr}
    \toprule
    Factor & RMS $|f_{1004}|$ \\
    \midrule
       -10 & $37,308,159$         \\ 
        -4 &  $6,721,270$          \\ 
         0 &  $3,533,796$          \\ 
        -1 &  $2,333,384$          \\
    \bottomrule
    \end{tabular}
    \caption{RMS of $|f_{1004}|$ depending on the factor of the $Q''$ corrections.}
    \label{table:decapoles:corrections_dq2_f1004_rms}
\end{table}

Large RDT shifts due to octupoles are relevant to the operation of the LHC, as
resonances can be greatly deteriorated, especially when powering landau octupoles.
A better understanding of the interaction between Landau octupoles and octupolar correctors could
lead to improved corrections in not only octupolar but also decapolar fields in the future.



% ------- Measurement ----------
%\FloatBarrier
\subsubsection{\review{Experimental Measurements}}

To confirm what is observed in simulations, measurements were performed by varying $Q'$ and kicking
the beam with the AC-Dipole. Limited by losses, up to three measurements with distinct $Q'$ were
taken, as shows \cref{fig:decapoles:rdts:measured_f1004_from_sextupoles}.

\begin{figure}[!htb]
    \centering
    \includegraphics[width=0.85\textwidth]{./images/f1004/f1004x_q2_q10_q15.pdf}
    \caption{Measured change of the decapolar RDT $f_{1004}$ depending on the desired linear
    chromaticity $Q'$ generated by sextupoles.}
    %It is to be noted that the vertical axis is one
    %order of magnitude higher than the previous simulations' plot.}
    \label{fig:decapoles:rdts:measured_f1004_from_sextupoles}
\end{figure}

Like in simulations, it is observed that an increase in $Q'$ translates to an increase in 
$|f_{1004}|$. The observed scale of the amplitude is though one order of magnitude higher than that
of simulations. An offset for all measurements could be explained by non-included field-errors. The
shift between them however should be similar between machine and simulations, this could be due by
the interaction of the sextupolar fields with octupoles, as detailed in the following section. 
%More data points with varying $Q'$ at similar kick amplitudes would be required to further
%investigate.

% ====

Measurements were performed to confirm and quantify the effect of octupoles coupled with sextupoles
on the decapolar fields. Previous corrections, aimed at correcting the second order chromaticity
$Q''$, via octupolar correctors \textit{MCO}, were applied with varying factors. Such corrections
use a uniform trim on all correctors of $\approx +2.5K_4$.
\Cref{decapoles:rdts:measured_f1004_mco} shows a comparison of the resulting RDT with those
corrections at factors $-10$, $-4$, $-1$ and $0$.

\begin{figure}[!htb]
    \centering
    \includegraphics[width=0.8\textwidth]{./images/f1004/f1004x_mco_corr.pdf}
    \caption{Shift of the decapolar RDT $f_{1004}$ depending on the factor applied on octupolar
    corrections for $Q''$.}
    \label{decapoles:rdts:measured_f1004_mco}
\end{figure}

  
\cref{table:decapoles:corrections_dq2_f1004_rms} shows the RMS of the amplitude of this RDT for the 
various configurations. Similar to the shift observed when powering the Landau octupoles in
simulations, the shift is of one order of magnitude between factors $-0$ and $-10$. Measurements
with Landau octupoles were also attempted, but losses made it impossible to obtain high enough
amplitudes to correctly measure the RDT.




%% ---------------------------------------
%%        Higher Order Contribution
%% ---------------------------------------
%\subsection{\review{Feed-Down Contributions}}
%
%% Measurements in 
%% /afs/cern.ch/work/m/mlegarr2/public/beta_beat_output/2024-05-21
%
%To produce collisions at top energy, \textit{crossing angles} are introduced via the orbit
%correctors located in the triplets, before the separation dipoles and the matching section of the
%interaction regions (\texttt{MCBX}, \texttt{MCBY} and \texttt{MCBC})~\cite{de_maria_lhc_2008}. Those
%collisions happen with a small $\beta*$, currently 30cm, requiring strong quadrupolar fields from
%the triplets.
%
%At such $\beta$, those triplets also generate strong dodecapolar field errors. Because of the
%crossing-angles, feed-down appears and lower-order fields can be observed.
%Such feed-down to decapolar fields was observed during the first commissioning of Run~3, in
%2022~\cite{maclean_prospects_2022}.
%\Cref{fig:decapoles:f1004_from_feeddown} shows how the RDT $f_{1004}$, normally affected by
%decapoles, varies with the application of crossing angles.
%
%\begin{figure}[!htb]
%    \centering
%    \includegraphics[width=0.9\textwidth]{./images/f1004x_feed-down_b6_triplets.pdf}
%    \caption{Varying amplitude of the decapolar RDT $f_{1004}$ at top energy depending on the
%    activation or not of the crossing angle at the IP. Offsets in orbit create feed-down from higher
%    orders.}
%    \label{fig:decapoles:f1004_from_feeddown}
%\end{figure}
%
%Such a contribution is though not expected at injection energy, as the triplets aren't powered a
%much as at top energy, $\beta*$ being set at around $10$m




% ===== Replicating Contribution
\subsection{\review{Replicating the Landau Octupoles Contribution}}

\begin{table}[!htb]
    \centering
    \begin{tabular}{rr}
        \toprule
        $\Delta K_5$         & RMS $|f_{1004}|$ \\
        \midrule
        $0$                  &            $618,947$ \\
        $\pm10500$             &         $17,566,377$ \\
        $\mp10500$             &         $17,623,867$ \\
        \bottomrule
    \end{tabular}
    \caption{RMS of $|f_{1004}|$ relative to the powering scheme of decapolar correctors.}
    \label{table:decapoles:impact:rdt_amplitude}
\end{table}

As seen previously in \cref{fig:decapoles:rdts:tune_diagram}, the resonance $1Q_x - 4Q_y$ passes
through the beam in tune space, deteriorating the lifetime of the nearby particles.
In order to measure the impact of this resonance on the beam, a knob was created, alternating the 
current of all decapole correctors in the machine arc by arc. Such a powering scheme has no impact
on chromaticity as the sum of the strengths $K_5$ is zero. Rather, the RDT $f_{1004}$ is impacted.
Is it to be noted that this is not a correction, but purely a way to artificially increase the RDT
in order to quantify the effect of the resonance.

Starting with nominal corrections for $Q'''$, a delta of $\pm 10500\;K_5$ is applied on
each decapolar correctors. \Cref{fig:decapoles:impact:alternating_knob} shows the response of the
real part of the RDT for this scheme and its inverse. The amplitude of the RDT is on a similar level
as the shift is significantly larger than the original level of the RDT.
\Cref{table:decapoles:impact:rdt_amplitude} indicates the amplitude of the RDT created with each
knob value.

\begin{figure}[!htb]
    \centering
    \includegraphics[width=0.8\textwidth]{./images/f1004/f1004x_knob_alt_lifetime_real.pdf}
    \caption{Measured real part of the RDT $f_{1004}$ depending on the powering scheme of the decapolar
    correctors.}
    \label{fig:decapoles:impact:alternating_knob}
\end{figure}

In order to measure the lifetime, a long time window is allocated to capture a clean signal, as that
returned from monitors can be jittery. The beam lifetime is expressed in hours and decreases over
time due to particles being lost. It is adjusted in real-time by the beam loss monitors.
\Cref{fig:decapoles:impact:b5_lifetime} shows this lifetime depending on the decapolar strength
scheme applied. The current of only one circuit is shown for readability. A current of $\approx 230$
A corresponds to a knob value of $+10500\;K_5$ while a current of $-45$ A corresponds to $0$.

\begin{figure}[!htb]
    \centering
    \includegraphics[width=0.8\textwidth]{./images/b5_lifetime.pdf}
    \caption{Measured lifetime of Beam 2 upon application of two different powering schemes for
    decapolar correctors. One trim keeps the RDT $f_{1004}$ at a low amplitude while the other greatly
    amplifies it.}
    \label{fig:decapoles:impact:b5_lifetime}
\end{figure}

It is clear from this measurement that a large RDT decreases the lifetime of the beam, more
particles being lost.
The first pair of trim sees the average lifetime decreasing of $0.31 \pm 0.03$ hours, while the
second one sees a decrease of $0.36 \pm 0.03$ hours. This observed decrease of 20 minutes accounts
for $10\%$ of the beam lifetime at injection energy.

