% ###################################
%      NON LINEAR CHROMATICITY
\section{Non-Linear Chromaticity}



% ===============================
%         Introduction
% ===============================
\subsection{Introduction}

% Expression
\paragraph{Expression}

Chromaticity is the tune shift $\Delta Q_{x,y}$ dependent on the momentum offset $\delta$ of a
particle. Its general expression is given by a Taylor expansion
in~\cref{eq:background_chromaticity}. The full third term is highlighted in the following,

\begin{equation} 
    Q (\delta) = Q_0 + Q' \delta + \frac{1}{2!} Q'' \delta^2 
                     + \colorbox{yellow!50}{$\displaystyle  \frac{1}{3!}  Q''' \delta^3$}
                     + \mathcal{O}(\delta^4).
\end{equation}

This third order is mainly contributed to by decapoles. It is related to the $\beta$-function, the
dispersion and the strength of the multipole, also shown
in~\cref{eq:detuning_effects:chromaticity_strength}:

\begin{equation}
    \begin{aligned}
        \Delta Q_x''' &=  &\frac{1}{4\pi} K_{5} L \beta_x D_x^{3}\\
        \Delta Q_y''' &= -&\frac{1}{4\pi} K_{5} L \beta_x D_x^{3}.
    \end{aligned}
\end{equation}


% Correction
\paragraph{Correction}

$Q'''$ is linear with the decapole strength. As such, it can be easily corrected via global trims
presented in~\cref{subsection:correction_chromaticity}.
A change of decapole strength $K_5 = 1000$ would for example have the following impact with the
injection optics used in 2022:

\begin{equation}
    \begin{aligned}
        \Delta Q_x =  1.5 \times 10^6 \quad;\quad
        \Delta Q_x = -0.9 \times 10^6.
    \end{aligned}
\end{equation}




% ===============================
%         Measurement
% ===============================
\subsection{Measurement}

Chromaticity is measured by varying the frequency of the radio-frequency (RF) cavities, used to
create buckets and to accelerate the beam. The change in $\delta$ related to this frequency is, as a
reminder:

\begin{equation}
    \delta = - \frac{1}{\alpha_c} \cdot \frac{\Delta f_{RF}}{f_{RF,nominal}}.
\end{equation}

More details on the measurements and analysis can be found in
earlier~\cref{subsection:optics_corrections_chromaticity}.

\paragraph{Momentum Compaction Factor} Rather than a constant, the momentum compaction factor is an 
expansion, as detailed in~\cref{subsection:coordinates_systems:momentum_compaction_factor}. It is
assumed here though to be constant as the induced difference in chromaticity is negligible.
\todo{Is it actually? Check}



\subsection{blabla}

Measurements were taken during 2022 Commissioning for 
\begin{itemize}
    \item Beam Test
    \item Commissioning
    \begin{itemize}
        \item FiDeL
        \item Q''' corr
        \item Q'' corr
    \end{itemize}
    \item 60° optics
\end{itemize}

Also during MD6864, 2022-10-19, for the bare machine \\
Also 2022-11-06, measurement at 30cm, flat top.

High order $\alpha_c$. \\
Radial loop instead of formula.



% ===============================
%        Bare Chromaticity
% ===============================
\subsection{Bare Chromaticity}