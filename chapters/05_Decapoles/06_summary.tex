
\section{\review{Summary}}
This chapter examines the role of decapolar fields in the Large Hadron Collider (LHC) at injection
energy. First is addressed the previously observed discrepancy between measurements and model
regarding the third-order chromaticity. To investigate these issues, various measurements and
simulations were conducted. By introducing novel observables, such as the bare chromaticity and, for the
first time, chromatic amplitude detuning, a clearer understanding of these discrepancies was
achieved. Simulations indicate that the decay of the decapolar component in the main dipoles is a
major factor contributing to the discrepancies.

For the first time at injection energy, measurements and corrections of the decapolar Resonance
Driving Term (RDT) $f_{1004}$ were carried out. Further simulations and measurements explored how
sextupoles and octupoles interact to create decapolar-like fields. The findings revealed that
sextupoles, both alone and in combination with Landau octupoles, generate substantial decapolar RDTs
during machine operation that could benefit from corrections.

Applying combined corrections for third-order chromaticity, chromatic amplitude detuning, and the
RDT $f_{1004}$ led to a $3\%$ improvement in beam lifetime. Additionally, a broader impact of
decapolar RDTs on beam stability was investigated. Specifically, intentionally degrading the RDT
$f_{1004}$ resulted in a decrease in beam lifetime of about $10\%$. This underscores the importance
of these corrections for stable beam operation and suggests that further advancements in correction
methods could lead to even greater improvements. 