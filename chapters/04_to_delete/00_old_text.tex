\chapter{Theory}
\thumbforchapter{}
\chaptertoc{}

\section{Units}\label{units}

\begin{itemize}
\item
  Joules: \[1 J = 1 N \cdot m = 1 kg \cdot m^2 \cdot s^{-2}\]
\item
  Electronvolt:

  \begin{itemize}
  \tightlist
  \item
    \(eV = e \cdot V\)

    \begin{itemize}
    \tightlist
    \item
      e: elementary charge
    \item
      V: volt
    \end{itemize}
  \item
    \(1 eV = 1.602176634 \cdot 10^{-19} J\)
  \end{itemize}
\item
  Momentum:

  \begin{itemize}
  \tightlist
  \item
    \(p\) in \(kg.m.s^{-1}\)
  \item
    \(p\) in \(eV.c^{-1}\): \(J \cdot c^{-1} = kg.m.s^{-1}\)
  \end{itemize}
\item
  Tesla \[T = \frac{V \cdot s}{m}\]
\end{itemize}

\newpage

\hypertarget{definitions}{%
\section{Definitions}\label{definitions}}

\hypertarget{some-math-definitions}{%
\subsection{Some Math Definitions}\label{some-math-definitions}}

\begin{itemize}
\item
  Harmonic Oscillator \[F = -kx\] \[x'' = -\frac{k}{m}x \]
\item
  Magnetic rigidity, used for dipoles: \[B\rho = \frac{p}{q}\]

  \begin{itemize}
  \item
    B: magnetic field {[}\(T\){]}
  \item
    \(\rho\): radius of curvature of the orbit {[}\(m\){]}
  \item
    p: momentum {[}\(eV/c\){]}
  \item
    q: particle charge {[}\(e\){]}
  \item
    Can be expressed as \[B \rho = 3.3356 p\] if \(p\) is in GeV/c
  \end{itemize}
\item
  Magnets orders
\end{itemize}

\begin{center}
  \begin{tabular}{lllllll}
  Magnet Letter          & B & Q & S & O & D & T \\
  Number of Poles        & 2 & 4 & 6 & 8 & 10 & 12 \\
  a (skew) designation   & 1 & 2 & 3 & 4 & 5 & 6 \\
  b (normal) designation & 1 & 2 & 3 & 4 & 5 & 6 \\
  K designation          & 1 & 2 & 3 & 4 & 5 & 6 \\
  MADX-K                 & 0 & 1 & 2 & 3 & 4 & 5 \\
  \end{tabular}
\end{center}

\begin{itemize}
\item
  Magnet Strength:
  \begin{equation}K_{n} = \frac{q}{P} (n-1)!B_n\label{eq:magnet_strength}\end{equation}

  \begin{itemize}
  \item
    The unit is given by:

    \begin{itemize}
    \tightlist
    \item
      k1: dipole: \(m^{-1}\)
    \item
      k2: quadrupole: \(m^{-2}\)
    \item
      k3: sextupole: \(m^{-3}\)
    \item
      k4: octupole: \(m^{-4}\)
    \item
      k5: decapole: \(m^{-5}\)
    \item
      k6: dodecapole: \(m^{-6}\)
    \end{itemize}
  \item
    If interested in the \emph{integrated} strength, multiply by
    \emph{m}
  \end{itemize}
\item
  Dispersion is given via:
  \begin{equation}D = \frac{\Delta x}{\delta}\end{equation}
\item
  The relative momentum offset is defined via the reference momentum and
  the momentum
  \begin{equation}\delta = \frac{\Delta P}{P_0} = \frac{P - P_0}{P_0}\label{eq:dpp}\end{equation}
\item
  Relation between the action and the single particle emittance:
  \href{https://journals.aps.org/prab/pdf/10.1103/PhysRevSTAB.17.081002}{Non
  Linear Observables} (eq. 2):
  \begin{equation}2J_{x,y} = \epsilon_{x,y}\end{equation}
\item
  The tune is defined as the derivative of the Hamiltonian relative to
  the action.
  \begin{equation}Q_{x,y} = \frac{\partial \left< H \right>}{\partial J_{x,y}}\end{equation}
\item
  The tune can also be defined as:
  \begin{equation}Q_{x,y} = \frac{1}{2 \pi} \oint \frac{1}{\beta_{x,y}(s)} \,ds\end{equation}
\item
  The phase advance in the \(x\) plane, between two elements \(w\) and
  \(j\), in the accelerator is given by
  \href{https://arxiv.org/abs/1711.06589}{AnalyticFormulasFranchi}:
\end{itemize}

\begin{equation}
\begin{cases} 
  \Delta \phi_{x,wj} = \left(\phi_{x,j} - \phi_{x,w} \right)              & \mbox{if } \phi_{x,j} > \phi_{x,w}, \\
  \Delta \phi_{x,wj} = \left(\phi_{x,j} - \phi_{x,w} \right) + 2 \pi Q_x  & \mbox{if } \phi_{x,j} < \phi_{x,w}.
\end{cases}
\end{equation}

\newpage

\hypertarget{maths}{%
\section{Maths}\label{maths}}

\hypertarget{taylor-series}{%
\subsection{Taylor Series}\label{taylor-series}}

The Taylor series of a multivariable function at the point
\((a_1, \cdots, a_d)\) is defined as:

\begin{equation}\begin{aligned}
  f(x_1, \cdots, x_d) = &f(a_1, \cdots, a_d) \\
                &+ \frac{1}{1!} \sum_{j=1}^{d} \frac{\partial f(a_1, \cdots, a_d)}{\partial x_j} (x_j - a_j) \\
                &+ \frac{1}{2!}\sum_{j=1}^{d}\sum_{k=1}^{d} \frac{\partial^2 f(a_1, \cdots, a_d)}{\partial x_j\partial x_k} (x_j - a_j) (x_k - a_k) \\
                &+ \frac{1}{3!}\sum_{j=1}^{d}\sum_{k=1}^{d}\sum_{l=1}^{d} \frac{\partial^3 f(a_1, \cdots, a_d)}{\partial x_j\partial x_k\partial x_l} (x_j - a_j) (x_k - a_k) (x_l - a_l)\\
                &+ \cdots
\end{aligned}\label{eq:taylor}\end{equation}

\hypertarget{non-linear-transfer-maps}{%
\subsection{Non-Linear Transfer Maps}\label{non-linear-transfer-maps}}

For linear dynamics, linear transfer maps can be used to describe a
particle's position from the elements in the lattice and the particle's
current position. In the non linear case, such a formalism can't be
used. Instead, the Lie operator is used.

\hypertarget{lie-operator}{%
\subsubsection{Lie Operator}\label{lie-operator}}

Resources found in
\href{https://mlegarre.web.cern.ch/mlegarre/Courses/Meghan_RDTpresentation.pdf}{Meghan
McAteer's talk} and Wolski's book.

The Lie operator is defined as:

\[
\colon f \colon = \sum^n_{i=1} \left(\frac{\partial f}{\partial x_i} \frac{\partial}{\partial p_i} 
                           - \frac{\partial f}{\partial p_i} \frac{\partial}{\partial x_i}
                      \right)
\]

With \emph{i} the index of a dimension. So \(x_1\) would be \(x\) and
\(x_2\), \(y\).

The exponential Lie operator \(e^{:f:}\) acting on a function \(g\):

\begin{equation}e^{:f:}g = g + [f, g] + \frac{1}{2} [f, [f, g]] + \cdots\label{eq:Lie}\end{equation}

This is from the usual power series of an exponential:

\[e^{:f:} = \sum^{\infty}_{m=0} \frac{:f:^m}{m!}\]

Where \([f, g]\) is a poisson bracket, still somehow used for historical
reasons:

\begin{equation}[f, g] = \colon f \colon g
\label{eq:poisson_bracket}\end{equation}

Thus, the particle with coordinates \(\vec{x}\) passing through an
element can be mapped using the Hamiltonian of the element (Wolski, eq.
9.6):

\begin{equation}\vec{x}_f = e^{-\Delta s:H:} \vec{x}_0\label{eq:position_non_linear}\end{equation}

For a \emph{thin} lens, we can directly multiply by the length of the
magnet:

\begin{equation}\vec{x}_f = e^{-L:H:} \vec{x}_0\label{eq:position_non_linear_thin}\end{equation}

By combining \cref{eq:Lie} and \cref{eq:position_non_linear}, we obtain:

\begin{equation}
\begin{aligned}
\left[H, \vec{x}_0\right] =& \frac{\partial H}{\partial x} \frac{\partial \vec{x}_0}{\partial p_x} 
                                    -\frac{\partial H}{\partial p_x} \frac{\partial \vec{x}_0}{\partial x} \\
                           &\frac{\partial H}{\partial y} \frac{\partial \vec{x}_0}{\partial p_y} 
                                    -\frac{\partial H}{\partial p_y} \frac{\partial \vec{x}_0}{\partial y}
\end{aligned}
\label{eq:bracket_hamiltonian}\end{equation}

Thus giving us the complete equation:

\begin{equation}
\begin{aligned}
\vec{x}_f &= e^{:H:}\vec{x}_0 \\
          &= \vec{x}_0 + \left[H, \vec{x}_0\right] \\
          &= \begin{pmatrix} x \\ p_x \\ y \\ p_y\end{pmatrix}_0
             + \begin{pmatrix} -\frac{\partial H}{\partial p_x} \\ 
                               \frac{\partial H}{\partial x} \\
                               -\frac{\partial H}{\partial p_y} \\
                               \frac{\partial H}{\partial y}
               \end{pmatrix} \\
\begin{pmatrix} x \\ 
                p_{x} \\
                y \\
                p_{y} 
\end{pmatrix}_f
          &=  \begin{pmatrix} x_0 - \dfrac{\partial H}{\partial p_x} \\ 
                               p_{x0} +\dfrac{\partial H}{\partial x} \\
                               y_0 - \dfrac{\partial H}{\partial p_y} \\
                               p_{y0} +\dfrac{\partial H}{\partial y}
               \end{pmatrix}
\end{aligned}
\label{eq:non_linear_map}\end{equation}

This non linear transfer map is a general map without coupling.

\hypertarget{multipole-expansion}{%
\subsection{Multipole Expansion}\label{multipole-expansion}}

This is the Hamiltonian for a magnetic element with normal strength
component \emph{K} and skew component \emph{J}:

\begin{equation}H = \Re \left[\sum_{n>1} \left( K_{n} + iJ_{n} \right) \frac{(x+iy)^n}{n!} \right]\label{eq:hamiltonian}\end{equation}

This equation is an expansion of the magnetic field into its multipole
components. If we're only interested in one component, the sum can be
dropped and \(n\) set to the desired multipole.

Remark: When writing terms like \(K_1\) it is assumed it is \(K_1(s)\).
Is is important later for the chromaticity where the average is a
circular integral over s.

It follows that the normal and skew fields are:

\begin{equation}\mathcal{N}_n = \frac{1}{n!}K_n \Re \left[(x+iy)^n\right]\end{equation}
\begin{equation}\mathcal{I}_n = -\frac{1}{n!}J_n \Im \left[(x+iy)^n\right]\end{equation}

\hypertarget{quadrupole}{%
\subsubsection{Quadrupole}\label{quadrupole}}

\begin{equation}\mathcal{N_2}(x, y) = \frac{1}{2} K_2 (x^2 - y^2)\end{equation}

\hypertarget{sextupole}{%
\subsubsection{Sextupole}\label{sextupole}}

\begin{equation}\mathcal{N_3}(x, y) = \frac{1}{6} K_3 (x^3 - 3xy^2)\end{equation}

\hypertarget{octupole}{%
\subsubsection{Octupole}\label{octupole}}

\begin{equation}\mathcal{N_4}(x, y) = \frac{1}{24} K_4 (x^4 - 6x^2y^2 + y^4)\end{equation}

\hypertarget{decapole}{%
\subsubsection{Decapole}\label{decapole}}

\begin{equation}\mathcal{N_5}(x, y) = \frac{1}{120} K_5 (x^5 - 10x^3y^2 + 5xy^4)\end{equation}

\hypertarget{dodecapole}{%
\subsubsection{Dodecapole}\label{dodecapole}}

\begin{equation}\mathcal{N_6}(x, y) = \frac{1}{720} K_6 (x^6 - 15x^4y^2 + 15x^2y^4 - y^6)\end{equation}

\hypertarget{multinomial-expansion}{%
\subsection{Multinomial Expansion}\label{multinomial-expansion}}

For any positive integer \emph{m} and non-negative integer \emph{n}, the
multinomial expansion describes the expansion of a sum raised to the
power \emph{n}.

\begin{equation}
(x_1 + x_2 + \cdots + x_m)^n = \sum_{k_1 + k_2 + k_3 + \cdots + k_m = n} 
                               \frac{n!}{k_1!k_2!\cdots k_m!}
                               x_1^{k_1}x_2^{k_2}x_1^{k_2} \cdots x_m^{k_m}
\label{eq:multinomial_expansion}\end{equation}

It should be noted that the sum \(k_1 + k_2 + \cdots + k_m\) is
\emph{equal} to \(n\). This avoids unwanted terms in the expansion.
Another form of writing the multinomial expansion is with the Kronecker
delta, notice that here the sum is \emph{less than or equal} to \(n\):

\begin{equation}\delta_{i,j} = 
\begin{cases} 
  0 & \mbox{if } i \neq j, \\
  1 & \mbox{if } i = j.
\end{cases}
\label{eq:kronecker}\end{equation}

\begin{equation}
(x_1 + x_2 + \cdots + x_m)^n = \sum_{k_1 + k_2 + k_3 + \cdots + k_m \leq n} \delta_{j + k + l + m,n}
                               \frac{n!}{k_1!k_2!\cdots k_m!}
                               x_1^{k_1}x_2^{k_2}x_1^{k_2} \cdots x_m^{k_m}
\label{eq:multinomial_expansion_kronecker}\end{equation}

\hypertarget{example}{%
\subsubsection{Example}\label{example}}

\begin{equation}
\begin{aligned}
(a + b + c + d)^n &= \sum_{j + k + l + m = n} 
                      \frac{n!}{j!k!l!m!}
                      a^j b^k c^l d^m  \\
                  &= \sum_{j + k + l + m \leq n} 
                      \delta_{j + k + l + m,n}
                      \frac{n!}{j!k!l!m!}
                      a^j b^k c^l d^m  \\
                  &= \sum^n_{j=0}\sum^n_{k=0}\sum^n_{l=0}\sum^n_{m=0}
                      \delta_{j + k + l + m,n}
                      \frac{n!}{j!k!l!m!} 
                      a^j b^k c^l d^m
\end{aligned}
\end{equation}

This expansion is particularly useful for Resonance Driving Terms (RDTs)
to isolate specific resonances.

\newpage

\hypertarget{non-linear-transfer-maps-1}{%
\section{Non-Linear Transfer Maps}\label{non-linear-transfer-maps-1}}

This section details the transfer maps for each multipole. The used
transfer map is the non linear map defined by \cref{eq:non_linear_map}.

\hypertarget{quadrupole-1}{%
\subsection{Quadrupole}\label{quadrupole-1}}

The main normal field of a quadrupole is defined by:

\begin{equation}
\begin{aligned}
  H &= \frac{1}{2} K_2 (x^2 - y^2) \\
    &= \frac{1}{2} \frac{q}{P} B_2 (x^2 - y^2)
\end{aligned}
\end{equation}

We can now differentiate the Hamiltonian to find the needed terms:

\begin{equation}
\begin{aligned}
\frac{\partial H}{\partial x} &= \frac{q}{P_0} B_2 x &; \quad \frac{\partial H}{\partial y} &= -\frac{q}{P_0} B_2 y \\
\frac{\partial H}{\partial p_x} &= 0                 &; \quad \frac{\partial H}{\partial p_y} &= 0
\end{aligned}
\end{equation}

We then get the equation for a thin quadrupole we're familiar with.
Please note the addition of the term \(-L\) into the power series of the
exponential, which leads to reversed signs:

\begin{equation}\begin{aligned}
\begin{pmatrix}
x \\ p_x \\ y \\ p_y
\end{pmatrix}_f
& =
\begin{pmatrix}
x_0 \\
p_{x0} - \dfrac{q}{P_0} L B_2 x_0 \\
y_0 \\
p_{y0} + \dfrac{q}{P_0} L B_2 y_0 \\
\end{pmatrix}\\
& =
\begin{pmatrix}
x \\ p_x \\ y \\ p_y
\end{pmatrix}_0
-\frac{q}{P_0}LB_2
\begin{pmatrix}
0 \\ x_0 \\ 0 \\ y_0
\end{pmatrix}\\
& =
\begin{pmatrix}
1 & 0 & 0 & 0 \\ -\dfrac{q}{P_0}LB_2 & 1 & 0 & 0 \\ 0 & 0 & 1 & 0 \\ 0 & 0 & \dfrac{q}{P_0}LB_2 & 1
\end{pmatrix}
\begin{pmatrix}
x \\ p_x \\ y \\ p_y
\end{pmatrix}_0
\end{aligned}\end{equation}

\hypertarget{sextupole-1}{%
\subsection{Sextupole}\label{sextupole-1}}

The main normal field of a sextupole is defined by:

\begin{equation}
\begin{aligned}
  H &= \frac{1}{6} K_3 (x^3 - 3xy^2) \\
    &= \frac{1}{3} \frac{q}{p} B_3 (x^3 - 3 xy^2)
\end{aligned}
\end{equation}

We can now differentiate the Hamiltonian to find the needed terms:

\begin{equation}
\begin{aligned}
\frac{\partial H}{\partial x} &= \frac{q}{P_0} B_3 (x^2 - y^2) &; \quad \frac{\partial H}{\partial y} &= -2 \frac{q}{P_0} B_3 x y \\
\frac{\partial H}{\partial p_x} &= 0                 &; \quad \frac{\partial H}{\partial p_y} &= 0
\end{aligned}
\end{equation}

We then get:

\begin{equation}
\begin{pmatrix}
x \\ p_x \\ y \\ p_y
\end{pmatrix}_f
=
\begin{pmatrix}
x_0 \\
p_{x0} - \dfrac{q}{P_0} L B_3 (x_0^2 - y_0^2) \\
y_0 \\
p_{y0} + 2\dfrac{q}{P_0} L B_3 x_0 y_0 \\
\end{pmatrix}
\end{equation}

\newpage

\hypertarget{chromaticity}{%
\section{Chromaticity}\label{chromaticity}}

The chromaticity is the change of tune relative to the relative offset
momentum. It is thus given by Eq.(5.5) of
\href{https://cds.cern.ch/record/1951379/files/Thesis-2014-Ewen}{Ewen's
Thesis}:

\[Q'_{x,y} = \frac{\partial Q_{x,y}}{\partial \delta} = \frac{1}{2\pi}\left< \frac{\partial² H}{\partial J_{x,y}\partial \delta}\right>\]

For a single element, we can compute $\Delta Q'$:
\begin{equation}\Delta Q_{x,y}' = \frac{1}{2\pi} \int_L \frac{\partial^2 \left< H \right>}{\partial J_{x,y} \partial \delta} \diff s\label{eq:chroma}\end{equation}

By doing a Taylor Expansion on \(Q(\delta)\), we get:

\begin{equation} Q (\delta) = Q_0 + Q' \delta + \frac{1}{2!} Q'' \delta^2 + \frac{1}{3!} Q''' \delta^3 ... \label{eq:tune_chroma}\end{equation}

\newpage

\hypertarget{quadrupole-2}{%
\subsection{Quadrupole}\label{quadrupole-2}}

Full expression of the Hamiltonian for the magnet given by eq.
\ref{eq:hamiltonian} :
\begin{equation}H_2(x,y) = \frac{1}{2} K_2 (x^2 - y^2) - J_2 xy\label{eq:hamiltonian_quadrupole}\end{equation}

Magnets usually only have a main field and aren't skewed. We can keep
only the main normal field:
\begin{equation}\mathcal{N}_2(x,y) = \frac{1}{2} K_2 (x^2 - y^2)\label{eq:main_hamiltonian_quadrupole}\end{equation}

In a region of zero dispersion, we can write
\(x \rightarrow x + \Delta x \Longleftrightarrow x \rightarrow x + \eta \delta\)
with \(\eta\) being the dispersion and \(\delta\) the relative momentum
offset \(\delta p / p_0\). This bring us to:
\begin{equation}\mathcal{N}_2 (x,y) = \frac{1}{2} K_2 ((x + \eta \delta)^2 - y^2)\label{eq:hamiltonian_quadrupole_dpp}\end{equation}

By operating a variable change to the angle coordinates
(\(x = \sqrt{2 J_x \beta_x} \cos \phi_x\) and
\(y = \sqrt{2 J_y \beta_y} \cos \phi_y\)):
\begin{equation}\mathcal{N}_2 (x,y) = \frac{1}{2} K_2 \left[\left(\sqrt{2 J_x \beta_x} \cos \phi_x + \eta \delta\right)^2 - \left(\sqrt{2 J_y \beta_y} \cos \phi_y\right)^2\right]\label{eq:hamiltonian_quadrupole_angle}\end{equation}

We can then now differentiate relative the relative momentum offset
\(\delta\):
\begin{equation}\frac{\partial \mathcal{N_2}}{\partial \delta} =  \frac{1}{2} K_2 \left(\sqrt{2 J_x \beta_x} \cos \phi_x \eta + 2 \eta \delta\right)\label{eq:n2_dpp}\end{equation}

Differientiating relative to the action:
\begin{equation}\frac{\partial \mathcal{N_2}}{\partial J_x \partial \delta} =  \frac{1}{2} K_2 \left (\frac{2 \beta_x}{2 \sqrt{2 J_x \beta_x}} \cos \phi_x \eta \right) \end{equation}
\begin{equation}\frac{\partial \mathcal{N_2}}{\partial J_y \partial \delta} = 0\end{equation}

Averaging over the phase variables gives:
\begin{equation}\left< \frac{\partial \mathcal{N_2}}{\partial J_{x,y} \partial \delta} \right> = 0\end{equation}

From this approach, we can see we don't have any chromaticity induced by
quadrupoles. This is because we assumed the quadropole strength,
\(K_2\), was constant and not momentum dependant.

To find chromaticity from the \(K\) strength we need to change \(K\) and
express \(P\) as \(P_0(1 + \delta)\), combining eq.
\ref{eq:magnet_strength} and eq. \ref{eq:dpp}:
\begin{equation}K_n = \frac{q}{P_0} \frac{1}{1 + \delta} (n - 1)! B_n\label{eq:k_dpp}\end{equation}

Going back to eq. \ref{eq:hamiltonian_quadrupole_angle}, averaging it
and substituting \ref{eq:k_dpp}:
\begin{equation}\frac{\partial \left< \mathcal{N_2} \right>}{\partial J_x} = \frac{1}{2} \frac{q}{P_0} (1 - \delta) B_2 \beta_x\end{equation}
\begin{equation}\frac{\partial \left< \mathcal{N_2} \right>}{\partial J_y} = - \frac{1}{2} \frac{q}{P_0} (1 - \delta) B_2 \beta_y\end{equation}

Now that \(\delta\) is in the equation, we can differentiate by it:
\begin{equation}\frac{\partial^2 \left< \mathcal{N_2} \right>}{\partial J_x \partial \delta}  = - \frac{1}{2} \frac{q}{P_0} B_2 \beta_x\end{equation}
\begin{equation}\frac{\partial^2 \left< \mathcal{N_2} \right>}{\partial J_x \partial \delta}  = \frac{1}{2} \frac{q}{P_0} B_2 \beta_x\end{equation}

We can now get the chromaticity \(Q'\): \begin{equation}\begin{aligned}
\Delta Q'_x &= \frac{1}{2\pi} \int_L \frac{\partial^2 \left< \mathcal{N_2} \right> }{\partial J_x \partial \delta} \diff s \\
  & = - \frac{1}{4 \pi} \frac{q}{P_0} B_2 L \beta_x
\end{aligned}\end{equation}

\begin{equation}\begin{aligned}
\Delta Q'_y &= \frac{1}{2\pi} \int_L \frac{\partial^2 \left< \mathcal{N_2} \right>}{\partial J_y \partial \delta} \diff s \\
  & = \frac{1}{4 \pi} \frac{q}{P_0} B_2 L \beta_y
\end{aligned}\end{equation}

\newpage

\hypertarget{sextupole-2}{%
\subsection{Sextupole}\label{sextupole-2}}

Full expression of the Hamiltonian for the magnet given by eq.
\ref{eq:hamiltonian} :
\begin{equation}\mathcal{H_3}(x, y) = \frac{3 K_{2} \left(x^{2} - y^{2}\right)}{6} + \frac{K_{3} \left(x^{3} - 3 x y^{2}\right)}{6} - J_{2} x y - \frac{J_{3} \left(3 x^{2}y - y^{3}\right)}{6}\end{equation}

Main normal field:
\begin{equation}\mathcal{N_3}(x, y) = \frac{K_{3} \left(x^{3} - 3 xy^{2}\right)}{6}\end{equation}

With a displacement in x:
\begin{equation}\mathcal{N_3}(x + \Delta x, y) = \frac{K_{3} \left((x + \eta \delta)^{3} - 3 (x + \eta \delta)y^{2}\right)}{6}\end{equation}

Expanding:
\begin{equation}\mathcal{N_3}(x + \Delta x, y) = \frac{K_{3} \left(x^3 + 3 x^2 \eta \delta + 3 x \eta^2 \delta^2 + \eta^3 \delta^3 - 3xy^2 - 3 \eta \delta y^2 \right)}{6}\end{equation}

Changing variables:

\begin{equation}\begin{aligned}
  \mathcal{N_3}(x + \Delta x, y) = \frac{1}{6} K_3 \biggl[&
       \left(\sqrt{2 J_x \beta_x} \cos \phi_x\right)^3 \\
  &    + 3 \left(\sqrt{2 J_x \beta_x} \cos \phi_x\right)^2 \eta \delta \\
  &    + 3 \left(\sqrt{2 J_x \beta_x} \cos \phi_x\right) \eta^2 \delta^2 \\
  &    + \eta^3 \delta^3 \\
  &    - 3 \left(\sqrt{2 J_x \beta_x} \cos \phi_x \right) \left(\sqrt{2 J_y \beta_y} \cos \phi_y \right)^2 \\
  &    - 3 \eta \delta (\sqrt{2 J_y \beta_y} \cos \phi_y)^2 \biggl]
\end{aligned}\end{equation}

It is faster and easier here to average over the phase variables before
differentiating. Any odd cosine can be removed since its average is 0:
\begin{equation}\begin{aligned}
  \left< \mathcal{N_3}(x + \Delta x, y) \right> = \frac{1}{6} K_3 &\biggl(
       3 J_x \beta_x \eta \delta \\
  &    + \eta^3 \delta^3 \\
  &    - 3 \eta \delta J_y \beta_y \biggl)
\end{aligned}\end{equation}

\vspace{.5cm}

Differentiating relative to the action:

\begin{equation}\begin{aligned}
\frac{\partial \left< \mathcal{N_3} \right>}{\partial J_x} &= \frac{1}{2} K_3 \beta_x \eta \delta \\
\frac{\partial \left< \mathcal{N_3} \right>}{\partial J_y} &= - \frac{1}{2} K_3 \beta_y \eta \delta
\end{aligned}\end{equation}

The chromaticity is then: \begin{equation}\begin{aligned}
\Delta Q'_x &= \frac{1}{2\pi} \int_L \frac{\partial^2 \left< \mathcal{N_3} \right>}{\partial J_x \partial \delta} \diff s \quad; \quad \Delta Q'_y &&= \frac{1}{2\pi} \int_L \frac{\partial^2 \left< \mathcal{N_3} \right>}{\partial J_y \partial \delta} \diff s \\
&= \frac{1}{2 \pi} L \frac{1}{2} K_3 \beta_x \eta  &&= - \frac{1}{2 \pi} L \frac{1}{2} K_3 \beta_y \eta \\
&= \frac{1}{4 \pi}  K_3 L \beta_x \eta &&= - \frac{1}{4 \pi}  K_3 L \beta_y \eta
\end{aligned}\end{equation}

\hypertarget{octupole-1}{%
\subsection{Octupole}\label{octupole-1}}

The main field of an octupole is:

\begin{equation}\mathcal{N}_4(x, y) = \frac{1}{24} K_4 (x^4 - 6x^2y^2 + y^4)\end{equation}

With a displacement in x:

\begin{equation}\begin{aligned}
\mathcal{N}_4(x + \Delta x, y) &= \frac{1}{24} K_4 \bigl[(x + \eta \delta)^4 - 6(x + \eta \delta)^2y^2 + y^4 ] \\
  &\;\begin{aligned}= \frac{1}{24} K_4 &[(x^4 + 4x^3 \eta \delta + 6 x^2 \eta^2 \delta^2 + 4x \eta^2 \delta^2 + \eta^4 \delta^4)\\
 & - 6 y^2 (x^2 + 2x \eta \delta + \eta^2 \delta^2)\\
 & + y^4\bigr]
 \end{aligned}
\end{aligned}\label{eq:hamiltonian_displacement_octupole}\end{equation}

Differentiate relative to \(\delta\): \begin{equation}
\begin{aligned}
\frac{\partial \mathcal{N}_4}{\partial \delta} = \frac{1}{24} K_4 [& 4x^3 \eta + 12 x^2 \eta^2 \delta + 8 x \eta^2 \delta + 4 \eta^4 \delta^3 & \\
& - 6 y^2 (2x \eta + 2 \eta^2 \delta) ]
\end{aligned}
\label{eq:hamiltonian_octupole_delta}\end{equation}

Differentiate \emph{again} relative to \(\delta\):
\begin{equation}\frac{\partial \mathcal{N}_4}{\partial^2 \delta} = \frac{1}{24} K_4 \left( 12 x^2 \eta^2 + 8 x \eta^2+ 12 \eta^4 \delta^2 - 12 y^2 \eta^2  \right)\end{equation}

Changing variables: \begin{equation}\begin{aligned}
\frac{\partial^2 \mathcal{N}_4}{\partial^2 \delta} &\;\begin{aligned}
                                                   = \frac{1}{24} K_4 \biggl[12 \left(\sqrt{2J_x\beta_x}\cos\phi_x\right)^2 \eta^2
                                                   \end{aligned}\\
                                                   &\;\begin{aligned}
                                                   \phantom{  = \frac{1}{24} K_4 \biggl[}&+ 8 \left(\sqrt{2J_x\beta_x}\cos\phi_x\right) \eta^2\\
                                                                                         &+ 12 \eta^4 \delta^2 \\
                                                                                         &- 12 \left(\sqrt{2J_y\beta_y}\cos\phi_y\right)^2 \eta^2  \biggr] \\
                                                   \end{aligned}\\
&= \frac{1}{24} K_4 \left( 24 J_x \beta_x \cos^2 \phi_x \eta^2 + 8 \sqrt{2J_x\beta_x} \cos\phi_x \eta^2 + 12 \eta^4\delta^2 - 24 J_y\beta_y \cos^2\phi_y\eta^2\right)
\end{aligned}\end{equation}

Averaging over the phase variables to get rid of the cosines:
\begin{equation}\frac{\partial^2 \left<\mathcal{N}_4\right>}{\partial^2 \delta} = \frac{1}{24} K_4 \left( 12 J_x \beta_x \eta^2 + 12 \eta^4\delta^2 - 12 J_y\beta_y \eta^2\right)\end{equation}

Differentiating by the action: \begin{equation}\begin{aligned}
\frac{\partial^3 \left<\mathcal{N}_4\right>}{\partial^2 \delta \partial J_x} &= \frac{1}{24} K_4 \left( 12 \beta_x \eta^2 \right) \\
\frac{\partial^3 \left<\mathcal{N}_4\right>}{\partial^2 \delta \partial J_y} &= \frac{1}{24} K_4 \left(- 12 \beta_y \eta^2\right)
\end{aligned}\end{equation}

We can now get the chromaticity \(Q''\): \begin{equation}\begin{aligned}
\Delta Q''_x &= \frac{1}{2\pi} \int_L \frac{\partial^3 \left< \mathcal{N_4} \right>}{\partial J_x \partial^2 \delta} \diff s \quad; \quad \Delta Q''_y &&= \frac{1}{2\pi} \int_L \frac{\partial^3 \left< \mathcal{N_4} \right>}{\partial J_y \partial^2 \delta} \diff s \\
&= \frac{1}{2 \pi} L \frac{1}{2} K_4 \beta_x \eta^2  &&= - \frac{1}{2 \pi} L \frac{1}{2} K_4 \beta_y \eta^2 \\
&= \frac{1}{4 \pi}  K_4 L \beta_x \eta^2 &&= - \frac{1}{4 \pi}  K_4 L \beta_y \eta^2
\end{aligned}\end{equation}

\hypertarget{decapole-1}{%
\subsection{Decapole}\label{decapole-1}}

The main normal field of a decapole:
\begin{equation}\mathcal{N_5}(x, y) = \frac{1}{120} K_{5} \left(x^5 - 10 x^3y^2 + 5xy^4 \right)\end{equation}

With a displacement in x:
\begin{equation}\mathcal{N_5}(x, y) = \frac{1}{120} K_{5} \biggl[(x+\eta\delta)^5 - 10 (x+\eta\delta)^3y^2 + 5(x+\eta\delta)y^4 \biggr]\end{equation}

Expanding: \begin{equation}\begin{aligned}
\mathcal{N_5}(x, y) = \frac{1}{120} K_{5} \biggl[&
  \eta^5\delta^5 + 5\eta^4\delta^4x + 10\eta^3\delta^3x^2 + 10\eta^2\delta^2 x^3 + 5\eta\delta x^4 + x^5 \\
  & -10y^2 (\eta^3\delta^3 + 3\eta^2\delta^2x + 3\eta\delta x^2 + x^3)\\
  & +5y^4 (x + \eta\delta) \biggr]
\end{aligned}\label{eq:decapole_expanded}\end{equation}

Directly differentiating by \(\delta\) 3 times:
\begin{equation}\frac{\partial^3 \mathcal{N_5}}{\partial^3 \delta} = \frac{1}{120} K_5 \left( 60\eta^5\delta^2 + 120 \eta^4\delta x + 60\eta^3x^2 - 60y^2\eta^3\right)\end{equation}

Changing variable: \begin{equation}\begin{aligned}
\frac{\partial^3 \mathcal{N_5}}{\partial^3 \delta} = \frac{1}{120} K_5 \biggl(& 60\eta^5\delta^2 + 120 \eta^4\delta \sqrt{2J_x\beta_x}\cos\phi_x\\
    &+ 120\eta^3 J_x\beta_x\cos^2\phi_x - 120\eta^3 J_y\beta_y\cos^2\phi_y\biggr)
\end{aligned}\end{equation}

Averaging over the phase variables to get rid of the cosines:
\begin{equation}\frac{\partial^3 \left<\mathcal{N_5}\right>}{\partial^3 \delta} = \frac{1}{120} K_5 \left( 60\eta^5\delta^2
  + 60\eta^3 J_x\beta_x- 60\eta^3 J_y\beta_y\right)\end{equation}

Differentiating by the action: \begin{equation}\begin{aligned}
\frac{\partial^4 \left<\mathcal{N}_5\right>}{\partial^3 \delta \partial J_x} &= \frac{1}{120} K_5 \left( 60 \eta^3 \beta_x \right) \\
\frac{\partial^4 \left<\mathcal{N}_5\right>}{\partial^3 \delta \partial J_y} &= \frac{1}{120} K_5 \left(- 60 \eta^3 \beta_y \right)
\end{aligned}\end{equation}

We can now get the chromaticity \(Q'''\):
\begin{equation}\begin{aligned}
\Delta Q'''_x &= \frac{1}{2\pi} \int_L \frac{\partial^4 \left< \mathcal{N_5} \right>}{\partial J_x \partial^3 \delta} \diff s \quad; \quad \Delta Q'''_y &&= \frac{1}{2\pi} \int_L \frac{\partial^4 \left< \mathcal{N_5} \right>}{\partial J_y \partial^3 \delta} \diff s \\
&= \frac{1}{2 \pi} L \frac{1}{2} K_5 \beta_x \eta^3  &&= - \frac{1}{2 \pi} L \frac{1}{2} K_5 \beta_y \eta^3 \\
&= \frac{1}{4 \pi}  K_5 L \beta_x \eta^3 &&= - \frac{1}{4 \pi}  K_5 L \beta_y \eta^3
\end{aligned}\end{equation}

\hypertarget{dodecapole-1}{%
\subsection{Dodecapole}\label{dodecapole-1}}

Similarly to previous calculations, we can derive that decapoles (b6)
act on \(Q^{(4)}\):

\begin{equation}\begin{aligned}
\Delta Q^{(4)}_x &= \frac{1}{4\pi} K_6 L \beta_x \eta^4 \quad; \quad \Delta Q^{(4)}_y &&= -\frac{1}{4\pi} K_6 L \beta_y \eta^4
\end{aligned}\end{equation}

The decapoles in the LHC are located near the IPs, in a dispersion
suppressed zone, meaning that those decapoles can't act on \(Q^{(4)}\).
The chromaticity of this order, and higher up can't be corrected.

\hypertarget{tetradecapole}{%
\subsection{Tetradecapole}\label{tetradecapole}}

As before, we can derive that tetradecapole fields (b7) act on
\(Q^{(5)}\):

\begin{equation}\begin{aligned}
\Delta Q^{(5)}_x &= \frac{1}{4\pi} K_7 L \beta_x \eta^5 \quad; \quad \Delta Q^{(5)}_y &&= -\frac{1}{4\pi} K_7 L \beta_y \eta^5
\end{aligned}\end{equation}

Magnets of this order don't exist in the LHC. The fields though exist
via higher order fields of the other magnets.

\newpage

\hypertarget{amplitude-detuning}{%
\section{Amplitude Detuning}\label{amplitude-detuning}}

\colorbox{yellow}{Add octupole and dodecapoles formulas, along with an AC-Dipole}

Amplitude Detuning is the change of tune due to the change of the beam's
action \(\epsilon_{x,y} = 2J_{x,y}\).

Recap, the tune is:
\begin{equation}Q_{x,y} = \frac{1}{2 \pi} \frac{\partial \left< H \right>}{\partial J_{x,y}}\end{equation}

The amplitude detuning is given via a taylor expansion
(\cref{eq:taylor}) on the action dependent tune, via Ewen's talk
\href{https://indico.cern.ch/event/456856/contributions/1968808/attachments/1196949/1739919/2015-12-01_long-range-beam-beam.v2.pdf}{Amplitude
Detuning Measurements}:

\begin{equation}
\begin{aligned}
Q_z(\epsilon_x, \epsilon_y) = Q_{z0} &+ \left(\frac{\partial Q_z}{\partial \epsilon_x} \epsilon_x
                                                + \frac{\partial Q_z}{\partial \epsilon_y} \epsilon_y
                                                \right) \\
                                     &+ \frac{1}{2!} \left(\frac{\partial^2Q_z}{\partial \epsilon_x^2}\epsilon_x^2 
                                                          + 2 \frac{\partial^2Q_z}{\partial \epsilon_x \partial \epsilon_y}\epsilon_x \epsilon_y
                                                          + \frac{\partial^2Q_z}{\partial \epsilon_y^2}\epsilon_y^2\right) \\
                                     &+\frac{1}{3!}\biggl(\frac{\partial^{3} Q_z}{\partial \epsilon_{x}^{3}} \epsilon_{x}^{3}
                                          + 3 \frac{\partial^{3} Q_z}{\partial \epsilon_{y}\partial \epsilon_{x}^{2}} \epsilon_{x}^{2} \epsilon_{y} 
                                          + 3 \frac{\partial^{3} Q_z}{\partial \epsilon_{y}^{2}\partial \epsilon_{x}} \epsilon_{x} \epsilon_{y}^{2} 
                                          +  \frac{\partial^{3} Q_z}{\partial \epsilon_{y}^{3}} \epsilon_{y}^{3}
                                       \biggr) \\
\end{aligned}
\end{equation}

\newpage

\hypertarget{chromatic-amplitude-detuning}{%
\section{Chromatic Amplitude
Detuning}\label{chromatic-amplitude-detuning}}

Recap, the tune is:
\[Q_{x,y} = \frac{1}{2 \pi} \left< \frac{\partial H }{\partial J_{x,y}} \right>\]

Chromatic amplitude detuning is the change of tune relative to a change
of amplitude \emph{and} momentum. The tune then depends on
\(\epsilon_x, \epsilon_y, \delta\).\\
By doing a Taylor expansion (\cref{eq:taylor}), we can write the tune
as:

\begin{equation}
\begin{aligned}
Q_z(\epsilon_x, \epsilon_y, \delta) = Q_{z0} &+ \left[\frac{\partial Q_z}{\partial \epsilon_x} \epsilon_x
                                                 + \frac{\partial Q_z}{\partial \epsilon_y} \epsilon_y
                                                 + \colorbox{yellow!50}{$\displaystyle \frac{\partial Q_z}{\partial \delta}$} \delta
                                                \right] \\
                                             &+ \frac{1}{2!} \biggl[\frac{\partial^2Q_z}{\partial \epsilon_x^2}\epsilon_x^2 
                                                 + \frac{\partial^2Q_z}{\partial \epsilon_y^2}\epsilon_y^2
                                                 + \colorbox{yellow!50}{$\displaystyle \frac{\partial^2 Q_z}{\partial \delta^2}$} \delta^2  \\
                                             &\;\begin{aligned}
                                             \phantom{+ \frac{1}{2!} \biggl[}
                                               &+ 2 \frac{\partial^2Q_z}{\partial \epsilon_x \partial \epsilon_y}\epsilon_x \epsilon_y
                                                  + 2 \frac{\partial^2Q_z}{\partial \epsilon_x \partial \delta}\epsilon_x \delta
                                                  + 2 \frac{\partial^2Q_z}{\partial \delta \partial \epsilon_y} \delta \epsilon_y
                                             \biggr] \\
                                             \end{aligned} \\
                                             &+ \frac{1}{3!}
                                             \biggl[
                                                  \colorbox{yellow!50}{$\displaystyle \frac{\partial^3 Q_z}{\partial \delta^3}$}\delta^{3}
                                                  + \frac{\partial^{3} Q_z}{\partial \epsilon_{x}^{3}}  \epsilon_{x}^{3} 
                                                  + \frac{\partial^{3} Q_z}{\partial \epsilon_{y}^{3}}  \epsilon_{y}^{3} \\
                                             &\;\begin{aligned}
                                             \phantom{+ \frac{1}{3!} \biggl[} 
                                               &+ 3 \frac{\partial^{3} Q_z}{\partial \epsilon_{x}\partial \delta^{2}} \delta^{2} \epsilon_{x} 
                                                + 3  \frac{\partial^{3} Q_z}{\partial \epsilon_{y}\partial \delta^{2}}  \delta^{2} \epsilon_{y}
                                                + 3 \frac{\partial^{3} Q_z}{\partial \epsilon_{x}^{2}\partial \delta}  \delta \epsilon_{x}^{2} \\
                                               &+ 3 \frac{\partial^{3} Q_z}{\partial \epsilon_{y}^{2}\partial \delta} \delta \epsilon_{y}^{2}  
                                                + 3  \frac{\partial^{3} Q_z}{\partial \epsilon_{y}\partial \epsilon_{x}^{2}} \epsilon_{x}^{2} \epsilon_{y} 
                                                + 3 \frac{\partial^{3} Q_z}{\partial \epsilon_{y}^{2}\partial \epsilon_{x}} \epsilon_{x} \epsilon_{y}^{2} \\
                                               &+ 6 \frac{\partial^{3} Q_z}{\partial \epsilon_{y}\partial  \epsilon_{x}\partial \delta} \delta \epsilon_{x} \epsilon_{y} 
                                             \biggr]\\
                                             \end{aligned}
\end{aligned}
\end{equation}

We recognise the terms in yellow, seen before: the chromaticities
\(Q'\), \(Q''\), and \(Q'''\). It has been shown that only octupoles and
dodecapoles contribute to the amplitude detuning. The \emph{Chromatic}
Amplitude Detuning though also adds chromatic terms, we then have to
consider more multipoles.

\newpage

\hypertarget{sextupole-3}{%
\subsection{Sextupole}\label{sextupole-3}}

The change of tune induced by a sextupole is:

\begin{equation}Q_x = \frac{1}{4\pi} K_3 \beta_x \eta \delta L\end{equation}
\begin{equation}Q_y = - \frac{1}{4\pi} K_3 \beta_y \eta \delta L\end{equation}

We can now get the terms we're interested in:

\begin{equation}\begin{aligned}
\frac{\partial Q_x}{\partial J_x} = 0 \quad;&& \frac{\partial Q_x}{\partial J_y} = 0 \quad;&& \frac{\partial Q_x}{\partial \delta} = \frac{1}{4\pi}K_3\beta_x\eta L = Q_x'\\
\frac{\partial Q_y}{\partial J_x} = 0 \quad;&& \frac{\partial Q_y}{\partial J_y} = 0 \quad;&& \frac{\partial Q_y}{\partial \delta} = -\frac{1}{4\pi}K_3\beta_y\eta L = Q_y'\\
\end{aligned}\end{equation}

Contribution to the Chromatic Amplitude Detuning:

\begin{equation}
\begin{aligned}
Q_z(\epsilon_x, \epsilon_y, \delta) = \color{gray}Q_{z0} &+ \color{gray}\left[\frac{\partial Q_z}{\partial \epsilon_x} \epsilon_x
                                                 + \frac{\partial Q_z}{\partial \epsilon_y} \epsilon_y
                                                 + \textcolor{orange}{\frac{\partial Q_z}{\partial \delta} \delta}
                                                \right] \\
                                             &\color{gray}
                                             + \frac{1}{2!} \biggl[\frac{\partial^2Q_z}{\partial \epsilon_x^2}\epsilon_x^2 
                                                 + \frac{\partial^2Q_z}{\partial \epsilon_y^2}\epsilon_y^2
                                                 + \frac{\partial^2 Q_z}{\partial \delta^2} \delta^2  \\
                                             &\;\begin{aligned}
                                             \phantom{+ \frac{1}{2!} \biggl[}
                                               & \color{gray}
                                               + 2 \frac{\partial^2Q_z}{\partial \epsilon_x \partial \epsilon_y}\epsilon_x \epsilon_y
                                                  + 2 \frac{\partial^2Q_z}{\partial \epsilon_x \partial \delta}\epsilon_x \delta
                                                  + 2 \frac{\partial^2Q_z}{\partial \delta \partial \epsilon_y} \delta \epsilon_y
                                             \biggr] \\
                                             \end{aligned} \\
                                             &\color{gray}+ \frac{1}{3!}
                                             \biggl[
                                                  \frac{\partial^3 Q_z}{\partial \delta^3} \delta^{3}
                                                  + \frac{\partial^{3} Q_z}{\partial \epsilon_{x}^{3}}  \epsilon_{x}^{3} 
                                                  + \frac{\partial^{3} Q_z}{\partial \epsilon_{y}^{3}}  \epsilon_{y}^{3} \\
                                             &\;\begin{aligned}
                                             \phantom{+ \frac{1}{3!} \biggl[} 
                                               &\color{gray}
                                               + 3 \frac{\partial^{3} Q_z}{\partial \epsilon_{x}\partial \delta^{2}} \delta^{2} \epsilon_{x} 
                                                + 3  \frac{\partial^{3} Q_z}{\partial \epsilon_{y}\partial \delta^{2}}  \delta^{2} \epsilon_{y}
                                                + 3 \frac{\partial^{3} Q_z}{\partial \epsilon_{x}^{2}\partial \delta}  \delta \epsilon_{x}^{2} \\
                                               &\color{gray}
                                               + 3 \frac{\partial^{3} Q_z}{\partial \epsilon_{y}^{2}\partial \delta} \delta \epsilon_{y}^{2}  
                                                + 3  \frac{\partial^{3} Q_z}{\partial \epsilon_{y}\partial \epsilon_{x}^{2}} \epsilon_{x}^{2} \epsilon_{y} 
                                                + 3 \frac{\partial^{3} Q_z}{\partial \epsilon_{y}^{2}\partial \epsilon_{x}} \epsilon_{x} \epsilon_{y}^{2} \\
                                               &\color{gray}
                                               + 6 \frac{\partial^{3} Q_z}{\partial \epsilon_{y}\partial  \epsilon_{x}\partial \delta} \delta \epsilon_{x} \epsilon_{y} 
                                             \biggr]\\
                                             \end{aligned}
\end{aligned}
\end{equation}

\newpage

\hypertarget{octupole-2}{%
\subsection{Octupole}\label{octupole-2}}

From the hamiltonian of a normal octupole, with a displacement in \(x\)
(\cref{eq:hamiltonian_displacement_octupole}), we can change the
variable (\(x = \sqrt{2 J_x \beta_x} \cos \phi_x\) and
\(y = \sqrt{2 J_y \beta_y} \cos \phi_y\)):

\begin{equation}
\begin{aligned}
\mathcal{N}_4 = \frac{1}{24} K_4 \biggl[& \left(\sqrt{2J_x\beta_x} cos\phi_x\right)^4 \\
                                        & +4 \left(\sqrt{2 J_x \beta_x} \cos \phi_x\right)^3 \eta \delta \\
                                        & +6 \left(\sqrt{2 J_x \beta_x} \cos \phi_x\right)^2 \eta^2 \delta^2 \\
                                        & +4 \left(\sqrt{2 J_x \beta_x} \cos \phi_x\right) \eta^2 \delta \\
                                        & +\eta^4 \delta^4 \\
                                        & -6 \left(\sqrt{2 J_x \beta_x} \cos \phi_x\right)^2 \left(\sqrt{2 J_y \beta_y}\cos \phi_y\right)^2\\
                                        & -6 \left(\sqrt{2 J_y \beta_y} \cos \phi_y\right)^2 \cdot 2 \left(\sqrt{2 J_x \beta_x}\cos \phi_x\right) \eta \delta \\
                                        & -6 \left(\sqrt{2 J_y \beta_y} \cos \phi_y\right)^2 \eta^2 \delta^2 \\
                                        & + \left(\sqrt{2 J_y \beta_y} \cos \phi_y\right)^4
                                  \biggr]\\
\end{aligned}
\end{equation}

We can now average over the phase variables: \begin{equation}
\begin{aligned}
\left< \mathcal{N}_4 \right> = \frac{1}{24} K_4 \biggl[& \frac{3}{2} J_x^2\beta_x^2 \\
                                                       & +6 J_x \beta_x \eta^2 \delta^2 \\
                                                       & +\eta^4 \delta^4 \\
                                                       & -6 J_x \beta_x J_y \beta_y\\
                                                       & -6 J_y \beta_y \eta^2 \delta^2 \\
                                                       & + \frac{3}{2} J_y^2 \beta_y^2
                                                 \biggr]\\
\end{aligned}
\end{equation}

The tunes then are:

\begin{equation}\begin{aligned}
Q_x = \frac{1}{2\pi} \frac{\partial \left< \mathcal{N_4} \right>}{\partial J_x} &= \frac{1}{48\pi} K_4 \biggl[3 J_x \beta_x^2
                                                                                                             +6 \beta_x \eta^2 \delta^2
                                                                                                             -6 \beta_x J_y  \beta_y 
                                                                                                      \biggr]\\
Q_y = \frac{1}{2\pi} \frac{\partial \left< \mathcal{N_4} \right>}{\partial J_y} &= \frac{1}{48\pi} K_4 \biggl[-6 J_x \beta_x \beta_y
                                                                                                             -6 \beta_y \eta^2 \delta^2
                                                                                                             +3 J_y \beta_y^2
                                                                                                      \biggr]
\end{aligned}\end{equation}

\begin{equation}\begin{aligned}
  \frac{\partial Q_x}{\partial J_x} =& \frac{1}{16\pi} K_4 \beta_x^2 &&;\quad 
  \frac{\partial Q_x}{\partial J_y} =& -\frac{1}{8\pi} K_4 \beta_x\beta_y &&;\quad
  \frac{\partial^2 Q_x}{\partial \delta^2} =& \frac{1}{4\pi} K_4 \beta_x \eta^2  = Q_x''
\\
  \frac{\partial Q_y}{\partial J_x} =& -\frac{1}{8\pi} K_4 \beta_x \beta_y &&;\quad
  \frac{\partial Q_y}{\partial J_y} =& \frac{1}{16\pi} K_4 \beta_y^2 &&;\quad
  \frac{\partial^2 Q_y}{\partial \delta^2} =& -\frac{1}{4\pi} K_4 \beta_y \eta^2  = Q_y''
\\
\end{aligned}\end{equation}

Contribution to the Chromatic Amplitude Detuning:

\begin{equation}
\begin{aligned}
Q_z(\epsilon_x, \epsilon_y, \delta) = \color{gray}Q_{z0} &\color{gray}+
                                                \textcolor{orange}{\biggl[}
                                                   \textcolor{orange}{\frac{\partial Q_z}{\partial \epsilon_x} \epsilon_x}
                                                 + \textcolor{orange}{\frac{\partial Q_z}{\partial \epsilon_y} \epsilon_y}
                                                 + \frac{\partial Q_z}{\partial \delta} \delta
                                                \textcolor{orange}{\biggr]} \\
                                             &\color{gray}
                                             + \textcolor{orange}{\frac{1}{2!} \biggl[}
                                                   \frac{\partial^2Q_z}{\partial \epsilon_x^2}\epsilon_x^2 
                                                 + \frac{\partial^2Q_z}{\partial \epsilon_y^2}\epsilon_y^2
                                                 + \textcolor{orange}{\frac{\partial^2 Q_z}{\partial \delta^2} \delta^2}  \\
                                             &\;\begin{aligned}
                                             \phantom{+ \frac{1}{2!} \biggl[}
                                               & \color{gray}
                                               + 2 \frac{\partial^2Q_z}{\partial \epsilon_x \partial \epsilon_y}\epsilon_x \epsilon_y
                                                  + 2 \frac{\partial^2Q_z}{\partial \epsilon_x \partial \delta}\epsilon_x \delta
                                                  + 2 \frac{\partial^2Q_z}{\partial \delta \partial \epsilon_y} \delta \epsilon_y
                                             \textcolor{orange}{\biggr]} \\
                                             \end{aligned} \\
                                             &\color{gray}+ \frac{1}{3!}
                                             \biggl[
                                                  \frac{\partial^3 Q_z}{\partial \delta^3} \delta^{3}
                                                  + \frac{\partial^{3} Q_z}{\partial \epsilon_{x}^{3}}  \epsilon_{x}^{3} 
                                                  + \frac{\partial^{3} Q_z}{\partial \epsilon_{y}^{3}}  \epsilon_{y}^{3} \\
                                             &\;\begin{aligned}
                                             \phantom{+ \frac{1}{3!} \biggl[} 
                                               &\color{gray}
                                               + 3 \frac{\partial^{3} Q_z}{\partial \epsilon_{x}\partial \delta^{2}} \delta^{2} \epsilon_{x} 
                                                + 3  \frac{\partial^{3} Q_z}{\partial \epsilon_{y}\partial \delta^{2}}  \delta^{2} \epsilon_{y}
                                                + 3 \frac{\partial^{3} Q_z}{\partial \epsilon_{x}^{2}\partial \delta}  \delta \epsilon_{x}^{2} \\
                                               &\color{gray}
                                               + 3 \frac{\partial^{3} Q_z}{\partial \epsilon_{y}^{2}\partial \delta} \delta \epsilon_{y}^{2}  
                                                + 3  \frac{\partial^{3} Q_z}{\partial \epsilon_{y}\partial \epsilon_{x}^{2}} \epsilon_{x}^{2} \epsilon_{y} 
                                                + 3 \frac{\partial^{3} Q_z}{\partial \epsilon_{y}^{2}\partial \epsilon_{x}} \epsilon_{x} \epsilon_{y}^{2} \\
                                               &\color{gray}
                                               + 6 \frac{\partial^{3} Q_z}{\partial \epsilon_{y}\partial  \epsilon_{x}\partial \delta} \delta \epsilon_{x} \epsilon_{y} 
                                             \biggr]\\
                                             \end{aligned}
\end{aligned}
\end{equation}

\newpage

\hypertarget{decapole-2}{%
\subsection{Decapole}\label{decapole-2}}

The normal field of a decapole has been calculated in
\cref{eq:decapole_expanded}:

\[\begin{aligned}
\mathcal{N_5}(x, y) = \frac{1}{120} K_{5} \biggl[&
  \eta^5\delta^5 + 5\eta^4\delta^4x + 10\eta^3\delta^3x^2 + 10\eta^2\delta^2 x^3 + 5\eta\delta x^4 + x^5 \\
  & -10y^2 (\eta^3\delta^3 + 3\eta^2\delta^2x + 3\eta\delta x^2 + x^3)\\
  & +5y^4 (x + \eta\delta) \biggr]
\end{aligned}\]

Changing variables (\(x \rightarrow \sqrt{2 J_x \beta_x} \cos\phi_x\);
\(y \rightarrow \sqrt{2 J_y \beta_y} \cos\phi_y\)):

\begin{equation}\begin{aligned}
\mathcal{N_5}(x, y) = \frac{1}{120} K_{5} 
  \biggl[&
        \eta^5\delta^5 + 5\eta^4\delta^4\left(\sqrt{2 J_x \beta_x} \cos \phi_x\right) \\
        &+ 10\eta^3\delta^3\left(\sqrt{2 J_x \beta_x} \cos \phi_x\right)^2 + 10\eta^2\delta^2 \left(\sqrt{2 J_x \beta_x} \cos \phi_x\right)^3 \\
        &+ 5\eta\delta \left(\sqrt{2 J_x \beta_x} \cos \phi_x\right)^4 + \left(\sqrt{2 J_x \beta_x} \cos \phi_x\right)^5 \\
        &- 10\left(\sqrt{2 J_y \beta_y} \cos \phi_y\right)^2 \biggl[(\eta^3\delta^3 + 3\eta^2\delta^2\left(\sqrt{2 J_x \beta_x} \cos \phi_x\right) \\
        &\phantom{- 10\left(\sqrt{2 J_y \beta_y} \cos \phi_y\right)^2 \biggl[}+ 3\eta\delta \left(\sqrt{2 J_x \beta_x} \cos \phi_x\right)^2 + \left(\sqrt{2 J_x \beta_x} \cos \phi_x\right)^3\biggr]\\
        & +5\left(\sqrt{2 J_y \beta_y} \cos \phi_y\right)^4 \left(\left(\sqrt{2 J_x \beta_x} \cos \phi_x\right) + \eta\delta\right) 
  \biggr]
\end{aligned}\end{equation}

Averaging over the phase variables: \begin{equation}\begin{aligned}
\mathcal{N_5}(x, y) = \frac{1}{120} K_{5} 
  \biggl[
         & \eta^5\delta^5 
          + 10 \eta^3 \delta^3 J_x \beta_x \\
         & + \frac{15}{2} \eta \delta J_x^2 \beta_x^2
          - 10 J_y \beta_y \eta^3 \delta^3 \\
         & - 30 J_y \beta_y \eta \delta J_x \beta_x 
          + \frac{15}{2} J_y^2 \beta_y^2 \eta \delta
  \biggr]
\end{aligned}\end{equation}

The tunes then are:

\begin{equation}\begin{aligned}
Q_x = \frac{1}{2\pi} \frac{\partial \left< \mathcal{N_5} \right>}{\partial J_x} &= \frac{1}{240\pi} K_5 \biggl[10 \eta^3 \delta^3 \beta_x
                                                                                                               + 15 \eta \delta J_x \beta_x^2
                                                                                                               - 30 J_y \beta_y \beta_x \eta \delta
                                                                                                      \biggr]\\
Q_y = \frac{1}{2\pi} \frac{\partial \left< \mathcal{N_5} \right>}{\partial J_y} &= \frac{1}{240\pi} K_5 \biggl[-10 \eta^3 \delta^3 \beta_y
                                                                                                               + 15 \eta \delta J_y \beta_y^2
                                                                                                               -30 J_x \beta_y \beta_x \eta \delta
                                                                                                      \biggr]
\end{aligned}\end{equation}

We can now calculate our chromatic amplitude detuning terms:

\begin{equation}\begin{aligned}
  \frac{\partial^2 Q_x}{\partial J_x \partial \delta} =& \frac{1}{16 \pi} K_5 \beta_x^2 \eta &&;\quad 
  \frac{\partial^2 Q_x}{\partial J_y \partial \delta} =& -\frac{1}{8\pi} K_5 \beta_x \beta_y \eta&&;\quad
  \frac{\partial^3 Q_x}{\partial \delta^3} =& \frac{1}{4\pi} K_5 \beta_x \eta^3 = Q_x''' 
\\
  \frac{\partial^2 Q_y}{\partial J_x \partial \delta} =& -\frac{1}{8\pi} K_5 \beta_x \beta_y \eta&&;\quad
  \frac{\partial^2 Q_y}{\partial J_y \partial \delta} =& \frac{1}{16 \pi} K_5 \beta_y^2 \eta &&;\quad 
  \frac{\partial^3 Q_y}{\partial \delta^3} =& -\frac{1}{4\pi} K_5 \beta_y \eta^3 = Q_y'''
\\
\end{aligned}\end{equation}

Contribution to the Chromatic Amplitude Detuning:

\begin{equation}
\begin{aligned}
Q_z(\epsilon_x, \epsilon_y, \delta) = \color{gray}Q_{z0} &\color{gray}+
                                                \left[
                                                   \frac{\partial Q_z}{\partial \epsilon_x} \epsilon_x
                                                 + \frac{\partial Q_z}{\partial \epsilon_y} \epsilon_y
                                                 + \frac{\partial Q_z}{\partial \delta} \delta
                                                \right] \\
                                             &\color{gray}
                                             + \textcolor{orange}{\frac{1}{2!} \biggl[}
                                                   \frac{\partial^2Q_z}{\partial \epsilon_x^2}\epsilon_x^2 
                                                 + \frac{\partial^2Q_z}{\partial \epsilon_y^2}\epsilon_y^2
                                                 + \frac{\partial^2 Q_z}{\partial \delta^2} \delta^2  \\
                                             &\;\begin{aligned}
                                             \phantom{+ \frac{1}{2!} \biggl[}
                                               & \color{gray}
                                               + 2 \frac{\partial^2Q_z}{\partial \epsilon_x \partial \epsilon_y}\epsilon_x \epsilon_y
                                               + \textcolor{orange}{2 \frac{\partial^2Q_z}{\partial \epsilon_x \partial \delta}\epsilon_x \delta}
                                               + \textcolor{orange}{2 \frac{\partial^2Q_z}{\partial \delta \partial \epsilon_y} \delta \epsilon_y}
                                             \textcolor{orange}{\biggr]} \\
                                             \end{aligned} \\
                                             &\color{gray}+ \textcolor{orange}{\frac{1}{3!}
                                             \biggl[}
                                                  \textcolor{orange}{\frac{\partial^3 Q_z}{\partial \delta^3} \delta^{3}}
                                                  + \frac{\partial^{3} Q_z}{\partial \epsilon_{x}^{3}}  \epsilon_{x}^{3} 
                                                  + \frac{\partial^{3} Q_z}{\partial \epsilon_{y}^{3}}  \epsilon_{y}^{3} \\
                                             &\;\begin{aligned}
                                             \phantom{+ \frac{1}{3!} \biggl[} 
                                               &\color{gray}
                                               + 3 \frac{\partial^{3} Q_z}{\partial \epsilon_{x}\partial \delta^{2}} \delta^{2} \epsilon_{x} 
                                                + 3  \frac{\partial^{3} Q_z}{\partial \epsilon_{y}\partial \delta^{2}}  \delta^{2} \epsilon_{y}
                                                + 3 \frac{\partial^{3} Q_z}{\partial \epsilon_{x}^{2}\partial \delta}  \delta \epsilon_{x}^{2} \\
                                               &\color{gray}
                                               + 3 \frac{\partial^{3} Q_z}{\partial \epsilon_{y}^{2}\partial \delta} \delta \epsilon_{y}^{2}  
                                                + 3  \frac{\partial^{3} Q_z}{\partial \epsilon_{y}\partial \epsilon_{x}^{2}} \epsilon_{x}^{2} \epsilon_{y} 
                                                + 3 \frac{\partial^{3} Q_z}{\partial \epsilon_{y}^{2}\partial \epsilon_{x}} \epsilon_{x} \epsilon_{y}^{2} \\
                                               &\color{gray}
                                               + 6 \frac{\partial^{3} Q_z}{\partial \epsilon_{y}\partial  \epsilon_{x}\partial \delta} \delta \epsilon_{x} \epsilon_{y} 
                                             \textcolor{orange}{\biggr]}\\
                                             \end{aligned}
\end{aligned}
\end{equation}

\newpage

\hypertarget{dodecapole-2}{%
\subsection{Dodecapole}\label{dodecapole-2}}

The main normal field of a dodecapole is:

\begin{equation}\mathcal{N_6}(x,y) = \frac{1}{720} K_6 (x^6 - 15x^4y^2 + 15x^2y^4 -y^6)\end{equation}

With a displacement in \(x \rightarrow x + \eta \delta\):
\begin{equation}\mathcal{N_6}(x,y) = \frac{1}{720} K_6 \biggl[(x + \eta \delta)^6 - 15(x + \eta \delta)^4y^2 + 15(x + \eta \delta)^2y^4 -y^6\biggr]\end{equation}

Expanded form, afdter having removed odd exponents. Those exponents
would yield an average of \(0\) for the cosines:

\begin{equation}\begin{aligned}
  \mathcal{N_6}(x,y) = \frac{1}{720} K_6 
                                \biggl[
                                  &x^6 + 15x^2 \eta^4 \delta^4 + 15 x^4 \eta^2 \delta^2 + \eta^6 \delta^6 \\
                                  &-15y^2 (\eta^4 \delta^4 + 4x \eta^3 \delta^3 + 6x^2 \eta^2 \delta^2 + 4x^3 \eta \delta+ x^4) \\
                                  &+15y^4 (x^2 + \eta^2 \delta^2) \\
                                  &-y^6
                                \biggr]
\end{aligned}\end{equation}

Changing variables (\(x \rightarrow \sqrt{2 J_x \beta_x} \cos\phi_x\);
\(y \rightarrow \sqrt{2 J_y \beta_y} \cos\phi_y\)):
\begin{equation}\begin{aligned}
  \mathcal{N_6}(x,y) = \frac{1}{720} K_6 
                                \biggl[
                                  &\left(\sqrt{2 J_x \beta_x} \cos \phi_x\right)^6 \\
                                  &+ 15\left(\sqrt{2 J_x \beta_x} \cos \phi_x\right)^2 \eta^4 \delta^4 \\
                                  &+ 15 \left(\sqrt{2 J_x \beta_x} \cos \phi_x\right)^4 \eta^2 \delta^2 \\
                                  &+ \eta^6 \delta^6 \\
                                  &-15\left(\sqrt{2 J_y \beta_y} \cos \phi_y\right)^2 \bigg(\eta^4 \delta^4 \\
                                      &\;\begin{aligned}
                                      \phantom{-15\left(\sqrt{2 J_y \beta_y} \cos \phi_y\right)^2 \bigg(}
                                        &+ 6\left(\sqrt{2 J_x \beta_x} \cos \phi_x\right)^2 \eta^2 \delta^2 \\
                                        &+  \left(\sqrt{2 J_x \beta_x} \cos \phi_x\right)^4
                                      \biggr) \\
                                      \end{aligned} \\
                                  &+15\left(\sqrt{2 J_y \beta_y} \cos \phi_y\right)^4 \biggl(\left(\sqrt{2 J_x \beta_x} \cos \phi_x\right)^2 + \eta^2 \delta^2 \biggr) \\
                                  &-  \left(\sqrt{2 J_y \beta_y} \cos \phi_y\right)^6
                                \biggr]
\end{aligned}\end{equation}

Averaging over the phase variables: \begin{equation}\begin{aligned}
\left< \mathcal{N_6}(x,y) \right> = \frac{1}{720} K_6 
                                \biggl[
                                  &\frac{5}{2} J_x^3 \beta_x^3 \\
                                  &+ 15 \cdot J_x \beta_x \eta^4 \delta^4 \\
                                  &+ 15 \cdot J_x^2 \beta_x^2 \frac{3}{2} \eta^2 \delta^2 \\
                                  &+ \eta^6 \delta^6 \\
                                  &-15\left(J_y \beta_y \right) \bigg(\eta^4 \delta^4 \\
                                      &\;\begin{aligned}
                                      \phantom{-15\left(J_y \beta_y \right) \bigg(}
                                        &+ 6\left(J_x \beta_x \right) \eta^2 \delta^2 \\
                                        &+  \left(J_x^2 \beta_x^2 \frac{3}{2}\right)
                                      \biggr) \\
                                      \end{aligned} \\
                                  &+ 15\left(J_y^2 \beta_y^2 \frac{3}{2} \right) \biggl(J_x \beta_x + \eta^2 \delta^2 \biggr) \\
                                  &- \frac{5}{2} J_y^3 \beta_y^3
                                \biggr]
\end{aligned}\end{equation}

The tunes then are:

\begin{equation}\begin{aligned}
Q_x = \frac{1}{2\pi} \frac{\partial \left< \mathcal{N_6} \right>}{\partial J_x} = \frac{1}{1440\pi} K_6 \biggl[
                                 & \frac{15}{2} J_x^2 \beta_x^3\\
                                 & +15 \beta_x \eta^4 \delta^4 \\
                                 & +45 J_x \beta_x^2 \eta^2 \delta^2 \\
                                 & -90 J_y \beta_y \beta_x \eta^2 \delta^2 \\
                                 & -45 J_y \beta_y J_x \beta_x^2\\
                                 & + 15 \cdot \frac{3}{2} J_y^2 \beta_y^2 \beta_x
                                                                                                      \biggr]\\
Q_y = \frac{1}{2\pi} \frac{\partial \left< \mathcal{N_6} \right>}{\partial J_y} = \frac{1}{1440\pi} K_6 \biggl[
                                 & -15 \beta_y \eta^4 \delta^4 \\
                                 & -90 \beta_y J_x \beta_x \eta^2 \delta^2 \\
                                 & -15 \cdot \frac{3}{2}  \beta_y J_x^2 \beta_x^2 \\
                                 & +45 J_y \beta_y^2 J_x \beta_x \\
                                 & +45 J_y \beta_y^2 \eta^2 \delta^2 \\
                                 & -\frac{15}{2} J_y^2 \beta_y ^3 
                                                                                                      \biggr]
\end{aligned}\end{equation}

We can now calculate our chromatic amplitude detuning terms. Since there
are many terms, I'm going to split them here. First, \(Q_x\):

\begin{equation}\begin{aligned}
  \frac{\partial^2 Q_x}{\partial J_x^2} =&\; \frac{1}{96\pi} K_6 \beta_x^3 \\
  \frac{\partial^3 Q_x}{\partial J_x \partial \delta^2} =&\; \frac{1}{16\pi} K_6 \beta_x^2 \eta^2\\
  \cline{1-2}\\[-5\jot]
  \frac{\partial^2 Q_x}{\partial J_y^2} =&\; \frac{1}{32\pi} K_6 \beta_y^2\beta_x \\
  \frac{\partial^3 Q_x}{\partial J_y \partial \delta^2} =&\; -\frac{1}{8\pi} K_6 \beta_y \beta_x \eta^2\\
  \cline{1-2}\\[-5\jot]
  \frac{\partial^2 Q_x}{\partial J_x \partial J_y} =&\; -\frac{1}{32\pi} K_6 \beta_y \beta_x^2\\[2\jot]
\end{aligned}\end{equation}

Then \(Q_y\): \begin{equation}\begin{aligned}
  \frac{\partial^2 Q_y}{\partial J_y^2} =&\; -\frac{1}{96\pi} K_6 \beta_y^3 \\
  \frac{\partial^3 Q_y}{\partial J_y \partial \delta^2} =&\; \frac{1}{16\pi} K_6 \beta_y^2 \eta^2\\
  \cline{1-2}\\[-5\jot]
  \frac{\partial^2 Q_y}{\partial J_x^2} =&\; -\frac{1}{32\pi} K_6 \beta_y\beta_x^2 \\
  \frac{\partial^3 Q_y}{\partial J_x \partial \delta^2} =&\; -\frac{1}{8\pi} K_6 \beta_y \beta_x \eta^2\\
  \cline{1-2}\\[-5\jot]
  \frac{\partial^2 Q_y}{\partial J_y \partial J_x} =&\; \frac{1}{32\pi} K_6 \beta_y^2 \beta_x\\[2\jot]
\end{aligned}\end{equation}

Then the chromaticity: \begin{equation}\begin{aligned}
  \frac{\partial^4 Q_x}{\partial \delta^4} =&\; \frac{1}{4\pi} K_6 \beta_x \eta^4 = Q_x''''\\
  \frac{\partial^4 Q_y}{\partial \delta^4} =&\; -\frac{1}{4\pi} K_6 \beta_y \eta^4 = Q_y''''\\
\end{aligned}\end{equation}

\vspace{2cm}

Contribution to the Chromatic Amplitude Detuning:

\begin{equation}
\begin{aligned}
Q_z(\epsilon_x, \epsilon_y, \delta) = \color{gray}Q_{z0} &\color{gray}+
                                                \left[
                                                   \frac{\partial Q_z}{\partial \epsilon_x} \epsilon_x
                                                 + \frac{\partial Q_z}{\partial \epsilon_y} \epsilon_y
                                                 + \frac{\partial Q_z}{\partial \delta} \delta
                                                \right] \\
                                             &\color{gray}
                                             + \textcolor{orange}{\frac{1}{2!} \biggl[}
                                                   \textcolor{orange}{\frac{\partial^2Q_z}{\partial \epsilon_x^2}\epsilon_x^2}
                                                 + \textcolor{orange}{\frac{\partial^2Q_z}{\partial \epsilon_y^2}\epsilon_y^2}
                                                 + \frac{\partial^2 Q_z}{\partial \delta^2} \delta^2  \\
                                             &\;\begin{aligned}
                                             \phantom{+ \frac{1}{2!} \biggl[}
                                               & \color{gray}
                                               + \textcolor{orange}{2 \frac{\partial^2Q_z}{\partial \epsilon_x \partial \epsilon_y}\epsilon_x \epsilon_y}
                                               + 2 \frac{\partial^2Q_z}{\partial \epsilon_x \partial \delta}\epsilon_x \delta
                                               + 2 \frac{\partial^2Q_z}{\partial \delta \partial \epsilon_y} \delta \epsilon_y
                                             \textcolor{orange}{\biggr]} \\
                                             \end{aligned} \\
                                             &\color{gray}+ \textcolor{orange}{\frac{1}{3!}
                                             \biggl[}
                                                  \frac{\partial^3 Q_z}{\partial \delta^3} \delta^{3}
                                                  + \frac{\partial^{3} Q_z}{\partial \epsilon_{x}^{3}}  \epsilon_{x}^{3} 
                                                  + \frac{\partial^{3} Q_z}{\partial \epsilon_{y}^{3}}  \epsilon_{y}^{3} \\
                                             &\;\begin{aligned}
                                             \phantom{+ \frac{1}{3!} \biggl[} 
                                               &\color{gray}
                                                + \textcolor{orange}{3 \frac{\partial^{3} Q_z}{\partial \epsilon_{x}\partial \delta^{2}} \delta^{2} \epsilon_{x}}
                                                + \textcolor{orange}{3  \frac{\partial^{3} Q_z}{\partial \epsilon_{y}\partial \delta^{2}}  \delta^{2} \epsilon_{y}}
                                                + 3 \frac{\partial^{3} Q_z}{\partial \epsilon_{x}^{2}\partial \delta}  \delta \epsilon_{x}^{2} \\
                                               &\color{gray}
                                               + 3 \frac{\partial^{3} Q_z}{\partial \epsilon_{y}^{2}\partial \delta} \delta \epsilon_{y}^{2}  
                                                + 3  \frac{\partial^{3} Q_z}{\partial \epsilon_{y}\partial \epsilon_{x}^{2}} \epsilon_{x}^{2} \epsilon_{y} 
                                                + 3 \frac{\partial^{3} Q_z}{\partial \epsilon_{y}^{2}\partial \epsilon_{x}} \epsilon_{x} \epsilon_{y}^{2} \\
                                               &\color{gray}
                                               + 6 \frac{\partial^{3} Q_z}{\partial \epsilon_{y}\partial  \epsilon_{x}\partial \delta} \delta \epsilon_{x} \epsilon_{y} 
                                              \textcolor{orange}{\biggr]}\\
                                             \end{aligned}\\
                                         &\color{gray}
                                         + \textcolor{orange}{\frac{1}{4!} \biggl[ \frac{\partial^4 Q_z}{\partial \delta^4} \delta^4 \biggr] }
\end{aligned}
\end{equation}

\newpage

\hypertarget{ptc-check}{%
\subsection{PTC check}\label{ptc-check}}

A simulation has been done with PTC to assess that those equations are
correct. A dodecapole has been added to the lattice with a strength
\(KL = 1e^6\). Here are the results, confirming PTC works as intended.

The ANH numbers refer to the partial derivative relative to \(J_x\),
\(J_y\) and \(\delta\). So ANHX 021 would for example be
\(\dfrac{\partial^3 Q_x}{\partial J_y^2 \partial \delta}\).

\begin{center}
\begin{tabular}{lrrr}
\toprule
       Term &         Analytical &      Simulation & Rel. Diff [\%] \\
\midrule
  ANH X 200 &      4782639.96971 &      4782639.97 &           0.0 \\
  ANH X 102 &       86945.930342 &        86945.93 &          -0.0 \\
  ANH X 020 &   593469879.552116 &    593469880.01 &           0.0 \\
  ANH X 012 &    -1118366.433407 &    -1118366.433 &          -0.0 \\
  ANH X 110 &   -92277073.535598 &     -92277073.6 &           0.0 \\
  ANH X 004 &        1053.754809 &       1053.7548 &     -0.000001 \\
            &                    &                 &               \\
  ANH Y 200 &   -92277073.535598 &     -92277073.6 &           0.0 \\
  ANH Y 102 &    -1118366.433407 &    -1118366.433 &          -0.0 \\
  ANH Y 020 & -1272278817.264865 & -1272278818.913 &           0.0 \\
  ANH Y 012 &     3596325.539479 &     3596325.543 &           0.0 \\
  ANH Y 110 &   593469879.552116 &    593469880.01 &           0.0 \\
  ANH Y 004 &       -6777.108503 &      -6777.1085 &          -0.0 \\
\bottomrule
\end{tabular}
\end{center}

\newpage

\hypertarget{resonance-driving-terms}{%
\section{Resonance Driving Terms}\label{resonance-driving-terms}}

The Resonance Driving Terms are derived from the multinomial expansion
of the multipole expansion itself.

\hypertarget{derivation}{%
\subsection{Derivation}\label{derivation}}

This derivation is based on
\href{https://journals.aps.org/prab/pdf/10.1103/PhysRevSTAB.17.074001}{Andrea
Franchi's Resonance Driving Terms} paper. The derivation has been
completely redone here, only the notation is followed to stay
consistent.

As a reminder, the multipole expansion is given below, where we'll only
consider one type of multipole, so a fixed \(n\):
\begin{equation}H = \Re \left[(K_n + iJ_n)\frac{(x+iy)^n}{n!} \right]\end{equation}

The variables \(x\) and \(y\) can be expressed as phase action
variables:

\hl{this is $x_{\pm} = \hat{z} \pm i \hat{p}_z$ right?}

\begin{equation}
\begin{aligned}
  x(s) &= \sqrt{2 J_x \beta_x} \cos(\phi_x + \phi_{x0})\\
  y(s) &= \sqrt{2 J_y \beta_y} \cos(\phi_y + \phi_{y0})
\end{aligned}
\end{equation}

Where \(s\) indicates the observed point and \(x_0\) the location of the
magnet.

Using Euler's formula we can rewrite those variables: \begin{equation}
\begin{aligned}
  x &= \sqrt{2 J_x \beta_x} \frac{e^{i(\phi_x + \phi_{x0})} + e^{-i(\phi_x + \phi_{x0})}}{2}\\
  y &= \sqrt{2 J_y \beta_y} \frac{e^{i(\phi_y + \phi_{y0})} + e^{-i(\phi_y + \phi_{y0})}}{2}\\
\end{aligned}
\end{equation}

Which then gives us:

\begin{equation}
\begin{aligned}
H = \Re \biggl[\frac{1}{2^n \cdot n!}  (K_n + iJ_n)\biggl(&\sqrt{2 J_x \beta_x} e^{i(\phi_x + \phi_{x0})}\\
                                                      &+ \sqrt{2 J_x \beta_x} e^{-i(\phi_x + \phi_{x0})}\\
                                                      &+ i\sqrt{2 J_y \beta_y} e^{i(\phi_y+ \phi_{y0})} \\
                                                      &+ i\sqrt{2 J_y \beta_y} e^{-i(\phi_y + \phi_{y0})} 
                                                \biggr)^n\biggr]
\end{aligned}
\end{equation}

Now, we can do the multinomial expansion of last term via
\cref{eq:multinomial_expansion}:

\begin{equation}(a + b + c + d)^n = \sum_{j + k + l + m = n} \frac{n!}{j!k!l!m!} a^j b^k c^l d^m\end{equation}

\begin{equation}\begin{aligned}
a^j &= \left(\sqrt{2 J_x \beta_x} e^{i(\phi_x + \phi_{x0})} \right)^j\\
    &= 2^{\frac{j}{2}} J_x^{\frac{j}{2}} \beta_x^{\frac{j}{2}} e^{ij(\phi_x + \phi_{x0})},\\
b^k &= \left(\sqrt{2 J_x \beta_x} e^{-i(\phi_x + \phi_{x0})}\right)^k \\
    &= (2J_x)^{\frac{k}{2}} \beta_x^{\frac{k}{2}} e^{-ik(\phi_x + \phi_{x0})}, \\
c^l &= \left(i \sqrt{2 J_y \beta_y} e^{i(\phi_y + \phi_{y0}}\right)^l \\
    &= i^l  (2J_y)^{\frac{l}{2}} \beta_y^{\frac{l}{2}} e^{il(\phi_y + \phi_{y0})}, \\
d^m &= \left( \sqrt{2 J_y \beta_y} e^{i(\phi_y + \phi_{y0})}\right)^m \\
    &= i^m (2J_y)^{\frac{m}{2}} \beta_y^{\frac{m}{2}} e^{-im(\phi_y + \phi_{y0})} \\
\end{aligned}\end{equation}

\begin{equation}\begin{aligned}
a^j b^k &= \phantom{i^{l+m}} (2J_x)^{\frac{j+k}{2}} \beta_x^{\frac{j+k}{2}} e^{i(j-k)(\phi_x + \phi_{x0})}, \\
c^l d^m &= i^{l+m} (2J_y)^{\frac{l+m}{2}} \beta_y^{\frac{l+m}{2}} e^{i(l-m)(\phi_y + \phi_{y0})}
\end{aligned}\end{equation}

\begin{equation}\begin{aligned}
a^j b^k c^l d^m &= i^{l+m} 
                   (2J_x)^{\frac{j+k}{2}} (2J_y)^{\frac{l+m}{2}} 
                   \beta_x^{\frac{j+k}{2}} \beta_y^{\frac{l+m}{2}} 
                   e^{i\left[ (j-k)(\phi_x + \phi_{x0}) + (l-m)(\phi_y + \phi_{y0} \right]}
\end{aligned}\end{equation}

We can now isolate the terms that are independant of the position \(s\):
\begin{equation}\begin{aligned}
a^j b^k c^l d^m &= 
                   i^{l+m} 
                   \beta_x^{\frac{j+k}{2}} \beta_y^{\frac{l+m}{2}} 
                   e^{i\left[ (j-k)\phi_x + (l-m)\phi_y \right]}
                   (2J_x)^{\frac{j+k}{2}} (2J_y)^{\frac{l+m}{2}} 
                   e^{i\left[ (j-k)\phi_{x0} + (l-m)\phi_{y0} \right]}
\end{aligned}\end{equation}

Which gets us to \(H_w\), the Hamiltonian for an element \(w\), noting
that \(i\) is the imaginary unit (\(i^2 = -1\)):

\begin{equation}\begin{aligned}
  H_w &=  \Re\left[\frac{1}{2^n \cdot n!} (K_{nw} + iJ_{nw}) (a_w + b_w + c_w + d_w)^n \right] \\
    &=  \Re\left[ 
         \frac{1}{2^n \cdot n!} 
         (K_{nw} + iJ_{nw}) 
         \sum_{j + k + l + m = n} 
         \frac{n!}{j!k!l!m!} a^j_w b^k_w c^l_w d^m_w \right] \\
    &=  \Re\left[ 
         (K_{nw} + iJ_{nw}) 
         \sum_{j + k + l + m = n} 
         \frac{1}{2^{j+k+l+m} \cdot j!k!l!m!} a^j_w b^k_w c^l_w d^m_w \right] \\
    &=  \Re\biggl[ 
         (K_{nw} + iJ_{nw})
         \sum_{j + k + l + m = n}
         \frac{1}{2^{j+k+l+m} \cdot j!k!l!m!} \\
    &\begin{aligned}\phantom{ \Re\biggl[ (K_{nw} + iJ_{nw})  \sum_{j + k + l + m = n} \quad}
           &i^{l+m} 
           \beta_{xw}^{\frac{j+k}{2}} \beta_{yw}^{\frac{l+m}{2}} 
           e^{i\left[ (j-k)\phi_{x} + (l-m)\phi_{y} \right]}\\
           &(2J_x)^{\frac{j+k}{2}} (2J_y)^{\frac{l+m}{2}} 
           e^{i\left[ (j-k)\phi_{x0} + (l-m)\phi_{y0} \right]}
           \biggr]
    \end{aligned}
\end{aligned}\label{eq:hamiltonian_multinomial_expanded}\end{equation}

We can see that the term \(i^{l+m}\) in the sum will have a decisive
impact on the result, being multiplied either by \(K_{nw}\) or
\(iJ_{nw}\). The real part of the sum is then directly influenced by the
parity of (\(l+m\)).\\
The equation can be rewritten taking this into account, effectively
selecting between the two terms:

\begin{equation}
\Omega(i) =
\begin{cases} 
  1, & \mbox{if } i \mbox{ is even}, \\
  0, & \mbox{if } i \mbox{ is odd}.
\end{cases}
\end{equation}

\begin{equation}\begin{aligned}
 H_w  &=  
         \sum_{j + k + l + m = n}
        \biggl[ 
         \frac{ K_{nw}\Omega(l+m) + iJ_{nw}\Omega(l+m+1) }{2^{j+k+l+m} \cdot j!k!l!m!}
         i^{l+m} 
         \beta_{xw}^{\frac{j+k}{2}} \beta_{yw}^{\frac{l+m}{2}} \\
   &\begin{aligned}\phantom{\biggl[\sum_{j + k + l + m = n} \quad}
           &(2J_x)^{\frac{j+k}{2}} (2J_y)^{\frac{l+m}{2}} 
           e^{i\left[ (j-k)(\phi_{x} + \phi_{x0}) + (l-m)(\phi_{y} + \phi_{y0}) \right]}
           \biggr]
    \end{aligned}
\end{aligned}\label{eq:hamiltonian_multinomial_expanded}\end{equation}

We can then define \(h_{w,jklm}\) as:

\begin{equation}
h_{w,jklm} = \frac{K_{w,n} \Omega(l+m) + iJ_{w,n} \Omega(l+m+1)}{2^{j+k+l+m} \cdot j!k!l!m!} i^{l+m} \beta_{w,x}^{\frac{j+k}{2}}\beta_{w,y}^{\frac{l+m}{2}}
\end{equation}

Leading to:

\begin{equation}
H_w = \sum_{j+k+l+m=n} h_{w,jklm}
           (2J_x)^{\frac{j+k}{2}} (2J_y)^{\frac{l+m}{2}} 
           e^{i\left[ (j-k)(\phi_{x} + \phi_{x0}) + (l-m)(\phi_{y} + \phi_{y0}) \right]}
\end{equation}

Expressed via the Courant-Snyder coordinates, we can write it as:

\begin{equation}
H_w = \sum_{j+k+l+m=n} h_{w,jklm} h^j_{w,x,-}h^k_{w,x,+}h^l_{w,y,-}h^m_{w,y,+}
\end{equation}

The beam at the observed position \(b\) can be described by changing the
position variable.

\begin{equation}\begin{aligned}
  x_b &= x \cdot e^{i\Delta \phi_{x}^b} \\
  y_b &= y \cdot e^{i\Delta \phi_{y}^b}
\end{aligned}\end{equation}

Where \(\Delta \phi_{x,y}^b\) is the phase advance between the locations
b and w. This simply adds one term to our previous hamiltonian:

\begin{equation}\begin{aligned}
H_{bw} = \sum_{j+k+l+m=n}& h_{w,jklm}
           e^{i\left[ (j-k)(\Delta \phi_{w,x}^b) + (l-m)(\Delta \phi_{w,y}^b) \right]}\\
           &(2J_x)^{\frac{j+k}{2}} (2J_y)^{\frac{l+m}{2}} 
           e^{i\left[ (j-k)(\phi_{x} + \phi_{x0}) + (l-m)(\phi_{y} + \phi_{y0}) \right]}
\end{aligned}\end{equation}

\hypertarget{f-terms}{%
\subsection{f-terms}\label{f-terms}}

The \emph{f-terms} come from the generating function used to have action
dependant only hamilonians. The Hamiltonian depends then only on new
variables: \(I_x\) and \(I_y\).

\begin{equation}\begin{aligned}
  f_{jklm} &= \frac{h_{jklm}}{1 - e^{2\pi i \left[(j-k)Q_x + (l-m)Q_y\right]}} \\
           &= \frac{\sum_w h_{w,jklm} e^{i\left[(j-k)\Delta \phi_{w,x}^b) + (l-m)\Delta \phi_{w,y}^b) \right]}}
                   {1 - e^{2\pi i \left[(j-k)Q_x + (l-m)Q_y\right]}} \\
\end{aligned}
\label{eq:fterms}\end{equation}

\hypertarget{position}{%
\subsection{Position}\label{position}}

\hl{Add the equations for the position}

\hypertarget{resonances}{%
\subsection{Resonances}\label{resonances}}

A resonance occurs when the denominator of the \cref{eq:fterms}
diverges, i.e when:

\begin{equation}(j-k)Q_x + (l-m)Q_y = p \quad p \in \mathbb{N}\label{eq:resonance}\end{equation}



% ============================================
%        Measurements and Corrections
% ============================================
\section{Measurements and Corrections}

\todo{DO NOT DELETE THIS PART ACTUALLY}
\todo{THIS IS TODO}
w
% -----------------------
%  Skew Octupole RDTs
% -----------------------
\subsection{Skew Octupole RDTs}

Measurement and correction of the $a_4$ RDTs: $f_{1012,y}$ and $f_{1210,x}$.

\begin{itemize}
    \item Past measurements and corrections
    \item Measurement of the bare a4 RDTs
    \item Creation of the response matrix
    \item Check of the correction in MADX
    \item Check of the correction on the LHC
\end{itemize}

For the response matrix: \\
\footnotesize
\verb|/afs/cern.ch/work/m/mlegarre/public/jupyter/response_matrix/Test_a4_correction_2023.ipynb|
\normalsize

For the measurement with the corrections: \\
\footnotesize
\verb|/nfs/cs-ccr-nfs4/lhc_data/OP_DATA/Betabeat/2023-04-08/LHCB1/Results/OMC3_LHCB1_30cm_noa4correction|
\verb|/nfs/cs-ccr-nfs4/lhc_data/OP_DATA/Betabeat/2023-04-08/LHCB1/Results/OMC3_LHCB1_30cm_withMaelCorr|
\normalsize


% -----------------------
%  Normal Decapole RDTs
% -----------------------
\subsection{Normal Decapole RDTs}

blabla