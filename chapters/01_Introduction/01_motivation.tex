\section{\review{Motivations}} 


Optics control in accelerators is a key factor for optimizing the performance and stability
of high-energy particle colliders like the Large Hadron Collider (LHC). As accelerators push the
boundaries of operational parameters, the influence of higher-order magnetic fields becomes more
pronounced, requiring precise corrections to ensure optimal performance. Addressing non-linearities,
which arise from magnetic field errors, is crucial not only for the LHC's current operations but
also for future collider designs.


The LHC acts as a central testbed for investigating higher-order non-linearities that directly
influence beam stability, dynamic aperture, and beam lifetime. Correcting these non-linearities is
crucial for sustaining high performance in the LHC and for informing the design of future machines.
This research is driven by the need to develop advanced beam-based techniques for measuring and
correcting these higher-order effects, shifting from traditional empirical methods to more precise
and quantitative approaches. Accurately quantifying the fields present in the LHC is essential for
achieving a comprehensive understanding of the machine. Such confidence in managing non-linear
dynamics is vital for the development and operation of future accelerators like the High Luminosity
LHC (HL-LHC) and the Future Circular Collider (FCC).


A key issue highlighted during the LHC's Run 2 is the observed discrepancy at injection energy
between the measured third-order chromaticity $Q'''$ and the predictions made by existing models.
The magnetic measurements of the LHC's magnets, conducted during its construction phase, have served
as the foundation for simulations, beam steering, and non-linear correction computations.  However,
this discrepancy suggests the presence of previously unaccounted-for field errors not captured by
the initial magnetic measurements. Identifying and addressing these unknown sources of higher-order
magnetic errors is crucial for improving the LHC's operational parameters, particularly during beam
injection, to ensure optimal performance and stability.


This thesis aims to address these challenges by developing improved methods for characterizing
higher-order magnetic fields, such as decapolar components, and their impact on beam dynamics.
Through direct, quantitative approaches, the research seeks to refine correction strategies and
better understand the interplay of non-linear magnetic fields. These advancements will contribute
not only to enhancing the LHC's performance but also to informing the design and operation of future
high-energy colliders.