\section{\review{Motivations}} 

The motivation for this PhD research arises from the necessity to address higher-order
non-linearities in the Large Hadron Collider (LHC). Nonlinear corrections are essential for the
stable operation of circular colliders like the LHC, as they play a critical role in suppressing
resonances, improving dynamic aperture, and enhancing beam lifetime. As the LHC continues to push
the boundaries of its operational parameters, the influence of higher-order magnetic field errors,
particularly those beyond octupoles, becomes increasingly significant, leading to potential
degradation in performance.

One specific challenge highlighted during the LHC's Run 2 is the observed discrepancy at injection
energy between the measured third-order chromaticity $Q'''$ and the predictions made by existing
models. The magnetic measurements of the LHC's magnets, conducted during its construction phase,
have served as the foundation for simulations, beam steering, and nonlinear correction computations.
However, this discrepancy suggests the presence of previously unaccounted-for field errors that are
not captured by the initial magnetic measurements. Identifying and addressing these unknown sources
of higher-order magnetic errors is crucial for improving the LHC's operational parameters,
particularly during beam injection, to ensure optimal performance and stability.

To tackle these challenges, this research is driven by the need to develop and refine methods for
measuring and characterizing higher-order non-linearities and understanding their impacts on beam
dynamics. A key motivation is also to explore and improve techniques for measuring and
characterizing higher-order magnetic fields, such as dodecapolar fields. These efforts are crucial
for accurately modeling the complex interactions within the LHC and for implementing effective
correction strategies that enhance the collider's performance. By advancing the understanding of
these higher-order effects, this research aims to contribute to the continued success of the LHC and
inform the design of future accelerators.