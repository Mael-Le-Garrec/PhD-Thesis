\section{Particle Accelerators and CERN}

\subsection{Particle Accelerators}

%The first modern particle accelerators started to be conceived at the beginning of the 1900s. Those first accelerators, such as the Cockroft-Walton generator, used voltage multipliers to generate electric fields in order to accelerate particles.
%Voltage multipliers are nowadays still an important piece of equipment as they are used in devices requiring high voltage, such as X-Ray machines, CRT monitors, microwaves or as part of an accelerator complex.
%Those early technologies were namely used to perform the first artificial nuclear disintegration.

\todo{GPT}

The history of particle accelerators is a captivating narrative that spans over a century of scientific innovation and discovery. It is a journey that has fundamentally transformed our understanding of the universe's fundamental particles and their interactions. The concept of accelerating particles to high speeds originated in the late 19th century, with early experiments conducted by pioneers such as J.J. Thomson and Ernest Rutherford, who utilized basic devices like cathode ray tubes to propel electrons.

One of the earliest breakthroughs in accelerator technology was the Cockcroft-Walton accelerator, introduced in 1932 by John Cockcroft and Ernest Walton. This pioneering device employed voltage multipliers to accelerate protons and ions, enabling the first artificial nuclear disintegration—a milestone that earned them the Nobel Prize in Physics in 1951. Building upon this achievement, the development of the synchrotron in the 1940s and 1950s by scientists like Edwin McMillan and Vladimir Veksler marked a significant stride. Synchrotrons harnessed magnetic fields to bend and accelerate charged particles in circular paths, advancing the study of particle properties.

A key turning point emerged with the establishment of CERN (the European Organization for Nuclear Research) in 1954, which culminated in the creation of the Proton Synchrotron (PS) in 1959. This marked the emergence of a powerful era in accelerator science, enabling the discovery of novel particles and laying the groundwork for the formulation of the Standard Model of particle physics. Throughout the 1960s and 1970s, the advent of bubble chambers and bubble chamber detectors provided researchers with the ability to trace the paths of charged particles, leading to the revelation of various particles and their intricate interactions.

Yet, the true marvel of accelerator technology came to the forefront with the construction of the Large Hadron Collider (LHC) at CERN, which commenced operation in 2008. The LHC, an awe-inspiring 27-kilometer ring of superconducting magnets, propels protons and heavy ions to velocities nearing the speed of light. The LHC's monumental achievement—the discovery of the Higgs boson in 2012—marked a crowning moment in particle physics, solidifying the vital role of particle accelerators in unraveling the fabric of the cosmos.

As particle physicists peer into the future, the quest continues. Concepts such as linear colliders and advanced circular colliders are on the horizon, promising to delve even deeper into the enigmatic realm of fundamental particles and the forces that govern them. The history of particle accelerators underscores the profound human endeavor to explore the most intricate mysteries of the universe, revealing the intricate dance of particles that shape the cosmos and expanding the horizons of human knowledge.



\subsection{The CERN Complex}


\todo{GPT}
The CERN complex, located near Geneva, Switzerland, is a prominent center for particle physics research. Its centerpiece is the Large Hadron Collider (LHC), the world's largest particle accelerator with a 27-kilometer circumference. Here, protons and heavy ions are accelerated to near light speed and collide at various points for fundamental particle studies. Surrounding the LHC are significant particle detectors, including ATLAS, CMS, ALICE, and LHCb, designed to capture and analyze particles generated during these collisions.

CERN also includes linear accelerators, the Proton Synchrotron (PS), Super Proton Synchrotron (SPS), and Antiproton Decelerator (AD), contributing to particle acceleration and antimatter research. Alongside these facilities, CERN houses the Theoretical Physics Department, where theorists collaborate with experimentalists. With research, administrative buildings, laboratories, and workshops, CERN provides a comprehensive environment for scientific exploration. Its history, including the 2012 discovery of the Higgs boson, underscores its importance in advancing particle physics and highlighting international scientific cooperation.



%%%%%%%%%%%%%%%%%%%%%%%%%%%%%%%%%%%%%%%%%%%%%%%%%%%%%%%%%%%%%
%%%                       LHC
%%%%%%%%%%%%%%%%%%%%%%%%%%%%%%%%%%%%%%%%%%%%%%%%%%%%%%%%%%%%%
\subsection{The Large Hadron Collider}

\begin{figure}[H]
    \includegraphics[width=\textwidth]{chapters/01_Introduction/images/lhc_3D_cut.png}
    \caption{3D cut of a main LHC dipole. Courtesy of the CERN Resources Website~\cite{cern-resources}.}
    \label{fig:3d_cut_dipole}
\end{figure}

speed of light, 11 000 turns per second, 12 000 amps in dipoles, number dipoles, price, parameters
energy consumption, detectors and experiments, discoveries, collimators, optics, magnets, luminosity, arcs, IRs, schematics, cryostat, beta function FODO

\begin{enumerate}
    \color{red}
    \item Cycles \& types of bunches: pilot for measurements
\end{enumerate}