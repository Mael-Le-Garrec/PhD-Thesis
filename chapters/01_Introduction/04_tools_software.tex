\section{\review{Tools and Softwares}}

In order to perform the measurements, analysis and simulations presented in this thesis, various
tools and softwares have been developed, used and contributed to.

Optics simulations have been done mainly in MAD-X~\cite{deniau_mad-x_nodate} and PTC.
MAD-NG~\cite{deniau_mad-ng_2020} and
Xsuite~\cite{g_iadarola_xsuite_nodate} have also been explored for specific tasks such as free RDT
simulations and GPU tracking.

Analysis of chromaticity measurements are done via a newly developed graphical
interface~\cite{m_le_garrec_non-linear_2022} written in Python. This tool makes cleaning of the raw
signal data, its analysis and results export more reliable and easier.

Overall, analysis of turn-by-turn measurements is supported by a large panel of libraries written by
the OMC team in Python and Java. Contributions have mainly been made to extend the following
packages:


\begin{itemize}
    \item \textbf{Beta-Beat GUI}~\cite{omc-team_beta-beat_2008}, Graphical interface for turn-by-turn measurements visualization and
    analysis.
    \item \textbf{OMC3}~\cite{omc-team_omc3_2021}, Main optics analysis and corrections software.
    \item \textbf{Beta-Beat.src}~\cite{omc-team_beta-beatsrc_2018}, Old analysis software, now
    replaced by OMC3.
    \item \textbf{pylhc.github.io}~\cite{omc-team_omc_2020}, Website of the OMC team with package
    documentation, examples and useful resources.
\end{itemize}
