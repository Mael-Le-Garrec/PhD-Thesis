\section{\review{Thesis Outline}}

The thesis starts by giving the motivations for this thesis work, as well as its outline, in
\cref{chapter:introduction}. Key concepts of accelerator physics are presented in
\cref{chapter:background}. The CERN accelerator complex and the LHC are then detailed. Measurement
and correction techniques are presented in \cref{chapter:optics_meas}.

The first results chapter, \cref{chapter:skew_octupole_fields} examines the skew octupolar fields
which have been shown to limit the dynamic aperture, especially during beam excitation with the
AC-Dipole. A response matrix method was developed to correct skew octupolar Resonance Driving Terms
(RDTs) at top energy. The study also explores the influence of Landau octupoles on skew octupolar
RDTs at injection energy, revealing the importance of accurate coupling modeling in predicting these
effects.

The second chapter, \cref{chapter:decapoles}, delves into the decapolar fields at injection energy,
addressing discrepancies between measurements and model predictions of third-order chromaticity.
Through a series of novel measurements and simulations, including the introduction of chromatic
amplitude detuning, the research identifies the decay of the decapolar component in the main dipoles
as a key factor in these discrepancies. Corrective strategies were developed for decapolar RDTs,
leading to measurable improvements in beam lifetime and stability.

The third chapter, \cref{chapter:high_order_fields}, focuses on the measurement and analysis of
dodecapolar and decatetrapolar fields, using an tailored post-processing technique. The study
successfully measures higher-order chromaticity terms and dodecapolar RDTs, demonstrating their
significant contribution to the overall field errors in the LHC. The findings underscore the need
for further investigation into these higher-order fields and their impact on beam dynamics to
optimize the LHC's performance.

The final chapter, \cref{chapter:superkekb}, which serves as a supplementary section, explores the
application of optics measurement techniques employed at CERN to the SuperKEKB rings (HER and LER)
at KEK in Tsukuba, Japan. This study is the outcome of a one-month secondment as part of the EAJADE
collaboration.

Finally, conclusions for these studies are drawn in \cref{chapter:conclusions}.