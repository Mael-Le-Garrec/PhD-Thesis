\usepackage{amssymb,amsmath}
\usepackage{ifxetex,ifluatex}
\usepackage{fix-cm}
\usepackage{microtype}  % better justifications, amongst others
\usepackage{longtable,booktabs}

% Make hyperref silent, it's always complaining for tokens like \beta
\usepackage{silence}
%\WarningFilter*{hyperref}{Token not allowed in a PDF string (Unicode)}

\usepackage[unicode=true,pdfa]{hyperref}
\hypersetup{breaklinks=true,
            pdfauthor={Maël Le Garrec},
            pdftitle={LHC Effective Model for Optics Corrections},
            colorlinks=true,
            citecolor=blue,
            urlcolor=blue,
            linkcolor=black,
            pdfborder={0 0 0}}
\usepackage[capitalise]{cleveref}
\usepackage[english]{babel}

% == FONTS
\usepackage{calligra}  % font for quote page
%\usepackage{libertinust1math}
\usepackage{libertine}  % font for the whole document
\usepackage[libertine]{newtxmath}
%\usepackage{lmodern}  % font based on Computer Modern for the whole doc
% ==


\usepackage{wasysym}
\usepackage{enumitem}  % to define new lists
\usepackage{tikz}  % for drawing figures by code
\usepackage{siunitx}
\usepackage{graphicx}
\usepackage{caption}
\usepackage{ragged2e}
\usepackage{atveryend}
%\usepackage{subfigure} (???)
\usepackage{subcaption}
\usepackage{pgf}  % fileformat for the flipbook
\usepackage{xcolor,soul}
\usepackage{lscape}  % landscape
\usepackage{changepage}
\usepackage[nonumberlist,acronyms,nogroupskip]{glossaries}
\usepackage{glossary-longbooktabs}
\usepackage{setspace}
\usepackage[english]{babel}
\usepackage{float}
%\usepackage[subfigure]{tocloft}
\usepackage{titletoc}
\usepackage{etoc}  % local tables of content
\usepackage{imakeidx}
\usepackage{lipsum}
\usepackage{geometry}
\usepackage[automark]{scrlayer-scrpage} % for headers and footers
\usepackage{scrhack}  % to remove some warnings
\usepackage{blindtext}
%\usepackage{unicode-math} % Setup an unicode font for regular typing and for maths  // clashes with OverBrace from nicematrix
\usepackage{amsmath}
\usepackage{mathtools}  % for some functions, like DeclarePairedDelimiter
\usepackage{svg}
\usepackage{wrapfig}
\usepackage{nicematrix}
\usepackage[
  height={2cm},
  topthumbmargin={auto},
  bottomthumbmargin={auto},
  eventxtindent={4mm},
  oddtxtexdent={2.5mm}]{thumbs}  % markers on the side to show the chapter
\usepackage[%
  backend=bibtex        % biber or bibtex
%,style=authoryear      % Alphabeticalsch
 ,style=numeric-comp    % numerical-compressed
 ,sorting=none          % no sorting
 ,sortcites=true        % some other example options ...
 ,block=none
 ,indexing=false
 ,citereset=none
 ,isbn=false
 ,url=true
 ,doi=true              % prints doi
 ,natbib=true           % if you need natbib functions
]{biblatex}
\AtEveryBibitem{%  % in the bibliography
    \clearlist{language}%  % remove language from citations
    \clearfield{urlyear}%  % to remove the "visited on"
    \clearfield{urlmonth}%
    \clearfield{urlday}%
    \clearfield{day}%  % only print the year
    \clearfield{month}%
    \clearfield{endday}%
    \clearfield{endmonth}%
    \clearfield{note}%  % information about the PDF, like size and number of pages
    \clearfield{pages}%  % which pages in the book
}
\AtEveryCitekey{%  % for \fullcite
    \clearlist{language}%
    \clearfield{urlyear}%
    \clearfield{urlmonth}%
    \clearfield{day}%
    \clearfield{month}%
    \clearfield{endday}%
    \clearfield{endmonth}%
    \clearfield{note}%
    \clearfield{pages}%
}
\addbibresource{library.bib}  % better than \bibliography
\usepackage{csquotes}  % Changes the quotestyle depending on the language, useful for bibliography

% Flipbook
\usepackage{flipbook}
%\cfoot*{
%  \flipbookframe[1][1]{../flipbook/frames/frame_}[pgf][0.2]
%  % start frame
%  % speed of the animation per page
%  % scale
%}


% To have big numbers at the start of each chapter
%\definecolor{gray75}{gray}{0.75}
%\newcommand{\hsp}{\hspace{0pt}}
%\titleformat{\chapter}[hang]
%    {\flushright\fontseries{b}\fontsize{80}{100}\selectfont}
%    {\fontseries{b}\fontsize{100}{130}\selectfont \textcolor{gray75}\thechapter\hsp}
%    {0pt}
%    {\\ \Huge\bfseries}[]
%
%% Same but un-numbered
%\titleformat{name=\chapter, numberless}[hang]
%    {\flushright\fontseries{b}\fontsize{80}{100}\selectfont}
%    {\fontseries{b}\fontsize{100}{130}\selectfont \textcolor{gray75}\hsp}
%    {0pt}
%{\\ \Huge\bfseries}[]
%
%% And now the other titles
%\titleformat{\section}
%{\normalfont\Large\bfseries}{\thesection}{1em}{}
%\titleformat{\subsection}
%{\normalfont\large\bfseries}{\thesubsection}{1em}{}
%\titleformat{\subsubsection}
%{\normalfont\normalsize\bfseries}{\thesubsubsection}{1em}{}
%\titleformat{\paragraph}[runin]
%{\normalfont\normalsize\bfseries}{\theparagraph}{1em}{}
%\titleformat{\subparagraph}[runin]
%{\normalfont\normalsize\bfseries}{\thesubparagraph}{1em}{}



% ===========================================================================
%                            SOME OPTIONS
% ===========================================================================

% Set the geometry of the pages
% For a twoside book like this one, `left` and `right` mean respectively `inner` and `outer`
% The numbers are based on the "Canon des Ateliers", https://etnadji.fr/rsc/canon/calcul.php
% https://www.alain.les-hurtig.org/varia/empagement.html
%
% tête = top, pied = bottom, petit fond = left, grand fond = right
%
%\geometry{a4paper, top=2.625cm, left=2.1cm, right=3.15cm, bottom=3.675cm, includehead, includefoot}
\geometry{b5paper, top=2.2cm, left=1.76cm, right=2.64cm, bottom=3.08cm, includehead, includefoot} % courant
%\geometry{b5paper, top=2.935cm, left=2.348cm, right=3.522cm, bottom=4.109cm, includehead, includefoot} % luxe
\geometry{paperheight=235mm, paperwidth=165mm,   % printhub.eu "B5" paper
          top=20.625mm, bottom=28.875mm, left=16.5mm, right=24.75mm,
          includehead, includefoot} % courant

% Vertical space before chapters
\RedeclareSectionCommand[beforeskip=5pt,
afterskip=2cm]{chapter}

% Fancy chapter headings, needs to be loaded after geometry
%\usepackage[Bjornstrup]{fncychap}  % issues a lot of warnings with scrbook, replaced in commands below

% Factor spacing between lines
\linespread{1.1}

\numberwithin{equation}{chapter}
\numberwithin{table}{chapter}
\numberwithin{figure}{chapter}
\newcommand*\diff{\mathop{}\!\mathrm{d}}

% Set some lengths
\setlength{\parindent}{12pt}
\setlength{\parskip}{6pt plus 2pt minus 1pt}
\setlength{\emergencystretch}{3em}  % prevent overfull lines
\setcounter{secnumdepth}{2}  %  set up to which point a sub[..]subsection is numbered

\urlstyle{same}  % don't use monospace font for urls

% Make the captions closer to table, figures, etC.
%\setlength{\belowcaptionskip}{-5pt}
%\setlength{\abovecaptionskip}{-5pt}

% Define some colors
\definecolor{thumb_color}{HTML}{D9D9D9}

% LaTeX will stretch the page to fit vertically on the whole page as part of the "book" style
% This prevents it
%\raggedbottom

% ===========================================================================
%                            SOME COMMANDS
% ===========================================================================

% Create a \tightlight command for itemize environments to have the bullet points closer together
\providecommand{\tightlist}{%
  \setlength{\itemsep}{0pt}\setlength{\parskip}{0pt}}

% Command to create highlights easily in colors
\DeclareRobustCommand{\hlcyan}[1]{{\sethlcolor{cyan}\hl{#1}}}

% Display a table of contents for the current chapter
\newcommand{\chaptertoc}[1][Contents]{%
  % Set the indent so it's a bit tighter and looks better
  %\setlength{\cftsecindent}{0.2cm}
  %\setlength{\cftsubsecindent}{0.8cm}
  %\setlength{\cftsubsubsecindent}{1.4cm}

  %\etocmulticolstyle{\addsec*{#1\\\rule{\textwidth}{0.4pt}}}%
  \setcounter{tocdepth}{3}

  % Reduce the spacing between the list items
  %\addsec*{\rule{\textwidth}{0.4pt}}
  \begin{spacing}{0.1}
    \localtableofcontents%
  \end{spacing}
}

% Define a wrapper for \addthumb, to add markers on the side
\newcommand{\thumbforchapter}{\addthumb{Chapter \thechapter}{\Large{\thechapter}}{black}{thumb_color}}
% Same but with letters for the appendices
\newcommand{\thumbforappendix}{\addthumb{Chapter \thechapter}{\Large{\thechapter}}{black}{thumb_color}}

% When using align from amsmath, each line is numbered
% This allows to use align* and the manually number the last equation
\newcommand\numberthis{\addtocounter{equation}{1}\tag{\theequation}}

% Some commands to display text in color
% To do in red
\newcommand*{\todo}[1]{{\bfseries\color{red}#1}}
% To be reviewed in orange
\newcommand*{\review}[1]{{\bfseries\color{orange}#1}}
% OK in green
\newcommand*{\done}[1]{{\bfseries\color{green}#1}}

% For commands floor and ceil to be easier to type
\DeclarePairedDelimiter\ceil{\lceil}{\rceil}
\DeclarePairedDelimiter\floor{\lfloor}{\rfloor}


% Nice looking chapters
%\renewcommand*{\chapterformat}{\thechapter}
%\renewcommand*{\raggedchapter}{\raggedleft}
%\setkomafont{chapter}{\LARGE}
%\setkomafont{chapterprefix}{\Huge}
%\newcommand*{\ChapterCase}[1]{#1}
%%\newcommand*{\ChapterCase}[1]{\MakeUppercase{#1}}%  ugly
%%\newcommand*{\ChapterCase}[1]{\MakeUppercase{\textls[75]{#1}}}% better
%\newsavebox\chapternumberbox
%\renewcommand*{\chapterlinesformat}[3]{% #1 = chapter command name
%                                       % #2 = number (or empty)
%                                       % #3 = text
%  \rule[-\dp\strutbox]{\linewidth}{.4pt}%
%  \sbox\chapternumberbox
%  {%
%    \makebox[0pt][l]{%
%      \hspace{-\linewidth}\hspace{.5em}%
%      \colorbox{black}{%
%        \parbox[c][1.5em][c]{1.5em}{%
%          \centering
%          \textcolor{white}{%
%            \usekomafont{chapterprefix}{%
%              \strut #2%
%            }%
%          }%
%          \par
%        }%
%      }%
%    }%
%  }%
%  \IfArgIsEmpty{#2}{%
%    \vphantom{\usebox\chapternumberbox}%
%  }{\usebox\chapternumberbox}%
%  \par
%  \ChapterCase{\strut\ignorespaces #3}%
%  \rule[.5em]{\linewidth}{.4pt}\par
%}

% =====================
%        Fonts
% =====================
% Font for the main title, chapter and sections titles
\newfontfamily{\chapterfont}{Tex Gyre Adventor}
\newfontfamily{\subtitlefont}{Tex Gyre Adventor}
% fontsize for the overall chapter's definition, sets the line reference size, etc.
\setkomafont{chapter}{\LARGE}
% fontsize for the number in the box
\setkomafont{chapterprefix}{\Huge}
% fonts for the sections, subsections etc
\setkomafont{section}{\chapterfont\Large\bfseries}
\setkomafont{subsection}{\chapterfont\large\bfseries}
\setkomafont{subsubsection}{\chapterfont\normalsize\bfseries}
\setkomafont{paragraph}{\chapterfont\small\bfseries}
\setkomafont{pagenumber}{\normalfont\chapterfont\small}
% fonts for the TOC
\setkomafont{disposition}{\normalfont\chapterfont}
\RedeclareSectionCommands[
  tocentryformat=\usekomafont{disposition}\bfseries,
  tocpagenumberformat=\usekomafont{disposition}\bfseries
]{chapter,section,subsection,subsubsection}
\RedeclareSectionCommands[
  tocentryformat=\usekomafont{disposition},
  tocpagenumberformat=\usekomafont{disposition}
]{section,subsection,subsubsection,paragraph,subparagraph}
\RedeclareSectionCommands[
  tocentryformat=\usekomafont{disposition},
  tocpagenumberformat=\usekomafont{disposition}
]{paragraph,subparagraph}
% == TOC
\DeclareTOCStyleEntry[
  beforeskip=.5em plus 1pt,
  pagenumberformat=\hfill
]{tocline}{chapter}



% ============================
%    Nice looking chapters
% ============================
\renewcommand*{\chapterformat}{\thechapter}
\renewcommand*{\raggedchapter}{\raggedleft}
\newcommand*{\ChapterCase}[1]{#1}
\newsavebox\chapternumberbox
\renewcommand*{\chapterlinesformat}[3]{% #1 = chapter command name
                                       % #2 = number (or empty)
                                       % #3 = text
  % Line                                       
  \vspace{1pt}\rule{\linewidth}{1pt}% First rule
  \vspace{-18pt} % Small vertical space between the rules
  \rule{\linewidth}{1pt}% Second rule
  %\rule[-\dp\strutbox]{\linewidth}{.4pt}%
  %
  % Create the box
  \sbox\chapternumberbox
  {%
    \makebox[0pt][l]{%
      \hspace{-\linewidth}\hspace{2em}% spacing of the box from the left
      %
      \colorbox{black}{%
        \parbox[c][1.5em][c]{1.5em}{%
          \centering
          \textcolor{white}{%
            \usekomafont{chapterprefix}{%
              \vspace{-0.2em}%
              \strut #2%
            }%
          }%
          \par
        }%
      }%
    }%
  }%
  % And now place it if the chapter is numbered
  \IfArgIsEmpty{#2}{%
    \vphantom{\usebox\chapternumberbox}%
  }{%
    \usebox\chapternumberbox%
  }%
  \par
  \vspace{1em}
  % Display the chapter's title
  %\hspace{2em}
  \parbox{0.85\textwidth}{%
    \raggedright% deactivate wrapping, align to the left
    %\raggedleft% deactivate wrapping, align to the right
    \ChapterCase{\strut\ignorespaces\chapterfont\fontsize{30pt}{30pt}\selectfont\bfseries #3}%
  }
  \par
  % Line
  \vspace{1em}
  \rule[.5em]{\linewidth}{1.2pt}
  \par
}
