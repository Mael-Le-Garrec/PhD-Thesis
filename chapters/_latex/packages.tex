\usepackage{lmodern}
\usepackage{amssymb,amsmath}
\usepackage{ifxetex,ifluatex}
\usepackage{fix-cm}
\usepackage{microtype}  % better justifications, amongst others
\usepackage{longtable,booktabs}
\usepackage[unicode=true,pdfa]{hyperref}
\hypersetup{breaklinks=true,
            pdfauthor={Maël Le Garrec},
            pdftitle={LHC Effective Model for Optics Corrections},
            colorlinks=true,
            citecolor=blue,
            urlcolor=blue,
            linkcolor=black,
            pdfborder={0 0 0}}
\usepackage{cleveref}
\usepackage[english]{babel}
\usepackage{calligra}
\usepackage{lmodern}
\usepackage{wasysym}
\usepackage{siunitx}
\usepackage{graphicx}
\usepackage{caption}
\usepackage{ragged2e}
\usepackage{atveryend}
%\usepackage{subfigure} (???)
\usepackage{subcaption}
\usepackage{xcolor,soul}
\usepackage{changepage}
\usepackage[nonumberlist,acronyms,nogroupskip]{glossaries}
\usepackage{glossary-longbooktabs}
\usepackage{setspace}
\usepackage[english]{babel}
\usepackage{float}
%\usepackage[subfigure]{tocloft}
\usepackage{titletoc}
\usepackage{etoc}
\usepackage{imakeidx}
\usepackage{lipsum}
\usepackage{geometry}
\usepackage[automark]{scrlayer-scrpage} % for headers and footers
\usepackage{scrhack}  % to remove some warnings
\usepackage{blindtext}
%\usepackage{unicode-math} % Setup an unicode font for regular typing and for maths  // clashes with OverBrace from nicematrix
\usepackage{amsmath}
\usepackage{nicematrix}
\usepackage[
  height={2cm},
  topthumbmargin={auto},
  bottomthumbmargin={auto},
  eventxtindent={4mm},
  oddtxtexdent={2.5mm}]{thumbs}  % markers on the side to show the chapter
\usepackage[%
  backend=bibtex      % biber or bibtex
%,style=authoryear    % Alphabeticalsch
 ,style=numeric-comp  % numerical-compressed
 ,sorting=none        % no sorting
 ,sortcites=true      % some other example options ...
 ,block=none
 ,indexing=false
 ,citereset=none
 ,isbn=false
 ,url=true
 ,doi=true            % prints doi
 ,natbib=true         % if you need natbib functions
]{biblatex}
\addbibresource{library.bib}  % better than \bibliography


% To have big numbers at the start of each chapter
%\definecolor{gray75}{gray}{0.75}
%\newcommand{\hsp}{\hspace{0pt}}
%\titleformat{\chapter}[hang]
%    {\flushright\fontseries{b}\fontsize{80}{100}\selectfont}
%    {\fontseries{b}\fontsize{100}{130}\selectfont \textcolor{gray75}\thechapter\hsp}
%    {0pt}
%    {\\ \Huge\bfseries}[]
%
%% Same but un-numbered
%\titleformat{name=\chapter, numberless}[hang]
%    {\flushright\fontseries{b}\fontsize{80}{100}\selectfont}
%    {\fontseries{b}\fontsize{100}{130}\selectfont \textcolor{gray75}\hsp}
%    {0pt}
%{\\ \Huge\bfseries}[]
%
%% And now the other titles
%\titleformat{\section}
%{\normalfont\Large\bfseries}{\thesection}{1em}{}
%\titleformat{\subsection}
%{\normalfont\large\bfseries}{\thesubsection}{1em}{}
%\titleformat{\subsubsection}
%{\normalfont\normalsize\bfseries}{\thesubsubsection}{1em}{}
%\titleformat{\paragraph}[runin]
%{\normalfont\normalsize\bfseries}{\theparagraph}{1em}{}
%\titleformat{\subparagraph}[runin]
%{\normalfont\normalsize\bfseries}{\thesubparagraph}{1em}{}



% ===========================================================================
%                            SOME OPTIONS
% ===========================================================================

% Set the geometry of the pages
% For a twosided book like this one, `left` and `right` mean respectively `inner` and `outer`
% The numbers are based on the "Canon des Ateliers", https://etnadji.fr/rsc/canon/calcul.php
%\geometry{a4paper, top=2.625cm, left=2.1cm, right=3.15cm, bottom=3.675cm, includehead, includefoot}
\geometry{b5paper, top=2.2cm, left=1.76cm, right=2.64cm, bottom=3.08cm, includehead, includefoot}

% Fancy chapter headings, needs to be loaded after geometry
\usepackage[Bjornstrup]{fncychap}

% Change the headers to display both the current chapter and section, from fancyhdr
%\pagestyle{fancy} 

% Factor spacing between lines
\linespread{1.1}

% Gives to the equation "number" the section number as well, makes it easier to find where it is in the book
\numberwithin{equation}{chapter}
\numberwithin{table}{chapter}
\numberwithin{figure}{chapter}
\newcommand*\diff{\mathop{}\!\mathrm{d}}

% Set some lengths
\setlength{\parindent}{12pt}
\setlength{\parskip}{6pt plus 2pt minus 1pt}
\setlength{\emergencystretch}{3em}  % prevent overfull lines
\setcounter{secnumdepth}{2}  %  set up to which point a sub[..]subsection is numbered

\urlstyle{same}  % don't use monospace font for urls

% Make the captions closer to table, figures, etC.
%\setlength{\belowcaptionskip}{-5pt}
%\setlength{\abovecaptionskip}{-5pt}

% Define some colors
\definecolor{thumb_color}{HTML}{D9D9D9}



% ===========================================================================
%                            SOME COMMANDS
% ===========================================================================

% Create a \tightlight command for itemize environments to have the bullet points closer together
\providecommand{\tightlist}{%
  \setlength{\itemsep}{0pt}\setlength{\parskip}{0pt}}

% Command to create highlights easily in colors
\DeclareRobustCommand{\hlcyan}[1]{{\sethlcolor{cyan}\hl{#1}}}

% Display a table of contents for the current chapter
\newcommand{\chaptertoc}[1][Contents]{%
  % Set the indent so it's a bit tighter and looks better
  %\setlength{\cftsecindent}{0.2cm}
  %\setlength{\cftsubsecindent}{0.8cm}
  %\setlength{\cftsubsubsecindent}{1.4cm}

  %\etocmulticolstyle{\addsec*{#1\\\rule{\textwidth}{0.4pt}}}%
  \setcounter{tocdepth}{4}

  % Reduce the spacing between the list items
  \addsec*{\rule{\textwidth}{0.4pt}}
  \begin{spacing}{0.1}
    \localtableofcontents%
  \end{spacing}
}

% Define a wrapper for \addthumb, to add markers on the side
\newcommand{\thumbforchapter}{\addthumb{Chapter \thechapter}{\Large{\thechapter}}{black}{thumb_color}}
% Same but with letters for the appendices
\newcommand{\thumbforappendix}{\addthumb{Chapter \thechapter}{\Large{\thechapter}}{black}{thumb_color}}

% Some commands to display text in color
% To do in red
\newcommand*{\todo}[1]{{\bfseries\color{red}#1}}
% To be reviewed in orange
\newcommand*{\review}[1]{{\bfseries\color{orange}#1}}
% OK in green
\newcommand*{\done}[1]{{\bfseries\color{green}#1}}