\section{Coordinate Systems}

In circular accelerators, particle dynamics are represented using a traveling coordinate system.
A reference orbit is determined by the lattice and its magnet strengths, forming the
\textit{optics}. In the case of a synchrotron, like the LHC, where the particles return to their
original location after some turns, the reference orbit is also called the closed orbit.  
The Frenet-Serret coordinate system is used, moving along the ring on the reference orbit. The
coordinates are then transverse: $x$ and $y$, and longitudinal in the direction of travel: $s$.
Figure~\ref{fig:coordinate_systems:frenet_serret} shows those coordinates.

\begin{figure}[H]
    \centering
    \includegraphics[width=0.6\textwidth]{example-image-a}
    \caption{\todo{Frenet-Serret coordinates commonly used in accelerator physics.}}
    \label{fig:coordinate_systems:frenet_serret}
\end{figure}



% ============================================
%               Linear Lattice 
% ============================================
\subsection{Linear Lattice}

A circular accelerator is composed of many multipoles of different orders. A basic
design only requires dipoles and quadrupoles in order to operate. Dipoles are used to bend the
particles in order to form the ring, whereas quadrupoles are used to focus the beam to a focal
point, similar to light optics.
Those elements can be arranged in a particular order, to form a \text{FODO} cell. Such cells present
an alternating placement of focusing an defocusing quadrupoles with dipoles in between, as shown in
Fig.\ref{fig:coordinate_systems:fodo}, and are usually repeated many times along the ring.

\begin{figure}[H]
    \centering
    \includegraphics[width=0.6\textwidth]{example-image-a}
    \caption{\todo{FODO cell, a repeated basic block present in most circular accelerators.}}
    \label{fig:coordinate_systems:fodo}
\end{figure}

A lattice composed on of only dipoles and quadrupoles, is referred to as a \textit{linear} lattice.

\todo{
    Courant Snyder/Twiss
    Tune, beta function
% 
    Lie, normal form
    Resonance Driving Terms
}