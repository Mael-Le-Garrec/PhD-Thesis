\section{Detuning Effects}

\todo{feed down?}

% ============================================
%                Chromaticity
% ============================================
\subsection{\review{Chromaticity}}
\label{subsection:concepts:chromaticity}

Chromaticity is the tune change $\Delta Q$ relative to the momentum offset $\delta$. Chromaticity
can be described by a Taylor expansion, given by

\begin{equation} 
    Q (\delta) = Q_0 + Q' \delta + \frac{1}{2!} Q'' \delta^2 + \frac{1}{3!} Q''' \delta^3 + \mathcal{O}(\delta^4).
    \label{eq:background_chromaticity}
\end{equation}

Or, more generally, rewritten as a sum to include all orders up to $n$:

\begin{equation}
    Q (\delta) = Q_0 + \sum_{i=1}^n \frac{1}{i!} Q^{(i)} \delta^i.
    \label{eq:background_chromaticity_sum}
\end{equation}


The first order of the chromaticity expansion, $Q'$, is generally simply referred to as 
\textit{chromaticity}, sometimes as \textit{linear chromaticity}.
The other terms are thus referred to as \textit{non-linear chromaticity}.
It is to be noted when referring to a chromaticity order, that the preceding fraction and
$\delta$ are usually \textit{not} included in the term.

The chromaticity change induced by a single element of order $n$ and length $L$ can be derived from
the Hamiltonian of \cref{eq:hamiltonian_magnet}, averaging over the phase variables and
differentiating relative to the actions $J_{x,y}$ and the momentum offset $\delta$:

\begin{equation}
    \Delta Q^{(n)}_{x,y} = \frac{\partial^n Q_{x,y}}{\partial^n \delta} = 
      \frac{1}{2\pi} \int_L \left< \frac{\partial^{(n+1)} H}{\partial J_{x,y}\partial^n \delta}\right> \diff s.
    \label{eq:background_chroma_order_hamiltonian}
\end{equation}

Detailed derivations can be found in~\cite{dilly_derivation_2023}.

% =========
\paragraph{\review{Natural Chromaticity from Quadrupoles}}

In a purely linear lattice, the linear chromaticity, $Q'$, is a result of the momentum dependence 
of the quadrupoles' focusing. It is in this case called the \textit{natural chromaticity} and can be
derived from the normal hamiltonian of \cref{eq:normal_skew_hamiltonian_magnet} and expressing
the normalized magnet strength $K$ with a dependence on $\delta$ via $P$ as $P_0(1+\delta)$:

\begin{equation}
    K_n = \frac{q}{P_0} \frac{1}{1 + \delta} (n - 1)! B_n
    \label{eq:k_dpp}
\end{equation}

The normal field of a quadrupole is then given by

\begin{equation}
    \mathcal{N}_2(x,y) = \frac{1}{2} \frac{q}{P_0} \frac{1}{1+\delta} B_2 (x^2 - y^2)
\end{equation}

By operating a variable change to the angle coordinates 
(\(x \rightarrow \sqrt{2 J_x \beta_x} \cos \phi_x\) and
\(y \rightarrow \sqrt{2 J_y \beta_y} \cos \phi_y\)), the following equation linking the $\beta$-function
and $\delta$ to the normal field is obtained:
\begin{equation}\mathcal{N}_2 (x,y) = \frac{1}{2} \frac{q}{P_0} \frac{1}{1+\delta} B_2 
        \left[
            \left(\sqrt{2 J_x \beta_x} \cos \phi_x \right)^2 
            - \left(\sqrt{2 J_y \beta_y} \cos \phi_y\right)^2
        \right].
    \label{eq:hamiltonian_quadrupole_angle}
\end{equation}

Following \cref{eq:background_chroma_order_hamiltonian}, the natural chromaticity $Q'$ induced by
quadrupoles is given by:
\begin{equation}
    \begin{aligned}
        \Delta Q'_x &= \frac{1}{2\pi} \int_L \frac{\partial^2 \left< \mathcal{N}_2 \right> }{\partial J_x \partial \delta} \diff s
        \quad;\quad
        \Delta Q'_y &= \frac{1}{2\pi} \int_L \frac{\partial^2 \left< \mathcal{N}_2 \right>}{\partial J_y \partial \delta} \diff s
        \\
        & = - \frac{1}{4 \pi} \frac{q}{P_0} B_2 L \beta_x
        & = \frac{1}{4 \pi} \frac{q}{P_0} B_2 L \beta_y
    \end{aligned}
\end{equation}


% =========
\paragraph{\review{Linear Chromaticity from Sextupoles}}
\label{subsubsection:linear_chroma}

The first order chromaticity $Q'$ is contributed to by sextupoles in the presence of off-momentum
particles. The normal field of a sextupole, following \cref{eq:normal_skew_hamiltonian_magnet}
is given by

\begin{equation}
    \mathcal{N}_3(x,y) = \frac{1}{3!} (x^3 - 3xy^2).
    \label{eq:detuning:linear_chromaticity}
\end{equation}

As the momentum offset $\delta$ introduces a change in orbit via
dispersion~\cite{keintzel_second-order_2019}, a variable change can be operated on both $x$ and $y$,
as shown in \cref{eq:detuning:offset}. In this thesis, vertical dispersion will be though
neglected.

\begin{equation}
    \begin{aligned}
        x &\rightarrow x + \Delta x = x + D_x\delta \\
        y &\rightarrow y + \Delta y = y + D_y\delta
    \end{aligned}
    \label{eq:detuning:offset}
\end{equation}

The positions $x$ and $y$ can once be replaced, using the twiss parameters, giving the full
expression:

\begin{equation}\begin{aligned}
  \mathcal{N}_3(x + \Delta x, y) = \frac{1}{6} K_3 \biggl[&
       \left(\sqrt{2 J_x \beta_x} \cos \phi_x\right)^3 \\
  &    + 3 \left(\sqrt{2 J_x \beta_x} \cos \phi_x\right)^2 D_x \delta \\
  &    + 3 \left(\sqrt{2 J_x \beta_x} \cos \phi_x\right) D_x^2 \delta^2 \\
  &    + D_x^3 \delta^3 \\
  &    - 3 \left(\sqrt{2 J_x \beta_x} \cos \phi_x \right) \left(\sqrt{2 J_y \beta_y} \cos \phi_y \right)^2 \\
  &    - 3 D_x \delta (\sqrt{2 J_y \beta_y} \cos \phi_y)^2 \biggl]
\end{aligned}\end{equation}

Averaging over the phase variables removes any odd cosine:
\begin{equation}\begin{aligned}
  \left< \mathcal{N}_3(x + \Delta x, y) \right> = \frac{1}{6} K_3 &\biggl(
       3 J_x \beta_x D_x \delta \\
  &    + D_x^3 \delta^3 \\
  &    - 3 D_x \delta J_y \beta_y \biggl)
\end{aligned}\end{equation}


The chromaticity can then be obtained by differentiating relative to the action $J_{x,y}$ to
retrieve the tune, and finally by the momentum offset $\delta$.
\begin{equation}
    \begin{aligned}
        \Delta Q'_x &= \frac{1}{2\pi} \int_L \frac{\partial^2 \left< \mathcal{N}_3 \right>}{\partial J_x \partial \delta} \diff s \quad; \quad \Delta Q'_y &&= \frac{1}{2\pi} \int_L \frac{\partial^2 \left< \mathcal{N}_3 \right>}{\partial J_y \partial \delta} \diff s \\
        &= \frac{1}{2 \pi} L \frac{1}{2} K_3 \beta_x D_x  &&= - \frac{1}{2 \pi} L \frac{1}{2} K_3 \beta_y D_x \\
        &= \frac{1}{4 \pi}  K_3 L \beta_x D_x &&= - \frac{1}{4 \pi}  K_3 L \beta_y D_x
    \end{aligned}
\end{equation}

From this last equation, it is apparent that sextupoles are not a source of chromaticity of higher
orders in the presence of linear dispersion. In the presence of second order
dispersion~\cite{keintzel_second-order_2019}, sextupoles can generate some amount of $Q''$, usually
negligible.


% =========
\paragraph{\review{Non-Linear Chromaticity}}

Higher orders of the chromaticity function are described in~\cite{dilly_derivation_2023} and follow
the same logic as for the linear chromaticity from sextupoles.
A general formula can be found for the chromaticity of order $n, n > 2$:

\begin{equation}
    \begin{aligned}
        \Delta Q_x^{(n)} &= &\frac{1}{4\pi} K_{n+2} L \beta_x D_x^{n}\\
        \Delta Q_y^{(n)} &= -&\frac{1}{4\pi} K_{n+2} L \beta_x D_x^{n}\\
    \end{aligned}
    \label{eq:detuning_effects:chromaticity_strength}
\end{equation}




% ============================================
%             Amplitude Detuning
% ============================================
\subsection{\review{Amplitude Detuning}}

Amplitude detuning is a tune shift induced by the amplitude of oscillations of a particle. This
detuning is directly related to the emittance and can be described via a Taylor expansion around the
emittance of both planes, $\epsilon_x$ and $\epsilon_y$.
Equation~\eqref{eq:detuning_effects:ampdet} shows this expansion up to the second order. 
Further expansions can be found in~\cite{dilly_derivation_2023}.

\begin{equation}
\begin{aligned}
Q_z(\epsilon_x, \epsilon_y) = Q_{z0} &+ \left(\frac{\partial Q_z}{\partial \epsilon_x} \epsilon_x
                                                + \frac{\partial Q_z}{\partial \epsilon_y} \epsilon_y
                                                \right) \\
                                     &+ \frac{1}{2!} \left(\frac{\partial^2Q_z}{\partial \epsilon_x^2}\epsilon_x^2 
                                                          + 2 \frac{\partial^2Q_z}{\partial \epsilon_x \partial \epsilon_y}\epsilon_x \epsilon_y
                                                          + \frac{\partial^2Q_z}{\partial \epsilon_y^2}\epsilon_y^2\right)
                                     + ... 
                                     %&+\frac{1}{3!}\biggl(\frac{\partial^{3} Q_z}{\partial \epsilon_{x}^{3}} \epsilon_{x}^{3}
                                     %     + 3 \frac{\partial^{3} Q_z}{\partial \epsilon_{y}\partial \epsilon_{x}^{2}} \epsilon_{x}^{2} \epsilon_{y} 
                                     %     + 3 \frac{\partial^{3} Q_z}{\partial \epsilon_{y}^{2}\partial \epsilon_{x}} \epsilon_{x} \epsilon_{y}^{2} 
                                     %     +  \frac{\partial^{3} Q_z}{\partial \epsilon_{y}^{3}} \epsilon_{y}^{3}
                                     %  \biggr) \\
\end{aligned}
\label{eq:detuning_effects:ampdet}
\end{equation}

The first order terms of amplitude detuning are generated by octupoles, and to some extent by
sextupoles when considering their higher order contributions. Those higher contributions are usually
measurable but small compared to the ones of normal octupoles.  
It is to be noted that each order does not correspond directly to a multipole order, like for
chromaticity seen previously.  While it is the case for the simple partial derivatives, the
crossterms are instead generated by multipoles of higher orders.
Further derivations can be found in \cref{chromatic-amplitude-detuning}.



% ============================================
%        Chromatic Amplitude Detuning
% ============================================
\subsection{\review{Chromatic Amplitude Detuning}}
\label{subsection:detuning_effects:chromatic_amplitude_detuning}

Similar to amplitude detuning, \textit{chromatic amplitude detuning} is a tune shift induced by the
amplitude of oscillations of a particle but with an additional dependence on the momentum offset.
This effect can be described by a Taylor expansion around the emittance of both planes $\epsilon_x$,
$\epsilon_y$, and the momentum offset $\delta$.
\cref{eq:detuning_effects:chromatic_ampdet} shows this expansion up to the second order. 
Both the emittance $\epsilon$ and the action $J$ can be seen to describe the chromatic amplitude
detuning. Terms are interchangeable with $\epsilon_{x,y} = 2J_{x,y}$.

\begin{equation}
\begin{aligned}
Q_z(\epsilon_x, \epsilon_y, \delta) = Q_{z0} &+ \left[\frac{\partial Q_z}{\partial \epsilon_x} \epsilon_x
                                                 + \frac{\partial Q_z}{\partial \epsilon_y} \epsilon_y
                                                 + \colorbox{yellow!0}{$\displaystyle \frac{\partial Q_z}{\partial \delta}$} \delta
                                                \right] \\
                                             &+ \frac{1}{2!} \biggl[\frac{\partial^2Q_z}{\partial \epsilon_x^2}\epsilon_x^2 
                                                 + \frac{\partial^2Q_z}{\partial \epsilon_y^2}\epsilon_y^2
                                                 + \colorbox{yellow!0}{$\displaystyle \frac{\partial^2 Q_z}{\partial \delta^2}$} \delta^2  \\
                                             &\;\begin{aligned}
                                             \phantom{+ \frac{1}{2!} \biggl[}
                                               &+ 2 \frac{\partial^2Q_z}{\partial \epsilon_x \partial \epsilon_y}\epsilon_x \epsilon_y
                                                  + 2 \frac{\partial^2Q_z}{\partial \epsilon_x \partial \delta}\epsilon_x \delta
                                                  + 2 \frac{\partial^2Q_z}{\partial \delta \partial \epsilon_y} \delta \epsilon_y
                                             \biggr] \\
                                             \end{aligned}  \\
                                             &+ ...
                                             %&+ \frac{1}{3!}
                                             %\biggl[
                                             %     \colorbox{yellow!50}{$\displaystyle \frac{\partial^3 Q_z}{\partial \delta^3}$}\delta^{3}
                                             %     + \frac{\partial^{3} Q_z}{\partial \epsilon_{x}^{3}}  \epsilon_{x}^{3} 
                                             %     + \frac{\partial^{3} Q_z}{\partial \epsilon_{y}^{3}}  \epsilon_{y}^{3} \\
                                             %&\;\begin{aligned}
                                             %\phantom{+ \frac{1}{3!} \biggl[} 
                                             %  &+ 3 \frac{\partial^{3} Q_z}{\partial \epsilon_{x}\partial \delta^{2}} \delta^{2} \epsilon_{x} 
                                             %   + 3  \frac{\partial^{3} Q_z}{\partial \epsilon_{y}\partial \delta^{2}}  \delta^{2} \epsilon_{y}
                                             %   + 3 \frac{\partial^{3} Q_z}{\partial \epsilon_{x}^{2}\partial \delta}  \delta \epsilon_{x}^{2} \\
                                             %  &+ 3 \frac{\partial^{3} Q_z}{\partial \epsilon_{y}^{2}\partial \delta} \delta \epsilon_{y}^{2}  
                                             %   + 3  \frac{\partial^{3} Q_z}{\partial \epsilon_{y}\partial \epsilon_{x}^{2}} \epsilon_{x}^{2} \epsilon_{y} 
                                             %   + 3 \frac{\partial^{3} Q_z}{\partial \epsilon_{y}^{2}\partial \epsilon_{x}} \epsilon_{x} \epsilon_{y}^{2} \\
                                             %  &+ 6 \frac{\partial^{3} Q_z}{\partial \epsilon_{y}\partial  \epsilon_{x}\partial \delta} \delta \epsilon_{x} \epsilon_{y} 
                                            %\biggr]\\
                                             %\end{aligned}
\end{aligned}
\label{eq:detuning_effects:chromatic_ampdet}
\end{equation}


% =========
\paragraph{Sextupolar contributions}

To the first order, the terms of the chromatic amplitude detuning are shared with the classic
amplitude detuning, which are not contributed to by sextupoles. 
The last term however is the linear chromaticity, seen previously in~\ref{subsubsection:linear_chroma}.


% =========
\paragraph{Octupolar contributions}

Similar to the sextupolar contributions, to the first order, the terms are shared with amplitude
detuning. The first terms $\frac{\partial Q_z}{\partial \epsilon_x}\epsilon_x$ and $\frac{\partial
Q_z}{\partial \epsilon_y}\epsilon_y$ are then contributed to by octupoles.
The second order chromaticity $Q''$ appears when expanding to the second order.


% =========
\paragraph{Decapolar contributions}

So far, only terms with amplitude detuning and chromaticity have been seen.  
The terms highlighted in orange, in \cref{eq:detuning_effects:decapole_chromatic_ampdet}, are
the terms contributed to by decapoles. Terms depending on both the emittance and the momentum offset
are present, as well as the third order chromaticity $Q'''$.

\begin{equation}
\begin{aligned}
Q_z(\epsilon_x, \epsilon_y, \delta) = \color{gray}Q_{z0} &\color{gray}+
                                                \left[
                                                   \frac{\partial Q_z}{\partial \epsilon_x} \epsilon_x
                                                 + \frac{\partial Q_z}{\partial \epsilon_y} \epsilon_y
                                                 + \frac{\partial Q_z}{\partial \delta} \delta
                                                \right] \\
                                             &\color{gray}
                                             + \textcolor{orange}{\frac{1}{2!} \biggl[}
                                                   \frac{\partial^2Q_z}{\partial \epsilon_x^2}\epsilon_x^2 
                                                 + \frac{\partial^2Q_z}{\partial \epsilon_y^2}\epsilon_y^2
                                                 + \frac{\partial^2 Q_z}{\partial \delta^2} \delta^2  \\
                                             &\;\begin{aligned}
                                             \phantom{+ \frac{1}{2!} \biggl[}
                                               & \color{gray}
                                               + 2 \frac{\partial^2Q_z}{\partial \epsilon_x \partial \epsilon_y}\epsilon_x \epsilon_y
                                               + \textcolor{orange}{2 \frac{\partial^2Q_z}{\partial \epsilon_x \partial \delta}\epsilon_x \delta}
                                               + \textcolor{orange}{2 \frac{\partial^2Q_z}{\partial \delta \partial \epsilon_y} \delta \epsilon_y}
                                             \textcolor{orange}{\biggr]} \\
                                             \end{aligned} \\
                                             &\color{gray}+ \textcolor{orange}{\frac{1}{3!}
                                             \biggl[}
                                                  \textcolor{orange}{\frac{\partial^3 Q_z}{\partial \delta^3} \delta^{3}}
                                                  + \frac{\partial^{3} Q_z}{\partial \epsilon_{x}^{3}}  \epsilon_{x}^{3} 
                                                  + \frac{\partial^{3} Q_z}{\partial \epsilon_{y}^{3}}  \epsilon_{y}^{3} 
                                            + \cdots \textcolor{orange}{\biggr]}\\
\end{aligned}
\label{eq:detuning_effects:decapole_chromatic_ampdet}
\end{equation}

Further derivations can be found in \cref{chromatic-amplitude-detuning}.


% ============================================
%                Feed-Down
% ============================================
\subsection{Feed-Down}

When a particle passes off-center through a magnet, an effect called \textit{feed-down} appears.
Feed-down is a lower order contribution created by ether mis-aligned magnets or an off-center orbit
of the beam.
A particle with an orbit offset will then experience the main field of the magnet and effects
similar to those of lower order multipoles.