\section{Magnetic Fields}

% ============================================
%                Nomenclature
% ============================================
\subsection{\review{Nomenclature}}

Several notations coexist to denote magnetic fields. In this thesis, the
\textit{European Convention}~\cite{dilly_corrections_2022} is used for field indices, as shown
in Tab.~\ref{tab:magnetic_fields:relation_indices}. MAD-X, and MAD-NG, however, use the
\textit{American Convention}.

\begin{table}[H]
    \centering
    \begin{tabular}{l|c|c|c}
        Multipole     &      MAD-X        &     Index        & Normalized Strength \\
    \hline            
        Dipole        &     0             &     1            & $K_1$   \\       
        Quadrupole    &     1             &     2            & $K_2 $  \\
        Sextupole     &     2             &     3            & $K_3 $  \\
        Octupole      &     3             &     4            & $K_4 $  \\
        Decapole      &     4             &     5            & $K_5 $  \\
        Dodecapole    &     5             &     6            & $K_6 $  \\
        Decatetrapole &     6             &     7            & $K_7 $  \\
    \end{tabular}
    \caption{Relation between field indices and multipoles.}
    \label{tab:magnetic_fields:relation_indices}
\end{table}

As such, unless explicitly stated, quantities such as the magnetic strength $b$ and normalized
strength $K$ will be expressed with this notation. 


% ============================================
%              Multipole Expansion
% ============================================
\subsection{\review{Multipole Expansion}}

A 2 dimension magnetic field in the planes \textit{x} and \textit{y} can be described as a sum of
the normal and skew field gradients $\mathcal{B}$ and $\mathcal{A}$ with multipoles of order $n$,
given by~\cite{wolf_engineering_2001}:
\begin{equation}
    B_y + iB_x = \sum_{n=1}^\infty \left(\mathcal{B}_n + i\mathcal{A}_n \right)  (x+iy)^{n-1}
\end{equation}

An ideal magnet would produce either a sole normal or skew field. However, this is not applicable 
to real-life magnets that are imperfect, due to design and manufacturing constraints.
Field errors are thus introduced, relative to the main field of the ideal 2N-pole magnet at a
reference radius $r_{ref}$~\cite{dilly_corrections_2022}, as shown in 
Eq.~\eqref{eq:magnetic_field:relative_errors}. The coefficients of the normal and skew relative 
field errors, referred to as $a_n$ and $b_n$, are dimensionless but often given in \textit{units}
of $10^{-4}$.

\begin{equation}
    B_y + iB_x = 
        \begin{cases}
            \mathcal{B}_N \cdot \sum_{n+1}^\infty (b_n + ia_n) \left(\frac{x+iy}{r_{ref}}\right)^{n-1}\text{, for normal magnets}\\
            \mathcal{A}_N \cdot \sum_{n+1}^\infty (b_n + ia_n) \left(\frac{x+iy}{r_{ref}}\right)^{n-1}\text{, for skew magnets}
        \end{cases}
    \label{eq:magnetic_field:relative_errors}
\end{equation}


The normal and skew field components of order $n$ for an imperfect 2N-pole magnet is thus given by
the following equation:

\begin{equation}
    \begin{aligned}
        \mathcal{B}_n &= \mathcal{B}_N \cdot \frac{b_n}{r_{ref}^{n-1}}, \\
        \mathcal{A}_n &= \mathcal{A}_N \cdot \frac{a_n}{r_{ref}^{n-1}}.
    \end{aligned}
\end{equation}

The unit of the field is relative to the multipole order $n$: $[\text{Tm}^{1-n}]$.


% ============================================
%              Normalization
% ============================================
\subsection{\review{Beam Rigidity and Normalization}}

\subsubsection{\review{Beam Rigidity}}

The beam rigidity refers to the resistance of a particle moving through the accelerator to the
bending applied by the magnetic fields. It is derived from the Laurentz force~\cite{dilly_corrections_2022}
and relates the magnetic field $B$, the radius of curvature $\rho$ to the momentum $p$ and charge $q$
of the particle:

\begin{equation}
    B \rho = \frac{p}{q}
    \label{eq:magnetic_fields_beam_rigidity}
\end{equation}

It is of interest when designing an accelerator to set the maximum field as well as the required
radius of curvature for a specific momentum and particle.
An interesting metric of an accelerator is also its \textit{filling factor}, or percentage of
dipoles in the machine. It can be calculated via the radius of curvature: $f = \rho / r$. A low 
filling factors means more space for other magnets, collimators, beam instrumentation, etc.

\subsubsection{\review{Field Normalization}}

The Beam Rigidity is also used as a way to normalize magnetic field strengths in particle
accelerators where the momentum of the particle changes (i.e. acceleration).
Normalized Normal and Skew components $K_n$ and $J_n$ are given by~\cite{wolf_engineering_2001}:

\begin{equation}
    \begin{aligned}
        K_n =  \frac{q}{p} &(n-1)! \mathcal{B}_n, \\ 
        J_n =  \frac{q}{p} &(n-1)! \mathcal{A}_n.
    \end{aligned}
    \label{eq:magnetic_fields_normalized}
\end{equation}



% ============================================
%            Hamiltonian Dynamics
% ============================================
\subsection{\review{Hamiltonian Dynamics}}

The Hamiltonian describing the motion for the transverse planes of a given multipole or order $n$ is
given by~\cite{keintzel_jacqueline_beam_nodate,tomas_direct_2003,franchi_studies_2006}:

\begin{equation}
    \begin{aligned}
        H &= \frac{q}{p} \Re \left[ \sum_{n>1} (\mathcal{B}_n + i\mathcal{A}_n) \frac{(x+iy)^n}{n} \right] \\
          &= \Re \left[ \sum_{n>1} (K_n + iJ_n) \frac{(x+iy)^n}{n!} \right].
    \end{aligned}
    \label{eq:hamiltonian_magnet}
\end{equation}

Quite often, when studying the effect of a magnet on the beam, only one component is required, and
the sum can thus be dropped.
The normal and skew fields can also be isolated in order to consider their effect only:

\begin{equation}
    \begin{aligned}
        N_n &= \frac{1}{n!} K_n \Re \left[ (x+iy)^n \right] \\
        S_n &= -\frac{1}{n!} J_n \Im \left[ (x+iy)^n \right].
    \end{aligned}
    \label{eq:normal_skew_hamiltonian_magnet}
\end{equation}



