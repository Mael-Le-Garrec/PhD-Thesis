\documentclass{article}
\usepackage{amsmath}
\usepackage{amsfonts}
\usepackage{geometry}
\usepackage{fontspec}
\usepackage{tgadventor}
\usepackage{libertine}
\usepackage{titling}

\setlength{\droptitle}{-5em}
\title{\scshape{LHC Effective Model for Optics Corrections}}
\author{Maël Le Garrec}
\date{}

\geometry{a4paper, top=2.625cm, left=3.15cm, right=3.15cm, bottom=3.675cm, includehead, includefoot}
\setmainfont{TeX Gyre Adventor}
\setlength{\parindent}{12pt}
\setlength{\parskip}{6pt plus 2pt minus 1pt}

\begin{document}

\maketitle

\fontspec{Linux Libertine O}

The Large Hadron Collider (LHC) at CERN, the world's largest and most powerful particle accelerator,
operates by accelerating protons and heavy ions to nearly the speed of light and colliding them to
recreate conditions similar to those just after the Big Bang. This undertaking allows
exploring fundamental forces and particles, advancing our understanding of the universe's building
blocks. However, key to the effective operation of the LHC is addressing the complex effects of
higher-order magnetic fields, which significantly impact beam stability and dynamics. These
non-linearities, arising from small errors in the magnetic fields used to guide and focus the
particle beams, present substantial challenges that must be effectively managed to improve the
overall performance and reliability of the accelerator.

This thesis employs advanced mathematical tools, such as Lie Algebra and Poisson Brackets, to model
the intricate dynamics of particle motion under the influence of various multipole fields. Utilizing
the Hamiltonian formalism, the study derives higher-order transfer maps that characterize non-linear
effects in the beam dynamics, leading to a deeper understanding of phenomena such as chromaticity,
amplitude detuning, and resonances. Key measurements taken from the LHC include data obtained from
Beam Position Monitors (BPMs). By applying Fourier analysis to the turn-by-turn data, crucial
parameters such as phase advance, beta function, and Resonance Driving Terms (RDTs) can be
identified, thereby revealing how field errors impact the stability of the particle beam.

To enhance the understanding and control of higher-order field errors, new tools and methods have
been developed. For instance, the Non-Linear Chromaticity GUI has been introduced to assist
operators in analyzing and adjusting chromaticity in real time during operation. Furthermore, the
introduction of a response matrix approach has enabled systematic corrections of specific RDTs,
marking a significant shift from empirical corrections to data-driven solutions. This approach
allows for more refined control over higher-order multipoles, such as octupoles and decapoles,
ultimately leading to improved beam lifetime and stability.

The first study presented in the thesis focuses on skew octupolar fields and their influence on the
LHC's dynamic aperture, which defines the range of amplitudes within which the particle beam remains
stable. New correction methods based on response matrix techniques were implemented and tested,
replacing older empirical approaches. These corrections, applied across various beam energies,
address skew RDTs directly, although limitations were noted due to the absence of certain corrector
magnets. Additionally, the study uncovered that Landau octupoles, which are crucial for managing
multi-particle coherent instabilities, contribute significantly to the generation of skew octupolar
RDTs via transverse coupling, underscoring the necessity for its precise modeling.

The second study centers on the behavior of decapolar fields, particularly their impact on
third-order chromaticity and chromatic amplitude detuning at injection energy. Understanding these
fields is essential for the successful operation of future accelerators like the FCC, which will
require meticulous high-order fields control to maintain beam stability. Observations from this
study revealed the source of discrepancies between measurements and simulations of third order
chromaticity, coming from the decay of decapolar components in the main dipoles, which had
previously been neglected. By directly measuring and implementing corrections for decapolar RDTs
for the first time, the study achieved notable improvements in beam lifetime, demonstrating the
effectiveness of adapted correction techniques and their importance for stable beam operation.
Further studies on the combined effect of lower-order multipoles such as sextupoles and octupoles
were conducted, revealing a large impact of the Landau octupoles on decapolar RDTs.

The third study investigates very-high-order fields, specifically dodecapolar and decatetrapolar
fields. Utilizing a newly implemented collimation setup alongside custom post-processing techniques,
this research successfully observed these higher-order fields for the first time. Measurements
conducted within this framework identified the presence of fourth and fifth-order chromaticity
terms, suggesting that non-linear errors in the main dipoles and quadrupoles are significant
contributors to these phenomena. Moreover, the study emphasizes that accurately characterizing the
lower-order terms necessitates a comprehensive understanding of these higher-order effects, which
can enhance dynamic aperture and beam control.

A EAJADE secondment at SuperKEKB in Japan during its commissioning utilized optics
measurement techniques from CERN on the HER and LER rings. Linear optics measurements showed good
agreement with the Closed Orbit Distortion (COD) method, demonstrating repeatability over time. For
the first time, vertical plane measurements with an injection offset were conducted, yielding
promising results. The study extended to non-linear optics, including chromaticity and amplitude
detuning, revealing some discrepancies between measurements and model predictions, particularly
concerning potential unmodeled sources. Resonance Driving Terms (RDTs) were measured successfully
for the first time, although challenges remained due to factors like decoherence and damping.
Overall, the findings align with alternative KEK methods, indicating that CERN's techniques are
effective for enhancing understanding of SuperKEKB and future accelerators like the FCC-ee.

In summary, the research detailed in this thesis underscores the importance of
understanding and managing higher-order multipole effects to enhance the performance of the LHC.
Skew octupolar fields, decapolar fields, and very-high-order fields all significantly impact beam
stability and lifetime, necessitating the development and implementation of advanced measurement
techniques and correction methods to address these challenges.

As particle accelerators continue to evolve, the challenges associated with higher-order multipole
components will remain relevant. Ongoing research in this field is vital for tackling these
challenges and ensuring that future accelerators achieve the precision required for groundbreaking
scientific discoveries. The lessons learned from the LHC's experience with the complex interactions
of multipole fields will inform the design and operation of next-generation accelerators, such as
the HL-LHC and FCC, which will increasingly depend on precise control of non-linear optics. By
developing robust solutions for current operational challenges, this work contributes to advancing
the field of accelerator physics and lays the groundwork for future innovations.

\end{document}
